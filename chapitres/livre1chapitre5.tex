
\section{Du souverain bien, et de ses parties}

Il n'y a presque personne, soit en commun soit en particulier, qui dans la vie ne se propose un certain
but. Et pour arriver à ce but, que sans cesse on a en vue, chacun de son côté fait tout ce qu'il peut afin
d'acquérir et d'éviter certaines choses. Or, ce but, en un mot, est ce que nous appelons \emph{souverain
bien}, \emph{félicité}, \emph{souverain bonheur}, et tout ce qui en dépend. Afin donc qu'on en ait quelque
idée, disons en gros ce que c'est que cette félicité ou ce souverain bien, et ce qui en fait partie, puisque
tout ce qu'on emploie, et a persuader, et à dissuader, regarde toujours ou la félicité elle-même, ou ce qui
se rapporte à elle, ou qui lui est opposé. Et de fait tout ce qui est capable de nous rendre heureux absolument
ou en partie, ou qui, d'un petit bien, en peut faire naître un plus grand est toujours ce que nous devons nous
proposer de faire, comme nous devons toujours nous abstenir d'entreprendre des choses qui peuvent détruire
notre bonheur, ou l'empêcher, ou nous faire passer à un état contraire. 

Supposons donc que la félicité se rencontre:

\begin{emphpar}
	À mener une vie dont toutes les actions réussissent au contentement de celui qui les fait, sans pourtant
	s'éloigner en rien de la vertu ni du devoir d'un honnête homme;
\end{emphpar}

Ou encore, 

\begin{emphpar}
	À se voir en tel état qu'on n'ait affaire de rien;
\end{emphpar}

Ou bien,

\begin{emphpar}
	À passer si agréablement ses jours que les plaisirs n'en puissent être troublés;
\end{emphpar}

Ou enfin,

\begin{emphpar}
	À jouir d'une possession si parfaite de toutes choses qu'on soit en puissance également de les conserver
	dans le besoin et de les acquérir de nouveau si elles étaient perdues.
\end{emphpar}

Car sans doute tout le monde demeure d'accord que le souverain bien consiste: ou dans la possession de quelqu'une
de ces choses, ou de plusieurs ensemble,

Que si la félicité est véritablement ce que nous venons de dire, on doit mettre au nombre de ce qui en fait partie:
la naissance, le crédit et l'amitié des honnêtes gens, les richesses, l'avantage d'avoir des enfants parfaits et en
grand nombre, et enfin, la jouissance d'une vieillesse exempte de toute sorte d'incommodités. De plus, il y faudra
ajouter toutes les qualités excellentes du corps, par exemple: la santé, la beauté, la force, la taille, l'adresse à
toutes sortes d'exercices, et encore la gloire et les honneurs, la bonne fortune; en un mot, la vertu et tout ce qui
en dépend, savoir la prudence, la valeur, la tempérance, la justice; car il est certain qu'un homme sera souverainement
content lorsqu'il se verra possesseur, et des biens qui se trouvent dans nous-mêmes et que nous possédons en propre,
et de ceux qu'on emprunte d'ailleurs et qui sont hors de nous, puisque après ces deux sortes de biens, il n'en faut
point chercher d'autres. J'appelle biens qu'on trouve dans soi-même tout ce qui sert à l'embellissement de l'âme et à
perfectionner le corps. Et j'appelle biens étrangers et hors de nous la noblesse, les amis, les honneurs et les richesses.
Outre ces avantages, néanmoins, nous croyons encore que pour assurer entièrement le bonheur de notre vie, il est bon
d'avoir de la puissance et d'être favorisé de la fortune.

Examinons en particulier quelle est la nature de toutes ces choses, et premièrement en quoi consiste la noblesse.

\subsection{Les parties qui composent le souverain bien}

La noblesse se peut considérer en deux façons: ou à l'égard de tout un peuple ou d'un particulier seulement. Un peuple
sera remarquable par sa noblesse s'il est originaire du pays qu'il habite, ou du moins fort ancien; si ses fondateurs
ont été illustres, et s'il en est sorti quantité de grands hommes qui aient éclaté par leur sagesse, par leur valeur,
par leur justice et par tous les autres avantages qui donnent de l’émulation. La noblesse d'un particulier peut venir:
ou du côté des hommes, ou du côté des femmes, ou de tous les deux ensemble, surtout si sa naissance est légitime. Et
cette noblesse sera toujours d'autant plus considérable si, de même que nous venons de remarquer touchant les fondateurs
des états, les premiers de sa race ont été illustres pour leur vertu ou leurs grands biens, ou pour quelqu'une des autres
choses qui ont du crédit dans le monde; et non seulement si les premiers de sa race ont été illustres, mais encore si depuis,
on en peut compter beaucoup d'autres dans sa famille, aussi bien parmi les femmes que parmi les hommes; parmi les jeunes
gens et les vieillards qui aient ajouté à cette première gloire.

\bigbreak

Il n'est pas difficile de connaître en quoi consiste ce que nous appelons \emph{être heureux en enfants}. En général, donc,
ce bonheur se rencontre dans une ville ou dans un état s'il y a beaucoup de jeunesse, et qui ait de bonnes qualités, soit
que ces qualités regardent le corps, comme font la taille et la beauté, la force, et l'adresse à toutes sortes d'exercices,
soit qu'elles regardent l’âme, comme la tempérance et la valeur; car à proprement parler, ces deux vertus appartiennent aux
jeunes gens. En particulier, nous appellerons un homme heureux en enfants celui qui en aura un grand nombre, tant de l'un que
de l'autre sexe, et remarquables par toutes les qualités que nous venons de dire. Au reste les qualités qui rendent les femmes
recommandables, premièrement quant au corps, sont la beauté et la taille; en second lieu, pour l’âme et pour l'esprit, nous
recherchons aux femmes, particulièrement, la tempérance, et de plus cet amour du ménage qui ne tient point de la bassesse et
qui n'est pas indigne d'une femme de condition. De quelque façon, donc, que nous considérions la possession des enfants, tant
de l'un que de l'autre sexe, soit que nous la considérions en général ou en particulier, jamais elle ne pourra être heureuse
entièrement si ces enfants, autant les filles que les mâles, n'ont toutes les vertus et toutes les qualités que nous avons
remarquées. Et pour cela, on peut assurer de tous ceux qui ont des filles et des femmes aussi mal élevées que les Lacédémoniens
en ont qu'ils ne sont heureux en enfants qu'à demi.

\bigbreak

Quant aux richesses, ce qui en fait partie est l'argent comptant, la quantité des héritages et des belles terres; les meubles,
les troupeaux, les esclaves, surtout s'ils sont remarquables par la grandeur, par la beauté et par le nombre. Or, non seulement
pour être riche il faudra posséder toutes ces choses, mais encore il faudra que la possession en soit sûre, honnête et profitable
tout ensemble. Une chose est profitable lorsqu'elle est de rapport, et elle est honnête lorsqu'on ne s'en sert que pour le plaisir.
J'appelle possession de rapport celle dont nous tirons du revenu, et je nomme possession pour le plaisir simplement, celle qui n'a
rien de plus considérable que l'usage. Enfin nous possédons en assurance une chose, lorsque nous en jouissons en tel lieu et de
telle sorte que nous pouvons en user comme il nous plaît, et de plus, quand la propriété nous en appartient. On possède en propre une
chose lorsqu'on la peut aliéner. J'appelle aliéner la vendre ou la donner. Après tout, il ne faut pas penser que la qualité de riche
dépende plus de la possession des richesses que de leur usage, car tant s'en faut que cela soit, que même se servir de son bien est
proprement ce que nous appelons être riche.

\bigbreak

La gloire et la réputation consistent à passer pour homme de bien dans l'esprit de tous les hommes; et encore à être cru possesseur
d'un avantage ou que tout le monde souhaite passionnément, ou du moins les plus honnêtes gens, ou les personnes d'esprit. 

\bigbreak

L'honneur est un témoignage d'estime qu'on rend à ceux qui sont bienfaisants. De là vient qu'on honore principalement les personnes
qui ont du bien; et quoi qu'il fût juste de ne porter de l'honneur qu'à ces gens-là, on ne laisse pas d'honorer encore ceux qui sont
en puissance de bien faire. Au reste, le bienfait regarde toujours ou la vie et tout ce qui peut être cause de sa conservation, ou les
richesses, ou enfin quelqu'un des autres avantages dont l'acquisition est difficile à faire, soit absolument, soit en certain lieu ou
en certain temps. Et c'est aussi pourquoi souvent nous voyons rendre beaucoup d'honneur et faire de grandes soumissions à des personnes
pour de très petites choses en apparence, seulement à cause que l'occasion ou la difficulté de les faire les avaient rendus considérables.
Les parties de l'honneur, ou les manières différentes d'honorer sont: les sacrifices, les inscriptions publiques soit en vers ou en prose,
les récompenses, les lieux consacrés, les préséances, les tombeaux, les statues, les pensions qu'on a du public; à quoi l'on peut ajouter
ce que pratiquent les nations étrangères quand elles veulent honorer quelqu'un, par exemple se prosterner contre terre ou se retirer d'un
chemin quand on passe. Il faut encore mettre les présents au nombre des choses qui sont en honneur. De vrai, le présent est de telle nature
qu'en même temps, il est: et la donation d'une chose et une marque d'estime; aussi les avares et les ambitieux en sont-ils grands amateurs,
à cause qu'ils y trouvent ce qu'ils cherchent. Les avares y rencontrent l'acquisition, et les ambitieux l'honneur, qui est ce que tous deux
demandent. 

\bigbreak

La santé est proprement la vertu du corps. Il faut néanmoins la posséder de manière que nous puissions faire toutes sortes
de fonctions sans en être malades; car il y en a beaucoup qui jouissent de la santé comme faisait Hérodicos, qu'on ne peut
pas dire être heureux en cet état, à cause qu'il faut qu'ils s'abstiennent de tout qui rend notre vie commode et agréable,
ou de la plus grande partie.

\bigbreak

Pour la beauté, elle est différente à raison des âges différents. La beauté d'un jeune homme est d'avoir le corps propre
à toutes sortes d'exercices, soit à la course et aux autres actions qui demandent de la force. Il faut encore qu'il soit
agréable à voir, et si agréable même, qu'on ne puisse se lasser de le regarder. Pour cette raison, les athlètes propres à
la course et à se battre sont très beaux. La beauté d'un homme fait est de pouvoir supporter toutes les fatigues de la guerre
et d'avoir je ne sais quoi dans le visage qui le rende agréable à voir et redoutable tout ensemble. Enfin, celle d'un
vieillard consiste à pouvoir faire toutes les fonctions nécessaires, et cela sans se plaindre, comme ne sentant aucune des
incommodités qui affligent d'ordinaire la vieillesse.

\bigbreak

La force consiste à tourner et manier quelqu'un comme on veut, ce qui se fait de cinq façons: ou en le tirant, ou en le
poussant, ou en l'élevant, ou en le terrassant, ou en l’étreignant. Car on ne peut pas dire qu'un homme soit fort, s'il
ne fait tout ceci, ou une partie.

\bigbreak

À l'égard de la belle taille, c'est quand on surpasse presque tous les autres, ou en hauteur, ou en largeur, ou en
grosseur, en sorte néanmoins que cet excès ne rende pas le corps plus pesant ni plus tardif dans tous ses mouvements.

\bigbreak

Pour réussir au métier d'athlète, qui comprend trois sortes d'exercices --- savoir la lutte, la course et le combat des poings
--- le corps doit avoir ces trois avantages: la taille, la force et l'agilité. Car tout homme qui est agile est fort.

Au reste, quiconque peut jeter les jambes d'une certaine manière, les avancer loin et promptement, est propre à la course.

Celui qui peut étreindre son homme et le tenir ferme, est né pour la lutte.

Enfin, pouvoir, à force de poings, repousser un adversaire et le faire toujours reculer, c'est ce qu'il faut au combat des
poings.

Entre les athlètes, quelques-uns réussissent aux poings et à la lutte tout ensemble, et d'autres sont adroits à toutes ces
trois sortes d'exercices.

\bigbreak

La vieillesse commode est celle qui vient tard et qui ne fait rien souffrir. Pour en jouir, donc, il ne faudra pas vieillir
de trop bonne heure. Aussi ne suffira-t-il pas de vieillir tard si en même temps on n'est exempt de toutes sortes
d'incommodités. Or, cet avantage ne dépendra pas seulement des qualités excellentes du corps, mais encore de la bonne
fortune. Car, qu'un homme soit sujet aux maladies et de faible complexion, le moyen qu'il ne souffre jamais? Et s'il doit
être incommodé, comment est-il possible que sans un grand bonheur, il puisse vivre longtemps en cet état. J'avoue
véritablement que, sans la santé et la bonne constitution, on ne laisse pas de vivre quelquefois assez longtemps puisque,
tous les jours, il se voit des gens privés de tous les avantages du corps arriver à de longues années. Mais ce n'est pas
ici qu'il faut donner une exacte connaissance de cette matière.

\bigbreak

Pour ce qui est du crédit et d'avoir l'amitié des honnêtes gens, ceci sera facile à connaître quand nous aurons déclaré ce
que nous entendons par le mot d'ami. Tout homme, donc, qui tâchera par toutes sortes de moyens de procurer à un autre ce
qu'il juge lui être avantageux, sans autre motif que de le vouloir obliger et seulement parce qu'il l'aime, c'est là
proprement ce que nous appelons être ami. Or, quiconque aura beaucoup de personnes disposées de cette sorte à son égard,
se pourra vanter d'avoir du crédit et beaucoup d'amis. Et si ces mêmes personnes ont du mérite et de la vertu, pour lors,
il aura l'amitié des honnêtes gens.

\bigbreak

On appelle bonne fortune quand il arrive à une personne: ou qu'il lui est arrivé tous les biens et les avantages dont la
fortune est la cause ordinairement, ou du moins quand de tous ces biens, il lui en est arrivé la meilleure partie et les
plus considérables. La fortune, au reste, peut être cause quelquefois des mêmes biens et nous procurer les mêmes avantages
que ceux que notre adresse et les arts nous procurent, quoi que d'ordinaire, la plupart de ceux qui viennent d'elle ne
soient nullement au pouvoir des arts, comme sont les biens de la nature. Quelquefois encore, elle est cause de certains
biens qui arrivent extraordinairement et en quelque façon contre le dessein de la nature même. Par exemple, la fortune est
quelquefois cause de la santé qui est un bien dépendant de la médecine, et cette même fortune, bien souvent, est cause de
la beauté et de la taille, qui sont des avantages purement dépendants de la nature. Mais en général, on peut nommer biens
de la fortune tous ceux qui sont sujets à l'envie. Outre ces biens, la fortune en donne encore d'autres quelquefois contre
toute sorte de raison et d'apparence, comme il arrive, quand entre plusieurs frères qui sont très laids, il s'en rencontre
un parfaitement beau; ou lorsque de plusieurs qui cherchent un trésor, il n'y en a qu'un qui le trouve; ou encore quand
une flèche qui a été tirée épargne celui-ci et en blesse un autre tout contre; ou enfin, lorsqu'une personne qui avait
accoutumé d'aller seule en certain lieu s'abstient d'y aller dans le temps que plusieurs qui étaient allés pour la première
fois y périssent. Car il semble que toutes ces choses-là soient de purs effets de la bonne fortune.

\bigbreak

Touchant la vertu, parce qu'elle regarde la louange particulièrement, nous remettons à faire savoir ce que c'est quand nous
serons au genre démonstratif.

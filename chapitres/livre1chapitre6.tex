
\section{De la fin du genre délibératif}
\subsection{Avec les lieux qui servent à prouver qu'une chose est bonne et utile}


Présentement, l'on voit quelles sont les choses à quoi il faut avoir égard lorsqu'on a à persuader, soit que ces choses-là
soient arrivées, ou aient à arriver. Et de même en est-il pour dissuader, puisqu'il n'y a qu'à prendre le contraire. 

Mais parce que celui qui délibère a toujours pour but ce qui est utile; d'ailleurs que jamais personne ne délibère de la
fin, mais seulement des moyens pour y arriver, et que ces moyens-là, c'est ce qui est utile touchant le dessein qu'on a;
enfin parce que ce qui est utile est toujours un bien et un avantage. Pour cela il faut que nous donnions ici quelques
notions du bien en général, et de ce qui est utile afin d'en tirer des propositions.

Supposons donc: 

\begin{emphpar}
     Que le bien est une chose souhaitable à cause d'elle-même,

     Ou, qui, pour l'avoir, oblige à en rechercher d'autres auxquelles on ne penserait jamais,

     Ou généralement, que c'est ce que souhaite tout ce qui est au monde, ou du moins tout ce qui a des sentiments ou de
     la raison; et même ce que souhaiterait tout ce qui est privé de raison s'il en avait.
\end{emphpar}

Disons encore:

\begin{emphpar}
     Que le bien est tout ce que la raison nous représente comme tel,
\end{emphpar}

Et encore:

\begin{emphpar}
     Que tout ce qu'elle nous représente comme un bien en chaque rencontre particulière, cela même nous est toujours
	 avantageux\footnote{Chaque fois qu'il m'arrive une couille, faut que je picole! Sans quoi je ne suis pas heureux.}.
\end{emphpar}

Ajoutons:

\begin{emphpar}
     Que le bien est ce qui, par sa présence, fait qu'on se trouve tout autre et si content qu'on ne souhaite rien au
	 delà\footnote{Tout se passe comme si tu avais perdu toute forme de désir. Tu étais beau, avant, mais tu ressembles
	 à un robot, programmé pour jouir.},

     Ou, ce qui tout seul nous suffit,
\end{emphpar}

Et même:

\begin{emphpar}
     Que c'est ce qui peut être cause de tous ces biens que nous venons de dire,

	 Ou qui les peut conserver,

	 Ou qui en est toujours suivi.
\end{emphpar}

\bigbreak

Supposons enfin:

\begin{emphpar}
     Que le bien est tout ce qui peut éloigner ou détruire ce qui est contraire aux avantages que nous avons remarqués.
\end{emphpar}

En passant, nous observerons qu'une chose peut être suivie d'une autre en deux manières, ou en même temps, ou quelque temps
après. Par exemple l'étude est suivie de la science quelque temps après, parce que pour être savant il faut auparavant
avoir étudié. Et la vie suit toujours la santé en même temps, puisqu'il n'est pas possible qu'on jouisse de la santé et
qu'en même temps, on soit privé de la vie.

Nous observerons aussi qu'une chose peut être cause d'une autre en trois façons: Ou \emph{formellement et par elle-même},
ainsi la santé est toujours cause qu'une personne est saine; ou \emph{en réparant ce que cette chose perd}, ainsi les
aliments sont cause encore de la santé. Ou bien enfin \emph{en la conservant}, et de cette sorte l'exercice est cause de
la santé, parce que d'ordinaire la santé en dépend.

\bigbreak

Suppose donc que le bien soit véritablement ce que nous venons de dire. Il sera nécessaire de tirer ces conséquences:
Premièrement, que l'acquisition d'un bien et la délivrance d'un mal seront des choses avantageuses puisque d'un côté,
acquérant un bien, on n'aura pas en même temps le mal qui lui est contraire; et d'un autre côté, étant délivré d'un
mal, on aura après le bien qui le suit. 

\bigbreak

En second lieu:

\begin{emphpar}
     Que l'échange d'un petit bien pour un plus grand, ou l'échange d'un grand mal pour un plus petit seront encore de
	 grands avantages\footnote{Notez qu'il m'est impossible de me plier à la règle du jeu avec un truisme. Et pourtant,
	 beaucoup d'usages de ce lieu, ainsi que d'autres truismes, seront trompeurs. En effet, la première prémisse
	 impliquera que la comparaison est certaine, ou presque, ce qui n'arrive pas souvent. Ce qui arrive souvent, en
	 revanche, c'est qu'on croie à cette certitude sur la base des données passées. Pensez à l’œuvre de Nassim Taleb,
	 ou plus simplement, aux vendeurs de miracles: <<Si vous suivez ma formation à 1000 euros, votre épargne sera
     multipliée par dix en dix ans! Réfléchissez-y: il suffit que vous possédiez 1600 euros pour y gagner!>>},
\end{emphpar}

Puisque d'une part, il sera vrai d'assurer qu'autant que ce grand bien aura d'avantage sur le petit, autant de bien
aura-t-on acquis qu'on n'avait pas; et d'autre part, qu'autant que ce petit mal sera moindre que le plus grand,
d'autant de mal sera-t-on délivré qu'on n'aura plus.

\bigbreak

On pourra aussi inférer:

\begin{emphpar}
     Que généralement, toutes les vertus seront des biens\footnote{Le Galérien: \begin{emphpar}\normalfont
	 Hélas! Que ne me suis-je épargné ces travaux pénibles en présence des rats et de la peste?
	 Eussé-je abjuré la religion de mes frères, quels honneurs n'eussé-je pas reçu?\end{emphpar}},
\end{emphpar}

Puisque ceux qui les possèdent se trouvent contents en cet état et que, d'ailleurs, elles sont cause qu'il leur arrive
ensuite beaucoup d'autres avantages; et même qu'elles les rendent capables de faire du bien aux autres. Mais nous
parlerons de cette matière à part en un autre endroit, où il sera traité de chaque vertu en particulier, et de sa
différence.

\bigbreak

De plus on soutiendra:

\begin{emphpar}
     Que le plaisir est un bien,
\end{emphpar}

Parce que naturellement, tous les animaux le recherchent.

\bigbreak

Et par la même raison:

\begin{emphpar}
     Toutes les belles choses, et qui sont agréables,
\end{emphpar}

Car tout ce qui est agréable nous apporte du plaisir. Quant aux choses qui sont belles, il faut remarquer que les unes sont
agréables simplement, et les autres honnêtes et souhaitables pour l'amour d'elles-mêmes. 

\bigbreak

Enfin, pour ne rien oublier, et pour nommer tous les biens les uns après les autres, il faudra mettre encore au nombre des
biens: premièrement,

\begin{emphpar}
     Le souverain bien,
\end{emphpar}

Vu qu'il est souhaitable à cause de lui-même et qu'il peut satisfaire pleinement, et que pour l'acquérir nous n'épargnons rien
de tout ce qui est en notre pouvoir.

\bigbreak

Secondement, il y faudra mettre:

\begin{emphpar}
     La justice, la valeur, la tempérance, la grandeur de courage, la magnificence, et pareilles habitudes,
\end{emphpar}

À cause que ce sont là les vertus de l'âme. 

\bigbreak

Il y faudra encore ajouter:

\begin{emphpar}
     La santé la beauté, et telles choses semblables,
\end{emphpar}

Puisque non seulement ce sont les vertus du corps et les qualités qui le perfectionnent, mais encore parce qu'elles sont capables
de nous faire entreprendre beaucoup de choses, et même de les exécuter. Par exemple, la santé est un bien, parce qu'elle est la
source de tous les plaisirs et de la vie même. Aussi est-ce principalement ce qui la fait passer pour un bien excellent, à cause
qu'elle est en même temps le principe des deux seules choses que le vulgaire estime le plus au monde, qui est de vivre, et de vivre
avec plaisir, 

\bigbreak

\begin{emphpar}
     Les richesses encore sont à mettre au rang des biens, 
\end{emphpar}

Puisqu'il y a de la vertu à s'en bien servir et que par leur moyen, on peut faire bien des choses.

\bigbreak

Pareillement:

\begin{emphpar}
      Les amis et l'amitié,
\end{emphpar}

Car un ami est toujours souhaitable à cause de lui-même, joint qu'il peut beaucoup servir.

\bigbreak

\begin{emphpar}
      L'honneur encore et la gloire sont des biens,
\end{emphpar}

Car outre qu'il est agréable de les posséder, et qu'ils nous peuvent servir beaucoup, c'est qu'il arrive d'ordinaire que les mêmes choses
qui nous font rendre de l'honneur se trouvent véritablement en nous.

\bigbreak

\begin{emphpar}
      Savoir parler et agir sont encore des biens,
\end{emphpar}

Puisque ces choses-là peuvent nous procurer de très grands avantages. 

\bigbreak

Il faut dire le même:

\begin{emphpar}
      Du bel esprit, de la mémoire, de la docilité, de la vivacité et de telles autres qualités,
\end{emphpar}

Car tout cela peut beaucoup contribuer à notre fortune et nous mettre en état de faire de grandes choses. 

\bigbreak

On doit aussi mettre au rang des biens:

\begin{emphpar}
      Toutes les sciences et les arts, comme aussi la vie,
\end{emphpar}

Puisque quand nous n'aurions autre avantage que de vivre, il ne faudrait pas laisser de souhaiter la vie à cause d'elle-même.

\bigbreak

Enfin, nous devons tenir pour bien:

\begin{emphpar}
      Tout ce qui est juste,
\end{emphpar}

Attendu qu'il regarde l'utilité publique.

\subsection{Biens douteux ou controversés et pour les faire valoir}

Quant aux autres choses à qui la qualité de bien est contestée, la preuve s'en pourra faire ainsi.

\bigbreak

Premièrement:

\begin{emphpar}
      Que tout ce qui a pour son contraire un mal est un bien\footnote{La déflation est un appauvrissement, mais je
	  connais un truc: Faites chauffer la planche à billets!}. 
\end{emphpar}

En second lieu:

\begin{emphpar}
      Tout ce qui a pour son contraire une chose dont les ennemis tirent de l'avantage,
\end{emphpar}

Par exemple s'il est utile aux ennemis que nous soyons poltrons, sans doute la valeur nous sera fort avantageuse.

Et généralement enfin:

\begin{emphpar}
      Tout ce qui sera contraire aux choses que les ennemis souhaitent, ou qui leur donnent de la joie, apparemment,
	  nous doit être utile.
\end{emphpar}

De là vient que Nestor, dans Homère, voulant réconcilier Achille et Agamemnon de qui la division allait ruiner
l'entreprise des grecs devant Troie, allègue d'abord comme un moyen très capable de les toucher:

\begin{emphpar}
      Quelle joie a Priam, s'il apprend ce désordre!
\end{emphpar}

Il faut pourtant remarquer que ceci n'est pas toujours vrai, mais seulement pour l'ordinaire, puisque enfin, rien
n'empêche que quelquefois une même chose ne soit utile à deux ennemis en même temps, d'où est venu le proverbe que:
\emph{souvent, les maux portent à la réconciliation et rendent les hommes amis}; ce qui se doit entendre lorsque la
même chose est dommageable également aux uns et aux autres.

\bigbreak

De plus, il y aura lieu de soutenir:

\begin{emphpar}
      Que tout ce qui n'est point dans l’excès est un bien, puisque tout ce qui est excessif et plus grand qu'il ne
	  faut est un mal.
\end{emphpar}

\bigbreak

Comme encore:

\begin{emphpar}
      Tout ce qui nous aura fait prendre beaucoup de peine et obligé à une grande dépense\footnote{25 décembre 2009,
	  au repas de noël: <<~Garde tes actions de la Société Générale. Maintenant que Kerviel est en prison, ça ne peut
	  que remonter!~>> Depuis, le même conseil revient à chaque noël et s'en trouve renforcé.}.
\end{emphpar}

Et certainement pourrait-on dire que ces choses-là n'eussent pas toutes les apparences d'un véritable bien, puisque en
effet, elles seront le but et la fin de toute cette dépense et de tous ces grands travaux. Car ce qui tient lieu de fin
est toujours un bien. Aussi est-ce la raison qui oblige Homère de faire dire à Junon, lorsque les Grecs sont prêts de s'en
retourner et de lever honteusement le siège de devant Troie:

\begin{emphpar}
      Quoi donc? De leur retour, les Grecs trop désireux,
	  
	  Oublieront en fuyant tant d'exploits généreux?

	  Les Troyens à leur honte auront donc la victoire,
	  
	  Et Priam pour jamais se verra plein de gloire?
\end{emphpar}

Il fait dire encore à Ulysse en un autre endroit, parlant à l'armée des Grecs pour les faire opiniâtrer à ce siège:

\begin{emphpar}
      Quelle honte, guerriers, à tant de combattants,

	  De n'être pas vainqueurs après un si long temps,
	  
	  Et de s'en retourner sans honneur et sans gloire!
\end{emphpar}

C'est encore ce qui a donné lieu au Proverbe:

\begin{emphpar}
	   Casser sa cruche à la porte. 
\end{emphpar}

\bigbreak

On pourra soutenir de même:

\begin{emphpar}
	   Que ce que quantité de personnes souhaitent passionnément, ou qui mérite en apparence qu'on en dispute la
	   possession et qu'on se batte pour l'avoir, est un bien\footnote{D'où le truc très connu: écrire <<plus d'un
	   million d'exemplaires vendus!>> en couverture\dots{} Et sinon, le désir mimétique de René Girard, on en
	   parle?}.
\end{emphpar}

Cette proposition doit passer pour certaine, suivant une des définitions du bien que nous avons données, vu qu'alors
il a été dit que le bien était \emph{une chose que généralement tous les hommes souhaitaient}. Il est vrai que cette
proposition est conçue en des termes moins universels, mais quand on dit un grand nombre, ou la plupart, il semble en
quelque façon qu'on veuille dire: tout le monde.

\bigbreak

Ce raisonnement encore sera plausible:

\begin{emphpar}
	   Que tout ce qui est louable est un bien,
\end{emphpar}

À cause que personne ne se met en peine de louer une chose qui n'a rien de bon en soi. 

\bigbreak

Toute action encore passera pour bonne:

\begin{emphpar}
	   Qui tire des louanges de la bouche même des ennemis et des plus méchants.
\end{emphpar}

Car qui pourrait dire alors que cette action ne fût pas dans une approbation générale, quand ceux qui ont le plus d'intérêt
d'en dire du mal pour leur avoir été préjudiciable eux-mêmes en disent du bien? Il est certain que jamais ils n'en auraient
fait cette estime si la vérité ne les y avait forcés. Ce fondement est si vrai, que c'est par cette raison qu'on tient pour
méchants ceux qui sont blâmés de leurs amis, et tout au contraire pour honnêtes gens et pour vertueux ceux qui obligent même
leurs propres ennemis à les louer. Et c'est de cette manière que Simonide loua un jour les Corinthiens dont néanmoins ils se
tinrent fort offensés; c'est quand il dit:

\begin{emphpar}
	   Et quoique tu sois Grecque, ô fameuse Corinthe,
	   
	   Ce n'est point contre toi qu'Illion fait sa plainte.
\end{emphpar}

\bigbreak

On pourra encore proposer comme excellent:

\begin{emphpar}
	   Tout ce qu'une personne très sage ou un très homme de bien ou une honnête femme auront jugé tel\footnote{À l'instar de
	   Descartes, nous croyons que la révolution de la terre tire son origine de ce qu'elle est entraînée par un tourbillon dû
	   à la rotation du soleil.}.
\end{emphpar}

Ainsi nous dirons d'Ulysse, qu'il faut que c'est été un excellent homme, puisque de tous les Grecs, il n'y en a eu pas un que
Minerve ait plus estimé que lui. Ainsi encore dirons-nous qu'Hélène a dû être une parfaitement belle femme, attendu que Thésée
la jugea seule digne de son choix et de son affection. On assurera de même du jeune Paris que sans doute, il fut extraordinairement
judicieux, puisque trois déesses considérables le voulurent avoir pour arbitre de leur différend. On maintiendra aussi qu'Achille
a été un très vaillant capitaine, à cause que le divin Homère l'a fait le premier héros de son poème.

\bigbreak

On mettra encore de ce rang:

\begin{emphpar}
	   Tout ce que d'ordinaire, on préfère aux autres choses.
\end{emphpar}

Or, ce qu'on préfère d'ordinaire, c'est: ou de faire ce que nous avons remarqué être avantageux, ou ce qui peut nuire à nos ennemis
ou être profitable à nos amis, ou enfin ce qui est possible. Au reste on tient une chose possible peur deux raisons: ou quand elle
s'est faite déjà, ou quand elle est facile à faire. Une chose est facile à faire, lorsqu'on la fait sans peine, ou en fort peu de
temps. Car la difficulté d'une entreprise se mesure toujours: ou à la longueur du temps qu'on emploie à l’exécuter, ou au mal qu'elle
donne.

\bigbreak

Il y aura lieu encore de soutenir:

\begin{emphpar}
	  Que tout ce qui se fait comme on veut est un bien\footnote{Demandez le journal! Justice est faite contre Batman! Le criminel
	  a été tué par le Joker en légitime défense!}.
\end{emphpar}

Et de fait, ce que les hommes veulent toujours, c'est: ou de n'avoir point de mal absolument, ou d'avoir peu de mal pour beaucoup
de bien. Ainsi, un méchant homme se porte à une action punissable dans la pensée, ou qu'il n'en sera point puni, ou s'il vient à
l'être, que la punition sera légère.

\bigbreak

Cet autre raisonnement encore pourra servir:

\begin{emphpar}
	  Que les choses que nous posséderons en propre ou que personne n'aura que nous, ou qui excelleront par-dessus toutes les autres
	  seront bonnes.
\end{emphpar}

À cause qu'il y aura plus d'honneur à les posséder.

\bigbreak

Comme aussi:

\begin{emphpar}
	  Tout ce qui nous conviendra particulièrement\footnote{De père en fils, on est médecin, depuis sept générations. Il serait temps
	  de changer, vous ne croyez pas?}.
\end{emphpar}

Par exemple, tout ce qui nous sera bienséant, ou à cause de notre naissance, ou à cause de nos grands emplois.

\bigbreak

Pareillement:

\begin{emphpar}
	  Toutes les choses que nous croirons nous manquer, pour petites qu'elles soient,
\end{emphpar}

Puisqu'on ne se met pas moins en peine d'acquérir celles-là que les autres qui sont d'une plus grande importance.

\bigbreak

On fera passer encore pour de bonnes choses:

\begin{emphpar}
    Celles dont on peut venir à bout aisément{Vous voulez vous débarrasser de ce pays d'emmerdeur, Monsieur le
	président? Vous avez les codes et le bouton est là.},
\end{emphpar}

Car non seulement elles sont possibles, mais encore faciles à faire. Au reste, nous croyons pouvoir aisément, venir à
bout d'une chose, lorsque tout le monde l'a déjà faite, ou quantité de personnes, du moins nos pareils, ou ceux qui ne
nous valent pas.

\bigbreak

Une chose encore paraîtra avantageuse et à entreprendre:

\begin{emphpar}
	  Qui sera agréable à nos amis ou fera dépit à nos ennemis.
\end{emphpar}

\bigbreak

Et encore: 

\begin{emphpar}
	 Tout ce que les personnes d'un haut mérite et qu'on estime infiniment au-dessus des autres d’ordinaire se
	 proposent de faire.
\end{emphpar}

\bigbreak

De plus:

\begin{emphpar}
	 Toutes les chose pour lesquelles il semble qu'on soit né, ou dont on a une très grande expérience,
\end{emphpar}

Puisque c'est d'ordinaire en de telles rencontres que les hommes se promettent plus de succès. 

\bigbreak

Nous pourrons encore faire valoir:

\begin{emphpar}
	 Tout ce que les personnes de néant et de basse condition ne peuvent faire,
\end{emphpar}

Vu qu'alors, il y aura d'autant plus de gloire à entreprendre ces choses qu'elles seront hors du commun et au-dessus
de la portée des hommes ordinaires. 

\bigbreak

Enfin, l'on fera passer pour bon:

\begin{emphpar}
	 Tout ce qu'ordinairement, on souhaite,
\end{emphpar}

Car outre qu'on y trouve du plaisir, c'est que même on ne croit pas qu'il y ait rien de meilleur.

\bigbreak

Mais surtout, une chose sera aisée à proposer comme excellente à une personne:

\begin{emphpar}
	 Si c'est particulièrement sa passion et ce qu'elle souhaite le plus au monde,
\end{emphpar}

Par exemple comme est la victoire a un ambitieux, l'argent à un avare, et ainsi des autres.

\bigbreak

Ce sont donc là les lieux qui doivent fournir des propositions quand on aura à montrer qu'une chose est bonne et utile.

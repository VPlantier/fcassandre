
\chapter{Le genre judiciaire}
\section{Ce que c'est que faire tort ou injure}

Maintenant, il s'agit de l'accusation et de la défense et de donner à connaître le nombre et la qualité
des lieux dont le genre judiciaire se sert pour tirer ses arguments. Mais auparavant, il est important de
savoir ces trois points. 

\begin{emphpar}
	Quelles choses portent les hommes à se nuire et combien il y en a;

	Qui sont ceux qui les font et les dispositions qu'ils ont à ceci;

	Enfin, à quelles personnes ils s'attaquent et en quel état il faut qu'ils les trouvent.
\end{emphpar}

Ce que nous tâcherons d'éclaircir après avoir expliqué ce que c'est que faire injure à quelqu'un. 

On appelle faire injure \emph{quand volontairement, on nuit à un autre contre la défense de la loi}. Or,
il y a deux sortes de lois: les unes particulières et les autres communes. J'appelle lois particulières celles
qui sont écrites et qui servent de règle dans un état. Et j'appelle communes toutes celles qui ne sont point
écrites et qui semblent avoir été établies du commun consentement de tous les peuples.

Toute personne, au reste, agit volontairement lorsqu'elle sait bien ce qu'elle fait, et qu'elle n'y est point
forcée. Ce n'est pas que tout ce qui se fait volontairement se fasse toujours de dessein et de propos délibéré,
mais bien ce qui se fait de propos délibéré et de dessein se fait toujours volontairement et avec connaissance
de cause; puisque enfin, il n'est pas possible qu'un homme ignore la chose qu'il se propose de faire plutôt qu'une
autre et à laquelle il se détermine.

\bigbreak

Or, de savoir pourquoi on est porté à faire du mal et à nuire contre la défense des lois, cela vient de deux causes:
du vice ou de la passion. Car il se remarque que tous les vicieux généralement, soit qu'ils aient plusieurs vices ou
qu'ils  n'en aient qu'un, jamais, presque, ne sont injustes ni malfaisants qu'en ce qui touche le vice qui leur
commande. Ainsi, \emph{l'avare} n'est guère porté à mal faire qu'à cause de l'argent, ni le \emph{débauché} que parce
qu'il espère jouir de quelque plaisir, ni le \emph{fainéant} qu'afin d'avoir de quoi flatter sa paresse, ni le
\emph{poltron} que pour s'éviter quelque danger (car la crainte fait toujours abandonner aux lâches leurs compagnons
dans le danger), ni \emph{l'ambitieux}, que lorsqu'il y va de l'honneur, ni un \emph{homme prompt} que dans les
transports de sa colère, ni \emph{celui qui aime à vaincre} que lorsqu'il s'agit dé la victoire, ni un \emph{vindicatif}
qu'a cause de la vengeance, ni \emph{celui qui n'a point d'esprit} que par sa bêtise et sa stupidité pour ne pas discerner
ce qui est juste d'avec ce qui ne l'est pas et sans cesse s'y tromper, ni \emph{l'effronté}, enfin, et \emph{l'impudent},
que parce qu'il a toute honte perdue et qu'il se moque de la réputation. Et ainsi en est-il de tous les autres vicieux à
l'égard du vice auquel ils sont sujets. Tout ceci au reste nous est déjà connu en partie par ce qui a été dit de la vertu
et le sera pleinement quand nous aurons à traiter des passions. Il nous reste à faire voir pour quelle raison et de quel
esprit sont portés ceux qui font du tort, et à quelles personnes ils s'attaquent.

\subsection{Actions humaines et leurs causes}

Donnons donc à connaître en premier lieu les choses que, souhaitant d'avoir ou d'éviter, ensuite nous tâchons de nuire à
notre prochain et lui faire tort. Car, sans difficulté, tout avocat qui accuse doit regarder principalement combien de ces
choses-là qui tentent les hommes et les portent à faire du tort donnent prise sur sa partie adverse. Comme au contraire,
c'est à l'avocat qui défend, d'examiner en sa partie combien elle est éloignée de soupçon touchant ces choses-là, afin
de la justifier.

\bigbreak

Il faut donc savoir, qu'absolument, \emph{il n'y a rien que les hommes ne fassent}, et que de tout ce qu'ils font, il y en
a une partie \emph{qu'ils ne font point d'eux-mêmes} et une autre \emph{qu'ils font d'eux-mêmes} et de leur propre mouvement.

De plus, que ce qu'ils ne font point d'eux-mêmes, une partie se fait par hasard et l'autre par nécessité.

Et enfin, que ce qu'ils font par nécessité, il y en a encore une partie ou ils sont violentés, et l'autre qu'ils font par nature.

De manière que tout ce que les hommes ne font point d'eux-mêmes peut être rapporté à trois causes principales: au hasard, à la
nature et à la contrainte.

\bigbreak

Pour les choses que les hommes font de leur propre mouvement et dont ils sont eux-mêmes les auteurs, elles sont de quatre
sortes. Car il y en a une qu'ils font par coutume et l'autre par inclination parce que leur appétit les y porte, mais
comme il y a deux sortes d'appétits dans l'homme, l'un d'animal ou sensuel et l'autre raisonnable, il se trouve que tantôt
ils suivent l'appétit raisonnable et tantôt le sensuel.

L'appétit raisonnable, au reste, n'est autre chose que notre volonté, qui est définie, \emph{un appétit ou un désir du bien
conduit et réglé par la raison}, car il est certain que jamais personne ne veut une chose, que parce qu'il croit que c'est
son bien. 

L'appétit sensuel se partage en deux et reconnaît deux principes différents: l'un de la colère et l'autre de la convoitise.

\bigbreak

Si bien qu'à examiner les causes de toutes nos actions, il ne s'en trouve que sept, puisque enfin, tout ce que les hommes
font, en leur vie doit être rapporté:

\begin{emphpar}
	Ou au hasard,

	Ou à la contrainte,

	Ou à la nature,

	Ou à l'accoutumance,

	Ou au raisonnement,

	Ou à la colère,

	Ou à la convoitise,
\end{emphpar}

Car de faire une plus longue division et de vouloir encore distinguer les actions des hommes par les ages différents,
par les habitudes et par telles autres qualités, ce serait une chose superflue, étant certain que s'il arrive aux
jeunes gens d'être colères ou adonnés à leurs plaisirs, ce n'est point à leur jeunesse qu'il s'en faut prendre, mais
à leur passion bouillante et à leur convoitise déréglée. Ainsi en est-il des riches et des pauvres, car ce n'est point
leur pauvreté ni leurs richesses qu'il faut accuser. Et quoi qu'on puisse dire que, quelquefois, les pauvres ne souhaitent
avec passion d'avoir de l'argent qu'à cause qu'ils sont dans la nécessité, ni les riches que parce qu'étant dans l'abondance
et en puissance de faire tout ce qu'ils veulent, ils viennent alors à former une infinité de vains désirs et à rechercher
des plaisirs qui ne sont point nécessaires, ce n'est pas une conséquence pour cela que de tels désirs soient des effets
de leurs richesses simplement, ou de leur pauvreté, mais bien de leur passion et de leur convoitise. On doit assurer le
même des personnes justes et injustes, et généralement de tous ceux que nous disons agir par quelque habitude de cette
qualité, vu que tout ce que ces gens-là font ne peut être rapporté à d'autres causes qu'à celles qui sont été remarquées,
puisqu'il faut toujours que ce soit: ou parce qu'ils sont persuadés de quelque raison, ou parce qu'ils se laissent emporter
à leurs passions. Toute la différence qu'il y a, c'est que les uns ont des mœurs et des passions louables, les autres au
contraire en ont de mauvaises. À la vérité, je demeure d'accord que chaque habitude ayant toujours des accompagnements et
des suites conformes à sa nature ordinaire, il arrive que telles ou telles actions en particulier ne font faites que parce
ce qu'un homme a contracté telle ou telle habitude. Car, par exemple, il se peut faire que cet homme ici qui est tempérant
n'aura d'abord tels désirs honnêtes et ne fera persuadé des sentiments qu'il faut avoir touchant les plaisirs de la vie
que parce qu'il est tempérant. Et tout au contraire, que ce débauché n'aura tels désirs et tels sentiments, déshonnêtes que
parce qu'il est attaché à la débauche. Néanmoins, comme ces distinctions ne sont pas considérables, on les peut laisser là.

\bigbreak

Ce que nous aurions à faire ici maintenant, ces serait d'examiner ce qui a accoutumé d'arriver à telles ou telles sortes de
personnes, ensuite de telle ou telle qualité. Et cela à cause que tout ce qui met de la différence dans les hommes n'apporte
pas toujours du changement dans leurs sentiments et leurs mœurs. Car, par exemple, qu'un homme soit blanc ou noir, grand ou
petit, tout cela de soi n'exige point qu'il ait telles mœurs en particulier, ni telles passions, mais bien s'il est vieux ou
jeune, s'il est homme de bien ou méchant. En un mot, nous aurions à examiner tout ce qui arrivant à une personne pour la fortune
fait que, d'ordinaire, elle vient à changer de mœurs et de sentiments; comme quand cette personne s'imagine qu'elle est riche ou
pauvre, heureuse ou malheureuse. Mais ce n'est pas encore ici le lieu de traiter cette matière. 

Achevons seulement d'expliquer ce qui nous reste à dire touchant les principes et les causes des actions humaines.

\bigbreak

Premièrement, donc, on attribue une chose à la fortune, si l'on croit qu'elle a été faite par hasard, quand la cause qui l'a
produite est purement incertaine et indéterminée, ou qu'on ne voit pas ni à quel dessein ni pourquoi elle a été faite, ou quand
elle n'arrive ni toujours ni d'ordinaire de cette façon-là, ou enfin quand elle ne se fait point règlement et avec un certain
ordre: toutes lesquelles conditions ont été remarquées exactement dans la définition que nous avons données ailleurs du hasard
et de la fortune. 

\bigbreak

En second lieu, la nature est cause d'une chose lorsque cette chose a au-dedans d'elle-même le principe qui la produit et
que ce principe, en la produisant observe un certain ordre. Au reste, il n'importe que telle chose arrive nécessairement ou
pour l'ordinaire, puisque de quelque façon qu'elle arrive, toujours, elle se fera de la même sorte. Quant aux effets
extraordinaires de la nature et qui semblent violer ses règles ou n'arriver pas selon son dessein, ce n'est point ici le
lieu de les examiner particulièrement ni de voir si leur production se fait véritablement selon le dessein de la nature ou
s'il faut en rechercher une autre cause; quoi qu'après tout, on puisse dire qu'assez souvent, tels effets sont un pur ouvrage
de la fortune et du hasard.

\bigbreak

A l'égard de la force et de la contrainte, nous tenons qu'une chose est faite avec contrainte et par force quand celui qui la
fait agit contre sa propre inclination, ou contre son avis.

\bigbreak

Un homme agit par coutume quand il ne fait une chose que parce qu'il l'a déjà faite plusieurs fois.

\bigbreak

On agit par raisonnement lorsqu'on ne fait une chose qu'en intention d'acquérir quelqu'un des biens que nous avons remarqués
devoir apparemment apporter du profit et de l'avantage, soit qu'on borne les prétentions à posséder un tel bien et qu'on se le
propose pour but et pour fin, soit qu'on ne le considère que comme un moyen pour arriver à quelque autre chose qui tienne lieu
de fin et qui soit de plus grande importance; à condition, néanmoins, comme j'ai dit, que cela se fasse à dessein seulement
d'attirer de l'utilité. Car il ne faut pas ici confondre ce que nous voyons faire assez souvent aux débauchés qui, en apparence,
semblent faire beaucoup d'entreprises pour le gain seulement et pour le profits, puisqu'en effet, leur intention alors n'est
autre que de jouir après plus à leur aise des plaisirs qu'ils aiment, et pour donner lieu à une plus longue débauche, 

\bigbreak

Pour ce qui est de la colère et de l'animosité, elles regardent simplement la vengeance. Or il faut remarquer qu'il y a grande
différence entre la vengeance et le châtiment, puisque le châtiment est toujours pour le bien de celui qu'on châtie, et qu'au
contraire, la vengeance n'a pour but que la satisfaction et le contentement de celui qui se venge. De savoir maintenant ce que
c'est que la colère et ce qui la fait naître, nous en parlerons amplement quand nous serons au traité des passions.

\bigbreak

Enfin, on fait par convoitise tout ce qui en apparence doit donner du plaisir. Au reste, je mets au nombre des choses agréables
et qui donnent du plaisir toutes celles qu'on a accoutumé de faire, et auxquelles on s'est habitué. Car l'accoutumance a cela
qu'elle fait trouver du plaisir même dans ce qui n'est point plaisant à faire, si tôt qu'on y est accoutumé.

\bigbreak

Donc, pour trancher en un mot cette matière, il est certain:

\begin{emphpar}
	Que de toutes les choses que les hommes font d'eux-mêmes, il n'en pas une qui ne soit ou véritablement bonne et utile, ou
	telle en apparence, et encore qui ne soit, ou en effet ou apparemment, agréable.
\end{emphpar}

Et parce que tout ce que les hommes font d'eux-mêmes, ils le font toujours volontairement, et au contraire, que ce qu'ils ne
font point d'eux-mêmes, c'est toujours contre leur volonté et malgré eux qu'ils le font, il est certain encore, au regard de
ceux qui agissent volontairement:

\begin{emphpar}
	Que de tout ce qu'ils se proposent de faire, il n'y a rien qui ne soit, en effet ou apparemment, utile; et encore qui ne
	soit: ou agréable véritablement, ou du moins en apparence.
\end{emphpar}

\bigbreak

Après tout, je mets au nombre des biens et des avantages, non seulement la délivrance de quelque mal que ce soit, réel ou
apparent; mais aussi l'échange d'un grand mal pour un petit, parce que tout cela est à souhaiter. Et tout de même, je mets
au nombre de ce qui est agréable la délivrance de tout ce qui est fâcheux, soit qu'en effet il soit tel ou seulement en
apparence; et encore l'échange d'une chose très fâcheuse pour une qui sera moins. 

\bigbreak

Ainsi donc, puisque ce qui est utile et ce qui peut apporter du plaisir est toujours ce qui porte les hommes à faire ce qu'ils
font, pour cela, l'orateur doit s'étudier à connaître et le nombre, et la qualité des choses qui font utiles et qui sont
agréables. Pour ce qui est de l'utile, nous en avons déjà parlé en traitant du genre délibératif. Il ne reste plus qu'à faire
voir celles qui sont agréables et qui apportent du plaisir. J'avertirai ici en passant qu'on ne doit pas prendre garde de si
près aux définitions que nous donnons, puisqu'il importe peu à la rhétorique qu'elles soient si exactes, pourvu qu'elles ne
paraissent pas obscures. 

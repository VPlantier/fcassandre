
\section{Lieux pour connaître quand un bien est plus grand ou plus petit qu'un autre}

Mais parce qu'assez souvent, il arrive que les mêmes personnes qui demeurent d'accord que deux choses,
véritablement, sont utiles ne laissent pas d'être en contestation sur le plus et le moins, il faut encore
que nous enseignions à connaître quand un bien est plus grand qu'un autre, et quand une chose sera plus
utile. 

Supposons donc, premièrement, que tout ce qui surpasse une chose en quoi que ce soit est ce qui déjà contient
en soi tout autant que cette chose là contient, et qui a encore quelque chose de plus.

Et, au contraire, que tout ce qui est surpassé et moindre est ce qui est renfermé et compris dans la chose
qui le surpasse.

Supposons en second lieu que tout ce qu'on dit être plus grand ou en plus grand nombre n'est tel qu'à cause
qu'on en fait comparaison avec quelque chose qui est plus petite. Et tout de même, qu'autant de fois qu'on
se sert des termes de grand et petit, de peu et de beaucoup, est toujours à l'occasion de choses qu'on fait
rapporter à d'autres dont on veut faire savoir la grandeur et en quelle quantité elles sont.

Supposons enfin que tout ce qui surpasse une autre chose est proprement ce que nous appelons grand; et, au
contraire, que tout ce qui en est surpassé est proprement ce que nous appelons petit. Ainsi du peu et du
beaucoup.

\bigbreak

Donc, puisqu'il a été remarqué:

\begin{emphpar}
	Que le bien est une chose qu'on doit souhaiter à cause d'elle-même, non pas à cause d'une autre,
\end{emphpar}

Et encore:

\begin{emphpar}
	Que c'est généralement ce que désire tout ce qui est au monde, ou qui a de la raison, ou que rechercherait
	tout ce qui est privé de raison s'il avait de la connaissance et du jugement,
\end{emphpar}

En un mot:

\begin{emphpar}
	Que c'est tout ce qui est capable de nous procurer de pareils avantages ou de nous les conserver, ou qui
	en est suivi,
\end{emphpar}

Supposé encore ce que nous avons dit: 

\begin{emphpar}
	Qu'une chose à laquelle nous en rapportons d'autres tient toujours lieu de fin,
\end{emphpar}

Puisque c'est à la fin seule à quoi on rapporte tout ce qu'on fait, et pour laquelle tout le reste est recherché.

Supposé, en dernier lieu:

\begin{emphpar}
	Que tout ce que chaque particulier se propose comme un bien, jamais a son égard il ne peut être tel qu'il n'ait
	en lui quelqu'une des bonnes qualités que nous venons de dire.
\end{emphpar}

Tout cela présupposé, on pourra tirer les conséquences qui suivent.

\bigbreak 

Premièrement:

\begin{lieu}
	Que plus de choses qu'une seule prise à part, ou qu'un petit nombre, et cela comparé de sorte l'un avec l'autre
	que dans ce plus grand nombre se trouve aussi compris ce même petit nombre ou cette seule chose, sans doute le plus
	grand nombre en cet état l'emportera et sera à préférer\footnote{Jeff Bezos peut profiter à peu près mille fois
	plus du bonheur d’être riche que Will Smith.}.
\end{lieu}

En effet, les deux conditions que nous avons remarquées, pour avoir l'avantage et être considéré comme le meilleur s'y
rencontrent. Car déjà, de soi et en qualité de plus grand nombre, on ne peut pas douter qu'il ne les surpasse. Et
d'ailleurs, ces autres choses ici, pour être comprises en lui, en sont surpassées.

\bigbreak 

Secondement cette conséquence sera bonne:

\begin{lieu}
	Que si une chose, qui est la plus excellente dans son genre, l'emporte sur une autre qui soit aussi la plus excellente
	dans le sien, sans difficulté, le genre de la plus excellente l'emportera sur le genre de l'autre\footnote{En affirmant
	d'Hitler qu'il est le pire chef d'état de tous les temps, n'ai-je pas prouvé que les chefs d'états allemands sont pires
	que les français?}.
\end{lieu}

Et réciproquement, 
\begin{lieu}
	Que si un genre est plus excellent qu'un autre genre, ce qu'il y aura de plus excellent dans ce plus parfait genre
	l'emportera sur tout ce qu'il y aura de plus excellent dans l'autre.
\end{lieu}

Par exemple, s'il est vrai de dire en particulier que le plus excellent de tous les hommes est plus parfait que la plus
excellente de toutes les femmes, en général, il sera vrai de dire encore que tous les hommes seront plus parfaits que
toutes les femmes généralement. 

Et au contraire, par la même raison, si l'on peut dire que tous les hommes généralement sont plus excellents et plus parfaits
que toutes les femmes en général, on pourra dire en particulier aussi que le plus excellent de tous les hommes sera plus
parfait que la plus excellente de toutes les femmes; puisque les degrés d'excellence de chaque genre, et des choses qui sous
eux tiennent le premier rang, ont toujours un parfait rapport entre eux, et sont dans une juste proportion\footnote{Sans parler
de l'image de la femme au temps d'Aristote, puisque je ne veux pas répondre aux prémisses explicites mais seulement aux lieux,
notez cette insistance d'Aristote à justifier ses lieux et à les prendre comme argent comptant.}.

\bigbreak 

De plus on pourra inférer:

\begin{lieu}
	Qu'un bien qui en aura un second à sa suite vaudra mieux qu'un autre qui n'en aura point,
\end{lieu}

À cause que, jouissant de l'un, on jouira aussi de celui qui le suit. Au reste, nous avons déjà fait
savoir qu'une chose peut venir en suite d'une autre en deux façons: ou \emph{en même temps} ou \emph{
quelque temps après}. Il se remarque encore une troisième façon que nous appelons \emph{suivre en 
puissance}. Donnons des exemples:

La vie suit la santé en même temps. Il n'en est pas toujours de même de la santé à l'égard de la vie.

La science encore suit l’étude, mais ce n'est que quelque temps après.

Enfin, le larcin suit en puissance le sacrilège, puisque quiconque a la hardiesse de voler sur les autels
et de piller les temples, celui-là ne fera pas difficulté de dérober ailleurs.

\bigbreak

Cette conséquence aussi aura lieu:

\begin{lieu}
 Que de deux choses qui en surpasseront une troisième, celle qui la surpassera de davantage sera la
 meilleure,
\end{lieu}

Attendu que, pour être en cet état, il est nécessaire qu'elle surpasse aussi l'autre qui était plus grande.

\bigbreak

On pourra dire encore:

\begin{lieu}
	Que tout ce qui produira un plus grand bien vaudra mieux et sera plus digne de notre choix\footnote{La
	guerre émerge de la paix. La paix est la fin de la guerre. Si, donc, la paix est préférable à la guerre,
	alors la guerre vaut mieux que la paix.},
\end{lieu}
 
Puisque c'est à cela principalement qu'on connaît quand un bien est plus grand qu'un autre.

Et réciproquement:

\begin{lieu}
	Tout ce qui sera produit par une plus excellente chose,
\end{lieu}

Car si tout ce qui est bon pour la santé est plus souhaitable, et est un plus grand bien que ce qui apporte
simplement du plaisir, la conséquence est nette que le plaisir est bien moins considérable que la santé.

\bigbreak

Il y aura lieu encore d'inférer:

\begin{lieu}
	Que tout ce qui sera souhaitable de soi-même vaudra mieux que ce qui ne sera souhaitable qu'à cause d'une
	autre chose.
\end{lieu}
 
Par exemple, la force doit être tenue pour un plus grand bien que tout ce qui regarde la santé seulement, puisque,
sans la santé, on ne souhaiterait jamais pas une de ces chose. Au lieu que la force est toujours désirable d'elle-même,
en quoi nous avons dit que consistait principalement la nature du bien, entre les définitions que nous en avons données.

\bigbreak

On pourra aussi prétendre:

\begin{lieu}
	Que tout ce qui tient lieu de fin est meilleur que ce qui n'est point considérable en cette qualité,
\end{lieu}

Vu que celui-ci n'est recherché qu'à cause d'une autre chose et que l'autre est recherché pour l'amour de lui-mème. Par
exemple, l'exercice le doit céder à la santé, à cause qu'on n'aime à faire de l'exercice, qu'afin de se bien porter.

\bigbreak

Cette conséquence encore sera bonne:

\begin{lieu}
	Que ce qui n'aura pas tant de besoin d'une chose ou de plusieurs qu'une autre sera meilleur\footnote{Non, on va plutôt
	jouer au Kem's, parce que Mario Kart, faut du matériel, donc c'est moins bien.},
\end{lieu}

Attendu qu'il sera beaucoup plus parfait et plus capable de satisfaire tout seul. Au reste, une chose a moins de besoin
qu'une autre en deux manières: ou quand elle n'a pas affaire de tant, ou que ce qui lui manque est plus aisé à trouver.

\bigbreak

On aura aussi raison d'assurer:

\begin{lieu}
	Que de deux biens dont l'un sera tellement dépendant de l'autre que sans lui, il ne serait pas ou ne pourrait être,
	l'autre au contraire ne dépendra point de ce bien en aucune façon, l'indépendant vaudra beaucoup mieux,
\end{lieu}

Car, comme il n'aura que faire de rien, c'est une marque qu'il sera: et plus capable de satisfaire tout seul, et plus
parfait de lui-même.

La même conséquence encore aura lieu à l'égard de deux choses comparées ensemble:

\begin{lieu}
	Si l'une a la qualité de principe et que l'autre ne l'ait pas.
\end{lieu}

Et tout de même de deux autres:

\begin{lieu}
	Si l'une est cause et que l'autre non,
\end{lieu}

Vu qu'on sera obligé d'en faire d'autant plus d'état qu'absolument, il est impossible que sans aucune cause et sans aucun
principe, quelque chose que ce soit puisse être jamais, ni être faite.

\bigbreak

Cette conséquence aussi sera nécessaire:

\begin{lieu}
	Que de deux biens qui reconnaîtront chacun un principe différent, celui qui sera produit par le plus excellent principe
	sera aussi le plus excellent.
\end{lieu}

Et encore celle-ci:

\begin{lieu}
	Que de deux biens qui reconnaîtront chacun une cause différente, celui qui sera l'effet de la plus noble cause sera aussi
	le plus noble.
\end{lieu}

\bigbreak

Et réciproquement, il sera vrai de dire en renversant ces deux mêmes conséquences:

\begin{lieu}
	Que de deux principes différents, celui qui produira un plus grand bien sera aussi le meilleur,
\end{lieu}

Comme aussi:

\begin{lieu}
	De deux causes, celles qui produira un plus grand effet.
\end{lieu}

Par ce que nous venons de dire, il se voit que, de quelque façon qu'on puisse raisonner en ce sens, toujours de part et d'autre
il sera aisé de faire paraître une chose plus considérable, car non seulement un bien en paraîtra plus considérable \emph{si
étant reconnu pour principe, on le compare avec un autre qui ne soit pas tel}, mais encore \emph{Si n'étant point principe, on
en fait comparaison avec un autre qui soit principe véritablement}. Et, de fait, dans tout ce qu'on se propose, la fin est
toujours la chose la plus considérable et ce qui tient le premier rang et cependant, ce n'est point un principe. Donnons
quelque exemple. Léodamas, accusant Callistratos, soutenait que \emph{celui qui avait conseillé de faire une mauvaise action 
était plus coupable que celui qui l'avait commise, parce que cette action n'aurait jamais été faite si premièrement elle n'avait
été conseillée}. Ici le conseil est considéré comme le principe de l'action. Et tout au contraire, une autre fois, le même,
accusant Chabrias, soutînt que \emph{celui qui avait commis une injustice était beaucoup plus coupable que celui qui l'avait
conseillée, puisque tout conseil demeure inutile si un autre ne l’exécute, et que que ceux qui conseillent de faire une chose
la conseillent toujours à dessein que d'autres la mettent à exécution}. Ici l’exécution est considérée comme la fin.

\bigbreak

On pourra encore tirer cette conséquence:

\begin{lieu}
	Que ce qui se trouve rarement est plus excellent que ce qui se trouve communément et en abondance.
\end{lieu}

Ainsi, l'or est plus excellent que le fer, car quoi qu'on n'en tire pas tant d'usage que du fer, il semble néanmoins plus
précieux, à cause que l'acquisition en est plus difficile à faire.

Dans un autre sens, aussi, on pourra soutenir:

\begin{lieu}
	Qu'une chose qu'on aura en abondance sera meilleure qu'une autre qui sera plus rare,
\end{lieu}

Puisque en effet, on se servira beaucoup plus de l'une que de l'autre, et que tout ce qui sert très souvent vaut mieux que
ce qui ne sert que quelques fois et très peu. C'est ce qui a fait dire à \textsc{Pindare} dans une de ses Odes:

\begin{emphpar}
	Il n'est rien de si bon que l'eau.
\end{emphpar}

\bigbreak

Et tout de même on pourra prétendre:

\begin{lieu}
	Que ce qui est plus difficile à acquérir est préférable à tout ce qui s'acquiert aisément,
\end{lieu}

Parce qu'il sera plus rare que l'autre, et tout au contraire:

\begin{lieu}
	Que ce qui est plus facile à acquérir vaut mieux que ce qui ne peut être acquis qu'avec difficulté,
\end{lieu}

Puisque nous avons ces choses-là comme nous voulons et quand bon nous semble.

\bigbreak

Pareillement:

\begin{lieu}
	Tout ce qui aura pour son contraire un plus grand mal.
\end{lieu}

\bigbreak

Et encore:

\begin{lieu}
	Toutes les choses dont la privation nous apportera plus de dommage ou d'incommodité.
\end{lieu}

\bigbreak

On prendra aussi:

\begin{lieu}
	Que tout ce qui est vertueux vaut mieux que ce qui n'est point une vertu.
\end{lieu}

Et au contraire:

\begin{lieu}
	Que ce qui est vicieux est pire que ce qui n'est point un vice et qui n'y a aucune disposition,
\end{lieu}

Attendu que ces choses sont arrivées à leur terme et à leur fin, et que les autres ne sont pas en cet état-là.

\bigbreak

Cette conséquence encore aura lieu:

\begin{lieu}
	Que ce qui produira des effets plus louables ou plus blâmables sera aussi plus blâmable ou plus louable
	lui-même.
\end{lieu}

Et par la même raison:

\begin{lieu}
	Que les plus hautes vertus et les plus grands vices produiront aussi des actions plus vicieuses et plus
	vertueuses,
\end{lieu}

Puisque ce qu'un principe et une cause sont à l’égard de leurs effets, tels sont toujours effets à l'égard de
leurs principes.

\bigbreak

De plus on pourra raisonner ainsi:

\begin{lieu}
	Que toutes les choses dont l'excès sera plus souhaitable ou plus honnête, ces choses seront elles-mêmes
	plus honnêtes et plus souhaitables.
\end{lieu}

par exemple, à cause qu'il est plus souhaitable d'avoir une excellente vue que d'avoir l'odorat excellent, il
s'ensuit que la bonne vue est quelque chose de plus souhaitable que le parfait odorat.

Et pareillement, que s'il est beaucoup plus honnête d'aimer à faire des amis que d'aimer à acquérir des richesses,
l'amour des richesses sera moins honnête que l'amour des amis.

Et réciproquement, on pourra dire en renversant les deux propositions précédentes:

\begin{lieu}
	Que plus une chose sera excellente et honnête, et plus l'excès en sera honnête et excellent,
\end{lieu}

Et tout de même:

\begin{lieu}
	Que plus le désir d'un bien sera honnête et raisonnable, et plus ce bien-là aussi sera honnête,
\end{lieu}

Car il est certain que plus les choses que nous souhaitons sont grandes en elles-mêmes, et plus à proportion nos désirs
croissent et sont grands pour l'ordinaire.

\bigbreak

Tout au contraire on dira:

\begin{lieu}
	Que d'autant plus qu'une chose sera honnête et bonne, d'autant plus aussi le désir en sera bon et honnête.
\end{lieu}

Et encore:

\begin{lieu}
	Que plus une science sera honnête et belle, et plus les matières qu'elle traitera seront belles aussi et honnêtes,
\end{lieu}

Attendu que telle qu'est la nature d'une science, telle est la doctrine; puisque chaque science n'enseigne rien que ce qui
est de son sujet.

Et par la même raison, à cause de l'analogie et du parfait rapport:

\begin{lieu}
	Que plus une chose sera belle et honnête, et plus la science qui en traitera sera telle.
\end{lieu}

\bigbreak

Ce raisonnement encore aura lieu:

\begin{lieu}
	Que tout ce que des hommes prudents et très judicieux, ou tous les hommes ensemble en un fort grand nombre de personnes,
	ou la plupart, ou les plus habiles gens d'une profession jugeraient sans doute ou auront déjà jugé être un bien ou un plus
	grand bien, assurément ce doit passer pour tel.
\end{lieu}

Ah reste, il n'importe que ç'ait été simplement leur avis ou qu'ils aient rendu ce jugement en qualité de maîtres et d'experts.

Or, non seulement on pourra se servir de cette proposition quand il sera question de juger si un bien sera plus grand qu'un autre,
mais encore de quelque matière que ce soit. Car on s'en pourra servir également: et en raisonnant sur la nature d'une chose, et en
traitant de sa quantité, de sa qualité, et ainsi du reste; puisque, toujours, il y aura lieu d’assurer qu'une chose ne sera jamais
autre que ce que la prudence, ou la science qui en doit juger, en aura déterminé. Ce que nous avons déjà remarqué être vrai entre
les définitions du bien que nous avons données, vu qu'il a été dit que \emph{le bien était une chose que tout ce qui est au monde
rechercherait sil avait du sens et de la prudence}. D'où il s'ensuit qu'un bien sera toujours d'autant plus grand et à souhaiter
que celui qui le jugera tel aura de prudence et de jugement.

\bigbreak

Sur ce fondement, on pourra dire encore:

\begin{lieu}
	Que tout ce qui se rencontre dans les honnêtes gens, soit absolument, soit en qualité d’honnêtes gens, est plus à rechercher.
\end{lieu}

Par exemple, à cause que la valeur se rencontre plus ordinairement dans un honnête homme que la force du corps, il sera vrai de
dire que la valeur est quelque chose de plus considérable que la force.

\bigbreak

\begin{lieu}
	Que ce que le plus homme de bien choisirait et préférerait à toute autre chose, soit absolument, soit en qualité de plus homme de
	bien, on le doit croire meilleur.
\end{lieu}

Ainsi, nous pourrons assurer qu'il vaut mieux souffrir l'injustice que de la faire, à cause que le plus homme de bien qui soit au monde
sera de ce sentiment.

On pourra dire encore:

\begin{lieu}
	Que ce qui donnera plus de plaisir sera préférable à tout ce qui en donnera moins,
\end{lieu}

Car ce raisonnement paraîtra d'autant plus vrai qu'il n'y a rien dans le monde qui ne recherche le plaisir, et qu'on souhaite
toujours le plaisir à cause de lui-même, qui sont deux qualités essentielles que nous avons attribuées au bien et à la fin en
apportant leurs définitions. Au reste une chose apporte plus de plaisir qu'une autre en deux façons: et quand elle est
accompagnée de moins de douleur, et quand le plaisir qu'elle donne est d'une plus longue durée.

Il y aura lieu encore de conclure:

\begin{lieu}
	Que ce qui sera plus honnête vaudra mieux que ce qui le sera moins
\end{lieu}

Vu que tout ce qui est honnête ou apporte du plaisir est souhaitable à cause de lui-même.

\bigbreak

Et pareillement:

\begin{lieu}
	Que tout ce que nous aimerions mieux nous procurer à nous-mêmes ou à nos amis sera un plus grand avantage,
\end{lieu}

Comme au contraire un plus grand mal:

\begin{lieu}
	Tout ce que nous aimerions mieux éviter ou faire éviter à nos amis.
\end{lieu}

\bigbreak

Il y aura lieu encore de soutenir:

\begin{lieu}
	Que ce qui sera d'une plus longue durée doit être préféré à ce qui ne durera pas tant,
\end{lieu}

Et tout de même:

\begin{lieu}
	Que ce qui sera moins sujet au changement vaudra mieux que ce qui sera d'une naturel plus changeant,
\end{lieu}

Attendu que l'usage de ces deux choses l'emportera sur celui des deux autres; puisque ce qui sera d'une plus
longue durée apportera plus d'utilité, à cause qu'on s'en servira plus longtemps. Et par la même raison, tout
ce qui sera d'une nature moins changeante, vu que nous aurons la liberté de nous en servir toutes et quantes
fois qu'il nous plaira. Car c'est seulement de ce qui ne change point qu'on se peut servir quand on veut, parce
qu'on le trouve toujours en même état.

\bigbreak

On pourra dire encore:

\begin{lieu}
	Que telles que seront entre elles deux chose comprises sous quelqu'un des termes que nous appelons \emph{
	conjugués et de cas semblables}, telles seront entre elles aussi toutes les autre qui seront de leur suite
	et de leur dépendance.
\end{lieu}

Par exemple, s'il est vrai de dire que ce que signifie le mot \emph{vaillamment}, qui est un terme conjugué, est
quelque chose de plus honnête et plus à souhaiter que ce qui est signifié par ce mot \emph{tempéramment}\footnote{
Avec deux m. Paléonéologisme de François \textsc{Cassandre}. Exemple: <<~Pour votre santé, buvez tempéramment.~>>},
qui est un autre terme conjugué, il faudra conclure que la valeur sera plus souhaitable que la tempérance, et qu'être
vaillant sera une vertu beaucoup plus considérable que d'être tempérant.

\bigbreak

Ce raisonnement encore pourra servir:

\begin{lieu}
	Qu'une chose que tout le monde souhaitera, ou beaucoup de personnes, vaudra mieux qu'une autre que tout le monde ne
	souhaitera pas, ou peu de personnes seulement.
\end{lieu}

Et cela conformément à la définition du bien que nous avons donnée. Car s'il est vrai que le bien soit \emph{une chose
que tout le monde souhaite généralement}, la conséquence est nécessaire, que tout ce qui sera davantage souhaité sera
aussi un plus grand bien.

\bigbreak

On pourra encore faire valoir ce raisonnement:

\begin{lieu}
	Que ce que nos parties adverses, ou nos ennemis, ou nos juges, ou des experts ayant commission d'eux de nous juger,
	auront déclaré être un plus grand bien, sans doute il doit passer pour tel.
\end{lieu}

Car, quant à l'approbation de nos parties adverses et de nos ennemis, on pourra soutenir qu'elle doit tenir lieu d'une
approbation générale et que c'est autant que si tout le monde en était demeuré d'accord. A l'égard des juges, leur jugement
encore sera très considérable, tant parce qu'il n'y aura qu'eux qui aient autorité de prononcer sur de semblables matières
que parce qu'ils y seront très intelligents.

\bigbreak

Quelquefois encore on pourra soutenir:

\begin{lieu}
	Qu'une chose à laquelle tout le monde participe est digne d'une plus grande estime,
\end{lieu}

Puisque en quelque façon, il y a de la honte à n'y pas participer comme les autres.

Quelquefois, le contraire aura lieu, par exemple, si une chose est de telle qualité:

\begin{lieu}
	Qu'aucun autre ne la possède que nous, ou fort peu de personnes,
\end{lieu}

Attendu qu'elle en sera beaucoup plus rare.

\bigbreak

On pourra dire encore:

\begin{lieu}
	Que ce qui est plus digne de louange est aussi plus considérable,
\end{lieu}

Puisque pour être tel, il faut qu'il soit plus honnête.

\bigbreak

Et tout de même:

\begin{lieu}
	Qu'une chose à qui on rend plus d'honneur doit être plus estimée,
\end{lieu}

À cause que l'honneur qu'on lui rend fait comme voir ce qu'elle vaut.

Et par la même raison au contraire:

\begin{lieu}
	Que tout ce qui est suivi d'un plus grand châtiment est un plus grand mal.
\end{lieu}

\bigbreak

Et encore il sera aisé de représenter comme meilleur:

\begin{lieu}
	Ce qui surpassera une chose reconnue généralement pour un très grand avantage, ou du moins qui paraîtra telle.
\end{lieu}

\bigbreak

Pour rendre une chose plus grande qu'une autre en apparence, on se pourra servir d'adresse qui est de la diviser en
plusieurs parties, parce que toutes ces parties la feront paraître comme multipliée et surpasser par un plus grand
nombre d'effets. De cette adresse s'est servi le poète à l'endroit où la femme de Méléagre veut persuader son mari de
prendre les armes pour la défense de son pays, car faisant la peinture du malheur d'une ville prise par force, c'est
ainsi qu'elle parle:

\begin{emphpar}
  Hélas! combien répand et de sang et de larmes

  Une ville exposée à la fureur des armes?

  Partout ce n'est que meurtre et que feux allumés

  Maisons, temples, palais brûlent, sont consumés.

  On voit traîner captifs par des troupes barbares, femmes, filles, enfants\dots
\end{emphpar}

\bigbreak

On se pourra aussi servir de l'adresse contraire, en assemblant plusieurs choses en une et en les entassant, comme fait
\textsc{Épicharme}, et cela pour la même raison que nous venons d'alléguer touchant la division d'une chose en ses
parties; car d'assembler ainsi plusieurs choses, non seulement l'objet grossit à la vue et parait beaucoup plus, mais
encore on se figure qu'il cause de très grands effets. 

\bigbreak

Outre ces adresses, parce que nous avons remarqué qu'un bien qui est plus difficile à acquérir, ou plus rare, est aussi
ordinairement plus considérable, on pourra encore faire paraître une chose plus grande en faisant valoir toutes les circonstances
qui l’accompagnent, comme sont les occasions, les lieux, le temps, l'âge et les forces des personnes qui l'auront faite. Car
si quelqu'un, par exemple, a réussi dans une entreprise qui passait de beaucoup ses forces et son âge, ou à laquelle pas un
de ses pareils n'eût jamais osé penser, ou encore s'il l'a exécutée d'une certaine façon ou en certain lieu, sans doute
qu'alors cette entreprise doit être tenue pour bien plus glorieuse, que si elle était sans toutes ces circonstances. Or, non
seulement cette adresse pourra servir à faire estimer davantage une action juste, utile, on qui aura été faite pour acquérir
de l'honneur, mais encore, elle servira à rendre plus blâmable tout ce qui aura été fait au contraire. Il y a un exemple de
ceci dans l'épigramme composée à la louange du poissonnier d'Argos qui remporta le prix aux jeux olympiques. C'est lui-même
qui parle:

\begin{emphpar}
  Aurait-on jamais cru qu'un jour j'eusse la gloire

  D'obtenir en ce lieu cette illustre victoire?

  Moi qu'on vit mille fois, un panier sur le dos et l'épaule chargée,

  Apporter du poisson de la ville d'Argos pour le vendre à Tégée.
\end{emphpar}

C'est de cette façon que se louait Iphicrate lui-même, ce fameux général d'armée des Athéniens, qui, de fils de savetier
qu'il était, monta enfin a ce haut degré d'honneur: \emph{Qui étais-je autrefois}, disait-il, \emph{pour être maintenant
ce que je suis?}

\bigbreak

On pourra encore tirer cette conséquence:

\begin{lieu}
  Que ce qui nous vient naturellement et qui naît avec nous vaudra beaucoup mieux que tout ce que nous empruntons d'ailleurs
  et que nous pouvons acquérir, 
\end{lieu}

À cause que l'acquisition en est bien plus difficile. D'où vient qu'\textsc{Homère} fait dire à Phémios:

\begin{emphpar}
  Ce que je sais, je le sais de moi-même.
\end{emphpar}

\bigbreak

Cette conséquence encore sera bonne:

\begin{lieu}
  Que la partie la plus considérable d'une chose qui de soi est considérable doit être plus estimée que pas une des autres
  parties.
\end{lieu}

C'est aussi sur ce fondement que \textsc{Périclès}, dans l'oraison funèbre qu'il fit à l'honneur de ceux qui étaient morts
au service de l’état dit \emph{que la perte d'une jeunesse si vaillante n'était pas moins considérable à la république
d'Athènes que ne le serait a l'année le retranchement du printemps}.

\bigbreak

Il faudra mettre encore au rang des plus grands biens:

\begin{lieu}
  Les choses qui nous serviront davantage dans notre plus grand besoin.
\end{lieu}

Par exemple dans notre vieillesse, ou lorsque nous serons malades.

\bigbreak

On pourra aussi prétendre 

\begin{lieu}
  Que de deux choses qui se rapportent à une même fin, celle qui touche de plus prés cette fin est la meilleure.
\end{lieu}

\bigbreak

Et encore, il y aura lieu d'assurer:

\begin{lieu}
  Qu'un bien qui nous regardera particulièrement vaut mieux qu'un autre qui sera simplement un bien en général.
\end{lieu}

Et tout de même:

\begin{lieu}
  Qu'un bien qui sera en notre puissance d'acquérir si nous voulons sera préférable à un autre que de toute
  impossibilité, nous ne saurions jamais avoir,
\end{lieu}

Attendu que l'un nous regarde et que nous en pouvons jouir, mais non pas de l'autre.

\bigbreak

De plus ce raisonnement pourra servir:

\begin{lieu}
  Que les biens qu'on ne pourra obtenir que sur la fin de ses jours seront beaucoup plus à estimer,
\end{lieu}

À cause qu'étant plus proches de la fin, ils sembleront plus participer à sa nature.

On dira aussi:

\begin{lieu}
  Que ce qui tient plus de la vérité vaut mieux que tout ce qui ne dépend que de l’opinion.
\end{lieu}

Or, pour savoir quand une chose dépendra seulement de l'opinion, il faut examiner si celui qui la fait voudrait la
faire au cas qu'elle ne vint à la connaissance de personnes. Ainsi, on pourra dire que recevoir du bien de quelqu'un
est plus à souhaiter que d'en faire, à cause que volontiers on recevrait d'autrui, si l'on était assuré que cela fût
toujours secret. Il n'en est pas de même de donner, vu qu'il semble que jamais on ne voudrait rien donner, si en donnant
l'on croyait que cette libéralité demeurât toujours cachée.

\bigbreak

On pourra encore faire passer pour meilleur et plus avantageux:

\begin{lieu}
  Tout ce que l'on aimerait mieux avoir en effet qu'en apparence,
\end{lieu}

Puisque ces choses-là tiendront davantage de la vérité. D'où vient que quelques-uns soutiennent que la justice est une vertu
dont on ne doit pas faire grand état, à cause, disent-ils, qu'on aimerait beaucoup mieux paraître juste que de l'être. Il
n'en est pas ainsi de la santé, puisqu'il vaut mieux se bien porter en effet que de n'avoir la santé qu'en apparence.

\bigbreak

On pourra encore avancer:

\begin{lieu}
  Que ce qui sera utile à plus de choses doit être davantage estimé.
\end{lieu}

Par exemple, ce qui en même temps contribuera non seulement à nous faire vivre, mais encore à nous faire vivre agréablement,
à nous donner la jouissance de toutes sortes de plaisirs, et à nous faire entreprendre de grandes choses. Aussi est-ce pour
cela que les richesses et la santé sont si estimées dans le monde, parce qu'en elles se trouvent tous ces grands avantages.

\bigbreak

On dira le même:

\begin{lieu}
  De tout ce qui est à la fois exempt de douleur et accompagné de plaisir,
\end{lieu}

Car, sans doute, deux biens ensemble valent mieux qu'un. Or le plaisir est un bien, et aussi l'indolence. Par indolence,
j'entends la privation de toute sorte d'incommodité et de douleur.

\bigbreak

Il y aura encore lieu d'inférer:

\begin{lieu}
  Que de deux biens, dont l'un ajouté à une certaine chose sera un tout plus considérable que si on y ajoutait l'autre,
  celui qui sera un tout plus considérable sera beaucoup meilleur.
\end{lieu}

Et tout de même on dira:

\begin{lieu}
  Qu'un bien qui se fait sentir et apercevoir aussitôt qu'on l'a sera préférable à un autre qui ne se fait pas sentir ni
  apercevoir davantage quand on l'a que quand on ne l'a pas,
\end{lieu}

Car l'un, sans doute, tient beaucoup plus de la vérité que l'autre. Aussi estime t-on un bien plus grand avantage d'être
riche en effet, que de le paraître simplement.

\bigbreak

Enfin on soutiendra:

\begin{lieu}
  Que les choses qu'on tient plus chères que d'autres seront aussi beaucoup plus estimables; et plus, sans doute, à ceux à
  qui il n'en restera plus qu'une, de plusieurs qu'ils avaient auparavant, qu'aux autres qui n'en ont pas pour une seule.
\end{lieu}

D'où vient aussi que la loi est beaucoup plus sévère à un homme qui a crevé l'œil à un autre qui n'avait que celui-là que s'il
le crevait à un qui eût encore ses deux yeux; parce qu'en effet, il le prive alors d'une chose qui lui était d'autant plus chère,
que c'était la seule qui lui restait.

\bigbreak

Voilà à peu près les preuves dont il se faut servir quand on aura à persuader ou à dissuader.

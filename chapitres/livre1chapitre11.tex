
\section{Des choses qui sont agréables et donnent du plaisir}

Posons pour fondement que le plaisir est \emph{une certaine émotion de l’âme  ou un changement qui arrive
tout à coup, qui se rend sensible et qui met la nature en l'état qu'elle demande}, et pour la douleur, que
c'est tout le contraire. Que si le plaisir est tel que nous venons de dire, il s'ensuit:

\begin{lieu}
	Que tout ce qui sera capable de nous mettre en l'état que nous venons de remarquer sera très agréable;
\end{lieu}

et au contraire, très fâcheux:

\begin{lieu}
	Tout ce qui détruira ce même état, ou qui sera cause que nous tomberons dans l'autre qui lui est opposé.
\end{lieu}

\bigbreak

Il s'ensuivra aussi:

\begin{lieu}
	Que pour l'ordinaire, c'est une chose agréable de se sentir arriver à cet état ou nous devons être
	naturellement, surtout quand ce qui se fera selon le désir de la nature aura atteint toute la perfection
	qu'il peut avoir. 
\end{lieu}

\bigbreak

Il faudra mettre encore au nombre des choses qui apportent du plaisir:

\begin{lieu}
	Toutes sortes d'accoutumances;
\end{lieu}

Puisque l'accoutumance est en quelque façon une chose qui a passé en nature. Aussi n'y a-t-il rien qui
ressemble davantage à la nature que l'accoutumance, par la même raison qu'il n'y a rien qui approche plus
de ce qui se fait toujours que ce qui se fait très souvent. Et de vrai, l'accoutumance est pour les choses
qui se font très souvent, et la nature, pour celles qui se font toujours.

\bigbreak

De plus, il s'ensuivra:

\begin{lieu}
	Que tout ce qui ne se fera point avec violence sera agréable;
\end{lieu}

Puisque la violence est ennemie de la nature. Et c'est pour cette raison que toutes les contraintes sont fâcheuses,
et toutes les occasions où il y a nécessité de faire quelque chose, ce qu'un poète a très bien remarqué lorsqu'il
a dit:

\begin{emphpar}
	Tout ce qu'on fait par force incommode toujours.
\end{emphpar}

Or, si cela est, il faudra encore tenir pour fâcheux les inquiétudes, les soins, l'étude, les fortes applications
d'esprit; en un mot toutes sortes d'efforts, à cause que de semblables actions tiennent toujours de la contrainte, 
si l'on n'y est accoutumé, vu qu'en ce cas l'accoutumance les adoucit et les rend agréables.

\bigbreak

D'où il faut conclure:

\begin{lieu}
	Que ce qui sera contraire à tout ce que nous venons de dire apportera du plaisir.
\end{lieu}

Par exemple, la paresse, l'oisiveté, la négligence, les divertissements, le repos, le sommeil; puisqu'il n'y a
rien de plus éloigné ni de plus affranchi de la contrainte que cela.

\bigbreak

Il faudra encore tenir pour agréable:

\begin{lieu}
	Toutes les choses où le désir et l'appétit nous portent;
\end{lieu}

Vu que le désir n'est qu'un appétit de jouir de ce qui est agréable et qui peut donner du plaisir. Or, comme nous
avons déjà remarqué, il y a deux sortes d'appétits dans l'homme, l'un sensuel ou d'animal, et l'autre raisonnable.
Par \emph{appétit sensuel}, j'entends tout ce que les hommes désirent sans faire de réflexion dessus ni l'examiner.
Ces sortes de désirs s'appellent proprement naturels et ne regardent que la satisfaction et les nécessités du corps.

Tels sont, premièrement, la faim et la soif, qui sont donnés au corps pour lui faire songer en général aux
aliments nécessaires à l'entretien de sa vie. Et tous les autres encore qui regardent chaque espèce de nourriture
en particulier.

Tels sont, en second lieu, les désirs qui tendent à l'amour et à la bonne chère. Bref, tous ceux qui flattent les
autres sens, et qui peuvent contenter le toucher, l'odorat, l'oreille et la vue.

J'appelle \emph{appétit raisonnable} ce qui fait désirer une chose seulement à cause qu'on est persuadé de sa bonté.
Car il se trouve beaucoup de choses dont on ne vient à désirer et la vue et la possession que parce qu'on en a ouï
faire de l'estime et que, véritablement, on croit qu'elles méritent d'être possédées. 

\bigbreak

Or, puisque le plaisir consiste à se sentir émouvoir en soi-même et à être touché de quelque passion, outre cela
que notre imagination, à la bien considérer, soit je ne sais quelle sorte de sentiment débile et imparfait, en un
mot, puisqu'il n'est pas possible d'espérer, ni de se souvenir de quoi que ce soit qu'en même temps on ne forme
dans son imagination l'idée et l'image de la chose qu'on espère ou dont ou se souvient, cela présupposé, il
s'ensuit:

\begin{lieu}
	Qu'il y aura du plaisir à se souvenir parfaitement d'une chose et à être dans une très grande espérance de
	l'avoir;
\end{lieu}

Puisque, selon ce que nous venons de dire, ce sera en jouir alors en quelque sorte et l'avoir présente à ses
sens. Tellement qu'il est nécessaire que rien ne nous puisse donner du plaisir qu'en l'une de ces trois façons:
ou quand il sera présent à nos sens et qu'en effet nous en jouissons, ou quand il nous souviendra qu'autrefois,
nous en avons joui, ou enfin, lorsque nous aurons espérance d'en jouir quelque jour. Car la jouissance regarde
toujours le présent, la mémoire le passé, et l'espérance l'avenir.

\bigbreak

La mémoire, donc, ne représente jamais rien qu'elle n'apporte du plaisir; car non seulement elle donne du plaisir
lorsqu'elle rappelle les images des choses qui étaient agréables dans le temps qu'on en jouissait, mais même encore
lorsqu'elle en représente d'autres d'une nature toute contraire et qui autrefois étaient très fâcheuses à supporter,
principalement quand les personnes ont changé d'état, et qu'à leurs travaux passés et à toutes leurs disgrâces a
succédé un grand repos ou beaucoup de gloire. C'est aussi ce qui a fait dire a \textsc{Euripide}:

\begin{emphpar}
	D'un péril évité, le souvenir est doux;
\end{emphpar}

Et encore à \textsc{Homère}:

\begin{emphpar}
	Quiconque a vu ses jours autrefois traversés

	Prend plaisir de songer à ses malheurs passés.

	Surtout quand son adresse et son propre courage,

	Après beaucoup d'efforts, ont surmonté l'orage.
\end{emphpar}

Et la raison de ceci est qu'il y a même du plaisir à n'avoir point de mal.

\bigbreak

Quant à l’espérance, il est certain encore qu'on ne saurait jamais rien espérer de tout ce qui semble
devoir réjouir par sa présence, apporter quelque grand avantage ou simplement être utile sans incommoder
qu'en même temps il n'en vienne du plaisir. En un mot, tout ce qui par sa présence cause de la joie, pour
l'ordinaire, apporte du plaisir à ceux qui s'en souviennent ou qui sont dans l'espérance de l'avoir. Et
de fait c'est pour cela encore:

\begin{lieu}
	Qu'il y a un très grand plaisir à se mettre en colère;
\end{lieu}

Comme \textsc{Homère} a fort bien remarqué quand, parlant de cette passion, il a dit:

\begin{emphpar}
	Lorsqu'en nous elle accroît son feu séditieux,

	Le miel n'est pas si doux, ni si délicieux.
\end{emphpar}

À cause que jamais on ne se met en colère contre les personnes de qui, en apparence, il est impossible de se
venger, non plus que contre ceux qui ont incomparablement plus de pouvoir et de crédit que nous; car s'il
arrive que nous nous mettions en colère contre eux, c'est toujours bien moins que contre d'autres.

\bigbreak

Il est certain encore:

\begin{lieu}
	Que la plupart de nos désirs seront accompagnés de plaisir;
\end{lieu}

Car soit qu'alors on se souvienne d'avoir joui autrefois de ce que l'on souhaite, ou qu'on espère d'en jouir
bientôt, toujours en cet état on vient à goûter je ne sais quel plaisir. Par exemple, ceux qui sont travaillés
de la soif pendant une fièvre, soit qu'alors il leur souvienne d'avoir bu autrefois à souhait étant extrêmement
altérés ou qu'ils espèrent de boire encore de même, toujours à cela ils trouvent je ne sais quelle joie. Le même
se remarque en ceux qui sont passionnés d'amour, car soit que dans l'entretien, ils viennent à parler de la
personne qu'ils aiment, soit qu'ils lui écrivent, qu'ils songent à elle ou qu'ils fassent quelque autre chose
qui la regarde, toujours alors ils sont joyeux. Et ce qui fait leur joie en toutes ces rencontres est qu'ayant
cette personne présente à la mémoire, il leur semble que, véritablement, ils sont avec elle. Aussi est-ce à cela
principalement qu'on reconnaît si l'amour commence à prendre empire sur l'esprit quand non seulement on se plaît
à demeurer avec la personne qu'on aime, mais encore lorsque l'affection persiste dans l'absence, et qu'on ne se
peut empêcher d'y songer, et tout de même lorsqu'en étant éloigné, ou s’attriste de ne la plus voir.

\bigbreak

Il faut dire encore:

\begin{lieu}
	Qu'il y aura je ne sais quel plaisir au milieu des plaintes et des soupirs;
\end{lieu}

Car si d'un côté la tristesse nous donne un déplaisir sensible d'avoir perdu pour jamais la personne que nous
pleurons, d'un autre côté, elle nous fait trouver du plaisir et de la consolation à nous la représenter telle
qu'elle était dans toutes ses actions et comme si nous l'avions encore devant nos yeux. Ce qu'\textsc{Homère}
justifie par ces vers:

\begin{emphpar}
	Il dit, et son discours fit lors trouver des charmes

	À pousser des soupirs et répandre des larmes.
\end{emphpar}

\bigbreak

On ne peut pas douter non plus:

\begin{lieu}
	Que la vengeance ne soit très douce;
\end{lieu}

Puisque autant qu'il est fâcheux de ne pouvoir venir à bout de ce qu'on souhaite, autant y a-t-il de douceur à
le voir réussir. Or est-il que ceux qui sont en colère se fâchent toujours extraordinairement lorsqu'ils perdent
l'occasion de se venger, et témoignent au contraire être très contents lorsqu'ils conçoivent le moindre espoir
de vengeance. 

\bigbreak

Il faudra encore conclure:

\begin{lieu}
	Que la victoire sera agréable;
\end{lieu}

Non seulement à ceux qui aiment à vaincre, mais encore à toutes sortes de personnes, puisque alors on s'imagine
qu'on est plus excellent qu'un autre, qui est une chose pour laquelle tous les hommes sont passionnés; à la
vérité, les uns plus et les autres moins.

\bigbreak

Que si, en effet, il se trouve du plaisir à vaincre, il s'ensuivra encore:

\begin{lieu}
	Que toutes sortes de jeux et de divertissements où il y aura défi et partie faite seront très agréables;
\end{lieu}

Et cela sans distinction, soit que la partie ait été faite entre musiciens, athlètes ou savants, puisqu'il arrive
toujours en ces rencontres de remporter la victoire.

\bigbreak

Il en sera de même:

\begin{lieu}
	Des dés, de la paume et des échecs;
\end{lieu}

Et encore:

\begin{lieu}
	Des jeux les plus sérieux et les plus graves;
\end{lieu}

Car quoi qu'ils ne soient pas tous divertissants d'abord, on ne laisse pas néanmoins d'y trouver du plaisir
sitôt qu'on y est accoutumé. Ceux qui, d'abord, apportent du plaisir sont la chasse et toute autre adresse à
prendre des animaux. Ce qui fait donc qu'on trouve du plaisir à toutes ces sortes d'occupations, c'est que
partout où il y a du combat, là, il y a de la victoire.

Et ainsi il se voit encore;

\begin{lieu}
	Que la profession du barreau et la dispute des écoles sont très agréables à ceux qui y réussissent et
	y sont accoutumés.
\end{lieu}

\bigbreak

De plus, il faudra mettre au nombre des choses qui apportent un très grand plaisir:

\begin{lieu}
	L'honneur et la réputation;
\end{lieu}

À cause de l'opinion qu'alors chacun a de soi-même que, véritablement, il est honnête homme et tel qu'on le
publie, à laquelle opinion on se laisse toujours aller d'autant plus aisément qu'on pense que ceux chez qui
on est en estime ne louent que parce qu'en effet c'est leur sentiment et qu'ils croient dire la vérité; tels
que sont des voisins plutôt que ceux qui sont éloignés, et plutôt encore les personnes avec qui on converse
familièrement ou qui sont de connaissance, ou de la même ville que des étrangers et des gens de dehors, et
encore, plutôt ceux qui sont vivants, que ceux qui ne sont pas encore au monde; bref, les personnes sages et
d'une haute prudence plutôt que des étourdis et des gens sans jugement, et enfin, quantité de personnes
plutôt qu'un petit nombre. Parce qu'en effet il y a toujours plus d'apparence que ces personnes-là disent la
vérité que non pas les autres. Et pour montrer que toute sorte d'estime n'est pas également considérable, c'est
qu'on ne se soucie point d'être estimé ni honoré de ceux que tout le monde méprise et dont on ne tient compte,
comme sont les enfants et les bêtes, au moins eu égard simplement à leur estime et aux honneurs qu'ils peuvent
rendre, puisque s'il arrive quelquefois qu'on témoigne faire cas de leur estime et de s'en mettre en peine,
c'est toujours par intérêt, ou pour quelqu autre raison.

\bigbreak

Il faut dire encore:

\begin{lieu}
	Que la possession d'un ami est une chose très douce;
\end{lieu}

À cause qu'il y a beaucoup de plaisir à aimer. Et de fait, qui est l'ivrogne et la personne aimant le vin qui
ne se plaise pas à voir du vin?

D'où il s'ensuit:

\begin{lieu}
	Qu'il y aura aussi du plaisir à être aimé;
\end{lieu}

À cause qu'on ne peut être aimé sans s'imaginer en même temps qu'on a en soi quelque bonne qualité dont tous
ceux qui ont la connaissance sont amateurs. Être aimé, au reste, proprement, veut dire être chéri pour sa
personne, et non point par intérêt.

\bigbreak

\begin{lieu}
	Être admiré encore doit être très agréable;
\end{lieu}

Puisqu'on ne peut pas être admiré sans être honoré en même temps.

\begin{lieu}
	Être flatté aussi, et avoir des flatteurs, plaît encore beaucoup;
\end{lieu}

Car tout flatteur parait en même temps admirateur et ami de celui qu'il flatte.

\bigbreak

On pourra soutenir encore:

\begin{lieu}
	Que faire les mêmes actions très souvent apporte du plaisir;
\end{lieu}

Puisque, comme nous avons déjà remarqué, l'accoutumance est agréable.

Et tout au contraire:

\begin{lieu}
	Qu'il y aura du plaisir a ne pas toujours faire la même chose et à changer parfois;
\end{lieu}

Vu que tout changement semble s'accommoder au dessein de la nature. Et de vrai, faire toujours la même chose
engendre un certain dégoût et témoigne je ne sais quel excès dans l'habitude qu'on a contractée; ce qui a fait
dire à un poète:

\begin{emphpar}
	Le changement nous plaît en toutes choses. 
\end{emphpar}

En effet, c'est par cette raison que tout ce qu'on est quelque temps sans voir, par exemple un homme ou quelque
autre chose, en parait plus agréable. Car outre que ce qu'on n'a pas vu il y a longtemps, apporte du changement
par sa présence, c'est que même il en parait plus rare, à raison qu'on ne le voit pas toujours.

\bigbreak

\begin{lieu}
	Apprendre encore, et avoir de l'admiration pour quelque chose, d'ordinaire, apporte du plaisir;
\end{lieu}

Puisque tout ce qu'on admire fait naître à l'instant le désir de savoir ce que c'est. De sorte qu'on peut
assurer que tout ce qui se fait admirer est souhaitable. En apprenant aussi, on a cet avantage que l'esprit
se perfectionne, et arrive à cet état excellent où il aspire de sa nature.

\bigbreak

Ce sont encore deux choses très agréables:

\begin{lieu}
	que d'obliger et d’être obligé;
\end{lieu}

Puisqu'on ne peut être obligé qu'en même temps on n'acquière ce qu'on désir; et de plus, qu'en obligeant,
on fait voir que non seulement on a de quoi obliger, mais même qu'en ce point on surpasse celui qu'on
oblige, qui sont deux avantages que tous les hommes souhaitent passionnément.

\bigbreak

Or, les mêmes raisons qui font dire qu'il y a du plaisir à obliger, les mêmes font dire encore:

\begin{lieu}
	Qu'il est très agréable de remontrer à son prochain et de le corriger de ses fautes;
\end{lieu}

Comme aussi:

\begin{lieu}
	D'achever quelque chose qui aura été commencé.
\end{lieu}

\bigbreak

Que s'il y a du plaisir à apprendre, à admirer, et autres choses semblables, il s'ensuivra encore:

\begin{lieu}
	Que tout ce qui sera imité parfaitement sera très agréable;
\end{lieu}

Comme sont les ouvrages de peinture, de sculpture, de poésie; en un mot, tout ce qui consiste en imitation,
quand bien même ce qui aurait été imité serait très désagréable en soi. Car enfin, le plaisir qu'on a de
voir une belle imitation ne vient point précisément de ce qui a été imité, mais bien de notre esprit qui
fait alors en lui-même cette réflexion et ce raisonnement \emph{qu'en effet, il n'est rien de plus ressemblant
et qu'on dirait que c'est la chose même,  et non pas une simple représentation}; de sorte qu'en telle rencontre,
il arrive qu'on apprend je ne sais quoi de nouveau.

\bigbreak

\begin{lieu}
	Les revers de fortune encore, et ces événements qui arrivent contre toute sorte d'attente, tels que
	d'ordinaire, représentent les tragédies et les théâtres, doivent apporter du plaisir.
\end{lieu}

Comme aussi:

\begin{lieu}
	De s'être vu en très grand danger et si près de périr que peu s'en soit fallu que cela ne soit arrivé;
\end{lieu}

Car tout ceci est surprenant et donne de l'admiration.

\bigbreak

Et parce que tout ce qui est selon la nature et qui a de la conformité avec elle est agréable, et de plus, que
toutes les choses qui sont de même genre et de même nature sont très conformes entre elles, il s'ensuit encore:

\begin{lieu}
	Que tout ce qui sera de même nature et de même genre, et aussi toutes les choses qui auront de la ressemblance,
	se plairont entre elles pour l'ordinaire. 
\end{lieu}

Par exemple un homme avec un autre homme, un cheval avec un cheval, un enfant avec un enfant, et ainsi du reste.
Et de fait, c'est de la que sont venus tous ces proverbes:

\begin{emphpar}
	Que chacun se plaît avec son pareil;

	Qu'un semblable aime son semblable;

	Qu'une bête connaît une autre bête et la cherche;

	Que la corneille est toujours avec la corneille;
\end{emphpar}

Et beaucoup d'autres.

\bigbreak

Davantage parce que toutes les choses qui se ressemblent, ou qui sont de même genre, se plaisent entre elles,
et qu'il n'y a rien qui nous soit plus semblable ou qui approche plus de notre nature que nous-mêmes, il sera
encore nécessaire de conclure:

\begin{lieu}
	Que tous les hommes généralement, plus ou moins, s'aimeront eux-mêmes;
\end{lieu}

Puisqu'il n'y a rien qui ait plus les qualités de conformité et de ressemblance qu'une personne comparée à
elle-même.

\bigbreak

Or, s'il est vrai que tous les hommes s'aiment eux-mêmes, il s'ensuivra encore:

\begin{lieu}
	Qu'ils aimeront tout ce qui viendra d'eux et y prendront du plaisir;
\end{lieu}

Comme sont leurs ouvrages, leurs discours, leurs raisonnements, ce qui doit encor servir de preuve pour montrer
que d'ordinaire ils aimeront les flatteurs et tous ceux qui auront pour eux de l'amour; enfin, qu'ils seront
jaloux d'honneur, et qu'ils auront une inclination particulière pour leurs enfants; car il n'y a rien qui soit
plus l'ouvrage d'un homme que ceux qu'ils a mis au monde.

\bigbreak

En un mot il s'ensuivra:

\begin{lieu}
	Que tous les hommes seront ravis et auront du plaisir d'achever un ouvrage qui aura été laissé imparfait;
\end{lieu}

Puisque l'ayant achevé, il semblera qu'il leur appartienne tout entier.

\bigbreak

Et parce que l'autorité et le commandement sont les choses du monde les plus agréables, il faut dire encore:

\begin{lieu}
	Qu'il y aura un très grand plaisir à passer pour un homme sage et prudent;
\end{lieu}

Puisque la prudence et la sagesse sont des vertus royales sans lesquelles on est incapable de commander.
La sagesse, au reste, est définie \emph{une science qui éclate par la diversité et le grand nombre des
connaissances, et qui peut rendre raison des effets les plus curieux et les plus propres à donner de
l'admiration}. 

\bigbreak

De plus, parce que d'ordinaire les hommes sont ambitieux et très aises de recevoir de l'honneur, il sera
encore nécessaire de conclure:

\begin{lieu}
	Que non seulement il y aura du plaisir à reprendre autrui et à le corriger de ses fautes, comme nous
	avons déjà remarqué, mais encore à s'occuper aux choses où l'on croit réussir et être plus excellent
	que les autres;
\end{lieu}

Comme a fort bien remarqué \textsc{Euripide} a l'endroit où il dit:

\begin{emphpar}
	Un artisan savant se plaît à son ouvrage,

	Il travaille sans cesse et ne perd point courage.

	Le désir de la gloire et de se surpasser

	Lui fait cent fois le jour son travail repasser.
\end{emphpar}

\bigbreak

D'ailleurs, parce que nous avons montré que le jeu est du nombre des choses qui plaisent, comme encore toutes
sortes de relâches, et aussi le rire, il sera nécessaire encore de tirer cette conséquence:

\begin{lieu}
	Que tout ce qui sera facétieux et ridicule, soit hommes, discours, actions, apportera du plaisir.
\end{lieu}

Quant au ridicule, nous en avons traité à part dans nos livres de la \emph{Poétique}.Voila pour ce qui regarde
les choses qui sont agréables et qui apportent ds plaisir. À l'égard de celles qui peuvent être fâcheuses et qui
attristent, il n'y a qu'à prendre le contraire.

\bigbreak

Nous avons donc fait voir que1s sont les motifs qui d'ordinaire portent les hommes à faire injure à leur prochain.

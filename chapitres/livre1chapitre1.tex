
\section{Servant de préface à tout l'ouvrage}

\subsection{Que la rhétorique et la dialectique se  ressemblent}

La rhétorique et la dialectique  ont beaucoup de rapport, car  toutes deux traitent de matières, qui, pour être communes, tombent
en quelque façon sous la connaissance de tout le monde, et ne sont renfermées dans les bornes d'aucune science particulière: d'où
vient aussi qu'il n'y a personne qui n'ait quelque usage de l'une et de l'autre, puisque chacun, selon sa portée et jusqu'à un
certain point, tâche d'examiner et de soutenir une raison, d'accuser et de défendre.

\subsection{Que la rhétorique est un art}

Parmi le peuple, quelques-uns réussissent à ces choses par hasard, d'autres parce qu'ils s'y sont habitués. Que si cela se fait
de toutes les deux façons, sans doute on peut avoir des règles là-dessus, et trouver une méthode assurée pour y réussir toujours;
puisque enfin il y a lieu de découvrir la cause pourquoi, et ceux qui font ceci par un pur hasard, et ceux qui le font par habitude,
arrivent au but qu'ils se proposent; or on m'avouera que c'est à l'art à donner ces règles, et que c'est là proprement son ouvrage.

\subsection{Que l'adresse principale de la rhétorique consiste aux preuves.}

Tous ceux au reste qui jusques à présent ont écrit de la rhétorique, n'ont presque rien fait de ce qu'il fallait faire, parce que
toute l'adresse de cet art est renfermée dans la preuve, le reste n'en est que l'accessoire. Cependant ils ne  parlent point des
enthymèmes et des arguments, qui font tout le corps de la preuve, et se sont amusés à des choses éloignées de leur art, et purement
étrangères: Car l'invective, la compassion, la colère et les autres passions de cette nature dont ils traitent fort au long, ne
sont point du fait de l'orateur, mais regardent le juge; de sorte que si dans toutes les justices on faisait son devoir et que
partout on se gouvernât ainsi qu'en quelques républiques, et particulièrement les mieux policées, il se trouverait que ces gens-là,
lorsqu'ils voudraient parler en public, n'auraient rien a dire. Ce n'est pas que cela ne passe pour un abus, et qu'on ne croie
qu'il devrait y avoir des lois pour s'opposer à cette licence; mais peu de gens le mettent en pratique, et ce n'est qu'en certains
lieux qu'il est expressément défendu aux orateurs de sortir de leur sujet et de ne rien dire d'inutile, comme à Athènes, encore
n'est-ce que pour les jugements qui se rendent dans l'Aréopage. Et certainement, ceux qui le font ont grande raison d'en user ainsi,
puisque jamais il ne faut pervertir un juge, ni le porter à la compassion, à la colère ou à l'envie; vu que c'est faire la même
chose que  celui qui courberait une règle dont il se voudrait servir. D'ailleurs, il n'y a personne qui ne voie que l'emploi de celui
qui plaide, et à quoi il doit s'étudier, est de montrer simplement que la chose dont il s'agit est ou n'est pas, qu'elle a été faite ou
ne l'a pas été, car de savoir si elle  est de conséquence ou non, si véritablement elle est juste ou injuste, au cas que le législateur
ne s'en soit pas expliqué, c'est au juge à le connaître, sans l'apprendre de ceux qui parlent devant lui. 

\bigbreak

On voit par là qu'il serait à souhaiter que les lois sagement établies fussent si exactes qu'elles remarquassent jusqu'aux moindres
circonstances, afin de laisser peu de choses à la discrétion des juges; et cela pour plusieurs raisons.

Premièrement, \emph{eu égard aux personnes}, attendu qu'il n'est pas si aisé de trouver d'habiles gens, et que pour un ou deux qu'on
rencontre capables de faire des lois et d'exercer la judicature, il y  en a cent qui ne le sont pas.

Secondement, \emph{à raison du temps}, vu que les lois dans leur établissement dépendent d'une longue observation et de l'expérience de
plusieurs siècles, au lieu que les jugements qui se rendent se font sur le champ; de sorte qu'en cet état, il est difficile aux juges et
à ceux qui délibèrent dans les grandes assemblées de satisfaire  entièrement à l'intérêt public, et à celui des parties.

La dernière raison, et la plus importante, est tirée \emph{des choses mêmes}; puisque enfin tout législateur n'a point à prononcer sur des
matières particulières, ni pour des personnes qui soient présentes, mais en général, et pour des personnes qui ne sont pas encore. Le juge
au contraire, et ceux qui délibèrent, ne connaissent que des faits  particuliers, où le plus souvent leur propre intérêt se rencontre; et ne
regardant que des personnes présentes, pour qui tantôt ils ont de l'amour, et tantôt de la haine. D'où vient qu'alors la passion les
aveugle, et les empêche de bien voir la vérité. 

Il est donc à propos, comme nous venons de remarquer, que le législateur laisse peu de chose au pouvoir des juges, afin, s'il est possible,
qu'ils n'aient qu'a examiner si ce qu'on leur dit est ou n'est pas, s'il arrivera ou s'il ne doit point arriver, qui sont des cas qu'un
législateur ne peut prévoir et que, nécessairement, il faut laisser à la connaissance des juges. 

\bigbreak

Que si cela est ainsi, l'on voit manifestement que ceux-là sortent du sujet de la rhétorique qui enseignent, par exemple, comment il faut
faire un exorde, une narration, et ainsi de chacune des autres parties d'un discours; parce que tout ce qu'ils font, en telle rencontre, ne
tend qu'à altérer l'esprit du juge et ne montre point en quoi consiste l'artifice de la preuve, qui est de cultiver le raisonnement et
de rendre un homme fort en enthymèmes. Aussi est-ce pour cette considération qu'encore qu'il y ait deux parties principales dans la rhétorique,
dont l'une regarde les délibérations et l'autre les matières du Barreau --- et même que la partie qui sert à délibérer soit beaucoup plus noble
et plus politique que celle qui s'arrête seulement à examiner les clauses d'un simple contrat --- tous néanmoins ont abandonné la délibération,
sans en dire le moindre mot, et pour l'autre, c'est à qui en traitera plus au  long, à qui donnera plus de préceptes. Et la raison qui les
y a portés est qu'il est peu avantageux de sortir de son sujet en parlant dans un conseil, cette partie donnant beaucoup moins d'entrée à la
malice et à la finesse que ne fait le plaidoyer, à cause que l'intérêt qu'elle regarde est un intérêt commun, et que celui qui écoute est juge
en sa propre cause; de sorte qu'ici l'orateur n'a qu'à montrer simplement que ce qu'il dit est véritable. Il en va autrement du barreau, où il ne
suffit pas de prouver que la chose est, mais encore  qu'il est bon de gagner l'esprit de l'auditeur, et de le faire tourner de son côté; vu qu'il
s'agit là de l'intérêt d'autrui et qu'il n'a point à prononcer sur des choses qui le touchent. Ainsi, comme il ne regarde que sa propre
satisfaction, et qu'il n'écoute que pour faire faveur, il se laisse aisément emporter aux discours de ceux qui plaident et ne fait plus l'office
de juge. C'est aussi pour cela, comme nous disons auparavant, qu'en beaucoup de lieu, la loi défend aux orateurs de parler hors de leur sujet; ce
qu'il n'a point été nécessaire de faire dans la délibération, à cause que c'est une chose qui s'observe là assez d'elle-même.

\subsection{Que les plus fortes preuves de la rhétorique dépendent des enthymèmes}

Donc puisqu'il est certain:

\begin{emphpar}
      Que tout l'artifice de la rhétorique consiste dans la preuve,
\end{emphpar}

De plus:

\begin{emphpar}
       Que la preuve est une sorte de démonstration,
       
      Que le plus puissant moyen qu'il y ait pour démontrer, c'est l'enthymème,
\end{emphpar}

Enfin:

\begin{emphpar}
      Que l'enthymème est une manière de syllogisme,
\end{emphpar}

En un mot:

\begin{emphpar}
      Puisque c'est ou à la dialectique toute entière, ou à l'une de ses parties, à traiter du syllogisme pleinement,
\end{emphpar}

Il s'ensuit:

\begin{emphpar}
            Que quiconque sera bon dialecticien, c'est à dire qui saura comment le syllogisme se fait et de quelles propositions il est composé,
             celui-là encore aisément pourra faire des enthymème; n'ayant plus qu'à observer sur quelles matières ils s'appliquent, et en quoi ils
             sont différents des syllogismes de la logique.
\end{emphpar}

Et cela d'autant que c'est à la même faculté qui s'attache au \emph{vrai} qui est l'objet du syllogisme, de connaître encore le \emph{vraisemblable},
qui est l'objet de l'enthymème, joint que tous les hommes naturellement sont assez portés aux sciences et à la connaissance du vrai et qu'assez
souvent, hors ce qui regarde les sciences, ils découvrent la vérité en beaucoup de choses. Tellement que pour tirer de simples conjectures, et
découvrir la vraisemblance dans les matières douteuses, il ne faut point d'autre adresse ni d'autre lumière que celles qui dans les matières
certaines et infaillibles nous font raisonner régulièrement et trouver toujours la vérité.

\bigbreak

Il paraît donc, évidemment, que ceux qui ont écrit de la rhétorique jusques à présent n'ont point traité son sujet; et nous avons dit pourquoi ils ont
quitté le \emph{genre délibératif} pour s'attacher au \emph{judiciaire}.

\subsection{Que la rhétorique est utile}

On ne peut pas douter que la rhétorique ne soit utile, puisqu'elle a pour but de faire rendre la justice, et de faire connaître la vérité, qui est
une chose avantageuse et toute autre que de faire le contraire. Aussi toutes les fois qu'on ne juge pas comme il faut, cela n'arrive que parce que
l'injustice et le mensonge ont prévalu sur la justice et sur la vérité, ce qui mérite punition. 

De plus, la rhétorique est de telle conséquence que, quand nous serions les plus savants du monde, néanmoins, il nous serait difficile en parlant à
certaines personnes de les persuader, à cause que les sciences ont une façon particulière de s'expliquer et certains termes, dont il est impossible de
se servir devant des ignorants. De sorte que, pour se faire entendre à eux et pour les persuader, il faut avoir recours à des notions générales, ou
\emph{lieux communs}, ainsi que nous avons remarqué dans nos Livres des \emph{Topiques}, en traitant \emph{de la manière de parler au peuple}.

Un troisième avantage de la rhétorique est qu'il faut être capable de persuader les deux partis contraires, de même que dans la dialectique on doit
savoir argumenter de part et d'autre; non pas à la vérité qu'il faille faire tous les deux, car jamais on ne doit persuader ce qui est mauvais, mais --- la
chose est importante --- afin qu'au moins on n'ignore pas comment cela se fait, et qu'en même temps on puisse répondre à ceux qui voudraient s'en servir
pour favoriser l'injustice. Or est-il que de tous les arts il n'y a que la dialectique et la rhétorique qui fassent profession de défendre les deux partis
contraires. Ce n'est pas pourtant qu'il faille croire que les matières qui se traitent en telles rencontres soient également probables, puisque absolument
parlant, tout ce qui est véritable et meilleur de soi est aussi, et plus aisé à être prouvé, et plus capable de persuader.

Après tout, il serait ridicule de s'imaginer qu'il y eût de la honte à ne se pouvoir aider de son corps, et qu'il n'y en eût point à être privé du secours
de la parole, dont l'usage, bien plus que celui du corps, appartient à l'homme naturellement. 

De dire que la rhétorique peut beaucoup nuire si l'on s'en veut mal servir, c'est une objection qui regarde en commun toutes les bonnes choses et les plus
utiles mêmes, excepté la vertu; par exemple, la force, la santé, les richesses, les armes; puisque selon l'usage qu'on en fera, bon ou mauvais, il en viendra
un grand mal ou un grand bien. 

De tout ce que nous avons dit jusques ici, il se voit en premier lieu que la rhétorique est utile, et qu'elle n'a point un sujet particulier ni déterminé non
plus que la dialectique.

En second lieu que l'ouvrage de la rhétorique ne consiste point à persuader absolument, mais à découvrir en chaque chose ce qui est capable de le faire, et en
cela convient-elle avec tous les autres arts. Par exemple, la médecine ne promet pas de guérir infailliblement, mais seulement de contribuer à la santé autant
qu'il est possible; puisqu'on ne laisse pas de bien traiter certains malades, encore que la santé ne leur puisse être rendue.

Enfin, il se voit que c'est à la rhétorique à considérer également, et ce qui est capable de persuader en effet, et ce qui ne le peut faire qu'en apparence;
comme c'est à la dialectique à traiter du syllogisme apparent et du véritable. Je dis que c'est à la dialectique à traiter du syllogisme apparent, afin qu'on
ne croie pas que cela soit réservé au sophiste; vu que ce qui donne la qualité de \emph{sophiste} à un homme n'est point cette connaissance et cette adresse de
pouvoir user de semblables arguments, mais bien le but qu'il se propose et le dessein qu'il a de n'argumenter que pour tromper. Véritablement, il y a cette
différence entre la rhétorique et la dialectique quant à ce point que dans la rhétorique, autant est \emph{orateur} celui qui n'emploie que de faux arguments, et
qui n'en veut point employer d'autres, que celui qui ne se sert que de bons, et qui ne tâche qu'à faire connaître la vérité. Pour la dialectique, il n'en va pas
ainsi, puisque là, le dessein de tromper et de ne s'attacher qu'à de vaines subtilités est proprement ce que nous appelons être \emph{sophistes}; au lieu que le
\emph{dialecticien} ne s'attache qu'à l'art et à la vérité.

Mais traitons tout de bon maintenant de la rhétorique, et voyons de quelle façon nous pourrons venir à bout des choses que nous avons proposées. Comme si donc nous
n'avions encore rien dit de cet art, commençons par sa définition, et ensuite, nous examinerons le reste.



\section{Que la rhétorique a trois genres}

La rhétorique a sous soi trois genres, puisqu'il se trouve autant de sortes d'auditeurs. Car il faut savoir que
tout discours regarde trois choses: celui qui parle, le sujet que l'on traite et la personne à qui on parle, que
nous appelons \emph{l'auditeur}, et auquel se rapporte tout le discours. 

Tout auditeur, au reste, doit être nécessairement: ou simple auditeur, ou juge. S'il est juge, il faut que ce soit:
ou de choses qui aient été faites déjà, ou de choses qui ne le soient pas encore.

L'auditeur qui a son jugement à donner sur ce qui n'est pas encore arrivé mais qu'on propose de faire simplement
est, par exemple, le peuple d'Athènes, assemblé pour délibérer sur les affaires de la république.

Celui qui a à juger du passé et de ce qui a été fait est, proprement, le magistrat ou le juge.

Enfin, le simple auditeur est celui qui ne vient que pour contenter sa curiosité, et pour avoir le plaisir d'entendre
un excellent orateur.

De manière qu'il faut, par nécessité, qu'il y ait trois genres dans la rhétorique qui répondent à ces trois sortes
d'auditeurs:

\begin{emphpar}
Le genre délibératif;

Le genre judiciaire;

Le genre démonstratif.
\end{emphpar}

Le genre délibératif a deux parties: la \emph{persuasion} et la \emph{dissuasion}, car toujours ceux qui délibèrent font
l'une de ces deux choses, soit qu'ils délibèrent sur leurs affaires particulières ou sur les affaires publiques.

Le genre judiciaire a aussi deux parties sous soi: \emph{l'accusation} et la \emph{défense}, car il est nécessaire qu'en
plaidant, les avocats fassent l'un ou l'autre, qu'ils défendent ou qu'ils accusent.

Le genre démonstratif, pareillement, comprend deux parties: la \emph{louange} et le \emph{blâme}.

\bigbreak

Chacun de ces genres a aussi un \emph{temps} qui lui est particulièrement affecté.

\emph{L'avenir} appartient au délibératif, car tout homme qui délibère, soit qu'il conseille ou dissuade, délibère toujours
sur ce qui n'est pas encore arrivé. 

Le \emph{passé} convient au judiciaire, car on n'accuse et l'on ne défend jamais que les actions qu'une personne a faites. 

Enfin, le \emph{présent} est le plus propre au genre démonstratif, puisqu'on ne loue ou ne blâme que ce qui est effectivement.
Ce n'est pas, néanmoins, qu'assez souvent, en telle rencontre, les orateurs ne fassent aussi mention du passé afin d'en
renouveler la mémoire; et même par avance de ce qui n'est pas encore, comme par un préjugé de l'avenir. 

\bigbreak

De plus, chacun de ces genres se propose un but et une fin particulière; de sorte que, comme il y a trois genres, il se
trouve aussi trois fins différentes.

Celui qui délibère se propose pour but ce qui est \emph{utile} ou \emph{nuisible}, car tout orateur qui entreprend de
persuader une chose la propose toujours comme la meilleure, et s'il veut la dissuader, il tâche de faire voir que c'est
la pire. Ce n'est pas qu'il ne se serve encore de tout le reste que les autres genres se proposent, afin d'en fortifier
sa preuve. Par exemple, il tâche de montrer que cette même chose est encore juste ou injuste, honnête ou contre l'honneur. 

Ceux qui plaident se proposent toujours de faire voir que la chose donc il s'agit est juste ou injuste et pareillement,
se servent de tout le reste pour ce dessein.

Enfin, ceux qui ont à louer ou à blâmer prétendent seulement démontrer que ce qu'ils louent ou blâment est honnête ou
honteux, et tout de même, y rapportent les autres choses que nous venons de dire.

Et une preuve certaine que chacun de ces genres ne se propose point une autre fin que celle dont nous venons de parler,
c'est que bien souvent, il n'y aurait point de contestation touchant les autres points. Par exemple, ceux qui plaident
demeurent souvent d'accord qu'une chose a été faite, et même qu'elle a porté préjudice, mais jamais ils n'avouent qu'ils
aient fait une injustice, autrement il serait inutile de plaider. Le même se peut dire de ceux qui délibèrent. Souvent,
ils accordent tout le reste, mais jamais ils n'avouent que ce qu'ils persuadent de faire soit inutile, ou que l'entreprise
dont ils veulent détourner soit avantageuse. De savoir maintenant si ce qu'ils conseillent est contre la justice ou non,
par exemple, d'assujettir des peuples voisins et qui n'ont jamais fait de tort, c'est bien souvent à quoi ils ne pensent
pas seulement, tant ils s'en mettent peu en peine. Il en est de même de ceux qui louent ou blâment quelqu'un, tant s'en
faut qu'ils examinent s'il a fait des choses qui lui aient apporté du profit ou de la perte que, bien souvent, ils le
louent davantage quand il a méprisé son propre intérêt pour entreprendre quelque action glorieuse. Par exemple, ils
donnent des louanges à Achille, de ce qu'étant assuré de perdre la vie en vengeant la mort de Patrocle, son meilleur ami,
il aima mieux mourir que de laisser cette mort impunie. Cependant il est certain, que d'une part cette mort lui fut
glorieuse, d'un autre côté la vie lui était utile.

\subsection{De la nécessité des lieux propres et des lieux communs}

On voit par ce qui a été dit qu'il faut avoir, premièrement, un certain fonds, ou amas de propositions sur toutes les matières
dont nous venons de parler qui appartiennent aux trois genres; et de plus, on se doit souvenir que les propositions dont la
rhétorique se sert sont toutes tirées des signes, tant simples que nécessaires, et du vraisemblable. La nécessité, au reste,
d'avoir ainsi des propositions toutes prêtes vient de ce qu'absolument, on ne saurait faire de syllogismes sans propositions.
Et ainsi, l'enthymème étant une espèce de syllogisme, il faut aussi qu'il soit composé de propositions, mais de propositions
de la qualité de celles que nous avons remarquées.

\bigbreak

Mais parce qu'on ne peut pas dire que ce qui est du tout impossible puisse jamais avoir été fait, ni qu'il le puisse être, vu
que cela n'appartient qu'aux choses qui sont possibles de leur nature; outre cela parce qu'il est encore impossible que ce qui
n'a point été, ou qui jamais ne doit être, ait été fait déjà, ou soit fait à l'avenir, il sera encore nécessaire à l'orateur,
soit dans une délibération, soit en plaidant, soit dans les sujets qui regardent le genre démonstratif, d'avoir un autre fonds
ou amas de propositions, tant sur la matière du \emph{possible} que sur celle de \emph{l'impossible}, afin de pouvoir connaître
si une chose aura été faite ou non, si elle arrivera ou n'arrivera pas.

\bigbreak

Et d'autant encore que tout orateur, soit qu'il loue ou blâme, qu'il accuse ou défende, qu'il persuade ou dissuade, ne tâche
pas seulement de prouver les matières que nous venons de dire, mais assez souvent même de faire voir qu'une chose qui est bonne
ou mauvaise, honnête ou déshonnête, juste ou injuste, est encore grande ou petite, de conséquence ou non. Et cela indifféremment,
soit qu'il considère ces choses-là en elles-mêmes ou qu'il les compare entre elles; il est certain qu'il sera encore nécessaire
d'avoir des propositions et en général et en particulier, tant sur la \emph{grandeur} et la \emph{petitesse}, que sur ce que nous
appelons plus grand et plus petit, afin de savoir quel bien en particulier sera plus grand ou plus petit qu'un autre. Quelle
action sera plus juste, ou plus injuste, et ainsi du reste. 

Nous venons donc de montrer quelles sont les matières d'où se doivent tirer nécessairement les propositions dont il se faut servir.
Parlons ensuite de chacune en particulier, savoir de celles qui appartiennent au genre délibératif premièrement; en second lieu,
de celles qui appartiennent au genre démonstratif et enfin, des autres qui regardent le genre judiciaire. 



\section{Éléments de la rhétorique}

\subsection{Ce que c'est que la rhétorique}

Posons que la rhétorique est \emph{un art ou une faculté qui considère en chaque sujet ce qui est capable de le persuader}; car il n'est point d'art
qui fasse la même chose, puisque tous les autres arts et toutes les autres facultés ne traitent que leur sujet et ne persuadent que là-dessus. Par
exemple, la médecine ne raisonne et ne persuade que sur ce qui regarde la santé et la maladie, la géométrie que sur les changements et les différences
remarquables qui arrivent aux grandeurs, et enfin, l'arithmétique, que sur ce qui touche le nombre; Ainsi en est-il des autres arts et des autres
sciences. Mais pour la rhétorique, quelque sujet qu'on lui propose, elle a droit, pour ainsi dire, d'y voir ce qui peut persuader. Aussi avons nous
remarqué qu'elle n'a point un sujet particulier ni déterminé sur lequel elle travaille. 

\subsection{Qualité des preuves de la rhétorique}

La rhétorique a deux sortes de preuves, les unes sont \emph{artificielles} et les autres \emph{sans artifice}. J'appelle preuves sans artifice celles
qui ne dépendent point de notre industrie, mais que nous trouvons toutes faites, comme sont les témoins, les réponses faites à la torture, les contrats
et autres choses semblables. Et je nomme artificielles toutes celles que nous pouvons trouver de nous-mêmes et par les règles de la rhétorique, de
sorte qu'il faut inventer celles-ci, au lieu qu'on se sert simplement des autres.

Pour les preuves \emph{artificielles}, il s'en trouve de trois espèces.

La première est fondée sur les mœurs et sur la bonne opinion qu'on a de celui qui parle.

La seconde vient de la disposition de l'auditeur, et d'avoir préparé son esprit d'une certaine façon.

Et la dernière enfin naît du discours, soit que véritablement, on ait démontré son sujet, ou seulement en apparence.

\bigbreak

L'orateur persuade à l'occasion de sa personne et de ses mœurs, lorsqu'il parle de manière qu'il se rend digne de foi; car la vertu est d'un tel crédit
qu'absolument, nous ajoutons toujours plus de foi et plutôt aux gens de bien qu'aux autres, et cela généralement en tout, mais particulièrement dans les
matières douteuses, et où l'esprit de part et d'autre ne voit point de raison qu'il puisse suivre avec sûreté; vu qu'alors nous nous abandonnons à eux
entièrement et croyons tout ce qu'ils disent. Or, il faut remarquer que ce crédit doit aussi venir de l'adresse de notre discours, et non pas simplement
de la préoccupation de l'auditeur, ni parce qu'il avait déjà cette bonne opinion de nous avant que de nous écouter; car enfin on ne doit point s'arrêter
à ce que disent quelques-uns de ceux qui ont traité de la rhétorique, qui, à propos des bonnes mœurs et de cette probité qui doit éclater dans le discours
de l'orateur, soutiennent qu'absolument, elle est inutile et ne contribue en rien à gagner l'esprit; mais tant s'en faut que cela soit, que même, c'est un
des plus forts et des plus puissants moyens qu'il y ait pour persuader.

On persuade à l'occasion de ses Auditeurs  lorsque par le discours, on les porte à quelque passion; aussi jugeons nous bien autrement quand nous sommes
tristes que quand nous sommes joyeux, et bien autrement quand nous aimons que quand nous avons de la haine; Or, comme il a déjà été dit, c'est la seule
chose que tous les rhétoriciens d'aujourd'hui se sont efforcés de traiter, mais il en sera parlé plus particulièrement quand nous serons au discours des
passions. 

Enfin, on persuade par la force du discours lorsque employant tout ce qui peut servir à prouver le sujet que l'on traite, on fait voir que la chose dont
il s'agit est véritable en effet, ou en apparence.

\bigbreak

Que si les preuves \emph{artificielles} dépendent de ces trois points, il est certain qu'il faudra s'étudier à trois choses. Premièrement, à savoir faire
des syllogismes. Secondement, à connaître les mœurs et les vertus de chacun. En dernier lieu, à connaître les passions. Par exemple, quelle est la nature
de chaque passion en particulier, sa différence, ce qui la fait naître, et comment on le peut faire: de sorte qu'il se voie  par là que la rhétorique est
comme un germe et un rejeton, non seulement, de la dialectique, mais encore de cette partie de la morale qu'on peut avec raison nommer \emph{politique}. Et
de fait, c'est pour cela que la rhétorique affecte de paraître sous un habit emprunté et de passer pour politique; aussi bien que ceux qui font profession
d'orateur, qui d'ordinaire se flattent de cette vanité en partie par présomption, en partie par ignorance, et en partie pour d'autres considérations
humaines.

Je dis que la rhétorique est comme un rejeton de la dialectique, parce qu'elle en est une partie et une image, ainsi que nous avons remarqué dès le
commencement; vu que ni l'une ni l'autre, ne sont point des sciences qui s'attachent à un sujet particulier, mais bien certaines facultés qui cherchent
à trouver des raisons dans toutes sortes de matières. Mais c'est assez parler de leur pouvoir, et du rapport qu'elles ont.

\subsection{De l'exemple et de l'enthymème, de leur rapport avec le syllogisme et l'induction}

Quant aux preuves qui en effet démontrent  une chose, ou qui semblent la démontrer. Tout ainsi que pour démontrer dans la dialectique, l'on se sert toujours
de l'induction et du syllogisme, soit véritable ou apparent, de même, pour démontrer dans la rhétorique, l'on se sert  toujours de \emph{l'exemple}, qui est
la même chose que \emph{l'induction}, et encore de \emph{l'enthymème} qui répond au \emph{syllogisme}. Aussi est-ce pour cette  raison que je nomme l'enthymème
et l'exemple, l'un le syllogisme, et l'autre l'induction de la rhétorique.

Et de vrai pour montrer leur parfait rapport, c'est que tout orateur qui prouve une chose par démonstration apporte toujours, ou des exemples, ou des
enthymèmes, n'ayant point dans la rhétorique d'autres moyens pour démontrer que ceux-là. Et par conséquent, si la même nécessité se rencontre dans la
dialectique et que là, il soit impossible de rien prouver démonstrativement, quelque chose même que ce puisse être, sans se servir du syllogisme et de
l'induction --- comme nous l'avons fait voir dans nos Livres des \emph{Analytiques} --- il s'ensuit que chacun de ces deux moyens à l'égard des deux
autres, je veux dire, que le syllogisme à l'égard de l'enthymème, et que l'induction à l'égard de l'exemple, ne seront qu'une même chose.

\bigbreak

De savoir maintenant la différence qu'il y a entre l'exemple et l'enthymème, nous l'avons enseigné dans nos \emph{Topiques}. Car toutes les fois qu'on veut
montrer que quelque chose est d'une certaine façon, et qu'on apporte pour  preuve un grand nombre d'autres choses toutes semblables, dans la dialectique cela
s'appelle \emph{induction}, et dans la rhétorique, \emph{exemple}. Mais lorsqu'on établit certaines propositions, et que par une conséquence nécessaire, on
vient à tirer une autre proposition toute différente, à cause seulement que ces premières propositions ont été établies --- et cela indifféremment soit que
telles propositions soient vraies ou qu'elles ne soient que vraisemblables --- dans la dialectique, cela  s'appelle \emph{syllogisme} et dans la rhétorique,
\emph{enthymème}.

De là paraît, que l'un et l'autre de ces deux moyens, quand on sait bien s'en servir, sont de très grand usage et très considérables, chacun d'eux contenant
en soi comme une espèce de rhétorique à part. Car ce que nous avons remarqué de la dialectique, dans nos livres des \emph{Méthodes}, touchant sa façon de prouver,
est encore ici remarquable pour la rhétorique, attendu que la rhétorique aussi bien qu'elle a deux styles distingués ou deux manières différentes, dont l'une prouve
tout par les exemples et l'autre par les enthymèmes, comme il se trouve des orateurs qui ne se servent que d'enthymèmes et d'autres qui n'emploient que des exemples.
Et certainement, les discours qui prouvent par les exemples ne persuadent pas moins que les autres; toute la différence qu'il y a, c'est que ceux qui prouvent par
les enthymèmes font une plus forte impression sur l'esprit et troublent d'avantage. La raison en sera dite ailleurs, quand nous montrerons de quelle façon il se faut
servir de tous les deux. Pour maintenant, il suffit de nous expliquer et de débrouiller ces matières.

\subsection{Sur quelles matières s'appliquent les enthymèmes}

Donc,

\begin{emphpar}
     Puisque tout ce qui est propre à persuader est relatif aux personnes, c'est à dire propre à persuader quelqu'un,
\end{emphpar}

De plus,

\begin{emphpar}
     Puisque ces choses-là sont de deux sortes, les unes capables de persuader d'elles-mêmes et croyables d'abord, les
      autres simplement parce qu'elles semblent établies sur des preuves de la qualité de ces premières,
\end{emphpar}

Enfin,

\begin{emphpar}
     Puisqu'il n'est point d'art qui s'arrête à considérer la nature d'aucun particulier ni qui en fasse son objet; car
      la médecine, par exemple, ne se propose point de connaître en particulier ce qui est bon pour la santé de Socrate ou
      de Callias, mais seulement de tels en général, et de  tels, qui différent de tempérament, ou qui ont telles maladies;
      vu que c'est là proprement où l'art se fait voir, n'étant pas possible à quelque science ni à quelque art que ce soit
      de connaître tous les particuliers, le nombre en étant infini.
\end{emphpar}

De là, il s'ensuit:

\begin{emphpar}
     Que la rhétorique ne se proposera pas non plus, et ne considérera point ce qui est probable à l'égard d'un tel particulier
      et qui pourra le persuader --- par exemple ce qui est probable à l'égard de Socrate ou d'Hippias --- mais bien ce qui le
      sera à l'égard de telle ou de telle sorte d'esprits, qui ont des mœurs et des inclinations différentes.
\end{emphpar}

Et cela à l'imitation de la dialectique, car la dialectique ne s'amuse pas à argumenter ni à faire des syllogismes sur tout ce qui se présente
indifféremment, pour probable même qu'il puisse être à certaines personnes, vu qu'il y a des choses qui peuvent paraître probables à certains
particuliers, par exemple à des saouls et à des extravagants; mais seulement, elle argumente sur les matières qui ne sont pas assez établies
d'elles-mêmes, et qui ont besoin de preuve. 

Pour la rhétorique, elle s'attache seulement aux matières qui ont accoutumé de tomber en délibération, car c'est là proprement son ouvrage que
d'examiner les choses sur lesquelles ordinairement nous délibérons, et de qui nous n'avons aucun art, et même encore en présence de certains
auditeurs, qui pour être peu éclairés ne sont pas capables de comprendre ce qui embrasse plusieurs choses à la fois, ni suivre de l'esprit un
raisonnement de longue haleine.

Sur cela, il faut remarquer que jamais nous ne délibérons que sur ce qui nous paraît arriver diversement, n'y ayant point d'autre occasion de
délibérer que celle-là, puisque jamais on ne met en délibération ni le passé, quand il ne s'est pu faire autrement qu'il a été fait, ni l'avenir,
quand  il est impossible qu'il arrive d'une autre façon, ni le présent, quand on ne peut pas empêcher qu'il ne soit comme il est --- du moins
tandis qu'on demeure dans cette opinion et que la chose est crue ainsi.

\subsection{La manière d'argumenter en rhétorique}

Pour ce qui est d'argumenter et d'établir une chose par syllogismes et par conséquences, on s'y prend en deux manières. Car ou l'on tire des
conséquences de propositions qui ont déjà été prouvées par d'autres syllogismes et par d'autres arguments, ou bien de propositions qui ne l'ont
pas été, mais qui ont besoin de l'être parce qu'elles ne sont pas probables d'elles-mêmes. Or est il que ni l'une ni l'autre de ces deux manières
n'est point propre à la rhétorique. La première, comme trop difficile à suivre à cause de la longueur, vu qu'on suppose que l'auditeur est simple
et peu intelligent; et l'autre est incapable de persuader, parce qu'elle avance des choses qui ne sont pas avouées de tout le monde, et qui n'ont
aucune vraisemblance. 

\bigbreak

De ces observations, il s'ensuit premièrement, touchant la matière de l'exemple et de l'enthymème, que toujours ils seront employés sur les matières
incertaines et sur des choses qui pour ordinaire arrivent de différentes façons; l'exemple, qui comme il a été déjà remarqué est la même chose
que l'induction, et l'enthymème la même chose que le syllogisme.

De plus, il s'ensuit quant à la forme de l'enthymème que d'ordinaire, il ne pourra pas avancer tant de choses, ni être composé de tant de propositions
que le syllogisme parfait, attendu que si quelqu'une de ces propositions est connue, il faut l'omettre, puisque l'auditeur de lui-même la supplée alors.
Par exemple, on veut faire savoir que Doricus, ce fameux Athlète, a vaincu aux Jeux Olympiques et a été couronné; il suffit de dire que ce Doricus a
gagné le prix sans qu'il soit besoin d'ajouter cette proposition générale que ceux qui remportent la victoire à ces jeux y sont couronnés, parce qu'on
sait bien que cela se fait toujours. 

\subsection{De quelle sorte de propositions sont composés les enthymèmes}

Donc,

\begin{emphpar}
      Puisque entre les propositions, dont la rhétorique forme ses syllogismes, il s'en trouve peu de nécessaires, car la plupart des matières
      qui se jugent dans le barreau et qui se traitent dans les délibérations sont incertaines, et peuvent arriver de différentes façons, vu
       qu'on ne délibère jamais que sur les choses qu'on veut entreprendre et qu'on propose de faire, toutes les actions qui se font dans le
       monde étant de cette nature, et n'y en ayant pas une, pour ainsi dire, qui porte un effet nécessaire et dont l'événement soit certain,
\end{emphpar}

De plus,

\begin{emphpar}
      Puisque les propositions contingentes, et qui ne sont vraies que pour l'ordinaire, doivent toujours être prouvées par d'autres de même nature
      et incertaines comme elles, et tout au contraire les nécessaires, par des nécessaires, ainsi que nous avons fait voir dans nos Livres des
      \emph{Analytiques},
\end{emphpar}

Il s'ensuit:

\begin{emphpar}
      Que les matières d'où se tirent les enthymèmes seront pour la plupart incertaines ou contingentes, et qu'il y en aura fort peu de nécessaires.
\end{emphpar}

De vrai, tous les enthymèmes qui se font ont toujours leur preuve fondée ou sur le \emph{vraisemblable} ou sur les \emph{signes}; en sorte qu'il faut que
ces signes, et ce vraisemblable, eu égard au nécessaire et à l'incertain, ou contingent, ne soient entre eux qu'une même chose. Et de fait, proprement,
le vraisemblable est ce qui se fait d'ordinaire, non % "non pas à la vérité absolument": je coupe. Ce sera déjà un peu moins wtf... 
pas absolument, comme le prétendent quelques-uns dans la définition qu'ils en donnent, entendant par là indifféremment tout ce qui peut être compris sous
le mot de vraisemblable, de quelque nature que ces choses-là puissent être, soit que la qualité d'universel leur convienne ou ne leur convienne pas. Si
bien que dans la rhétorique, le \emph{vraisemblable} se doit seulement entendre des choses qui n'arrivent pas toujours de la même façon, et de plus le rapporter
à celles à l'égard desquelles il passe pour vraisemblable, de la même sorte que \emph{l'universel} se rapporte au \emph{particulier}.

\subsection{Des signes et de leur différence}

Pour les \emph{signes}, ils sont de deux sortes. Les uns se rapportent aux choses à qui ils servent de signes, comme le particulier se rapporte à l'universel,
c'est à dire que la preuve en est la même que si l'on prouvait une proposition générale par une proposition particulière; les autres, au contraire, ont
le rapport d'un universel à un particulier; et de ceux-ci, quelques-uns sont nécessaires à qui on donne le nom de \emph{tekmérion}; les autres ne sont
pas nécessaires, et sont simplement appelés signes, sans avoir d'autre nom qui les distinguent. J'appelle \emph{signes nécessaires}, ceux qui peuvent
servir de matière au syllogisme et dont la preuve est convaincante; c'est pourquoi le signe appelle \emph{tekmérion} est mis au nombre de ceux-là. Aussi
toutes les fois qu'un orateur allègue pour preuve des choses auxquelles il ne pense pas qu'on puisse répondre, alors il qualifie ces preuves du nom
\emph{tekmérion}, comme qui dirait une preuve démonstrative et qui \emph{termine} tout le différend. Car le mot de \emph{tekmar}, d'où est tiré celui de
\emph{tekmérion}, anciennement signifiait la même chose que le mot de \emph{terme}\footnote{300 ans après\dots{}
                                                                                        <<Le \emph{tekmérion} est l'indice sûr, le signe nécessaire
                                                                                        ou encore ``le signe indestructible'', celui qui est ce qu'il
                                                                                        est et qui ne peut pas être autrement>>. Roland \textsc{Barthes}, 
                                                                                        \emph{L'aventure sémiologique}, Seuil, 1985, p.\,134}.

\bigbreak

Mais donnons des exemples de ces signes et, premièrement, de celui que nous avons remarqué avoir le rapport du particulier à l'universel. Si donc on
raisonne ainsi:

\begin{emphpar}
      Un signe que tous les habiles gens sont gens de bien c'est que Socrate, qui était un habite homme, a été très homme de bien.
\end{emphpar}

Véritablement alors, ce serait apporter un signe pour sa preuve. Un tel signe, néanmoins, ne serait pas nécessaire ni convaincant, étant facile
d'y répondre. La raison est qu'on n'en peut pas faire un syllogisme, puisque le syllogisme ne tire jamais une conclusion universelle d'une simple
proposition particulière.

Mais si quelqu'un venait à raisonner de cette autre façon:

\begin{emphpar}
      Un signe que cet homme est malade, c'est qu'il a la fièvre,
\end{emphpar}

ou bien:

\begin{emphpar}
      Un signe que cette femme est mère, c'est qu'elle a du lait aux mamelles,
\end{emphpar}

Cette sorte de signe serait nécessaire, et le seul que nous appelions \emph{tekmérion}; car quand un signe est de telle qualité que lui seul suffit
pour faire connaître que ce qu'on dit est vrai; pour lors la preuve est convaincante et ne souffre point de réponse.

Quant aux autres signes qui ont le même rapport qu'a l'universel au particulier, mais qui ne sont pas nécessaires, c'est comme se quelqu'un disait:

\begin{emphpar}
      Un signe que cette personne a la fièvre, c'est  qu'elle respire comme si, elle était hors d'haleine.
\end{emphpar}

Certainement, ce signe serait véritable; il est aisé néanmoins d'y répondre, puisqu'il arrive quelque fois qu'un homme est hors d'haleine qui
pourtant n'a pas la fièvre.

\bigbreak

\begin{flushleft}
\noindent
\begin{tabular}{|p{0.35\textwidth}|p{0.25\textwidth}|p{0.30\textwidth}|}
\cline{2-3}
\multicolumn{1}{c|}{} & incertain & nécessaire \\
\hline
Le particulier comme signe d'un universel & Socrate \rightarrow tous les habiles gens &  \\
\hline
L'universel comme signe du particulier & Je tousse \rightarrow coronavirus & N'importe quel \emph{tekmérion}\\
\hline
\end{tabular}
\caption{Légende du tableau.}
\End{flushleft}


\bigbreak

Nous avons donc enseigné ce que c'est que \emph{vraisemblable}, et ce que c'est que \emph{signe}; de plus, nous avons remarqué la différence qu'il y
a entre les signes nécessaires et ceux qui ne le sont pas. Mais ces choses-là ont été expliqués plus clairement et plus au long dans nos Livres des
\emph{Analytiques}, où nous avons touché les raisons pourquoi quelques-uns de ces signes peuvent servir de matière aux syllogismes, et pourquoi les
autres en sont incapables.

\subsection{De l'exemple, et comment il s'en faut servir}

Pour ce qui est de l'exemple, nous avons remarqué qu'il était la même chose que l'induction, et de plus nous avons fait voir en quoi consistait l'induction.
Au reste, il ne faut pas considérer l'exemple à l'égard des choses à qui il sert d'exemple comme le particulier est considéré à l'égard de l'universel, ou
comme l'universel est à l'égard du particulier, encore moins comme un universel le peut-être à l'égard d'un autre universel, mais bien toujours comme une chose
particulière est considérée à l'égard d'une autre particulière, et comme un semblable l'est à l'égard d'un autre semblable. Toutes les fois, donc, que deux
choses se trouvent sous un même genre, que l'une est plus connue que l'autre, celle qui est la plus connue est proprement ce que nous appelions \emph{exemple}.
Car si je voulais montrer que Denys de Syracuse a dessein de se faire tyran lorsqu'il demande des gardes, je dirais que Pisistrate, comme lui, demanda des gardes
d'abord, et que si tôt qu'il en eut, il se saisit du gouvernement d'Athènes. Je dirais que Théagène fit la même chose à Mégare, et alléguerais ensuite les autres,
qu'on saurait être venus à la tyrannie par telle voie, qui tous serviraient d'exemple à l'égard de Denys de Syracuse, dont il ne paraîtrait pas encore si véritablement
c'est à ce dessein qu'il demande des gardes. Or, tous ces exemples particuliers sont compris sous cette proposition générale, que \emph{quiconque pense à la tyrannie
et à se saisir du gouvernement demande des gardes}.

Nous avons donc montré en quoi consistent les preuves de la rhétorique qui paraissent démonstratives.

\subsection{De la différence des enthymèmes}

Quant aux enthymèmes, leur différence est si grande qu'il y a peu de personnes qui se puissent vanter de les bien connaître, puisque enfin cette
différence est la même que celle des  syllogismes de la dialectique --- attendu que quelques-uns sont particuliers à la rhétorique, ni plus ni moins
qu'entre les syllogismes, quelques-uns sont particuliers à la dialectique --- les autres  appartiennent aux autres arts, et aux autres facultés, tant
de celles qui sont à inventer, que de celles que nous connaissons et qui sont déjà inventées, ce qui fait qu'ils paraissent obscurs à l'auditeur, et
que ceux qui s'en servent autrement que la rhétorique ou la dialectique n'enseignent s'écartent de leur art, et ne raisonnent  plus alors ni comme un
dialecticien doit faire, ni en qualité d'orateurs.

Mais sans doute que ceci sera plus clair quand nous l'aurons davantage expliqué. Il faut donc savoir que les syllogismes que j'attribue à la dialectique
sont ceux à qui nous assignons des \emph{lieux}. Or, il y a deux sortes de lieux. Les uns sont \emph{communs} et les autres \emph{propres}. J'appelle
\emph{lieux communs} ceux qui servent à prouver diverses matières, comme de jurisprudence, de physique, de politique et de beaucoup d'autres qui diffèrent
d'espèces. Tel est le lieu commun qui traite du plus et du moins, parce que de ce lieu là, nous pourrons aussitôt tirer des syllogismes et des enthymèmes
sur des matières de droit ou de physique que de quelque autre science que ce soit. Or est-il que toutes ces matières sont distingués d'espèces et
différentes entre elles. Pour les \emph{lieux propres}, ce sont ceux qui sont particuliers à chaque genre et à chaque espèce de propositions. Par exemple,
il y a des propositions tellement dépendantes de la physique qu'on n'en saurait faire d'enthymèmes ni de syllogismes pour prouver aucune proposition de la
morale, et d'autres, au contraire, tellement dépendantes de la morale qu'on ne s'en pourrait pas servir pour prouver aucune proposition de la physique; ce
qui se doit entendre également de toutes les autres propositions particulières et spécifiques.

\bigbreak

Il y a ceci a remarquer touchant les \emph{lieux communs} que jamais ils ne peuvent nous rendre savants sur aucune matière particulière, à cause qu'ils
sont vagues et ne traitent point un sujet déterminé. Il en est tout au contraire des \emph{lieux propres}, car plus les propositions que nous eu tirerons
seront choisies et particulières au sujet que nous traitons, et plus insensiblement nous nous éloignerons de la dialectique et de la rhétorique pour nous
approcher d'une autre science; parce qu'enfin, si nous ramenons ces propositions jusqu'aux principes, alors notre raisonnement et notre preuve ne seront
plus l'ouvrage de la dialectique ni de la rhétorique, mais seulement de la science dont nous aurons touché les principes. 

Ici, nous observerons encore que la plupart des enthymèmes se tirent des lieux propres seulement, et qu'il y en a fort peu qui soient tirés des lieux
communs. Nous diviserons donc ici les enthymèmes de la même façon que nous avons déjà fait dans les \emph{Topiques}, savoir en autant de lieux propres
qu'il y a de sortes de  propositions d'où ils peuvent être tirés. Au reste, j'appelle \emph{lieux propres d'enthymèmes} les propositions qui sont
particulières à chaque genre de la rhétorique, et je nomme \emph{lieux communs} les propositions communes à tous les genres, et qui servent à prouver
toute sorte de matières.  Parlons donc, premièrement, des lieux propres des enthymèmes, mais auparavant des genres de la rhétorique, afin qu'ayant montré
combien il y en a, nous puissions voir en particulier quels font les \emph{éléments} de chacun, et les propositions qui leur conviennent.


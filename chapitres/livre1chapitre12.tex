
\section{Ceux qui font injure à autrui}

Faisons à présent connaître l'état et le raisonnement de ceux qui proposent de faire injure à autrui, et de plus,
à quelles personnes ils s'attaquent ordinairement.

\bigbreak

Les hommes donc sont portés à faire injure, en quatre façons. 

\begin{emphpar}
	Ou quand ils croient que ce qu'ils veulent en reprendre est possible et qu'eux-mêmes en pourront venir à
	bout;

	Ou qu'ils pensent qu'après l'avoir fait, on n'en saura rien et qu'ils ne seront point découverts;

	Ou si l'on vient à découvrir que c'est eux, qu'ils n'en seront point punis;

	Enfin, s'ils en sont punis, que la punition n'égalera point le profit qui leur en reviendra, soit à eux en
	particulier, ou à ceux qui les touchent et pour qui ils s'intéressent.
\end{emphpar}

De savoir, maintenant, quelles choses sont possibles à faire ou impossibles, c'est une matière que nous ne
traiterons pas encore si tôt, à cause qu'elle regarde en commun toutes les parties de la rhétorique. 

\subsection{Ceux qui se promettent l'impunité}

Or, entre les personnes qui s'engagent à faire tort à autrui, ceux-là particulièrement croient le pouvoir faire
avec impunité qui sont éloquents ou entreprenants et gens d'exécution, ou qui ont acquis une grande expérience
dans le monde, et vu ou manié une infinité d'affaires; en un mot, ceux qui ont beaucoup d'amis, et qui sont
riches. Mais surtout, ils se promettront l'impunité, s'ils se voient fortifiés de tous les avantages que nous
venons de remarquer, ou du moins quelqu'un de leurs amis, ou de leurs Associés, ou même des personnes qui dépendent
d'eux et qui sont à leur service. Car par le moyen de tous ces avantages, non seulement ils exécuteront leur mauvais
dessein, mais encore ils pourront ni être découverts, ni punis. 

Ceux-là encore se promettront l'impunité qui seront amis des personnes mêmes à qui ils voudront faire injure, ou
des juges devant qui ils auront à répondre. Car quant aux amis, il n'est rien de si aisé que de leur faire tort, à
cause qu'ils ne s'en défient pont, joint qu'ils sont plutôt d'accord et réconciliés qu'ils n'ont songé à plaider
ni à faire aucune poursuite en jugement. À l'égard des juges, il est certain encore qu'ils font toujours faveur à
leurs amis; car de deux choses l'une, ou ils les renvoient absous, ou ils ne les condamnent que légèrement.

\subsection{Ceux qui croient qu'ils ne seront point découverts}

Ceux-là aussi auront espérance de n'être point découvert de qui l'apparence sera si trompeuse qu'à juger d'eux par
l'extérieur, jamais on ne les prendrait pour avoir fait ce qu'ils auront fait effectivement, comme quand quelqu'un
en apparence très faible de corps en aura battu un autre outrageusement, qui paraîtra de beaucoup plus fort que lui,
ou quand un gueux aura couché avec une dame de condition, ou un homme très laid avec une fort belle femme. 

Les choses encore qui sont trop en jour, et exposées aux yeux de trop de monde, pourront faire croire à un
méchant homme qu'il ne sera point découvert s'il les prend; la raison est que personne ne s'en donne de garde,
et qu'ordinairement, on ne s'imagine pas qu'il y ait des gens assez hardis pour oser seulement y penser. 

De plus, on croira n'être point découvert si le crime est de telle nature et si énorme qu'on n'ait pas même
connaissance que jamais il ait été commis, puisque c'est une chose à laquelle on ne songe point et dont personne
ne se défie. Car les hommes n'ont point accoutumé de se préparer autrement contre les injures qu'ils font contre
les maladies; pas un ne craignant et ne tâchant d'éviter celles qu'il n'a pas encore éprouvées.

Tout homme encore qui n'aura point d'ennemis, ou au contraire qui en aura beaucoup, croira n'être pas découvert.
Car d'un coté, n'ayant point d'ennemis, il lui sera très facile de surprendre et de faire son coup, parce qu'on
ne se défiera point de lui. D'un autre côté aussi, ayant beaucoup d'ennemis, on aura de la peine à s'imaginer
qu'il ait osé s'attaquer à des personnes qui étaient sans cesse sur leurs gardes; outre que pour sa défense,
il aura cette raison à alléguer qu'il se fût bien empêché d'entreprendre une action de cette qualité, quand
bien même il en aurait eu envie, a cause qu'il devait être soupçonné plutôt qu'un autre.

Ceux-là enfin se pourront persuader de n'être point découverts, en faisant tort à autrui, qui auront moyen: ou
de cacher leur larcin, ou de le détourner, ou bien de lui faire changer de forme et de nature, ou de s'en défaire
promptement.

\subsection{Ceux qui ne craignent pas d'être punis}

D'autres, au contraires seront assurés d'être découverts et poursuivis en justice qui n'entreprendront pas moins
de faire du torts par exemple s'ils espèrent: ou d'échapper aux juges et de décliner leur juridiction, ou de
faire durer le procès fort longtemps, ou enfin de gagner les juges et de les corrompre.

\bigbreak

D'autres encore verront leur condamnation inévitable, mais parce qu'au plus il n'ira que d'une amende, ils ne s'en
mettront pas en peine, à cause qu'ils sauront les moyens: ou de s'en défendre et de jamais n'en rien payer, ou de
se faire donner un long terme pour y satisfaire, ou bien même que leur pauvreté sera si grande qu'ils n'auront rien
à perdre.

Ceux-là encore ne craindront point d'être condamnés en faisant tort, à qui le larcin promettra présentement: ou
bientôt un profit assuré, ou quelque avantage important et cependant, si on vient à les condamner, qu'ils en
sortiront pour peu de choses, ou même qu'il ne leur en coûtera rien; en tout cas, s'il leur en doit coûter, qu'il
se passera bien du temps avant que d'y avoir satisfait. 

On mettra encore de ce nombre tous ceux qui se proposeront d'acquérir une chose si considérable que la punition,
pour grande quelle soit, supposé même qu'elle arrive, n'égalera jamais l'avantage ni le profit qu'ils en tireront.
Tel est l'avantage que semble promettre la tyrannie à ceux qui ont envie de se rendre maîtres d'un état.

Ceux-là aussi n'appréhenderont pas d'être condamnés pour leur injustice s'ils trouvent qu'il y ait à gagner pour
eux et, quant à la punition, qu'ils en seront quittes pour un affront et pour quelque peu d'injures. Ni tous ceux,
au contraire, dont le crime les fera estimer et leur tournera à honneur, comme quand un homme viendra à venger en
même temps la mort de son père ou de sa mère, ainsi qu'il arriva à Zénon; et cependant que la punition ne pourra
aller au plus qu'à une amende, à un simple bannissement ou à quelque autre peine semblable. Car il est certain
qu'en ces deux rencontres, ces personnes-là seront portées à exécuter leur mauvais dessein; quoi qu'entre elles,
il y ait cette différence que les dernières sont louables pour leurs mœurs et les autres dignes de punition.

Ceux-là encore volontiers se hasarderont à faire tort, qui jamais n'auront été pris sur le fait, ni découverts,
ni punis; et pareillement ceux qui auront manqué plusieurs fois leur coup. Car il en prend ici comme à la
guerre, où souvent, il arrive aux vaincus de tenter la fortune de nouveau et de retourner au combat.

Ceux-là encore seront hardis qui auront espérance de jouir présentement: ou de tel plaisir en particulier, ou
d'avoir un tel profit à cause que s'il y a quelque chose à souffrir en punition ou quelque perte à faire, ce
ne sera qu'après. Tels sont d'ordinaire les incontinents et les débauchés. L'incontinence, au reste est un vice
qui regarde les choses où nous portent toutes les passions agréables et le dérèglement de la convoitise.

Il s'en trouve d'autres qui font le contraire de ceux-ci: d'abord ils préfèrent d'endurer quelque chose, ou de
faire quelque perte, parce qu'ils espèrent à l'avenir ou d'avoir un établissement assuré, ou de jouir d'un plaisir
très durable. Et c'est ce que font ordinairement ceux qui ont de la prudence, et qui ne sont pas adonnés a leur
plaisir. 

D'autres encore ne se soucieront pas qu'on sache que c'est eux qui ont fait tort, à cause qu'ils ne paraîtront
l'avoir fait que par malheur: ou par nécessité, ou dans un transport et un premier mouvement, ou par accoutumance.
En un mot, parce qu'ils paraîtront avoir plutôt failli qu'offensé malicieusement.

Ceux-là encore seront de ce nombre qui espéreront en la bonté des juges et qu'on ne les traitera pas à là rigueur,
comme aussi ceux qui seront pauvres. Or il y a deux sortes de pauvres dans le monde: les uns le sont des choses
nécessaires à l'entretien de la vie, comme ceux qui mendient, et les autres des choses superflues, comme la plupart
des riches. 

Enfin, ceux-là ne craindront point de faire du tort qui seront en très bonne estime; ni ceux, au contraire qui
seront tout à fait perdus de réputation. Car quant à ceux qui auront de l'estime, jamais on ne voudra croire que
ce soit eux, et pour les autres, ils n en seront pas plus décriés.

Voila ce que nous avions a dire touchant les personnes qui entreprennent de faire tort et injure à autrui. Voyons
maintenant ceux a qui on s'attaque ordinairement.

\subsection{Les personnes à qui d'ordinaire on fait tort}

Les méchants, donc, d'ordinaire, s'attaquent aux personnes qui possèdent les choses qu'ils n'ont pas et dont ils
ont besoin; soit qu'elles soient nécessaires à l'entretien de la vie, ou superflues, ou seulement pour la jouissance
et le plaisir.

Ils attaquent encore également leurs voisins et ceux qui sont d'un pays éloigné. Leurs voisins? Parce que leur coup
est bientôt fait. Les étrangers? À cause que, d'ordinaire, la vengeance en est tardive et qu'il leur faut beaucoup de
temps pour tirer raison du tort qu'ils ont reçu. Tels sont ceux par exemple qui attendent les Carthaginois au passage,
afin de les piller.

On fait tort encore ordinairement aux personnes négligentes et qui ne se tiennent point sur leurs gardes, ou qui sont
si simples qu'on leur peut faire accroire tout ce qu'on veut, pour ce qu'il y a lieu de s'imaginer qu'on ne sera point
découvert.

On s'adresse encore assez souvent à ceux qui sont d'un naturel lâche et qui aiment à vivre en repos; telles personnes
n'étant pas d'humeur à s'embarrasser d'un procès, à cause que la poursuite en est difficile, et qu'il faut être agissant
pour en venir à bout. 

Il en est de même de ceux qui ont beaucoup de pudeur, parce qu'ils ont l'honneur en recommandation et seraient
honteux de paraître en jugement pour un léger intérêt, et de plaider pour peu de chose.

On s'attaque encore, d'ordinaire, aux personnes que d'autres ont déjà attaquées ou offensées plusieurs fois, sans
que jamais elles en aient fait de poursuite, comme étant du nombre de ceux que le proverbe appelle \emph{la proie
des Mysiens}.

Tout ceux encore à qui une personne n'a jamais fait tort, et ceux au contraire à qui, plusieurs fois déjà, elle en
a fait sont en grand danger d'en être attaqués, à cause que ni les uns ni les autres ne se tiennent point sur leurs
gardes. Ceux-ci, parce qu'ils ne croient pas qu'elle leur en veuille plus faire; les autres, parce qu'elle ne leur
en a pas encore fait.

On se propose aussi de faire injure à ceux qui ont déjà été traduits en justice pour plusieurs crimes, ou à qui il
est très facile de faire faire le procès, à cause que telles gens n'oseront pas s'en plaindre, soit pour la crainte
qu'ils auront des juges, soit pour n'être pas en état d'être crus; ce qui se peut dire encore de tous ceux qui sont
haïs ou enviés de tout le monde.

D'ordinaire, encore, on s'attaque à ceux contre qui on a quelque prétexte et quelque raison spécieuse, soit qu'on
aille rechercher l'histoire de leurs ancêtres et qu'on déterre des querelles mortes et ensevelies, soit qu'on se
plaigne d'eux en particulier ou de quelqu'un de leurs amis; par exemple: ou pour être en état d'en recevoir
présentement du tort, ou pour en avoir déjà reçu plusieurs fois, soit en sa propre personne, soit en celle de ses
amis ou de ceux de qui on prend les intérêts. Car comme dit fort bien le proverbe: \emph{la malice n'a besoin que
de prétexte}.

On fait tort encore indifféremment à les ennemis et à ses amis propres. À ses amis? parce qu'il est très facile de
le faire. À ses ennemis? À cause qu'il y a du plaisir.

Il en est de même de ceux qui n'ont point d'amis, et des autre qui ne sont ni éloquents ni gens d'exécution, car:
ou ces personnes-là n'auront pas seulement la hardiesse de poursuivre en justice ceux qui leur auront fait tort,
ou, si elles le font, elles s'accorderont bientôt, ou même, ne gagneront rien à plaider.

Ceux-là aussi seront sujets à être attaqués, à qui il n'est pas avantageux de s'arrêter longtemps en un même lieu
dans l'attente qu'un procès soit terminé, ou qu'ils soient dédommagés et remboursés de leurs frais. Tels sont,
d'ordinaire, les personnes de dehors, et ceux qui n'ont d'autre revenu que le travail de leurs mains, car, pour
peu de chose, on compose avec eux, étant facile de les contenter.

On attaque encore volontiers ceux qui ont fait beaucoup de tort en leur vie, ou qui ont fait la même injure à
d'autres qu'on a dessein de leur faire; à cause qu'il ne semble pas que ce soit une injustice de traiter un méchant
homme de la même sorte qu'il a accoutumé de traiter les autres, comme quand quelqu'un qui est connu pour un
querelleur et pour battre ordinairement viendra lui-même à être très bien battu.

On tâche aussi de faire injure a ceux de qui, autrefois, on a reçu quelque déplaisir, ou qui ont eu dessein d'en
faire, ou qui ne manquent pas de volonté pour cela, ou même qui s'y préparent et qui font tout ce qu'ils peuvent
pour en venir à bout. Car non seulement on y trouvera du plaisir, mais encore cela fera honneur, outre qu'il ne
semblera pas qu'on ait fait une injustice.

C'est encore une occasion de faire injure à quelqu'un si en attaquant on est assuré de faire une chose agréable
et qui plaira extrêmement: ou à ses amis, ou à ceux qu'on estime beaucoup, ou aux personnes pour qui on a de
l'amour, ou à ses maîtres; en un mot, à tous ceux donc on dépend ou de qui on attend quelque faveur.

On cherche encore à nuire aux personnes qu'on a autrefois accusées de quelque crime, ou à l’amitié desquelles
on a renonce, témoin ce que fit Calippe contre Dion. Et ce qui donne d'autant plus de hardiesse alors, c'est
que même il ne semble pas qu'on fasse une injustice.

On attaque encore les personnes qu'on sait que d'autres sont tout prêts d'attaquer si on ne les prévient, comme n'y
ayant plus lieu de délibérer si on le doit faire ou non. De là vient qu'Ænésidème envoya des présents à Gélon pour
l'avoir prévenu en la réduction de certains peuples qu'il avait dessein d'assujettir lui-même.

On s'adresse encore à ceux à qui on ne doit faire qu'une seule fois du tort pour être en état de leur faire après
beaucoup de bien, à cause qu'il sera facile alors de guérir le mal, et les récompenser de leur perte. C'était sur
ce fondement que Jason le Thessalien avait accoutumé de dire \emph{qu'il est bon quelquefois de faire un peu de mal,
pour être en état après de faire beaucoup de bien}.

\subsection{Les injustices qui se font d'ordinaire}

Pour ce qui est des injustices, d'ordinaire on se laisse aller à celles que la plupart ou tout le monde fait, vu
qu'alors on se persuade qu'on aura sa grâce aisément.

On cherche encore a faire tort dans les choses qu'il est facile de cacher. Or ces choses-là sont de plusieurs sortes.
Les unes se consument en peu de temps, comme tout ce qui est bon à manger; d'autres sont aisées à déguiser, soit
qu'on leur donne une nouvelle figure, ou qu'on leur fasse changer de couleur, ou qu'on les mêle; d'autres peuvent
être détournées en divers lieux, comme tout ce qui est facile à transporter, ou qui tient peu de place; et quelques-unes
enfin sont telles, que comme celui qui les veut dérober en a beaucoup chez lui toutes semblables, jamais on ne
pourra les reconnaître lorsqu'elles seront ensemble. 

On fait encore injure dans les choses qu'on sait être honteuses à dire aux personnes mêmes à qui l'injure est faite,
comme quand on a abusé de la femme de quelqu'un, ou que lui-même ou ses enfants ont été contraints de céder à la
brutalité d'un infâme. 

On fait tort enfin et injure dans les choses pour lesquelles on peut intenter des procès sans se décrier et passer
pour chicaneur: ou à cause qu'elles sont de peu d'importance, ou parce que ce sont des fautes pardonnables. 

C'est a peu prés ce qui se peut dire sur cette matière, soit à l'égard de ceux qui font tort, ou des choses qu'ils
recherchaient, ou des personnes qu'ils attaquent; soit à l'égard des motifs et des raisons qui d'ordinaire les portent
à exécuter leur mauvais dessein.

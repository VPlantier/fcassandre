
\section{De l'autorité souveraine, et de chaque sorte d'état en particulier}


Après tout, le plus excellent moyen et le plus fort pour persuader et parler avec succès dans
une assemblée publique où l'on délibère, c'est de connaître toutes les formes de gouvernement
qu'il y a, les mœurs de chacune, et encore de savoir distinguer leurs lois, leurs coutumes et
tout ce qui est utile à un état. Car les hommes ont cela  qu'ils se laissent aller à leur intérêt
quand on leur propose des choses qui doivent leur apporter de l'utilité. Or est-il que dans un état,
tout ce qui sert à le maintenir est ce qui est le plus utile. Faisons donc connaître, en peu de mots,
quelle est la nature de chaque état en particulier, et premièrement en quoi consiste l'autorité
souveraine. 

\bigbreak

L'autorité, ou la puissance, souveraine n'est autre chose que ce qu’établissent et ordonnent, dans un
état, ceux qui y commandent et qui en ont la conduite. Cette autorité, au reste, se divise en autant
d'espèces qu'il y a de formes de gouvernement, car autant qu'il y a de gouvernements différents, autant
y a-t-il de souverainetés différentes.

\bigbreak

Touchant les formes du gouvernement, il s'en trouve quatre:

\begin{emphpar}
	La démocratie;

	L'oligarchie;

	L'aristocratie;

	La monarchie;
\end{emphpar}

De manière que ce qui commande dans ces états et qui a l'autorité en main y doit être considéré: ou comme
partie simplement, ou comme le tout. C'est à dire que cette autorité souveraine est: ou partagée en
plusieurs, ou renfermée toute entière en une seule personne.

La démocratie, ou l'état populaire, est une forme de gouvernement où les charges se donnent au sort.

L'oligarchie, ou le gouvernement de peu de personnes, est un état où celui qui possède davantage a le plus
d'autorité.

L'aristocratie est une forme de gouvernement où commandent ceux qui ont eu une meilleure éducation. Par
éducation, j'entends cette instruction et ces mœurs que les lois d'un tel état prescrivent. Car il faut
savoir que dans l'aristocratie, il n'y a que ceux qui ont été parfaits observateurs des lois qui montent
aux charges et qui prennent le maniement des affaires. Et parce que des personnes qui vivent ainsi paraissent
très honnêtes gens, cette forme de gouvernement a emprunté son nom de là, car le mot d'\emph{aristocratie}
proprement veut dire un état où les plus honnêtes gens ont l'autorité.

Il est aisé de connaître ce que c'est que la monarchie par le nom qu'elle porte, puisqu'il marque un état où
un seul homme commande à tous les autres. Il y a pourtant cette différence à faire que lorsque celui qui
gouverne observe quelque ordre, on l'appelle royauté, et tyrannie, au contraire, quand celui qui commande
gouverne à sa fantaisie sans observer ni règles ni lois.

Pour être capable encore de persuader dans uns assemblée publique, il ne faudra pas ignorer quelle fin se
propose en particulier chacune de ces formes de gouvernement; puisque tout ce qui se fait dans un état est
toujours rapporté au but et à la fin que cet état se propose.

La fin que se propose la démocratie, c'est la liberté. 

L'oligarchie se propose les richesses. 

L'aristocratie, la bonne éducation et l'exacte observation des lois. 

Et la tyrannie a pour but d’entretenir des gardes pour la sûreté de celui qui commande.

Pour persuader, donc, dans les Assemblées où l'on délibère, il faudra savoir distinguer les lois, les coutumes,
et tout ce qui est utile et qui se rapporte à la fin que se propose chaque état, puisqu'on n'entreprend jamais
rien dans quelque état que ce soit qu'on ne le rapporte toujours au but et à la fin particulière que cet état
se propose.

\bigbreak

Mais parce que l'orateur ne persuade pas seulement quand il démontre et prouve son sujet, mais encore lorsqu'il
parle de sorte qu'on peut juger de ses mœurs par son discours --- et de fait, souvent, il arrive que nous n'ajoutons
foi aux paroles d'un homme qu'a cause qu'il nous parait tel en particulier, je veux dire honnête homme, ou affectionné,
ou tous les deux ensemble --- pour cette raison, il sera encore à propos de savoir quelles mœurs conviennent à chaque
forme du gouvernement, vu qu'il n'est rien de plus puissant pour persuader que de faire paraître en sa personne des mœurs
conformes à celles de l'état où on parle. Ainsi, ce sont ces mœurs-là même qu'il faudra prendre pour modèle, et ne point
recourir ailleurs. Or ce qui les donnera à connaître, est la façon d'agir qu'affecte chaque état dans tout ce qu'il
entreprend, et ce choix particulier auquel il se détermine; car ceci se rapporte toujours au but et à la fin qu'ils se
proposent tous.

\bigbreak

A l’égard donc des choses que doit connaître l'orateur qui a à délibérer, soit qu'elles ne soient pas encore arrivées
ou le soient déjà; de plus, pour ce qui est des propositions dont, il se doit servir quand il aura à montrer que tel ou
tel moyen qu'il propose est utile et avantageux; enfin, pour ce qui regarde les mœurs, les lois, les coutumes de chaque
forme de gouvernement, nous en avons parlé autant qu'il était à propos de le faire à présent, puisque cette matière a
déjà été examinée ailleurs et traitée à fond dans nos livres de la \emph{Politique}.

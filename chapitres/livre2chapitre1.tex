
\part{Le second livre}
\section{Que l'orateur doit avoir une connaissance particulière des moeurs et des passions}

Dein Syria per speciosam interpatet diffusa planitiem. hanc nobilitat Antiochia, mundo cognita civitas, cui non certaverit alia 
advecticiis ita adfluere copiis et internis, et Laodicia et Apamia itidemque Seleucia iam inde a primis auspiciis florentissimae.

Ipsam vero urbem Byzantiorum fuisse refertissimam atque ornatissimam signis quis ignorat? Quae illi, exhausti sumptibus bellisque 
maximis, cum omnis Mithridaticos impetus totumque Pontum armatum affervescentem in Asiam atque erumpentem, ore repulsum 
et cervicibus interclusum suis sustinerent, tum, inquam, Byzantii et postea signa illa et reliqua urbis ornanemta sanctissime 
custodita tenuerunt.

Qui cum venisset ob haec festinatis itineribus Antiochiam, praestrictis palatii ianuis, contempto Caesare, quem videri decuerat, ad 
praetorium cum pompa sollemni perrexit morbosque diu causatus nec regiam introiit nec processit in publicum, sed abditus multa in 
eius moliebatur exitium addens quaedam relationibus supervacua, quas subinde dimittebat ad principem.

Atque, ut Tullius ait, ut etiam ferae fame monitae plerumque ad eum locum ubi aliquando pastae sunt revertuntur, ita homines 
instar turbinis degressi montibus impeditis et arduis loca petivere mari confinia, per quae viis latebrosis sese convallibusque 
occultantes cum appeterent noctes luna etiam tum cornuta ideoque nondum solido splendore fulgente nauticos observabant quos 
cum in somnum sentirent effusos per ancoralia, quadrupedo gradu repentes seseque suspensis passibus iniectantes in scaphas 
eisdem sensim nihil opinantibus adsistebant et incendente aviditate saevitiam ne cedentium quidem ulli parcendo obtruncatis 
omnibus merces opimas velut viles nullis repugnantibus avertebant. haecque non diu sunt perpetrata.

Erat autem diritatis eius hoc quoque indicium nec obscurum nec latens, quod ludicris cruentis delectabatur et in circo sex vel 
septem aliquotiens vetitis certaminibus pugilum vicissim se concidentium perfusorumque sanguine specie ut lucratus ingentia 
laetabatur.

Superatis Tauri montis verticibus qui ad solis ortum sublimius attolluntur, Cilicia spatiis porrigitur late distentis dives bonis omnibus 
terra, eiusque lateri dextro adnexa Isauria, pari sorte uberi palmite viget et frugibus minutis, quam mediam navigabile flumen 
Calycadnus interscindit.

Quam quidem partem accusationis admiratus sum et moleste tuli potissimum esse Atratino datam. Neque enim decebat neque 
aetas illa postulabat neque, id quod animadvertere poteratis, pudor patiebatur optimi adulescentis in tali illum oratione versari. 
Vellem aliquis ex vobis robustioribus hunc male dicendi locum suscepisset; aliquanto liberius et fortius et magis more nostro 
refutaremus istam male dicendi licentiam. Tecum, Atratine, agam lenius, quod et pudor tuus moderatur orationi meae et meum 
erga te parentemque tuum beneficium tueri debeo.

Nemo quaeso miretur, si post exsudatos labores itinerum longos congestosque adfatim commeatus fiducia vestri ductante 
barbaricos pagos adventans velut mutato repente consilio ad placidiora deverti.

Ob haec et huius modi multa, quae cernebantur in paucis, omnibus timeri sunt coepta. et ne tot malis dissimulatis paulatimque 
serpentibus acervi crescerent aerumnarum, nobilitatis decreto legati mittuntur: Praetextatus ex urbi praefecto et ex vicario 
Venustus et ex consulari Minervius oraturi, ne delictis supplicia sint grandiora, neve senator quisquam inusitato et inlicito more 
tormentis exponeretur.

Eodem tempore etiam Hymetii praeclarae indolis viri negotium est actitatum, cuius hunc novimus esse textum. cum Africam pro 
consule regeret Carthaginiensibus victus inopia iam lassatis, ex horreis Romano populo destinatis frumentum dedit, pauloque 
postea cum provenisset segetum copia, integre sine ulla restituit mora.

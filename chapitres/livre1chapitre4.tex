
\chapter{Le genre délibératif}
\section{Des matières qui tombent en délibération}

Dans ce genre ici, ce qu'il y a à faire premièrement, c'est d'avoir égard à la qualité des \emph{biens} et des
\emph{maux} que celui qui a à délibérer examine d'ordinaire, et sur lesquels il donne son avis. Car assurément,
il ne les examine pas tous, n'y ayant que les incertains auxquels il s'arrête, et qu'il juge également pouvoir
arriver et ne pas arriver, puisque jamais on ne met en délibération ni tout ce qui arrive nécessairement et de
la même façon, ni ce qui de toute impossibilité ne peut être.

Il est encore certain qu'on ne met pas en délibération tous les biens qui sont incertains absolument, puisqu'il
y en a qui dépendent de la nature et d'autres de la fortune, qui tantôt arrivent et n'arrivent pas, sur lesquels
il serait inutile de délibérer. D'où il est facile de voir quels sont les biens ou les maux qui peuvent tomber
en délibération.

Ce sont donc tous ceux qui de leur nature se rapportent à nous et qui sans nous n'arriveraient point, comme ayant
en nous-mêmes le principe de leur production, car d'ordinaire nous délibérons sur une chose jusqu'à ce que nous
ayons reconnu si elle est en notre pouvoir ou s'il nous est impossible de la faire.

Au reste, ce n'est pas ici le lieu de faire une exacte recherche, ni un dénombrement particulier de toutes les
choses dont les hommes ont accoutumé de délibérer, bien loin d'en traiter à fond et d'en donner une parfaite
connaissance, puisque cet emploi appartient à un art plus excellent et plus intelligent que la rhétorique; car
tant s'en faut que la rhétorique soit capable de rien traiter à fond, que même on lui a attribué beaucoup plus de
connaissance qu'il ne lui en appartient naturellement. Aussi, ce que nous avons remarqué au commencement est-il
vrai, que la rhétorique est composée premièrement de l'analytique, qui est une portion de la logique, en second lieu,
de cette partie de la politique qui s'attache aux mœurs et de plus, qu'elle ressemble à la dialectique en partie, et
en partie à la manière trompeuse de raisonner des sophistes. Mais la plus forte preuve qu'on ait qu'elle ne peut rien
traiter à fond, c'est que plus un orateur prendra à tâche d'employer ou la dialectique ou la rhétorique, non pas
comme de simples facultés qui raisonnent en général, mais comme des sciences exactes, et plus sans y penser il détruira
leur nature, puisque alors, s'en servant comme de sciences, il les renfermera dans de certains sujets, au lieu qu'elles
font profession de discours sur toutes sortes de matières. Ne laissons pas néanmoins de traiter ces choses, de sorte
que nous n'omettions rien de tout ce qui peut servir à notre dessein, et qu'il en demeure encore assez pour occuper la
politique.

\bigbreak

Il y a donc cinq points principaux qui donnent lieu aux assemblées publiques et sur lesquels tout le monde délibère, car
on délibère toujours:

\begin{emphpar}
	Ou sur la matière des finances,

	Ou touchant les affaires de la guerre et de la paix, 

	Ou pour la garnison des places, 

	Ou sur le fait des vivres et des marchandises, qu'on apporte de dehors et qui se transportent ailleurs,

	Ou enfin pour l'établissement des lois.
\end{emphpar}

De manière que si un orateur est obligé de parler sur les finances, il faudra qu'il sache en premier lieu quels sont les
revenus de l'état et à combien ils montent, afin que, si quelque fonds est diverti, on le rétablisse, ou si quelque droit a
été diminué, qu'on l'augmente. Il faudra qu'il sache encore tout ce que l'état dépense chaque année, afin que si quelqu'une
de ces dépenses est superflue, on la retranche, ou qu'on diminue celle qui sera trouvée trop grande, car non seulement on
devient plus riche quand on ajoute à ce qu'on possède déjà, mais même quand on retranche les dépenses inutiles. Or pour parler
pertinemment de toutes ces matières, il ne suffira pas simplement de les comprendre par sa propre expérience et par ce qui
sera arrivé dans l'état où l'on est, mais encore, il sera nécessaire de savoir tout ce qui aura été inventé là-dessus, et tout
ce qu'en disent les histoires.

Ainsi en doit-il être si nous avons à délibérer touchant les affaires de la guerre et de la paix. Il faudra qu'il sache la
puissance de l'état, combien il a de forces présentement, et jusqu'à quel point on les peut accroître; de plus, en quoi elles
consistent et celles qu'il y faudrait ajouter. Il sera bon encore de savoir les guerres que l'état a soutenues autrefois, et
comment il les a terminées. Et non seulement il les faudra savoir en particulier, mais encore celles de tous les autres états
voisins. Il ne faudra pas non plus ignorer quels sont les peuples à qui il sera glorieux de faire la guerre, afin que faisant
la paix avec ceux qui seront plus puissants que nous, il soit après en notre disposition de prendre les armes contre les autres
qui seront plus faibles. Il faudra aussi pouvoir faire comparaison de nos forces avec celles des ennemis, afin de connaître si
elles sont égales ou inégales, puisque en ce point consiste assez souvent le gain ou la perte des batailles. Or pour cela, il ne
suffira pas d'avoir fait réflexion sur toutes nos guerres en particulier, ni d'en avoir remarqué les évènements, mais encore il
sera nécessaire d'avoir fait la même chose sur toutes les guerres des autres peuples; vu que d'ordinaire les entreprises qui se
ressemblent ont des succès semblables.

Pour ce qui est des garnisons, non seulement il ne faudra pas ignorer comment une province est gardée, mais encore il faudra connaître
et la qualité de la garnison, et le nombre de ceux qui la composent, et la situation même de chaque place forte, afin que si quelque
garnison est faible, on la renforce, ou si quelqu'une est trop grosse, qu'on la diminue, et encore afin que les places les plus
importantes soient aussi les mieux gardées. Or est-il qu'il est impossible de savoir toutes ces choses, si l'on n'a une connaissance
particulière du pays.

Quant aux vivres, il faudra savoir, et la quantité qui sera nécessaire pour l'entretien de l'état, et la qualité de ceux qui croissent
dans le pays ou qu'on apporte d'ailleurs, et de plus, quelles sont les marchandises qui viennent de dehors, ou qui doivent être transportées.
Et le tout, afin que nous fassions alliance et amitié avec les peuples, ou qui emporteront ce que nous aurons de trop, ou qui nous fourniront
les choses nécessaires à la vie, car il se faut donner de garde principalement d'offenser deux sortes de personnes: ceux qui sont plus
puissants que nous et ceux qui nous sont absolument utiles. Voila pour ce qui regarde la sûreté d'un état, et qu'il faut qu'un orateur connaisse.

\bigbreak

Il nous reste à parler du dernier point, qui n'est pas moins considérable et que celui qui délibère ne doit pas non plus ignorer, savoir est,
de l'établissement des lois. Car c'est principalement de l'observation des lois et de leur établissement que dépend le salut d'un état. Il
faudra donc qu'il sache encore combien il y a de formes de gouvernements, ce qui convient à chacun et ce qui les détruit, soit que ces choses-là
leur soient propres et essentielles, soit que de leur nature, elles leur soient contraires. Je dis que les états peuvent être détruits par les
choses mêmes qui leur sont propres et qui les établissent, puisque si nous en exceptons l'état seul qui est véritablement parfait, on peut dire
qu'il n'y en a pas un qu'on ne puisse détruire en lui donnant trop de ces choses, ou en ne lui en donnant pas assez. Par exemple, si nous
donnons à l'état populaire plus ou moins de liberté qu'il ne faut, aussitôt, il s'affaiblit et dégénère en oligarchie. Car il en est de même que
des nez que nous appelons camus et aquilins: non seulement ajoutant aux uns et ôtant aux autres, on les ramène à la médiocrité, mais encore si
l'on s'efforce de les rendre toujours plus camus ou plus aquilins, on les met en tel état qu'à la fin, il ne leur reste pas même la moindre
apparence de nez. Or, pour ce qui est de l'établissement des lois, ce ne sera pas assez à l'orateur de connaître par ce qui s'est passé dans l'état
où il parle, quelle façon de gouverner est la meilleure, mais encore il faudra qu'il sache par une exacte lecture de tout ce qui s'est fait chez
les autres peuples, quelles sortes de lois sont plus propres a telles ou à telles sortes de personnes.

D'où il s'ensuit évidemment deux choses. La première: que pour être capable d'établir des lois, les voyages sont utiles,
puisque c'est principalement dans les voyages et en pratiquant plusieurs nations qu'on fait expérience des lois. La seconde:
que pour être en état de persuader dans les assemblées publiques, il faut être versé dans l'histoire. Or, tout cela est
l'ouvrage de la politique seulement, et n'appartient en aucune façon à la rhétorique.

Nous avons donc remarqué quels sont les points principaux que doit connaître l'orateur qui a à délibérer. Parlons, désormais
de ce qu'il doit employer, non seulement afin de pouvoir persuader sur ces mêmes points, ou dissuader, mais encore afin de
le pouvoir faire sur quelque matière qui se présente.

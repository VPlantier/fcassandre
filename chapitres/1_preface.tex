
\tableofcontents

\newpage

\section*{Commentaire d'Aristote sur cette édition}
\subsection*{Rédigée lors d'une séance de spiritisme au ouija en présence d'un notaire anonyme}

Pas mal\dots{}

\newpage

\section*{Avant-propos hors de propos}

Si on voulait écrire un traité de rhétorique au XXIème siècle, l'idée centrale et essentielle serait
probablement de maintenir l'attention. Chaque orateur est une distraction en concurrence avec mille
autres, et dix secondes de creux suffisent à perdre l'attention de l'auditeur, qui se tournera aussitôt
vers une autre appli. 

Alors, du coup, ben\dots{} je me suis dit\dots{}

Si je veux vendre Aristote, il faut que je le présente comme un médicament. 

Considérez la lenteur de ses raisonnements et son attention à ce que chaque point soit une suite logique
de ce qui précède. On sent que ce monsieur vient d'ailleurs. Imaginez un monde où les notifications sont
désactivés pendant plusieurs mois, des conversations de plusieurs heures sans jamais passer du coq à l’âne,
un monde où l'attention reste posée en permanence sur la pertinence et la cohérence logique de celui qui
parle, et où l'orateur est plus attentif à la continuité de ses propres preuves qu'à ses techniques de
\emph{mind-hacking}. C'est le monde\footnote{Nous voulons bien consentir à en supposer le chiantisme, mais
pour notre défense, on s'en sert modérément et dans les limites de la finalité de déniaiser l'Orgon qui
sommeille en chacun de nous\dots{}} du Lycée (notez la majuscule: on s'en tanne le coquillard de ta ruine
de lycée Jean Monnet à Vaux-en-Sauce).

De quoi est-ce que je parlais? \emph{Attentif à\dots} Ah oui.

Je ne prétends pas y être pour quelque chose, mais j'ai entendu une rumeur\dots{} comme quoi Aristote avait,
lui aussi, ses faiblesses.

Je ne dis pas non plus qu'il y a vingt-trois siècles, son succès en pâtissait. Non. Je parle d'un défaut gênant
pour les futures générations de ses auditeurs.

Je me permets une dernière remarque, pour maintenir le suspense une seconde de plus: on peut se poser la
question pourquoi et comment il pût conserver un succès et une influence aussi solide sur un si grand nombre
de générations malgré ce défaut. Il faut croire qu'il se trouve quelque chose d'attirant dans ses écrits
puisque en dépit de tout, ils se transmettent.

Encore une fois, je ne fais que relayer la rumeur, puisque je ne lis pas le grec. Qu'il me soit pardonné si je
diffame mais, dans le doute, je signerai sous un faux nom. Bref. J'ose tout, je le dis quand-même. Quoique son
argumentation fût rigoureuse et remarquable par son suivi, son expression était confuse et sa présentation
brouillonne. 

D'où le choix de cette traduction.

\newcommand\rocquentin{\addfontfeatures{Ligatures = {Common, Rare}, Style = Historic}}

Parce que là aussi, ça peut interroger: choisir une traduction surannée, \emph{{\rocquentin reviser successi}vement
{\rocquentin \& s}ans {\rocquentin foiblir, \& Orthografe, \& Maniere d'escrire, \& Ponctuation, m'emploiant à
reformer}, Les Traits devolues à un Tems, {\rocquentin Où Jean Racine estoit encore jeune Estudiant}},
pour, en fin de compte, se retrouver avec un texte charmant mais scolaire par son style avec ses <<~ainsi~>>,
<<~à la vérité~>>, <<~d'une part~>> et <<~d'autre part~>>, formules que je me suis permis d'amputer arbitrairement
en de rares endroits, quand j'ai estimé qu'ils polluaient la compréhension (uniquement ces formules: le texte est
complet).

Je ne sais pas si cette traduction est fidèle, d'ailleurs, mais elle l'est suffisamment pour une initiation. Son
grand mérite, c'est sa clarté, donc si on juge par ce critère, j'ai choisi la meilleure traduction du domaine public,
voire la meilleure tout court (si on juge par la fidélité, au contraire, la <<~rumeur~>> veut qu'on s'interdise la
clarté).

Mais j'en ai dit assez sur le choix de cette vieille traduction. Il me reste à me justifier sur mes interventions.
Pourquoi agrémenter une œuvre si sérieuse avec des notes de bas de pages aussi bourrées de conneries?

Parce que c'est un jeu.

Je me suis dit que, quoi qu'Aristote ne soit pas joueur, il ne m'en voudra pas si je joue avec ses abstractions de la
manière la plus rigoureuse qu'il soit possible à un esprit dispersé du XXIème siècle.

Prenons quelques secondes pour comprendre les règles du jeu que je me suis imposé.

Nous savons tous ce qu'est un syllogisme. Comment, de deux prémisses, on en arrive à une conclusion nécessaire.
La base des raisonnements mathématiques. Aristote est celui qui étendit le concept à tous les discours avec sa
notion d'enthymème: à partir d'une seule prémisse, on trouve une conclusion nécessaire. Pourquoi? Parce qu'il
se trouve une deuxième prémisse, qui est implicite, et qu'Aristote appelle <<~lieu~>>. Ainsi, si je dis qu'il faut
absolument aller à la plage, parce qu'il fait beau, j'utilise le \emph{lieu} qui prétend que quand il fait beau,
il faut aller à la plage\footnote{Avec en plus un deuxième degré de l'implicite, puisque je ne précise pas que c'est
dimanche, que t'es tout pâle et qu'on s'emmerde.}. Par ce biais là, on est certain que:

\begin{emphpar}
	N'importe quel syllogisme ayant une seule prémisse est \emph{sequitur}, ce qui signifie qu'il suit les
	prémisses.
\end{emphpar}

Évidemment, puisqu'il n'y en a qu'une! Il se trouve toujours une prémisse qui rend ce raisonnement \emph{sequitur}.
Il suffit de la considérer comme implicite et le tour est joué\dots{}

Ajoutons que:

\begin{emphpar}
	Si la prémisse explicite est admise, la conclusion convaincra en proportion de l'efficacité du lieu implicite utilisé,
	laquelle dépendra du contexte du discours.
\end{emphpar}

Or, si on voulait résumer en trois mots un élément central de l’œuvre d'Aristote, c'est \emph{un inventaire de lieux
efficaces, classés par disciplines}.

Pour chaque domaine de la connaissance, Aristote listait les lieux implicites qui sont à la base des raisonnements
qu'on y trouve. Œuvre monumentale et travail de titan, il affûta son regard à l'abstrait. Composante essentiel du
langage, l'abstrait est souvent trompeur et, à ce titre, on a raison de s'y intéresser. Les sophismes sont rarement
malveillants: ils sont la conséquence d'un manque de rigueur.

Le seul tort d'Aristote, c'est que parfois, son intelligence l'abusait. Il avait une tendance à croire en ces lieux
abstraits qu'il découvrait, à se laisser tenter par l'hypothèse de l'universalité absolue de ses abstractions.
Le premier réflexe de ce digne héritier de Platon, c'était d'admettre que la beauté spirituelle de l'abstrait suffisait
à démontrer un fait universel. Ce n'est que depuis peu qu'on reconnaît le caractère animal de l'esprit de finesse, sa
subordination aux hasards des associations bayésiennes, la faillibilité de l'esprit humain. Il n'y a rien de plus
difficile à une âme pétrie de culture classique que de s'apercevoir que les esprits les plus polis et cultivés sont,
comme vous et moi, des cons. Quant à ces lieux d'Aristote, il faut croire que, s'ils sont convaincants, c'est parce
qu'empiriquement, ils fonctionnent plus souvent qu'ils n'échouent. Ils sont adaptés. La nature sélectionne ces biais.
Dans ce livre même, on trouve d'ailleurs des lieux qui correspondent à des biais connus, comme le biais d'aversion à
la perte, au chapitre VI du premier livre:

\begin{lieu}
	Tout ce qui nous aura fait prendre beaucoup de peine et obligé à une grande dépense est un bien,
\end{lieu}

Lieu qu'on peut utiliser dans des discours pour susciter ce biais, comme Aristote en donne des exemples dans l'Illiade
d'Homère.

Passons\dots{} J'en viens à mon jeu dans un instant. Un peu de patience, je vous prie. L'essentiel à retenir pour
bien comprendre le manuel, c'est que:

\begin{emphpar}
	Aristote avait plus de difficultés à questionner les lieux qu'il a découvert qu'à y croire sans réserve.

	Cette tendance est encore plus frappante chez sa postérité, puisque les aristotéliciens n'ont pas tous le génie
	d'Aristote.

	Ainsi, Aristote a influencé la philosophie autant par ses torts que par ses mérites. D'une part, des esprits
	bornés se sont servi d'Aristote comme caution de leurs préjugés sur la science, sur l'esclavage, sur les femmes
	etc. (tel Sganarelle dans \emph{Le Malade Imaginaire}). Mais d'autre part, d'autres l'ont pris comme caution de
	l'implicite, en usant abusivement des lieux dont il fit l'inventaire, et je préfère me focaliser sur ce deuxième
	abus, plus subtil.
\end{emphpar}

Du coup --- parce que, quand on est jeune, on peut dire du coup --- pour en revenir à mes (excellentes) notes de bas
de pages, ce que j'ai fait à travers elles, c'est que je me suis amusé à réagir à quelques lieux d'Aristote, glanés
ici et là, en suivant scrupuleusement la règle qui suit.

\bigbreak

Le jeu consiste à attribuer une note de bas de page à un maximum de lieux définis par Aristote. Cette note est
sarcastique et vise à diminuer la crédibilité du lieu. Elle doit impérativement embrasser une de ces deux
structures:

La première structure est ironique. Elle consiste à accepter sans réserve le lieu et à en donner un exemple d'assertion
qui ne convaincra personne, quoi que sa prémisse explicite soit valable. Exemple:

\begin{lieu}
	Toutes les vertus seront des biens\footnote{Mieux vaut périr tué par un parisien fou que de renier son engagement
	en faveur de l'OM.}.
\end{lieu}

La deuxième structure est contradictoire. Elle consiste à proposer une assertion susceptible de convaincre, mais dont le
lieu, par sa modernité ou par autre chose, est en contradiction avec le lieu d'Aristote. Exemple:

\begin{lieu}
	Plus de choses qu'une seule prise à part, ou qu'un petit nombre, et cela comparé de sorte l'un avec l'autre
	que dans ce plus grand nombre se trouve aussi compris ce même petit nombre ou cette seule chose, sans doute le plus
	grand nombre en cet état l'emportera et sera à préférer\footnote{Sacha \textsc{Guitry}, \emph{Faisons un rêve}:
	\begin{emphpar}\normalfont
	-- Ça y est ? On a toute la vie ?

	-- Ah! mieux que toute la vie.

	-- Mieux que toute la vie?

	-- Oui, nous avons deux jours!
	\end{emphpar}}.
\end{lieu}

\newfontfamily\DejaSans{DejaVu Sans}

En vérité, je me suis autorisé à ajouter des notes de bas de pages un peu moins baroques et qui ne répondent pas à
la règle --- je fais ce que je veux~{\DejaSans ☹}! --- mais ces notes seront aisées à reconnaître. Précisons que j'ai
numéroté les lieux en marge, afin de les distinguer des prémisses posées par Aristote pour les trouver, mais sans
m'interdire de jouer aussi avec ces prémisses en les prenant pour des lieux (ce que j'ai fait au commencement du
chapitre VI).

\bigbreak

Une petite mise au point pour la route: avec ce <<~jeu~>>, il ne s'agit pas de contredire Aristote, puisqu'il écrit au
chapitre VI: <<~Ce sont donc là les lieux qui doivent fournir des propositions quand on aura à montrer, etc.~>>. Il
ne dit pas qu'il faut y croire, mais qu'on peut s'en servir (même s'il croit à quelques uns à cause de son platonisme).
Foutez-vous de la gueule du vieux philosophe millénaire, avec mes notes de bas de pages, mais commençons par le défendre
un peu. Commençons par observer que, si ses lieux, parfois, nous paraissent absurdes, que dirions-nous de quelques lieux
plus contemporains qui ne lui seraient jamais venu à l'esprit d'utiliser s'il avait eu un compte twitter?

D'une part:

\begin{lieu}
	Les gens de droites sont des scélérats.
\end{lieu}

D'autre part:

\begin{lieu}
	Les gens de gauche sont des Tartuffe.
\end{lieu}

Vu à la télé:

\begin{lieu}
	Les personnes vertueuses sont celles qui, quand elles immigrent, abandonnent leurs racines en \emph{ex-voto}.
\end{lieu}

Bonne année 2020:

\begin{lieu}
	Toute information qu'un homme de pouvoir nous communique, si elle l'arrange, est un mensonge\footnote{Surpuissant ce
	lieu! Mais pas très rigoureux\dots}.
\end{lieu}

\bigbreak

Toutes mes excuses. Je m'éparpille encore et je n'ai toujours rien dit de la rhétorique, qui est le sujet de ce livre.
L’utilité de ce livre est de servir à l'\emph{inventio}: amasser des idées et des arguments. Rien de mieux qu'une accumulation
de lieux pour servir ce dessein. Supposons que, spontanément, vous avez envie d'écrire un panégyrique de Patrick Balkany.
Vous êtes disposé à écrire cet éloge; le désir est là mais l'inspiration ne vient pas. Par quoi commencer? Pardi: par
l'\emph{inventio}\dots{} Rendez vous au chapitre IX (qui traite de l'éloge) et il ne vous reste plus qu'à chercher la
grandeur et le mérite de Patrick Balkany dans les lieux que vous y lisez. Le plus dur est fait.

Cet exercice d'ironie est on ne peut plus classique. Lucien de \textsc{Samosate} s'y amusait souvent: il a fait l'éloge d'un
moustique. Pour en revenir à mon jeu, il est beaucoup plus facile, puisque je ne choisis pas mon sujet; je fais des phrases
éparpillées et absurdes, parlant de n'importe quoi. Je singe twitter.

Pourtant, parfois, j'ai des moments de grâce. Vous ne les verrez pas souvent, ces moments de grâce, mais, parfois, mes notes
de bas de pages sont des réponses aux préjugés d'Aristote, préjugés géométrique pour la plupart:

\begin{emphpar}
	Tout contraire s'étudie de manière symétrique à la chose dont il est le contraire.

	L'asymétrie n'existe pas.

	Les catégories sont des vérités qui s'étudient de manière linéaires.

	La pensée abstraite est un monde parfaitement ordonnée.
\end{emphpar}

Le culte de la symétrie se voit dans le chapitre IX qui ne propose que des lieux pour la louange, en précisant que pour le
blâme, il n'y aura qu'à prendre les lieux contraires, sans plus de précision. Un exemple de linéarité absurde:

\begin{lieu}
	Si une chose, qui est la plus excellente dans son genre, l'emporte sur une autre qui soit aussi la plus excellente
	dans le sien, sans difficulté, le genre de la plus excellente l'emportera sur le genre de l'autre.
\end{lieu}

Beaucoup d'erreurs philosophiques viennent du langage et de sa grammaire, qui sont des systèmes, et qui nous soufflent
d'inventer de nouveaux systèmes: les écoles et les sectes. Pourtant, le langage et sa grammaire ne sont que des outils
pour s'exprimer, ces outils datent de la préhistoire et ceux qui les analysent pour y trouver des vérités profondes
(d'Aristote à Lacan) n'y trouvent que la confirmation de leurs préjugés.

\bigbreak

Si vous avez des questions, relisez cette préface cinq fois. Si vous avez encore plus de questions, lisez l’œuvre
complète de Schopenhauer au moins trois fois avant de contacter mon secrétariat (Odéon 8400).


\begin{comment}

En terme d'éthique, les végans sont aux français moyens ce que que les français moyens sont aux cannibales par loisirs.

\end{comment}


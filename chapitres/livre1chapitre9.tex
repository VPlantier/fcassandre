
\chapter{Le genre démonstratif}
\section{De la vertu en général et en particulier}
\subsection{Avec les lieux et les adresses qui regardent la louange et le blâme}

Parlons maintenant du vice et de la, vertu, de ce qui est honnête et déshonnête, puisque c'est le but que se
proposent ceux qui ont à louer ou à blâmer quelqu'un. Au reste, en traitant ces matières, il se rencontrera
qu'en même temps, nous ferons connaître les choses dont il se faut servir pour se mettre bien dans l'esprit
de l'auditeur et lui faire avoir bonne opinion de nos mœurs, qui est la seconde sorte de preuve artificielle
que nous avons remarquée. Car enfin, les mêmes moyens que doit employer l'orateur pour faire croire honnête
homme et vertueux celui qu'il a dessein de louer, ces mêmes moyens là lui serviront encore pour faire croire
qu'il est honnête homme lui-même.

Et parce qu'assez souvent il nous arrive de louer aussi bien par plaisir que sérieusement, non seulement un
homme, ou une divinité, mais même les choses qui n'ont point de vie, ou des animaux le premier venu, Il est
nécessaire encore que nous fassions ici comme nous avons déjà fait dans le genre délibératif, c'est à dire:
établir des propositions sur toutes les matières de la louange ou du blâme, afin qu'elles nous servent. Mais
auparavant, donnons quelque idée de ce que nous appelons honnête et de ce que nous appelons vertu.

\bigbreak

Par le mot d'honnête, on entend:

\begin{emphpar}
	Une chose qui, étant souhaitable à cause d'elle-même, mérite qu'on la loue;
\end{emphpar}

Ou si l'on veut encore:

\begin{emphpar}
	Une chose qui étant un bien en soi, outre cela, est agréable à cause que c'est un bien.
\end{emphpar}

Que si cette supposition est vraie, il s'ensuivra:

\begin{lieu}
	Que la vertu est une chose honnête;
\end{lieu}

Puisque étant un bien, elle mérite encore qu'on la loue. Au reste, à juger de la vertu seulement par ce qu'elle
nous parait, elle peut être définie:

\begin{emphpar}
	Une puissance capable de nous faire acquérir de très grands avantages et de nous les conserver;
\end{emphpar}

Ou encore:

\begin{emphpar}
	Une puissance capable d'obliger beaucoup et en des occasions importantes, et même à qui rien n'est difficile
	dans ce qu'elle entreprend, et qui peut tout en toutes choses.
\end{emphpar}

\bigbreak

Les parties de la vertu, ou les vertus en particulier, sont la justice, la valeur, la tempérance, la magnificence,
la magnanimité, la libéralité, la mansuétude ou la clémence, la prudence et la sagesse.

\bigbreak

Or, supposé que la vertu soit telle que nous venons de dire, il faudra mettre au nombre des vertus les plus hautes:

\begin{lieu}
	Celles qui sont très utiles aux autres;
\end{lieu}

Puisque le propre de la vertu, c'est d'obliger. Aussi est-ce pour cette raison que les peuples honorent
principalement les hommes justes et les vaillants, à cause que la justice leur est utile en temps de paix
et la valeur durant la guerre. Après ceux-ci, le libéral est la personne qu'on honore le plus puisque, loin
de quereller pour l'argent que tous les autres recherchent avec tant d'avidité, jamais au contraire il n'est
plus joyeux que lorsqu'il le donne et qu'il en fait des largesses.

\bigbreak

Quant aux définitions de chaque vertu en particulier, premièrement, la justice est définie une vertu qui
conserve à chacun ce qui lui appartient conformément aux lois et aux ordres établis dans chaque état.

L'injustice au contraire est un vice qui nous fait usurper et retenir le bien d'autrui contre l'ordonnance et
l'intention de ces mêmes lois.

La valeur, ou le courage, est une vertu qui, au milieu des plus grands périls, fait entreprendre de belles
actions, ce qui se doit entendre lorsque toutes les circonstances que les lois prescrivent sont exactement
observées. Le courage aussi parait à faire valoir les lois et à les maintenir dans leur vigueur.

La lâcheté est le vice contraire.

La tempérance est une vertu qui fait que nous nous réglons sur la loi touchant les plaisirs sensuels. 

L'intempérance, ou la débauche, est le vice opposé.

La libéralité est une vertu qui ne regarde ses richesses que pour en faire du bien et pour obliger. 

L'avarice est le contraire. 

La magnanimité est une vertu qui se plaît à obliger dans les grandes choses et aux occasions importantes.

La pusillanimité, ou bassesse d'âme, est un vice qui lui est opposé.

La magnificence est une vertu qui aime l'éclat et à faire de grandes dépenses.

La mesquinerie est tout le contraire.

La prudence, enfin, est une vertu de l'esprit qui, touchant les biens et les maux que nous avons dit
contribuer à nous rendre heureux ou malheureux, nous les fait distinguer afin de ne pas prendre l'un pour l'autre.

\subsection{Lieux communs pour la louange}

Après avoir considéré le vice et la vertu en général et en particulier autant qu'il était à propos de le faire pour
notre dessein, il ne sera pas difficile de passer au reste et de tirer des conséquences. Premièrement donc il sera
nécessaire de conclure:

\begin{lieu}
	Que tout ce qui contribue à nous rendre vertueux est honnête;
\end{lieu}

Puisqu'il se rapporte à la vertu.

\bigbreak

Secondement:

\begin{lieu}
	Tout ce qui vient de la vertu et qui en est une suite;
\end{lieu}

Comme sont tous les signes et les marques qu'on aide la vertu, et tout ce qu'elle produit.

\bigbreak

Que si tout ce qui sert de signe pour faire connaître la vertu en général, même si tous les effets qu'elle produit,
et tout ce qu'on peut souffrir à son occasion, est honnête, il sera vrai d'assurer encore de chaque vertu en particulier,
par exemple de la valeur:

\begin{lieu}
	Que tout ce qui sera un effet ou une marque de valeur et tout ce qui aura été souffert en se portant vaillamment
	fera honneur.
\end{lieu}

On en dira autant de la justice et de tous ses effets, à l'exception néanmoins d'une partie des choses qu'elle fait souffrir,
car c'est de la justice seulement qu'il n'est pas toujours vrai de dire que ce qu'elle fait souffrir soit honnête, attendu
qu'il est beaucoup plus honteux d'être justement puni, que de l'être injustement. Au reste, ce que nous remarquons ici de ces
deux vertus se doit entendre également de toutes les autres.

\bigbreak

Il faudra mettre encore au rang des choses honnêtes:

\begin{lieu}
	Toutes les actions à qui l'honneur sera proposée pour récompense;
\end{lieu}

Comme aussi:

\begin{lieu}
	Celles qui nous apporteront beaucoup plus d'honneur que de profit.
\end{lieu}

On en doit dire autant:

\begin{lieu}
	De toutes les choses qui sont à rechercher pour elles-mêmes, si celui qui les fait ne le fait point pour
	lui, mais pour d'autres.
\end{lieu}

Et ainsi en est-il:

\begin{lieu}
	De celles qui seront bonnes simplement en général.
\end{lieu}

Par exemple, tout ce qu'une personne entreprendra pour le salut ou la gloire de sa patrie, au préjudice de
son intérêt.

On assurera la même chose:

\begin{lieu}
	De tout ce qui est bon naturellement;
\end{lieu}

Et encore:

\begin{lieu}
	De tout ce qui ne sera pas bon pour soi et dont on ne tirera aucun usage;
\end{lieu}

Puisqu'on ne le possédera point par intérêt ni en sa propre considération.

\bigbreak

Il en sera de même:

\begin{lieu}
	De tout ce qui arrivera à une personne plutôt après sa mort que durant sa vie;
\end{lieu}

Attendu que tout ce qu'on fait pour un homme durant sa vie, et toutes les déférences qu'on lui rend, viennent
ordinairement de gens intéressés qui regardent moins le mérite et la vertu de celui qu'ils honorent que son crédit
et le pouvoir qu'il a de les obliger.

\bigbreak

Il faudra encore tenir pour honorable:

\begin{lieu}
	Tous les ouvrages publics et ce qui se fait pour les autres;
\end{lieu}

Puisque l'auteur de telles choses y aura le moins de part.

\bigbreak

Et pareillement:

\begin{lieu}
	Toutes les entreprises que nous aurons achevées heureusement, et toutes les affaires que nous aurons conduites
	avec succès, où il n'y allait nullement de notre intérêt, mais seulement de l’intérêt d'autrui.
\end{lieu}

\bigbreak

Comme encore:

\begin{lieu}
	Tout ce que nous aurons fait à l'avantage des personnes à qui nous serons obligés;
\end{lieu}

À cause que c'est une chose juste d'être reconnaissant.

\begin{lieu}
	Les bienfait encore seront de cette nature;
\end{lieu}

Puisque tout bienfait regarde autrui et qu'il n'en revient rien à celui qui le fait.

\bigbreak

On sera passer encore pour honnête:

\begin{lieu}
	Tout ce qui est contraire aux choses qui font rougir et donnent de la honte;
\end{lieu}

Car tout homme qui témoigne de la honte et qui rougit, c'est toujours pour des choses sales et déshonnêtes, soit qu'en
effet, il soit trouvé s'en entretenant ou qu'il les fasse, ou seulement qu'il soit prêt de les faire ou de les dire; ce
qu'a fort bien remarqué \textsc{Sapphô} à l'endroit où elle introduit Alcée qui lui parle en ces termes:

\begin{emphpar}
	Je voudrais bien, Sapphô, vous dire quelque chose,
	
	Mais un respect honteux à mon désir s'oppose\dots
\end{emphpar}

Sapphô répond:

\begin{emphpar}
	C'est trop me dire, Alcée: un si honteux respect

	Accuse ton désir et me le rend suspect.

	Si ce désir était un désir légitime,

	Si ta langue trop prompte à se charger d'un crime

	N'avait à mettre au jour un propos vicieux,

	Tu n'abaisserais pas honteusement les yeux,

	Et tu serais hardi dans une cause juste. 
\end{emphpar}

\bigbreak

Il faudra encore tenir pour honnête:

\begin{lieu}
	Tout ce qui nous donne de l'inquiétude et du soin, sans pourtant qu'en cet état nous nous trouvions saisis
	d'aucune appréhension;
\end{lieu}

Puisque cela ne pourra venir que d'émulation seulement, et de ce que nous nous serons proposé d'acquérir quelqu'un
de ces biens et de ces avantages éclatants qui regardent la réputation et la gloire. 

\bigbreak

On pourra assurer aussi:

\begin{lieu}
	Que les vertus et les œuvres des personnes plus parfaites seront aussi plus remarquables et plus dignes d'honneur;
\end{lieu}

Par exemple celles de l'homme, plus que celles de la femme. 

\bigbreak

Et encore:

\begin{lieu}
	Toutes celles qui feront plus pour la jouissance et le profit des autres que de celui en qui elles se trouveront;
\end{lieu}

D'où vient que la justice particulièrement est en honneur, et tout ce qui est juste.

\bigbreak

Il en sera de même:

\begin{lieu}
	Du dessein qu'on aura de se venger de ses ennemis plutôt que de faire accord avec eux;
\end{lieu}

Car il est juste de rendre la pareille à ses ennemis. Et si la chose est juste, il y a de l'honneur, joint que le
propre d'un homme généreux est de ne point céder à ses ennemis, et de ne souffrir jamais d'en être vaincu.

\bigbreak

Il faudra encore conclure:

\begin{lieu}
	Que la victoire et l'honneur seront fort à estimer;
\end{lieu}

Car puisque ce font des choses à souhaiter quand bien même elles seraient infructueuses et qu'il n'en reviendrait
rien, nous aurons encore cet avantage en les possédant qu'elles feront paraître en nous un mérite extraordinaire
et un excès de vertu. 

\bigbreak

On fera encore passer pour honnête:

\begin{lieu}
	Tout se qui peut entretenir la mémoire d'un homme et faire parler de lui après sa mort.
\end{lieu}

D'où non seulement il s'ensuivra:

\begin{lieu}
	Que plus une chose sera capable de produire un tel effet, plus elle sera honorable;
\end{lieu}

Mais encore:

\begin{lieu}
	Tout ce qui ne pourra arriver à une personne qu'après sa mort.
\end{lieu}

Comme aussi:

\begin{lieu}
	Tout ce qui sera suivi d'honneur, de gloire et de réputation.
\end{lieu}

Pareillement:

\begin{lieu}
	Toutes les choses extraordinaires et qui excellent.
\end{lieu}

Enfin:

\begin{lieu}
	Tout ce qui ne sera possédé d'aucun autre que de nom;
\end{lieu}

Car comme ces choses-là seront plus remarquables d'elles-mêmes, aussi seront-elles plus propres à faire
passer de nous et à nous mettre en estime.

\bigbreak

On pourra encore proposer comme honnête:

\begin{lieu}
	Toutes les acquisitions qui ne seront d'aucun rapport;
\end{lieu}

Puisqu'elles feront éclater davantage la libéralité de celui qui les aura en sa possession.

\bigbreak

Il en sera de même:

\begin{lieu}
	De tout ce qui sera particulier à chaque peuple et à chaque nation;
\end{lieu}

Comme aussi:

\begin{lieu}
	De tout ce qui servira de marque à chaque peuple des choses qui sont particulièrement en estime chez lui.
\end{lieu}

Par exemple c'est un honneur chez les Lacédémoniens de porter de grands cheveux, à cause qu'ils les prennent
pour une marque de liberté et d'indépendance. Et sans doute, il y a quelque raison à cela, puisque enfin, il
n'est pas aisé à un homme qui a les cheveux grands de faire rien de servile. Ainsi encore en est-il chez eux
de n'exercer aucun art mécanique, comme étant encore de la liberté de ne dépendre point d'autrui et de n'être
assujetti à personne.

\subsection{Adresses pour louer ce qui ne sera pas louable}

Outre les propositions et les conséquences que nous venons d'alléguer, on se pourra encore servir d'adresse.
Premièrement, donc, au vu des qualités véritables qui se trouveront en la personne qu'on voudra louer ou blâmer,
on se servira de celles qui leur ressemblent ou en approchent. Par exemple, si nous avons à parler contre un homme
vaillant en effet, mais qui, à la guerre, emploie plus ordinairement la ruse que la force, nous dirons que c'est
un poltron qui n'a du courage que quand il faut prendre en trahison et dresser des embuscades. Au contraire, si
nous avons à louer un sot et un niais, nous ferons passer sa niaiserie pour une bonté. Et encore, nous appellerons
doux et pacifique un homme insensible à toutes sortes d'injures. En un mot, nous tâcherons de faire prendre en bonne
part chaque défaut et chaque vice, en leur attribuant le nom des choses qui les accompagnent d'ordinaire et qui sont
de leur suite. Aussi, parlant à l'avantage d'un colère et d'un rébarbatif, nous dirons que c'est un homme ouvert et
qui ne peut dissimuler. De même, nous dirons d'un orgueilleux ou d'un arrogant que sa façon d'agir est noble et sent
sa personne de qualité.

\bigbreak

Une seconde adresse dont on se pourra servir, c'est d'attribuer la qualité de vertueux à des personnes qui pêchent
par excès, comme de nommer vaillant un téméraire; ou d'appeler libéral un prodigue. Car outre que bien des gens y
seront trompés, c'est qu'il y aura lieu même d'apporter un faux raisonnement pour le faire croire tout de bon. Et de
fait, on dira:

\begin{emphpar}
	Un homme qui court au danger sans nécessité, que ne fera-t-il point quand l'honneur l'y appellera?
\end{emphpar}

On en dira autant du prodigue en raisonnant de la même sorte:

\begin{emphpar}
	Celui qui donne a tout venant et qui ne peut refuser à personne, est-il croyable qu'il abandonne ses amis au
	besoin et qu'il ne soit avare que pour eux?
\end{emphpar}

Et véritablement il semble que faire ainsi du bien à tout le monde est l'effet d'une vertu extraordinaire, et
d'une bonté qui va jusqu'à l'excès.

\bigbreak

Une autre observation encore à faire pour la louange, c'est de prendre garde à qui sont ceux devant qui on doit parler.
Car ce n'est pas sans raison que Socrate disait qu'il n'était pas difficile de louer les Athéniens en parlant aux Athéniens.
Ainsi donc, selon les personnes devant qui on aura à paraître, il faudra voir quelles choses particulièrement seront chez
elles en estime, et alors en parler comme si véritablement elles étaient à estimer; par exemple, chez les Scythes, au cas
qu'on ait à parler devant eux, ou chez les Lacédémoniens, ou devant des philosophes. En un mot il faudra ramener à l'honneur
et faire passer pour tel, ce qui ne sera que simplement honorable et estimé de quelques personnes. Et de fait, il semble
qu'il n'y ait pas grande différence de l'un à l'autre.

\bigbreak

Outre ceci, quand on aura à louer quelqu'un, il sera bon encore d'examiner s'il a fait en sa vie des actions bienséantes et
qui conviennent à une personne de sa qualité. Par exemple, s'il a fait des choses dignes de sa naissance et de ses ancêtres,
ou si ce qu'il vient de faire répond à ses actions passées et à l'attente qu'on avait de lui, car non seulement il y a du
bonheur à augmenter toujours sa réputation et à entasser honneur sur honneur, mais encore c'est une chose glorieuse.

\bigbreak

Il sera encore bon d'examiner le contraire; car sans doute, c'est une très belle occasion de louer un homme que d'avoir
à montrer qu'il a été vertueux au delà même de ce qu'on en devait attendre et que ce qu'il a fait, il l'a toujours fait
de mieux en mieux; comme de dire qu'au milieu de la prospérité, il ne s'est point oublié et qu'il a été aussi modeste
que devant. Ou au contraire, que dans l'adversité ou le malheur de ses affaires, il s'est toujours soutenu et n'a pas moins
paru généreux. Ou enfin, qu'étant sorti de bas lieu, à mesure qu'il est monté aux charges et aux honneurs il n'en est devenu
que plus honnête homme et plus facile à aborder. Et de vrai, c'est là-dessus qu'est fondée la louange qu'Iphicrate se donnait
à lui-même, comme nous avons déjà remarqué, lorsqu'il dit: <<~Qui étais-je autrefois pour être maintenant ce que je suis?~>>
Et encore celle qu'on lit dans l'épigramme du Poissonnier d'Argos qui remporta le prix aux Jeux Olympiques:

\begin{emphpar}
	Aurait-on jamais cru qu'un jour j'eusse la gloire,

	Moi qu'on vit mille fois un panier sur le dos\dots
\end{emphpar}

Telle est encore la louange que \textsc{Simonide} donne à Archedicé qui se montra si bonne et si obligeante à tout le monde,
quoi qu'elle fut d'une très haute naissance, et comme il l'assure lui-même, 

\begin{emphpar}
	Et fille et femme et sœur de monarques puissants.
\end{emphpar}

Après tout, parce que la louange regarde principalement les actions de la vie, et que le propre d'un homme vertueux c'est
d'agir toujours de dessein, il faudra tâcher en louant une personne de montrer que toutes les actions qu'elle a faites,
elle ne les a point faites par hasard mais de dessein, de propos délibéré. Pour cela, donc, il sera nécessaire de faire voir
que souvent, elle a fait de même, et alors on ramassera tout ce qui lui sera arrivé en sa vie, ou fortuitement, ou par bonheur,
le faisant valoir comme des choses qu'elle avait résolues de faire et auxquelles elle s'était étudiée particulièrement, car
quand on peut alléguer d'une même personne plusieurs actions toutes semblables, c'est en quelque façon un préjugé et une preuve
certaine que cette personne est vertueuse effectivement, et qu'elle n'a rien fait que de dessein et après s'être proposé de le 
faire\footnote{Vous aimez ça, vous, les biopics? Parce que c'est exactement ça\dots}.

\subsection{Espèces différentes de louange}

Au reste, il y a plusieurs sortes de louanges. La première espèce regarde les vertus héroïques et confirmées par de
longues habitudes. Elle est définie \emph{un discours qui donne à connaître une très haute vertu}. Or, pour faire qu'on
puisse ajouter foi à une louange de cette qualité, il faudra montrer que toutes les actions de la personne qu'on loue
viennent d'habitude et sont des effets d'une vertu éminente.

\bigbreak

La seconde espèce de louange regarde les œuvres et chaque action louable en particulier. Pour tout le reste qui a
accoutume d'entrer dans la louange, comme sont les circonstances, cela sert seulement à rendre une chose croyable
et à la persuader plus aisément. Telles sont la naissance et l'éducation, pour ce qu'il est vraisemblable qu'un
homme qui est sorti d'honnêtes gens est honnête homme; et encore que celui qui a eu une telle éducation est tel
qu'on l'a élevé. Aussi, pour cela, toujours louons-nous bien d'avantage ces personnes quand elles font des actions
qui répondent à leur éducation ou à leur naissance, puisque alors il y a lieu de faire voir que de semblables
actions viennent d'une nature confirmée au bien, et procèdent d'habitude. Ce fondement est si véritable que, même,
nous ne laisserions pas de louer un homme quoi qu'il n'ait rien fait de remarquable en sa vie, si nous étions
assurés qu'il fût tel que nous venons de dire.

\bigbreak

La troisième espèce de louange à qui les Grecs donnent deux noms, quoi qu'en effet, ces deux noms n'aient que la
même signification, consiste à féliciter une personne et à la louer comme souverainement heureuse. Cette louange
est différente des deux autres en ce qu'elle se propose un sujet plus vaste et plus étendu. Car tout aussi que le
souverain bonheur et la félicité comprennent en soi la possession de toutes sortes de vertus, de même cette espèce
de louange renferme les deux autres, puisqu'elle n'a pas simplement pour objet une habitude vertueuse comme la
première, ou quelqu'un de louable en particulier comme la seconde, mais toutes les vertus généralement et les
riches qualités de l'âme.

\subsection{Ressemblance du genre démonstratif avec le délibératif}

Une observation à faire touchant la louange et le conseil, c'est que tous deux ont beaucoup de conformité. Car
enfin, ce qu'on propose en conseillant quelqu'un et tout ce que l'orateur alors met en avant comme des avis à
suivre, cela même peut servir de louange en changeant simplement la façon de parler. D'où il s'ensuit qu'ayant
la connaissance, comme nous avons, de tout ce qu'il faut qu'un homme fasse pour être loué et des qualités qu'il
doit posséder, il nous sera très facile de former des préceptes de toutes ces matières de louange, puisqu'il n'y
aura qu'à changer un peu la phrase. Donnons quelque exemple. Si, donc, on disait ainsi:

\begin{emphpar}
	Jamais il ne se faut prévaloir, ni tirer avantage, des biens que la fortune nous donne, mais seulement de
	notre vertu et des biens qui nous appartiennent en propre.
\end{emphpar}

Cela, sans doute, exprimé de la sorte, est un précepte et un conseil tout pur. Cependant, qu'on change un peu
la façon de parler, ce sera une louange, car il n'y aura qu'à dire:

\begin{emphpar}
	Jamais cet homme n'a tiré avantage des faveurs qu'il a reçues de la fortune, et quand il s'est voulu faire
	valoir, il ne s'est servi que de son mérite et de sa propre vertu.
\end{emphpar}

Toutes les fois, donc, que vous aurez à louer quelqu'un, prenez garde au conseil que vous lui donneriez si vous
aviez à lui faire entreprendre quelque belle action, et au contraire, quand vous aurez à donner un conseil, ou
quelque avis, jugez en vous même et examinez quelle action mériterait en effet d'être louée. À la vérité, l'expression
sera différente et doit être opposée nécessairement, car pour le conseil, il faut qu'elle soit prohibitive, et
pour la louange, il ne le faut pas.

\subsection{De l'amplification}

Ce ne sera pas encore une petite adresse, quand on voudra louer quelqu'un, d'user d'amplification et de se servir
des circonstances qui agrandissent une action et la font paraître plus considérable; comme de dire qu'il a été le
seul ou le premier qui ait osé faire une telle entreprise, ou de montrer qu'il l'a exécutée avec fort peu de monde,
ou qu'il n'y en a point qui s'y soit plus signalé que lui; car ces circonstances sont glorieuses à remarquer, et
méritent une louange particulière.

\bigbreak

Le temps encore et les occasions peuvent beaucoup faire valoir une action, parce qu'alors elle paraîtra extraordinaire
et sera regardée comme une chose qui a passé l'attente de tout le monde et l'espérance qu'on en avait conçue.

\bigbreak

On pourra aussi agrandir la louange d'un homme en faisant voir qu'il a souvent réussi dans les mêmes entreprises, car
outre que l'action en sera plus considérée et se fera davantage admirer, jamais on ne pourra croire qu'elle ait été
faite par hasard, et on l'attribuera toujours à l'adresse et à la vertu de celui qui y aura réussi.

\bigbreak

Il sera encore avantageux de remarquer si quelqu'une des choses qui sont faites pour donner de l'émulation et pour
porter les hommes aux belles actions ont été inventées et établies pour faire honneur à celui que nous aurons à louer,
ou s'il est le premier à qui on ait donné des éloges en public, comme il est arrivé a Hippoloque, enfin si sa gloire
peut être égalée à celle d'Harmodios et d'Aristogiton, qui furent les premiers à qui les Athéniens dressèrent des
statues dans la place publique. Or, non seulement ceci aura lieu pour embellir une action et la faire davantage valoir,
mais encore pour faire le contraire, pouvant servir également à enlaidir la vie et les actions de ceux que nous
voudrons blâmer. 

\subsection{Adresses pour louer un homme qui n'a rien fait de louable}

Mais s'il arrive que la personne que nous aurons à louer n'ait rien fait qui puisse fournir de matière pour en parler
glorieusement, en ce cas, il faudra avoir recours aux parallèles, et la comparer à d'autres, ce qu'\textsc{Isocrate} a
fait souvent pour n'avoir pas pratiqué le barreau, ni s'être étudié au judiciaire. Il y a ceci à observer touchant ces
comparaisons qu'il faut que les personnes qu'on choisit soient illustres et d'une haute réputation, à cause qu'il n'y
a rien qui agrandisse davantage la louange d'un homme, que de faire voir qu'il a des qualités plus éclatantes, et qu'il
a fait des actions plus vertueuses, que ceux-même qui passent pour être très vertueux.

Or, pour montrer que ce n'est pas sans sujet que l'amplification a lieu particulièrement dans la louange, c'est que la
louange aime l’excès et ne cherche que ce qui est excellent et qui passe l'ordinaire. Or est-il que nous avons déjà
remarqué que tout ce qui est excellent et qui passe à un excès louable est du nombre des choses honnêtes. Pour cela
donc, si celui que nous aurons à louer n'est pas assez considérable de lui-même pour être comparé à des personnes
illustres, il ne faudra pas laisser de le comparer à d'autres; car enfin, de quelque façon qu'on élève un homme au
dessus d'un autre, toujours cette élévation et ce degré d'éminence témoignent qu'il a du mérite.

\subsection{Les choses qui sont particulières à chaque genre}

En un mot, donc, et pour prononcer en général chaque partie de la rhétorique, nous pouvons dire que de tous ses
trois genres, il n'y en a pas un à qui l'amplification soit plus nécessaire et plus propre qu'au genre démonstratif.
La raison est qu'un orateur qui loue, prend toujours pour son sujet des actions véritables et reconnues telles de
tout le monde. De sorte que ce qu'il lui reste à faire, c'est d'embellir ces actions-là et de leur donner de l'éclat.

Pour les exemples, ils s'accommodent mieux avec le genre délibératif, puisque les jugements que nous formons dans
nos entreprises et dans tous nos desseins se fondent sur les conjectures que le passé donne de l'avenir, et sur
le rapport qui se remarque entre ce qui s'est déjà fait et ce qui se peut faire. 

Quant aux enthymèmes, ils sont plus propres au genre judiciaire, car comme, là, il s'agit de fait et de juger du
passé, qui est une chose qu'on ne connaît pas toujours et qui aisément peut être révoquée en doute, pour cela il
est besoin de rendre raison particulièrement pourquoi une chose a été faite, et d'en faire la preuve.

Voilà à peu près ce qui se peut dire sur le sujet du blâme et de la louange, et tout ce que l'orateur se doit
proposer quand il aura à louer ou à blâmer quelqu'un. En un mot, tous les lieux et toutes les adresses qui peuvent
servir à embellir ou à enlaidir quelque action que ce soit, car pour blâmer et parler au désavantage d'une personne,
il ne faut point d'autres préceptes que ceux que nous avons donnés pour louer. Tout contraire ayant cela de propre
de donner la connaissance de son contraire en même temps qu'il le fait connaître.

\bigbreak

Le blâme, donc, aura pour son sujet tout ce qui est contraire et opposé à la matière de la louange.
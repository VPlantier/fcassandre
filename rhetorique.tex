
%\RequirePackage{luatex85}

\documentclass[11pt]{book}

\usepackage[utf8]{inputenc}
\usepackage[T1]{fontenc}

\usepackage{fontspec}

\setmainfont[Ligatures = {Common, Rare}]{Junicode}

\usepackage[a5paper,
			inner=29mm,outer=18mm,top=18mm,bottom=18mm,
			includefoot,
			includehead
			]{geometry}

\pagewidth=148mm
\pageheight=210mm
\marginparwidth=20mm

 \usepackage{comment}

\usepackage{polyglossia}
\setdefaultlanguage{french}

\usepackage[perpage]{footmisc}

\usepackage{import}
\usepackage{microtype}
\usepackage{chngcntr}

\setcounter{tocdepth}{1}
\setcounter{secnumdepth}{1} 

\counterwithout{section}{chapter}

\renewcommand{\thesection}{\Roman{section}}
\renewcommand{\thechapter}{}
\renewcommand{\thepart}{}

\usepackage{fancyhdr}

\pagestyle{fancy}
\fancyhf{}
\fancyfoot[CO, CE]{\thepage}

\usepackage[svgnames]{xcolor}

\usepackage{svg}

\usepackage{titlesec}

\usepackage{scrextend}

\titleformat
{\subsection}
[hang]
{\raggedright}
{}
{0pt}
{\large\itshape}


\titleclass{\section}{straight}
\titleformat
{\section} % command
[display] % shape
{
	\raggedright\bfseries\LARGE} % format
{	\def\svgwidth{8cm}
	\centering\input{lithographie.pdf_tex}\\
%\vspace{1pc}%
\raggedright\MakeUppercase{\chaptertitlename} \thesection} % label
{0pc} % sep
{} % before-code

\titleclass{\chapter}{top}
\titleformat
{\chapter}
[hang]
{\raggedright}
{}
{0pt}
{\Huge\MakeUppercase}
\titlespacing*{\chapter}{0pt}{0pt}{0pt}

\titleclass{\part}{top}
\titleformat
{\part}
[hang]
{\Huge\bfseries\raggedright}
{}
{0pt}
{\Huge\MakeUppercase}
\titlespacing*{\part}{0pt}{0pt}{0pt}

\renewcommand{\baselinestretch}{1.1}

\newenvironment{emphpar}
	{ 


	\itshape \leftskip=1cm
	}
	{ 


	\leftskip=0cm
	}

\newcounter{lieu}
\counterwithin{lieu}{section}
\renewcommand{\thelieu}{\Roman{lieu}}

\newenvironment{lieu}
	{ 
		\begin{emphpar} 
		\stepcounter{lieu}
		\hspace{-0.3em}~\marginpar{\ifthispageodd{\raggedright}{\raggedleft}\tiny\thelieu}
	}
	{ 
		%\ifthispageodd{\marginpar{\raggedright\tiny \thelieu}}{\marginpar{\raggedleft\tiny \thelieu}}
		\end{emphpar}
	}

\usepackage{multirow}
\usepackage{caption}

\title{LA RHÉTORIQUE D'ARISTOTE}

\author{François \textsc{Cassandre}}

%\usepackage{graphicx}

\begin{document}

\begin{titlepage}

	\centering
	
	~\\[\baselineskip] 

	\textcolor{Red}{\textbf{\Huge LA RHÉTORIQUE D'ARISTOTE}}\\[1.6\baselineskip] 
	
	{\LARGE Traduite par François \textsc{Cassandre}}\\{\small(16$**$ -- 1695)}\\[2.6\baselineskip]
	
	Revue et accompagnée de propos hasardeux\\[0.6\baselineskip] 
	
	\Large Par Vincent P\textsc{lantier}\\[0.6\baselineskip]
	
	{	\footnotesize \emph{Bullshit vendor} émérite.\\$*~~*$\\$*$\\
	}

	\vfill 
	
	\fbox{βήτα}\\[0.5\baselineskip]
	
	Éditions Béta

\end{titlepage}


\renewcommand{\headrulewidth}{0pt}


\tableofcontents

\newpage

\section*{Commentaire d'Aristote sur cette édition}
\subsection*{Rédigée lors d'une séance de spiritisme au ouija en présence d'un notaire anonyme}

Pas mal\dots{}

\newpage

\section*{Avant-propos hors de propos}

Si on voulait écrire un traité de rhétorique au XXIème siècle, l'idée centrale et essentielle serait
probablement de maintenir l'attention. Chaque orateur est une distraction en concurrence avec mille
autres, et dix secondes de creux suffisent à perdre l'attention de l'auditeur, qui se tournera aussitôt
vers une autre appli. 

Alors, du coup, ben\dots{} je me suis dit\dots{}

Si je veux vendre Aristote, il faut que je le présente comme un médicament. 

Considérez la lenteur de ses raisonnements et son attention à ce que chaque point soit une suite logique
de ce qui précède. On sent que ce monsieur vient d'ailleurs. Imaginez un monde où les notifications sont
désactivés pendant plusieurs mois, des conversations de plusieurs heures sans jamais passer du coq à l’âne,
un monde où l'attention reste posée en permanence sur la pertinence et la cohérence logique de celui qui
parle, et où l'orateur est plus attentif à la continuité de ses propres preuves qu'à ses techniques de
\emph{mind-hacking}. C'est le monde\footnote{Nous voulons bien consentir à en supposer le chiantisme, mais
pour notre défense, on s'en sert modérément et dans les limites de la finalité de déniaiser l'Orgon qui
sommeille en chacun de nous\dots{}} du Lycée (notez la majuscule: on s'en tanne le coquillard de ta ruine
de lycée Jean Monnet à Vaux-en-Sauce).

De quoi est-ce que je parlais? \emph{Attentif à\dots} Ah oui.

Je ne prétends pas y être pour quelque chose, mais j'ai entendu une rumeur\dots{} comme quoi Aristote avait,
lui aussi, ses faiblesses.

Je ne dis pas non plus qu'il y a vingt-trois siècles, son succès en pâtissait. Non. Je parle d'un défaut gênant
pour les futures générations de ses auditeurs.

Je me permets une dernière remarque, pour maintenir le suspense une seconde de plus: on peut se poser la
question pourquoi et comment il pût conserver un succès et une influence aussi solide sur un si grand nombre
de générations malgré ce défaut. Il faut croire qu'il se trouve quelque chose d'attirant dans ses écrits
puisque en dépit de tout, ils se transmettent.

Encore une fois, je ne fais que relayer la rumeur, puisque je ne lis pas le grec. Qu'il me soit pardonné si je
diffame mais, dans le doute, je signerai sous un faux nom. Bref. J'ose tout, je le dis quand-même. Quoique son
argumentation fût rigoureuse et remarquable par son suivi, son expression était confuse et sa présentation
brouillonne. 

D'où le choix de cette traduction.

\newcommand\rocquentin{\addfontfeatures{Ligatures = {Common, Rare}, Style = Historic}}

Parce que là aussi, ça peut interroger: choisir une traduction surannée, \emph{{\rocquentin reviser, successi}vement,
{\rocquentin \& s}ans {\rocquentin foiblir, \& Orthografe \& Maniere d'escrire, tout ainsi que Ponctuation,
m'emploiant à reformer}, les Traits devolues à un Tems, {\rocquentin où Jean Racine estoit encore jeune Estudiant}},
pour, en fin de compte, se retrouver avec un texte charmant, mais scolaire par son style, avec ses <<~ainsi~>>,
<<~à la vérité~>>, <<~d'une part~>> et <<~d'autre part~>>, formules que je me suis permis d'amputer arbitrairement
en de rares endroits, quand j'ai estimé qu'ils polluaient la compréhension (uniquement ces formules: le texte est
complet).

Je ne sais pas si cette traduction est fidèle, d'ailleurs, mais elle l'est suffisamment pour une initiation. Son
grand mérite, c'est sa clarté, donc si on juge par ce critère, j'ai choisi la meilleure traduction du domaine public,
voire la meilleure tout court (si on juge par la fidélité, au contraire, la <<~rumeur~>> veut qu'on s'interdise la
clarté).

Mais j'en ai dit assez sur le choix de cette vieille traduction. Il me reste à me justifier sur mes interventions.
Pourquoi agrémenter une œuvre si sérieuse avec des notes de bas de pages aussi bourrées de conneries?

Parce que c'est un jeu.

Je me suis dit que, quoi qu'Aristote ne soit pas joueur, il ne m'en voudra pas si je joue avec ses abstractions de la
manière la plus rigoureuse qu'il soit possible à un esprit dispersé du XXIème siècle.

Prenons quelques secondes pour comprendre les règles du jeu que je me suis imposé.

Nous savons tous ce qu'est un syllogisme. Comment, de deux prémisses, on en arrive à une conclusion nécessaire.
La base des raisonnements mathématiques. Aristote est celui qui étendit le concept à tous les discours avec sa
notion d'enthymème: à partir d'une seule prémisse, on trouve une conclusion nécessaire. Pourquoi? Parce qu'il
se trouve une deuxième prémisse, qui est implicite, et qu'Aristote appelle <<~lieu~>>. Ainsi, si je dis qu'il faut
absolument aller à la plage, parce qu'il fait beau, j'utilise le \emph{lieu} qui prétend que quand il fait beau,
il faut aller à la plage\footnote{Avec en plus un deuxième degré de l'implicite, puisque je ne précise pas que c'est
dimanche, que t'es tout pâle et qu'on s'emmerde.}. Par ce biais là, on est certain que:

\begin{emphpar}
	N'importe quel syllogisme ayant une seule prémisse est \emph{sequitur}, ce qui signifie qu'il suit les
	prémisses.
\end{emphpar}

Évidemment, puisqu'il n'y en a qu'une! Il se trouve toujours une prémisse qui rend ce raisonnement \emph{sequitur}.
Il suffit de la considérer comme implicite et le tour est joué\dots{}

Ajoutons que:

\begin{emphpar}
	Si la prémisse explicite est admise, la conclusion convaincra en proportion de l'efficacité du lieu implicite utilisé,
	laquelle dépendra du contexte du discours.
\end{emphpar}

Or, si on voulait résumer en trois mots un élément central de l’œuvre d'Aristote, c'est \emph{un inventaire de lieux
efficaces, classés par disciplines}.

Pour chaque domaine de la connaissance, Aristote listait les lieux implicites qui sont à la base des raisonnements
qu'on y trouve. Œuvre monumentale et travail de titan, il affûta son regard à l'abstrait. Composante essentiel du
langage, l'abstrait est souvent trompeur et, à ce titre, on a raison de s'y intéresser. Les sophismes sont rarement
malveillants: ils sont la conséquence d'un manque de rigueur.

Le seul tort d'Aristote, c'est que parfois, son intelligence l'abusait. Il avait une tendance à croire en ces lieux
abstraits qu'il découvrait, à se laisser tenter par l'hypothèse de l'universalité absolue de ses abstractions.
Le premier réflexe de ce digne héritier de Platon, c'était d'admettre que la beauté spirituelle de l'abstrait suffisait
à démontrer un fait universel. Ce n'est que depuis peu qu'on reconnaît le caractère animal de l'esprit de finesse, sa
subordination aux hasards des associations bayésiennes, la faillibilité de l'esprit humain. Il n'y a rien de plus
difficile à une âme pétrie de culture classique que de s'apercevoir que les esprits les plus polis et cultivés sont,
comme vous et moi, des cons. Quant à ces lieux d'Aristote, il faut croire que, s'ils sont convaincants, c'est parce
qu'empiriquement, ils fonctionnent plus souvent qu'ils n'échouent. Ils sont adaptés. La nature sélectionne ces biais.
Dans ce livre même, on trouve d'ailleurs des lieux qui correspondent à des biais connus, comme le biais d'aversion à
la perte, au chapitre VI du premier livre:

\begin{lieu}
	Tout ce qui nous aura fait prendre beaucoup de peine et obligé à une grande dépense est un bien,
\end{lieu}

Lieu qu'on peut utiliser dans des discours pour susciter ce biais, comme Aristote en donne des exemples dans l'Illiade
d'Homère.

Passons\dots{} J'en viens à mon jeu dans un instant. Un peu de patience, je vous prie. L'essentiel à retenir pour
bien comprendre le manuel, c'est que:

\begin{emphpar}
	Aristote avait plus de difficultés à questionner les lieux qu'il a découvert qu'à y croire sans réserve.

	Cette tendance est encore plus frappante chez sa postérité, puisque les aristotéliciens n'ont pas tous le génie
	d'Aristote.

	Ainsi, Aristote a influencé la philosophie autant par ses torts que par ses mérites. D'une part, des esprits
	bornés se sont servi d'Aristote comme caution de leurs préjugés sur la science, sur l'esclavage, sur les femmes
	etc. (tel Sganarelle dans \emph{Le Malade Imaginaire}). Mais d'autre part, d'autres l'ont pris comme caution de
	l'implicite, en usant abusivement des lieux dont il fit l'inventaire, et je préfère me focaliser sur ce deuxième
	abus, plus subtil.
\end{emphpar}

Du coup --- parce que, quand on est jeune, on peut dire du coup --- pour en revenir à mes (excellentes) notes de bas
de pages, ce que j'ai fait à travers elles, c'est que je me suis amusé à réagir à quelques lieux d'Aristote, glanés
ici et là, en suivant scrupuleusement la règle qui suit.

\bigbreak

Le jeu consiste à attribuer une note de bas de page à un maximum de lieux définis par Aristote. Cette note est
sarcastique et vise à diminuer la crédibilité du lieu. Elle doit impérativement embrasser une de ces deux
structures:

La première structure est ironique. Elle consiste à accepter sans réserve le lieu et à en donner un exemple d'assertion
qui ne convaincra personne, quoi que sa prémisse explicite soit valable. Exemple:

\begin{lieu}
	Toutes les vertus seront des biens\footnote{Mieux vaut périr tué par un parisien fou que de renier son engagement
	en faveur de l'OM.}.
\end{lieu}

La deuxième structure est contradictoire. Elle consiste à proposer une assertion susceptible de convaincre, mais dont le
lieu, par sa modernité ou par autre chose, est en contradiction avec le lieu d'Aristote. Exemple:

\begin{lieu}
	Plus de choses qu'une seule prise à part, ou qu'un petit nombre, et cela comparé de sorte l'un avec l'autre
	que dans ce plus grand nombre se trouve aussi compris ce même petit nombre ou cette seule chose, sans doute le plus
	grand nombre en cet état l'emportera et sera à préférer\footnote{Sacha \textsc{Guitry}, \emph{Faisons un rêve}:
	\begin{emphpar}\normalfont
	-- Ça y est ? On a toute la vie ?

	-- Ah! mieux que toute la vie.

	-- Mieux que toute la vie?

	-- Oui, nous avons deux jours!
	\end{emphpar}}.
\end{lieu}

\newfontfamily\DejaSans{DejaVu Sans}

En vérité, je me suis autorisé à ajouter des notes de bas de pages un peu moins baroques et qui ne répondent pas à
la règle --- je fais ce que je veux~{\DejaSans ☹}! --- mais ces notes seront aisées à reconnaître. Précisons que j'ai
numéroté les lieux en marge, afin de les distinguer des prémisses posées par Aristote pour les trouver, mais sans
m'interdire de jouer aussi avec ces prémisses en les prenant pour des lieux (ce que j'ai fait au commencement du
chapitre VI).

\bigbreak

Une petite mise au point pour la route: avec ce <<~jeu~>>, il ne s'agit pas de contredire Aristote, puisqu'il écrit au
chapitre VI: <<~Ce sont donc là les lieux qui doivent fournir des propositions quand on aura à montrer, etc.~>>. Il
ne dit pas qu'il faut y croire, mais qu'on peut s'en servir (même s'il croit à quelques uns à cause de son platonisme).
Foutez-vous de la gueule du vieux philosophe millénaire, avec mes notes de bas de pages, mais commençons par le défendre
un peu. Commençons par observer que, si ses lieux, parfois, nous paraissent absurdes, que dirions-nous de quelques lieux
plus contemporains qui ne lui seraient jamais venu à l'esprit d'utiliser s'il avait eu un compte twitter?

D'une part:

\begin{lieu}
	Les gens de droites sont des scélérats.
\end{lieu}

D'autre part:

\begin{lieu}
	Les gens de gauche sont des Tartuffe.
\end{lieu}

Vu à la télé:

\begin{lieu}
	Les personnes vertueuses sont celles qui, quand elles immigrent, abandonnent leurs racines en \emph{ex-voto}.
\end{lieu}

Bonne année 2020:

\begin{lieu}
	Toute information qu'un homme de pouvoir nous communique, si elle l'arrange, est un mensonge\footnote{Surpuissant ce
	lieu! Mais pas très rigoureux\dots}.
\end{lieu}

\bigbreak

Toutes mes excuses. Je m'éparpille encore et je n'ai toujours rien dit de la rhétorique, qui est le sujet de ce livre.
L’utilité de ce livre est de servir à l'\emph{inventio}: amasser des idées et des arguments. Rien de mieux qu'une accumulation
de lieux pour servir ce dessein. Supposons que, spontanément, vous avez envie d'écrire un panégyrique de Patrick Balkany.
Vous êtes disposé à écrire cet éloge; le désir est là mais l'inspiration ne vient pas. Par quoi commencer? Pardi: par
l'\emph{inventio}\dots{} Rendez vous au chapitre IX (qui traite de l'éloge) et il ne vous reste plus qu'à chercher la
grandeur et le mérite de Patrick Balkany dans les lieux que vous y lisez. Le plus dur est fait.

Cet exercice d'ironie est on ne peut plus classique. Lucien de \textsc{Samosate} s'y amusait souvent: il a fait l'éloge d'un
moustique. Pour en revenir à mon jeu, il est beaucoup plus facile, puisque je ne choisis pas mon sujet; je fais des phrases
éparpillées et absurdes, parlant de n'importe quoi. Je singe twitter.

Pourtant, parfois, j'ai des moments de grâce. Vous ne les verrez pas souvent, ces moments de grâce, mais, parfois, mes notes
de bas de pages sont des réponses aux préjugés d'Aristote, préjugés géométrique pour la plupart:

\begin{emphpar}
	Tout contraire s'étudie de manière symétrique à la chose dont il est le contraire.

	L'asymétrie n'existe pas.

	Les catégories sont des vérités qui s'étudient de manière linéaires.

	La pensée abstraite est un monde parfaitement ordonnée.
\end{emphpar}

Le culte de la symétrie se voit dans le chapitre IX qui ne propose que des lieux pour la louange, en précisant que pour le
blâme, il n'y aura qu'à prendre les lieux contraires, sans plus de précision. Un exemple de linéarité absurde:

\begin{lieu}
	Si une chose, qui est la plus excellente dans son genre, l'emporte sur une autre qui soit aussi la plus excellente
	dans le sien, sans difficulté, le genre de la plus excellente l'emportera sur le genre de l'autre.
\end{lieu}

Beaucoup d'erreurs philosophiques viennent du langage et de sa grammaire, qui sont des systèmes, et qui nous soufflent
d'inventer de nouveaux systèmes: les écoles et les sectes. Pourtant, le langage et sa grammaire ne sont que des outils
pour s'exprimer, ces outils datent de la préhistoire et ceux qui les analysent pour y trouver des vérités profondes
(d'Aristote à Lacan) n'y trouvent que la confirmation de leurs préjugés.

\bigbreak

Si vous avez des questions, relisez cette préface cinq fois. Si vous avez encore plus de questions, lisez l’œuvre
complète de Schopenhauer au moins trois fois avant de contacter mon secrétariat (Odéon 8400).


\begin{comment}

En terme d'éthique, les végans sont aux français moyens ce que que les français moyens sont aux cannibales par loisirs.

\end{comment}




\part{Le premier livre}

\fancyhead[LE]{La rhétorique d'Aristote}
\fancyhead[RO]{Le premier livre}
\renewcommand{\headrulewidth}{1pt}


\section{Servant de préface à tout l'ouvrage}

\subsection{Que la rhétorique et la dialectique se ressemblent}

La rhétorique et la dialectique ont beaucoup de rapport, car toutes deux traitent de matières, qui, pour être communes, tombent
en quelque façon sous la connaissance de tout le monde, et ne sont renfermées dans les bornes d'aucune science particulière: d'où
vient aussi qu'il n'y a personne qui n'ait quelque usage de l'une et de l'autre, puisque chacun, selon sa portée et jusqu'à un
certain point, tâche d'examiner et de soutenir une raison, d'accuser et de défendre.

\subsection{Que la rhétorique est un art}

Parmi le peuple, quelques-uns réussissent à ces choses par hasard, d'autres parce qu'ils s'y sont habitués. Que si cela se fait
de toutes les deux façons, sans doute on peut avoir des règles là-dessus, et trouver une méthode assurée pour y réussir toujours;
puisque enfin il y a lieu de découvrir la cause pourquoi, et ceux qui font ceci par un pur hasard, et ceux qui le font par habitude,
arrivent au but qu'ils se proposent; or on m'avouera que c'est à l'art à donner ces règles, et que c'est là proprement son ouvrage.

\subsection{Que l'adresse principale de la rhétorique consiste aux preuves.}

Tous ceux au reste qui jusques à présent ont écrit de la rhétorique, n'ont presque rien fait de ce qu'il fallait faire, parce que
toute l'adresse de cet art est renfermée dans la preuve, le reste n'en est que l'accessoire. Cependant ils ne parlent point des
enthymèmes et des arguments, qui font tout le corps de la preuve, et se sont amusés à des choses éloignées de leur art, et purement
étrangères: Car l'invective, la compassion, la colère et les autres passions de cette nature dont ils traitent fort au long, ne
sont point du fait de l'orateur, mais regardent le juge; de sorte que si dans toutes les justices on faisait son devoir et que
partout on se gouvernât ainsi qu'en quelques républiques, et particulièrement les mieux policées, il se trouverait que ces gens-là,
lorsqu'ils voudraient parler en public, n'auraient rien a dire. Ce n'est pas que cela ne passe pour un abus, et qu'on ne croie
qu'il devrait y avoir des lois pour s'opposer à cette licence; mais peu de gens le mettent en pratique, et ce n'est qu'en certains
lieux qu'il est expressément défendu aux orateurs de sortir de leur sujet et de ne rien dire d'inutile, comme à Athènes, encore
n'est-ce que pour les jugements qui se rendent dans l'Aréopage. Et certainement, ceux qui le font ont grande raison d'en user ainsi,
puisque jamais il ne faut pervertir un juge, ni le porter à la compassion, à la colère ou à l'envie; vu que c'est faire la même
chose que celui qui courberait une règle dont il se voudrait servir. D'ailleurs, il n'y a personne qui ne voie que l'emploi de celui
qui plaide, et à quoi il doit s'étudier, est de montrer simplement que la chose dont il s'agit est ou n'est pas, qu'elle a été faite ou
ne l'a pas été, car de savoir si elle est de conséquence ou non, si véritablement elle est juste ou injuste, au cas que le législateur
ne s'en soit pas expliqué, c'est au juge à le connaître, sans l'apprendre de ceux qui parlent devant lui. 

\bigbreak

On voit par là qu'il serait à souhaiter que les lois sagement établies fussent si exactes qu'elles remarquassent jusqu'aux moindres
circonstances, afin de laisser peu de choses à la discrétion des juges; et cela pour plusieurs raisons.

Premièrement, \emph{eu égard aux personnes}, attendu qu'il n'est pas si aisé de trouver d'habiles gens, et que pour un ou deux qu'on
rencontre capables de faire des lois et d'exercer la judicature, il y en a cent qui ne le sont pas.

Secondement, \emph{à raison du temps}, vu que les lois dans leur établissement dépendent d'une longue observation et de l'expérience de
plusieurs siècles, au lieu que les jugements qui se rendent se font sur le champ; de sorte qu'en cet état, il est difficile aux juges et
à ceux qui délibèrent dans les grandes assemblées de satisfaire entièrement à l'intérêt public, et à celui des parties.

La dernière raison, et la plus importante, est tirée \emph{des choses mêmes}; puisque enfin tout législateur n'a point à prononcer sur des
matières particulières, ni pour des personnes qui soient présentes, mais en général, et pour des personnes qui ne sont pas encore. Le juge
au contraire, et ceux qui délibèrent, ne connaissent que des faits particuliers, où le plus souvent leur propre intérêt se rencontre; et ne
regardant que des personnes présentes, pour qui tantôt ils ont de l'amour, et tantôt de la haine. D'où vient qu'alors la passion les
aveugle, et les empêche de bien voir la vérité. 

Il est donc à propos, comme nous venons de remarquer, que le législateur laisse peu de chose au pouvoir des juges, afin, s'il est possible,
qu'ils n'aient qu'a examiner si ce qu'on leur dit est ou n'est pas, s'il arrivera ou s'il ne doit point arriver, qui sont des cas qu'un
législateur ne peut prévoir et que, nécessairement, il faut laisser à la connaissance des juges. 

\bigbreak

Que si cela est ainsi, l'on voit manifestement que ceux-là sortent du sujet de la rhétorique qui enseignent, par exemple, comment il faut
faire un exorde, une narration, et ainsi de chacune des autres parties d'un discours; parce que tout ce qu'ils font, en telle rencontre, ne
tend qu'à altérer l'esprit du juge et ne montre point en quoi consiste l'artifice de la preuve, qui est de cultiver le raisonnement et
de rendre un homme fort en enthymèmes. Aussi est-ce pour cette considération qu'encore qu'il y ait deux parties principales dans la rhétorique,
dont l'une regarde les délibérations et l'autre les matières du Barreau --- et même que la partie qui sert à délibérer soit beaucoup plus noble
et plus politique que celle qui s'arrête seulement à examiner les clauses d'un simple contrat --- tous néanmoins ont abandonné la délibération,
sans en dire le moindre mot, et pour l'autre, c'est à qui en traitera plus au long, à qui donnera plus de préceptes. Et la raison qui les
y a portés est qu'il est peu avantageux de sortir de son sujet en parlant dans un conseil, cette partie donnant beaucoup moins d'entrée à la
malice et à la finesse que ne fait le plaidoyer, à cause que l'intérêt qu'elle regarde est un intérêt commun, et que celui qui écoute est juge
en sa propre cause; de sorte qu'ici l'orateur n'a qu'à montrer simplement que ce qu'il dit est véritable. Il en va autrement du barreau, où il ne
suffit pas de prouver que la chose est, mais encore qu'il est bon de gagner l'esprit de l'auditeur, et de le faire tourner de son côté; vu qu'il
s'agit là de l'intérêt d'autrui et qu'il n'a point à prononcer sur des choses qui le touchent. Ainsi, comme il ne regarde que sa propre
satisfaction, et qu'il n'écoute que pour faire faveur, il se laisse aisément emporter aux discours de ceux qui plaident et ne fait plus l'office
de juge. C'est aussi pour cela, comme nous disons auparavant, qu'en beaucoup de lieu, la loi défend aux orateurs de parler hors de leur sujet; ce
qu'il n'a point été nécessaire de faire dans la délibération, à cause que c'est une chose qui s'observe là assez d'elle-même.

\subsection{Que les plus fortes preuves de la rhétorique dépendent des enthymèmes}

Donc puisqu'il est certain:

\begin{emphpar}
	Que tout l'artifice de la rhétorique consiste dans la preuve;
\end{emphpar}

De plus:

\begin{emphpar}
	Que la preuve est une sorte de démonstration;
	
	Que le plus puissant moyen qu'il y ait pour démontrer, c'est l'enthymème;
\end{emphpar}

Enfin:

\begin{emphpar}
	Que l'enthymème est une manière de syllogisme;
\end{emphpar}

En un mot:

\begin{emphpar}
	Puisque c'est ou à la dialectique toute entière, ou à l'une de ses parties, à traiter du syllogisme pleinement;
\end{emphpar}

Il s'ensuit:

\begin{emphpar}
	Que quiconque sera bon dialecticien, c'est à dire qui saura comment le syllogisme se fait et de quelles propositions il est composé,
	celui-là encore aisément pourra faire des enthymème; n'ayant plus qu'à observer sur quelles matières ils s'appliquent, et en quoi ils
	sont différents des syllogismes de la logique.
\end{emphpar}

Et cela d'autant que c'est à la même faculté qui s'attache au \emph{vrai} qui est l'objet du syllogisme, de connaître encore le \emph{vraisemblable},
qui est l'objet de l'enthymème, joint que tous les hommes naturellement sont assez portés aux sciences et à la connaissance du vrai et qu'assez
souvent, hors ce qui regarde les sciences, ils découvrent la vérité en beaucoup de choses. Tellement que pour tirer de simples conjectures, et
découvrir la vraisemblance dans les matières douteuses, il ne faut point d'autre adresse ni d'autre lumière que celles qui dans les matières
certaines et infaillibles nous font raisonner régulièrement et trouver toujours la vérité.

\bigbreak

Il paraît donc, évidemment, que ceux qui ont écrit de la rhétorique jusques à présent n'ont point traité son sujet; et nous avons dit pourquoi ils ont
quitté le \emph{genre délibératif} pour s'attacher au \emph{judiciaire}.

\subsection{Que la rhétorique est utile}

On ne peut pas douter que la rhétorique ne soit utile, puisqu'elle a pour but de faire rendre la justice, et de faire connaître la vérité, qui est
une chose avantageuse et toute autre que de faire le contraire. Aussi toutes les fois qu'on ne juge pas comme il faut, cela n'arrive que parce que
l'injustice et le mensonge ont prévalu sur la justice et sur la vérité, ce qui mérite punition. 

De plus, la rhétorique est de telle conséquence que, quand nous serions les plus savants du monde, néanmoins, il nous serait difficile en parlant à
certaines personnes de les persuader, à cause que les sciences ont une façon particulière de s'expliquer et certains termes, dont il est impossible de
se servir devant des ignorants. De sorte que, pour se faire entendre à eux et pour les persuader, il faut avoir recours à des notions générales, ou
\emph{lieux communs}, ainsi que nous avons remarqué dans nos Livres des \emph{Topiques}, en traitant \emph{de la manière de parler au peuple}.

Un troisième avantage de la rhétorique est qu'il faut être capable de persuader les deux partis contraires, de même que dans la dialectique on doit
savoir argumenter de part et d'autre; non pas à la vérité qu'il faille faire tous les deux, car jamais on ne doit persuader ce qui est mauvais, mais --- la
chose est importante --- afin qu'au moins on n'ignore pas comment cela se fait, et qu'en même temps on puisse répondre à ceux qui voudraient s'en servir
pour favoriser l'injustice. Or est-il que de tous les arts il n'y a que la dialectique et la rhétorique qui fassent profession de défendre les deux partis
contraires. Ce n'est pas pourtant qu'il faille croire que les matières qui se traitent en telles rencontres soient également probables, puisque absolument
parlant, tout ce qui est véritable et meilleur de soi est aussi, et plus aisé à être prouvé, et plus capable de persuader.

Après tout, il serait ridicule de s'imaginer qu'il y eût de la honte à ne se pouvoir aider de son corps, et qu'il n'y en eût point à être privé du secours
de la parole, dont l'usage, bien plus que celui du corps, appartient à l'homme naturellement. 

De dire que la rhétorique peut beaucoup nuire si l'on s'en veut mal servir, c'est une objection qui regarde en commun toutes les bonnes choses et les plus
utiles mêmes, excepté la vertu; par exemple, la force, la santé, les richesses, les armes; puisque selon l'usage qu'on en fera, bon ou mauvais, il en viendra
un grand mal ou un grand bien. 

De tout ce que nous avons dit jusques ici, il se voit en premier lieu que la rhétorique est utile, et qu'elle n'a point un sujet particulier ni déterminé non
plus que la dialectique.

En second lieu que l'ouvrage de la rhétorique ne consiste point à persuader absolument, mais à découvrir en chaque chose ce qui est capable de le faire, et en
cela convient-elle avec tous les autres arts. Par exemple, la médecine ne promet pas de guérir infailliblement, mais seulement de contribuer à la santé autant
qu'il est possible; puisqu'on ne laisse pas de bien traiter certains malades, encore que la santé ne leur puisse être rendue.

Enfin, il se voit que c'est à la rhétorique à considérer également, et ce qui est capable de persuader en effet, et ce qui ne le peut faire qu'en apparence;
comme c'est à la dialectique à traiter du syllogisme apparent et du véritable. Je dis que c'est à la dialectique à traiter du syllogisme apparent, afin qu'on
ne croie pas que cela soit réservé au sophiste; vu que ce qui donne la qualité de \emph{sophiste} à un homme n'est point cette connaissance et cette adresse de
pouvoir user de semblables arguments, mais bien le but qu'il se propose et le dessein qu'il a de n'argumenter que pour tromper. Véritablement, il y a cette
différence entre la rhétorique et la dialectique quant à ce point que dans la rhétorique, autant est \emph{orateur} celui qui n'emploie que de faux arguments, et
qui n'en veut point employer d'autres, que celui qui ne se sert que de bons, et qui ne tâche qu'à faire connaître la vérité. Pour la dialectique, il n'en va pas
ainsi, puisque là, le dessein de tromper et de ne s'attacher qu'à de vaines subtilités est proprement ce que nous appelons être \emph{sophistes}; au lieu que le
\emph{dialecticien} ne s'attache qu'à l'art et à la vérité.

Mais traitons tout de bon maintenant de la rhétorique, et voyons de quelle façon nous pourrons venir à bout des choses que nous avons proposées. Comme si donc nous
n'avions encore rien dit de cet art, commençons par sa définition, et ensuite, nous examinerons le reste.



\section{Éléments de la rhétorique}

\subsection{Ce que c'est que la rhétorique}

Posons que la rhétorique est \emph{un art ou une faculté qui considère en chaque sujet ce qui est capable de le persuader}; car il n'est point d'art
qui fasse la même chose, puisque tous les autres arts et toutes les autres facultés ne traitent que leur sujet et ne persuadent que là-dessus. Par
exemple, la médecine ne raisonne et ne persuade que sur ce qui regarde la santé et la maladie, la géométrie que sur les changements et les différences
remarquables qui arrivent aux grandeurs, et enfin, l'arithmétique, que sur ce qui touche le nombre; Ainsi en est-il des autres arts et des autres
sciences. Mais pour la rhétorique, quelque sujet qu'on lui propose, elle a droit, pour ainsi dire, d'y voir ce qui peut persuader. Aussi avons nous
remarqué qu'elle n'a point un sujet particulier ni déterminé sur lequel elle travaille. 

\subsection{Qualité des preuves de la rhétorique}

La rhétorique a deux sortes de preuves, les unes sont \emph{artificielles} et les autres \emph{sans artifice}. J'appelle preuves sans artifice celles
qui ne dépendent point de notre industrie, mais que nous trouvons toutes faites, comme sont les témoins, les réponses faites à la torture, les contrats
et autres choses semblables. Et je nomme artificielles toutes celles que nous pouvons trouver de nous-mêmes et par les règles de la rhétorique, de
sorte qu'il faut inventer celles-ci, au lieu qu'on se sert simplement des autres.

Pour les preuves \emph{artificielles}, il s'en trouve de trois espèces.

La première est fondée sur les mœurs et sur la bonne opinion qu'on a de celui qui parle.

La seconde vient de la disposition de l'auditeur, et d'avoir préparé son esprit d'une certaine façon.

Et la dernière enfin naît du discours, soit que véritablement, on ait démontré son sujet, ou seulement en apparence.

\bigbreak

L'orateur persuade à l'occasion de sa personne et de ses mœurs, lorsqu'il parle de manière qu'il se rend digne de foi; car la vertu est d'un tel crédit
qu'absolument, nous ajoutons toujours plus de foi et plutôt aux gens de bien qu'aux autres, et cela généralement en tout, mais particulièrement dans les
matières douteuses, et où l'esprit de part et d'autre ne voit point de raison qu'il puisse suivre avec sûreté; vu qu'alors nous nous abandonnons à eux
entièrement et croyons tout ce qu'ils disent. Or, il faut remarquer que ce crédit doit aussi venir de l'adresse de notre discours, et non pas simplement
de la préoccupation de l'auditeur, ni parce qu'il avait déjà cette bonne opinion de nous avant que de nous écouter; car enfin on ne doit point s'arrêter
à ce que disent quelques-uns de ceux qui ont traité de la rhétorique, qui, à propos des bonnes mœurs et de cette probité qui doit éclater dans le discours
de l'orateur, soutiennent qu'absolument, elle est inutile et ne contribue en rien à gagner l'esprit; mais tant s'en faut que cela soit, que même, c'est un
des plus forts et des plus puissants moyens qu'il y ait pour persuader.

On persuade à l'occasion de ses Auditeurs  lorsque par le discours, on les porte à quelque passion; aussi jugeons nous bien autrement quand nous sommes
tristes que quand nous sommes joyeux, et bien autrement quand nous aimons que quand nous avons de la haine; Or, comme il a déjà été dit, c'est la seule
chose que tous les rhétoriciens d'aujourd'hui se sont efforcés de traiter, mais il en sera parlé plus particulièrement quand nous serons au discours des
passions. 

Enfin, on persuade par la force du discours lorsque employant tout ce qui peut servir à prouver le sujet que l'on traite, on fait voir que la chose dont
il s'agit est véritable en effet, ou en apparence.

\bigbreak

Que si les preuves \emph{artificielles} dépendent de ces trois points, il est certain qu'il faudra s'étudier à trois choses. Premièrement, à savoir faire
des syllogismes. Secondement, à connaître les mœurs et les vertus de chacun. En dernier lieu, à connaître les passions. Par exemple, quelle est la nature
de chaque passion en particulier, sa différence, ce qui la fait naître, et comment on le peut faire: de sorte qu'il se voie  par là que la rhétorique est
comme un germe et un rejeton, non seulement, de la dialectique, mais encore de cette partie de la morale qu'on peut avec raison nommer \emph{politique}. Et
de fait, c'est pour cela que la rhétorique affecte de paraître sous un habit emprunté et de passer pour politique; aussi bien que ceux qui font profession
d'orateur, qui d'ordinaire se flattent de cette vanité en partie par présomption, en partie par ignorance, et en partie pour d'autres considérations
humaines.

Je dis que la rhétorique est comme un rejeton de la dialectique, parce qu'elle en est une partie et une image, ainsi que nous avons remarqué dès le
commencement; vu que ni l'une ni l'autre, ne sont point des sciences qui s'attachent à un sujet particulier, mais bien certaines facultés qui cherchent
à trouver des raisons dans toutes sortes de matières. Mais c'est assez parler de leur pouvoir, et du rapport qu'elles ont.

\subsection{De l'exemple et de l'enthymème, de leur rapport avec le syllogisme et l'induction}

Quant aux preuves qui en effet démontrent  une chose, ou qui semblent la démontrer. Tout ainsi que pour démontrer dans la dialectique, l'on se sert toujours
de l'induction et du syllogisme, soit véritable ou apparent, de même, pour démontrer dans la rhétorique, l'on se sert  toujours de \emph{l'exemple}, qui est
la même chose que \emph{l'induction}, et encore de \emph{l'enthymème} qui répond au \emph{syllogisme}. Aussi est-ce pour cette  raison que je nomme l'enthymème
et l'exemple, l'un le syllogisme, et l'autre l'induction de la rhétorique.

Et de vrai pour montrer leur parfait rapport, c'est que tout orateur qui prouve une chose par démonstration apporte toujours, ou des exemples, ou des
enthymèmes, n'ayant point dans la rhétorique d'autres moyens pour démontrer que ceux-là. Et par conséquent, si la même nécessité se rencontre dans la
dialectique et que là, il soit impossible de rien prouver démonstrativement, quelque chose même que ce puisse être, sans se servir du syllogisme et de
l'induction --- comme nous l'avons fait voir dans nos Livres des \emph{Analytiques} --- il s'ensuit que chacun de ces deux moyens à l'égard des deux
autres, je veux dire, que le syllogisme à l'égard de l'enthymème, et que l'induction à l'égard de l'exemple, ne seront qu'une même chose.

\bigbreak

De savoir maintenant la différence qu'il y a entre l'exemple et l'enthymème, nous l'avons enseigné dans nos \emph{Topiques}. Car toutes les fois qu'on veut
montrer que quelque chose est d'une certaine façon, et qu'on apporte pour  preuve un grand nombre d'autres choses toutes semblables, dans la dialectique cela
s'appelle \emph{induction}, et dans la rhétorique, \emph{exemple}. Mais lorsqu'on établit certaines propositions, et que par une conséquence nécessaire, on
vient à tirer une autre proposition toute différente, à cause seulement que ces premières propositions ont été établies --- et cela indifféremment soit que
telles propositions soient vraies ou qu'elles ne soient que vraisemblables --- dans la dialectique, cela  s'appelle \emph{syllogisme} et dans la rhétorique,
\emph{enthymème}.

De là paraît, que l'un et l'autre de ces deux moyens, quand on sait bien s'en servir, sont de très grand usage et très considérables, chacun d'eux contenant
en soi comme une espèce de rhétorique à part. Car ce que nous avons remarqué de la dialectique, dans nos livres des \emph{Méthodes}, touchant sa façon de prouver,
est encore ici remarquable pour la rhétorique, attendu que la rhétorique aussi bien qu'elle a deux styles distingués ou deux manières différentes, dont l'une prouve
tout par les exemples et l'autre par les enthymèmes, comme il se trouve des orateurs qui ne se servent que d'enthymèmes et d'autres qui n'emploient que des exemples.
Et certainement, les discours qui prouvent par les exemples ne persuadent pas moins que les autres; toute la différence qu'il y a, c'est que ceux qui prouvent par
les enthymèmes font une plus forte impression sur l'esprit et troublent d'avantage. La raison en sera dite ailleurs, quand nous montrerons de quelle façon il se faut
servir de tous les deux. Pour maintenant, il suffit de nous expliquer et de débrouiller ces matières.

\subsection{Sur quelles matières s'appliquent les enthymèmes}

Donc,

\begin{emphpar}
     Puisque tout ce qui est propre à persuader est relatif aux personnes, c'est à dire propre à persuader quelqu'un,
\end{emphpar}

De plus,

\begin{emphpar}
     Puisque ces choses-là sont de deux sortes, les unes capables de persuader d'elles-mêmes et croyables d'abord, les
      autres simplement parce qu'elles semblent établies sur des preuves de la qualité de ces premières,
\end{emphpar}

Enfin,

\begin{emphpar}
     Puisqu'il n'est point d'art qui s'arrête à considérer la nature d'aucun particulier ni qui en fasse son objet; car
      la médecine, par exemple, ne se propose point de connaître en particulier ce qui est bon pour la santé de Socrate ou
      de Callias, mais seulement de tels en général, et de  tels, qui différent de tempérament, ou qui ont telles maladies;
      vu que c'est là proprement où l'art se fait voir, n'étant pas possible à quelque science ni à quelque art que ce soit
      de connaître tous les particuliers, le nombre en étant infini.
\end{emphpar}

De là, il s'ensuit:

\begin{emphpar}
     Que la rhétorique ne se proposera pas non plus, et ne considérera point ce qui est probable à l'égard d'un tel particulier
      et qui pourra le persuader --- par exemple ce qui est probable à l'égard de Socrate ou d'Hippias --- mais bien ce qui le
      sera à l'égard de telle ou de telle sorte d'esprits, qui ont des mœurs et des inclinations différentes.
\end{emphpar}

Et cela à l'imitation de la dialectique, car la dialectique ne s'amuse pas à argumenter ni à faire des syllogismes sur tout ce qui se présente
indifféremment, pour probable même qu'il puisse être à certaines personnes, vu qu'il y a des choses qui peuvent paraître probables à certains
particuliers, par exemple à des saouls et à des extravagants; mais seulement, elle argumente sur les matières qui ne sont pas assez établies
d'elles-mêmes, et qui ont besoin de preuve. 

Pour la rhétorique, elle s'attache seulement aux matières qui ont accoutumé de tomber en délibération, car c'est là proprement son ouvrage que
d'examiner les choses sur lesquelles ordinairement nous délibérons, et de qui nous n'avons aucun art, et même encore en présence de certains
auditeurs, qui pour être peu éclairés ne sont pas capables de comprendre ce qui embrasse plusieurs choses à la fois, ni suivre de l'esprit un
raisonnement de longue haleine.

Sur cela, il faut remarquer que jamais nous ne délibérons que sur ce qui nous paraît arriver diversement, n'y ayant point d'autre occasion de
délibérer que celle-là, puisque jamais on ne met en délibération ni le passé, quand il ne s'est pu faire autrement qu'il a été fait, ni l'avenir,
quand  il est impossible qu'il arrive d'une autre façon, ni le présent, quand on ne peut pas empêcher qu'il ne soit comme il est --- du moins
tandis qu'on demeure dans cette opinion et que la chose est crue ainsi.

\subsection{La manière d'argumenter en rhétorique}

Pour ce qui est d'argumenter et d'établir une chose par syllogismes et par conséquences, on s'y prend en deux manières. Car ou l'on tire des
conséquences de propositions qui ont déjà été prouvées par d'autres syllogismes et par d'autres arguments, ou bien de propositions qui ne l'ont
pas été, mais qui ont besoin de l'être parce qu'elles ne sont pas probables d'elles-mêmes. Or est il que ni l'une ni l'autre de ces deux manières
n'est point propre à la rhétorique. La première, comme trop difficile à suivre à cause de la longueur, vu qu'on suppose que l'auditeur est simple
et peu intelligent; et l'autre est incapable de persuader, parce qu'elle avance des choses qui ne sont pas avouées de tout le monde, et qui n'ont
aucune vraisemblance. 

\bigbreak

De ces observations, il s'ensuit premièrement, touchant la matière de l'exemple et de l'enthymème, que toujours ils seront employés sur les matières
incertaines et sur des choses qui pour ordinaire arrivent de différentes façons; l'exemple, qui comme il a été déjà remarqué est la même chose
que l'induction, et l'enthymème la même chose que le syllogisme.

De plus, il s'ensuit quant à la forme de l'enthymème que d'ordinaire, il ne pourra pas avancer tant de choses, ni être composé de tant de propositions
que le syllogisme parfait, attendu que si quelqu'une de ces propositions est connue, il faut l'omettre, puisque l'auditeur de lui-même la supplée alors.
Par exemple, on veut faire savoir que Doricus, ce fameux Athlète, a vaincu aux Jeux Olympiques et a été couronné; il suffit de dire que ce Doricus a
gagné le prix sans qu'il soit besoin d'ajouter cette proposition générale que ceux qui remportent la victoire à ces jeux y sont couronnés, parce qu'on
sait bien que cela se fait toujours. 

\subsection{De quelle sorte de propositions sont composés les enthymèmes}

Donc,

\begin{emphpar}
      Puisque entre les propositions, dont la rhétorique forme ses syllogismes, il s'en trouve peu de nécessaires, car la plupart des matières
      qui se jugent dans le barreau et qui se traitent dans les délibérations sont incertaines, et peuvent arriver de différentes façons, vu
       qu'on ne délibère jamais que sur les choses qu'on veut entreprendre et qu'on propose de faire, toutes les actions qui se font dans le
       monde étant de cette nature, et n'y en ayant pas une, pour ainsi dire, qui porte un effet nécessaire et dont l'événement soit certain,
\end{emphpar}

De plus,

\begin{emphpar}
      Puisque les propositions contingentes, et qui ne sont vraies que pour l'ordinaire, doivent toujours être prouvées par d'autres de même nature
      et incertaines comme elles, et tout au contraire les nécessaires, par des nécessaires, ainsi que nous avons fait voir dans nos Livres des
      \emph{Analytiques},
\end{emphpar}

Il s'ensuit:

\begin{emphpar}
      Que les matières d'où se tirent les enthymèmes seront pour la plupart incertaines ou contingentes, et qu'il y en aura fort peu de nécessaires.
\end{emphpar}

De vrai, tous les enthymèmes qui se font ont toujours leur preuve fondée ou sur le \emph{vraisemblable} ou sur les \emph{signes}; en sorte qu'il faut que
ces signes, et ce vraisemblable, eu égard au nécessaire et à l'incertain, ou contingent, ne soient entre eux qu'une même chose. Et de fait, proprement,
le vraisemblable est ce qui se fait d'ordinaire, non % "non pas à la vérité absolument": je coupe. Ce sera déjà un peu moins wtf... 
pas absolument, comme le prétendent quelques-uns dans la définition qu'ils en donnent, entendant par là indifféremment tout ce qui peut être compris sous
le mot de vraisemblable, de quelque nature que ces choses-là puissent être, soit que la qualité d'universel leur convienne ou ne leur convienne pas. Si
bien que dans la rhétorique, le \emph{vraisemblable} se doit seulement entendre des choses qui n'arrivent pas toujours de la même façon, et de plus le rapporter
à celles à l'égard desquelles il passe pour vraisemblable, de la même sorte que \emph{l'universel} se rapporte au \emph{particulier}.

\subsection{Des signes et de leur différence}

Pour les \emph{signes}, ils sont de deux sortes. Les uns se rapportent aux choses à qui ils servent de signes, comme le particulier se rapporte à l'universel,
c'est à dire que la preuve en est la même que si l'on prouvait une proposition générale par une proposition particulière; les autres, au contraire, ont
le rapport d'un universel à un particulier; et de ceux-ci, quelques-uns sont nécessaires à qui on donne le nom de \emph{tekmérion}; les autres ne sont
pas nécessaires, et sont simplement appelés signes, sans avoir d'autre nom qui les distinguent. J'appelle \emph{signes nécessaires}, ceux qui peuvent
servir de matière au syllogisme et dont la preuve est convaincante; c'est pourquoi le signe appelle \emph{tekmérion} est mis au nombre de ceux-là. Aussi
toutes les fois qu'un orateur allègue pour preuve des choses auxquelles il ne pense pas qu'on puisse répondre, alors il qualifie ces preuves du nom
\emph{tekmérion}, comme qui dirait une preuve démonstrative et qui \emph{termine} tout le différend. Car le mot de \emph{tekmar}, d'où est tiré celui de
\emph{tekmérion}, anciennement signifiait la même chose que le mot de \emph{terme}\footnote{300 ans après\dots{}
                                                                                        <<Le \emph{tekmérion} est l'indice sûr, le signe nécessaire
                                                                                        ou encore ``le signe indestructible'', celui qui est ce qu'il
                                                                                        est et qui ne peut pas être autrement>>. Roland \textsc{Barthes}, 
                                                                                        \emph{L'aventure sémiologique}, Seuil, 1985, p.\,134}.

\bigbreak

Mais donnons des exemples de ces signes et, premièrement, de celui que nous avons remarqué avoir le rapport du particulier à l'universel. Si donc on
raisonne ainsi:

\begin{emphpar}
      Un signe que tous les habiles gens sont gens de bien c'est que Socrate, qui était un habite homme, a été très homme de bien.
\end{emphpar}

Véritablement alors, ce serait apporter un signe pour sa preuve. Un tel signe, néanmoins, ne serait pas nécessaire ni convaincant, étant facile
d'y répondre. La raison est qu'on n'en peut pas faire un syllogisme, puisque le syllogisme ne tire jamais une conclusion universelle d'une simple
proposition particulière.

Mais si quelqu'un venait à raisonner de cette autre façon:

\begin{emphpar}
      Un signe que cet homme est malade, c'est qu'il a la fièvre,
\end{emphpar}

ou bien:

\begin{emphpar}
      Un signe que cette femme est mère, c'est qu'elle a du lait aux mamelles,
\end{emphpar}

Cette sorte de signe serait nécessaire, et le seul que nous appelions \emph{tekmérion}; car quand un signe est de telle qualité que lui seul suffit
pour faire connaître que ce qu'on dit est vrai; pour lors la preuve est convaincante et ne souffre point de réponse.

Quant aux autres signes qui ont le même rapport qu'a l'universel au particulier, mais qui ne sont pas nécessaires, c'est comme se quelqu'un disait:

\begin{emphpar}
      Un signe que cette personne a la fièvre, c'est  qu'elle respire comme si, elle était hors d'haleine.
\end{emphpar}

Certainement, ce signe serait véritable; il est aisé néanmoins d'y répondre, puisqu'il arrive quelque fois qu'un homme est hors d'haleine qui
pourtant n'a pas la fièvre.

\bigbreak

%\begin{flushleft}
\noindent
\begin{tabular}{|p{0.35\textwidth}|p{0.25\textwidth}|p{0.30\textwidth}|}
\cline{2-3}
\multicolumn{1}{c|}{} & Incertain & Nécessaire \\
\hline
Le particulier comme signe d'un universel & Socrate \rightarrow tous les habiles gens &  \\
\hline
L'universel comme signe du particulier & Je tousse \rightarrow coronavirus & N'importe quel \emph{tekmérion} \\
\hline
\end{tabular}
%\caption{Légende du tableau.}
%\End{flushleft}


\bigbreak

Nous avons donc enseigné ce que c'est que \emph{vraisemblable}, et ce que c'est que \emph{signe}; de plus, nous avons remarqué la différence qu'il y
a entre les signes nécessaires et ceux qui ne le sont pas. Mais ces choses-là ont été expliqués plus clairement et plus au long dans nos Livres des
\emph{Analytiques}, où nous avons touché les raisons pourquoi quelques-uns de ces signes peuvent servir de matière aux syllogismes, et pourquoi les
autres en sont incapables.

\subsection{De l'exemple, et comment il s'en faut servir}

Pour ce qui est de l'exemple, nous avons remarqué qu'il était la même chose que l'induction, et de plus nous avons fait voir en quoi consistait l'induction.
Au reste, il ne faut pas considérer l'exemple à l'égard des choses à qui il sert d'exemple comme le particulier est considéré à l'égard de l'universel, ou
comme l'universel est à l'égard du particulier, encore moins comme un universel le peut-être à l'égard d'un autre universel, mais bien toujours comme une chose
particulière est considérée à l'égard d'une autre particulière, et comme un semblable l'est à l'égard d'un autre semblable. Toutes les fois, donc, que deux
choses se trouvent sous un même genre, que l'une est plus connue que l'autre, celle qui est la plus connue est proprement ce que nous appelions \emph{exemple}.
Car si je voulais montrer que Denys de Syracuse a dessein de se faire tyran lorsqu'il demande des gardes, je dirais que Pisistrate, comme lui, demanda des gardes
d'abord, et que si tôt qu'il en eut, il se saisit du gouvernement d'Athènes. Je dirais que Théagène fit la même chose à Mégare, et alléguerais ensuite les autres,
qu'on saurait être venus à la tyrannie par telle voie, qui tous serviraient d'exemple à l'égard de Denys de Syracuse, dont il ne paraîtrait pas encore si véritablement
c'est à ce dessein qu'il demande des gardes. Or, tous ces exemples particuliers sont compris sous cette proposition générale, que \emph{quiconque pense à la tyrannie
et à se saisir du gouvernement demande des gardes}.

Nous avons donc montré en quoi consistent les preuves de la rhétorique qui paraissent démonstratives.

\subsection{De la différence des enthymèmes}

Quant aux enthymèmes, leur différence est si grande qu'il y a peu de personnes qui se puissent vanter de les bien connaître, puisque enfin cette
différence est la même que celle des  syllogismes de la dialectique --- attendu que quelques-uns sont particuliers à la rhétorique, ni plus ni moins
qu'entre les syllogismes, quelques-uns sont particuliers à la dialectique --- les autres  appartiennent aux autres arts, et aux autres facultés, tant
de celles qui sont à inventer, que de celles que nous connaissons et qui sont déjà inventées, ce qui fait qu'ils paraissent obscurs à l'auditeur, et
que ceux qui s'en servent autrement que la rhétorique ou la dialectique n'enseignent s'écartent de leur art, et ne raisonnent  plus alors ni comme un
dialecticien doit faire, ni en qualité d'orateurs.

Mais sans doute que ceci sera plus clair quand nous l'aurons davantage expliqué. Il faut donc savoir que les syllogismes que j'attribue à la dialectique
sont ceux à qui nous assignons des \emph{lieux}. Or, il y a deux sortes de lieux. Les uns sont \emph{communs} et les autres \emph{propres}. J'appelle
\emph{lieux communs} ceux qui servent à prouver diverses matières, comme de jurisprudence, de physique, de politique et de beaucoup d'autres qui diffèrent
d'espèces. Tel est le lieu commun qui traite du plus et du moins, parce que de ce lieu là, nous pourrons aussitôt tirer des syllogismes et des enthymèmes
sur des matières de droit ou de physique que de quelque autre science que ce soit. Or est-il que toutes ces matières sont distingués d'espèces et
différentes entre elles. Pour les \emph{lieux propres}, ce sont ceux qui sont particuliers à chaque genre et à chaque espèce de propositions. Par exemple,
il y a des propositions tellement dépendantes de la physique qu'on n'en saurait faire d'enthymèmes ni de syllogismes pour prouver aucune proposition de la
morale, et d'autres, au contraire, tellement dépendantes de la morale qu'on ne s'en pourrait pas servir pour prouver aucune proposition de la physique; ce
qui se doit entendre également de toutes les autres propositions particulières et spécifiques.

\bigbreak

Il y a ceci a remarquer touchant les \emph{lieux communs} que jamais ils ne peuvent nous rendre savants sur aucune matière particulière, à cause qu'ils
sont vagues et ne traitent point un sujet déterminé. Il en est tout au contraire des \emph{lieux propres}, car plus les propositions que nous en tirerons
seront choisies et particulières au sujet que nous traitons, et plus insensiblement nous nous éloignerons de la dialectique et de la rhétorique pour nous
approcher d'une autre science; parce qu'enfin, si nous ramenons ces propositions jusqu'aux principes, alors notre raisonnement et notre preuve ne seront
plus l'ouvrage de la dialectique ni de la rhétorique, mais seulement de la science dont nous aurons touché les principes. 

Ici, nous observerons encore que la plupart des enthymèmes se tirent des lieux propres seulement, et qu'il y en a fort peu qui soient tirés des lieux
communs. Nous diviserons donc ici les enthymèmes de la même façon que nous avons déjà fait dans les \emph{Topiques}, savoir en autant de lieux propres
qu'il y a de sortes de  propositions d'où ils peuvent être tirés. Au reste, j'appelle \emph{lieux propres d'enthymèmes} les propositions qui sont
particulières à chaque genre de la rhétorique, et je nomme \emph{lieux communs} les propositions communes à tous les genres, et qui servent à prouver
toute sorte de matières.  Parlons donc, premièrement, des lieux propres des enthymèmes, mais auparavant des genres de la rhétorique, afin qu'ayant montré
combien il y en a, nous puissions voir en particulier quels font les \emph{éléments} de chacun, et les propositions qui leur conviennent.



\section{Que la rhétorique a trois genres}

La rhétorique a sous soi trois genres, puisqu'il se trouve autant de sortes d'auditeurs. Car il faut savoir que
tout discours regarde trois choses: celui qui parle, le sujet que l'on traite et la personne à qui on parle, que
nous appelons \emph{l'auditeur}, et auquel se rapporte tout le discours. 

Tout auditeur, au reste, doit être nécessairement ou simple auditeur, ou juge. S'il est juge, il faut que ce soit
ou de choses qui aient été faites déjà, ou de choses qui ne le soient pas encore.

L'auditeur qui a son jugement à donner sur ce qui n'est pas encore arrivé mais qu'on propose de faire simplement
est, par exemple, le peuple d'Athènes, assemblé pour délibérer sur les affaires de la république.

Celui qui a à juger du passé et de ce qui a été fait est, proprement, le magistrat ou le juge.

Enfin, le simple auditeur est celui qui ne vient que pour contenter sa curiosité, et pour avoir le plaisir d'entendre
un excellent orateur.

De manière qu'il faut, par nécessité, qu'il y ait trois genres dans la rhétorique qui répondent à ces trois sortes
d'auditeurs:

\begin{emphpar}
Le genre délibératif,

Le genre judiciaire,

Le genre démonstratif.
\end{emphpar}

Le genre délibératif a deux parties: la \emph{persuasion} et la \emph{dissuasion}, car toujours ceux qui délibèrent font
l'une de ces deux choses, soit qu'ils délibèrent sur leurs affaires particulières ou sur les affaires publiques.

Le genre judiciaire a aussi deux parties sous soi: \emph{l'accusation} et la \emph{défense}, car il est nécessaire qu'en
plaidant, les avocats fassent l'un ou l'autre, qu'ils défendent ou qu'ils accusent.

Le genre démonstratif, pareillement, comprend deux parties: la \emph{louange} et le \emph{blâme}.

\bigbreak

Chacun de ces genres a aussi un \emph{temps} qui lui est particulièrement affecté.

\emph{L'avenir} appartient au délibératif, car tout homme qui délibère, soit qu'il conseille ou dissuade, délibère toujours
sur ce qui n'est pas encore arrivé. 

Le \emph{passé} convient au judiciaire, car on n'accuse et l'on ne défend jamais que les actions qu'une personne a faites. 

Enfin, le \emph{présent} est le plus propre au genre démonstratif, puisqu'on ne loue ou ne blâme que ce qui est effectivement.
Ce n'est pas, néanmoins, qu'assez souvent, en telle rencontre, les orateurs ne fassent aussi mention du passé afin d'en
renouveler la mémoire; et même par avance de ce qui n'est pas encore, comme par un préjugé de l'avenir. 

\bigbreak

De plus, chacun de ces genres se propose un but et une fin particulière; de sorte que, comme il y a trois genres, il se
trouve aussi trois fins différentes.

Celui qui délibère se propose pour but ce qui est \emph{utile}, ou \emph{nuisible}, car tout orateur qui entreprend de
persuader une chose la propose toujours comme la meilleure, et s'il veut la dissuader, il tâche de faire voir que c'est
la pire. Ce n'est pas qu'il ne se serve encore de tout le reste que les autres genres se proposent, afin d'en fortifier
sa preuve. Par exemple, il tâche de montrer que cette même chose est encore juste ou injuste, honnête ou contre l'honneur. 

Ceux qui plaident se proposent toujours de faire voir que la chose donc il s'agit est juste ou injuste et pareillement,
se servent de tout le reste pour ce dessein.

Enfin, ceux qui ont à louer ou à blâmer prétendent seulement démontrer que ce qu'ils louent ou blâment est honnête ou
honteux, et tout de même, y rapportent les autres choses que nous venons de dire.

Et une preuve certaine que chacun de ces genres ne se propose point une autre fin que celle dont nous venons de parler,
c'est que bien souvent, il n'y aurait point de contestation touchant les autres points. Par exemple, ceux qui plaident
demeurent souvent d'accord qu'une chose a été faite, et même qu'elle a porté préjudice, mais jamais ils n'avouent qu'ils
aient fait une injustice, autrement il serait inutile de plaider. Le même se peut dire de ceux qui délibèrent. Souvent,
ils accordent tout le reste, mais jamais ils n'avouent que ce qu'ils persuadent de faire soit inutile, ou que l'entreprise
dont ils veulent détourner soit avantageuse. De savoir maintenant si ce qu'ils conseillent est contre la justice ou non,
par exemple, d'assujettir des peuples voisins et qui n'ont jamais fait de tort, c'est bien souvent à quoi ils ne pensent
pas seulement, tant ils s'en mettent peu en peine. Il en est de même de ceux qui louent ou blâment quelqu'un, tant s'en
faut qu'ils examinent s'il a fait des choses qui lui aient apporté du profit ou de la perte que, bien souvent, ils le
louent davantage quand il a méprisé son propre intérêt pour entreprendre quelque action glorieuse. Par exemple, ils
donnent des louanges à Achille, de ce qu'étant assuré de perdre la vie en vengeant la mort de Patrocle, son meilleur ami,
il aima mieux mourir que de laisser cette mort impunie. Cependant il est certain, que d'une part cette mort lui fut
glorieuse, d'un autre côté la vie lui était utile.

\subsection{De la nécessité des lieux propres et des lieux communs}

On voit par ce qui a été dit qu'il faut avoir, premièrement, un certain fonds, ou amas de propositions sur toutes les matières
dont nous venons de parler qui appartiennent aux trois genres; et de plus, on se doit souvenir que les propositions dont la
rhétorique se sert sont toutes tirées des signes, tant simples que nécessaires, et du vraisemblable. La nécessité, au reste,
d'avoir ainsi des propositions toutes prêtes vient de ce qu'absolument, on ne saurait faire de syllogismes sans propositions.
Et ainsi, l'enthymème étant une espèce de syllogisme, il faut aussi qu'il soit composé de propositions, mais de propositions
de la qualité de celles que nous avons remarquées.

\bigbreak

Mais parce qu'on ne peut pas dire que ce qui est du tout impossible puisse jamais avoir été fait, ni qu'il le puisse être, vu
que cela n'appartient qu'aux choses qui sont possibles de leur nature; outre cela parce qu'il est encore impossible que ce qui
n'a point été, ou qui jamais ne doit être, ait été fait déjà, ou soit fait à l'avenir, il fera encore nécessaire à l'orateur,
soit dans une délibération, soit en plaidant, soit dans les sujets qui regardent le genre démonstratif, d'avoir un autre fonds
ou amas de propositions, tant sur la matière du \emph{possible} que sur celle de \emph{l'impossible}, afin de pouvoir connaître
si une chose aura été faite ou non, si elle arrivera ou n'arrivera pas.

\bigbreak

Et d'autant encore que tout orateur, soit qu'il loue ou blâme, qu'il accuse ou défende, qu'il persuade ou dissuade, ne tâche
pas seulement de prouver les matières que nous venons de dire, mais assez souvent même de faire voir qu'une chose qui est bonne
ou mauvaise, honnête ou déshonnête, juste ou injuste, est encore grande ou petite, de conséquence ou non. Et cela indifféremment,
soit qu'il considère ces choses là en elles-mêmes ou qu'il les compare entre elles; il est certain qu'il sera encore nécessaire
d'avoir des propositions et en général et en particulier, tant sur la \emph{grandeur} et la \emph{petitesse}, que sur ce que nous
appelons plus grand et plus petit, afin de savoir quel bien en particulier sera plus grand ou plus petit qu'un autre. Quelle
action sera plus juste, ou plus injuste, et ainsi du reste. 

Nous venons donc de montrer quelles sont les matières d'où se doivent tirer nécessairement les propositions dont il se faut servir.
Parlons ensuite de chacune en particulier, savoir de celles qui appartiennent au genre délibératif premièrement; en second lieu,
de celles qui appartiennent au genre démonstratif et enfin, des autres qui regardent le genre judiciaire. 



\chapter{Le genre délibératif}
\section{Des matières qui tombent en délibération}

Dans ce genre ici, ce qu'il y a à faire premièrement, c'est d'avoir égard à la qualité des \emph{biens} et des
\emph{maux} que celui qui a à délibérer examine d'ordinaire, et sur lesquels il donne son avis. Car assurément,
il ne les examine pas tous, n'y ayant que les incertains auxquels il s'arrête, et qu'il juge également pouvoir
arriver et ne pas arriver, puisque jamais on ne met en délibération ni tout ce qui arrive nécessairement et de
la même façon, ni ce qui de toute impossibilité ne peut être.

Il est encore certain qu'on ne met pas en délibération tous les biens qui sont incertains absolument, puisqu'il
y en a qui dépendent de la nature et d'autres de la fortune, qui tantôt arrivent et n'arrivent pas, sur lesquels
il serait inutile de délibérer. D'où il est facile de voir quels sont les biens ou les maux qui peuvent tomber
en délibération.

Ce sont donc tous ceux qui de leur nature se rapportent à nous et qui sans nous n'arriveraient point, comme ayant
en nous-mêmes le principe de leur production, car d'ordinaire nous délibérons sur une chose jusqu'à ce que nous
ayons reconnu si elle est en notre pouvoir ou s'il nous est impossible de la faire.

Au reste, ce n'est pas ici le lieu de faire une exacte recherche, ni un dénombrement particulier de toutes les
choses dont les hommes ont accoutumé de délibérer, bien loin d'en traiter à fond et d'en donner une parfaite
connaissance, puisque cet emploi appartient à un art plus excellent et plus intelligent que la rhétorique; car
tant s'en faut que la rhétorique soit capable de rien traiter à fond, que même on lui a attribué beaucoup plus de
connaissance qu'il ne lui en appartient naturellement. Aussi, ce que nous avons remarqué au commencement est-il
vrai, que la rhétorique est composée premièrement de l'analytique, qui est une portion de la logique, en second lieu,
de cette partie de la politique qui s'attache aux mœurs et de plus, qu'elle ressemble à la dialectique en partie, et
en partie à la manière trompeuse de raisonner des sophistes. Mais la plus forte preuve qu'on ait qu'elle ne peut rien
traiter à fond, c'est que plus un orateur prendra à tâche d'employer ou la dialectique ou la rhétorique, non pas
comme de simples facultés qui raisonnent en général, mais comme des sciences exactes, et plus sans y penser il détruira
leur nature, puisque alors, s'en servant comme de sciences, il les renfermera dans de certains sujets, au lieu qu'elles
font profession de discours sur toutes sortes de matières. Ne laissons pas néanmoins de traiter ces choses, de sorte
que nous n'omettions rien de tout ce qui peut servir à notre dessein, et qu'il en demeure encore assez pour occuper la
politique.

\bigbreak

Il y a donc cinq points principaux qui donnent lieu aux assemblées publiques et sur lesquels tout le monde délibère, car
on délibère toujours:

\begin{emphpar}
     Ou sur la matière des finances,

     Ou touchant les affaires de la guerre et de la paix, 

     Ou pour la garnison des places, 

     Ou sur le fait des vivres et des marchandises, qu'on apporte de dehors et qui se transportent ailleurs,

     Ou enfin pour l'établissement des lois.
\end{emphpar}

De manière que si un orateur est obligé de parler sur les finances, il faudra qu'il sache en premier lieu quels sont les
revenus de l'état et à combien ils montent, afin que, si quelque fonds est diverti, on le rétablisse, ou si quelque droit a
été diminué, qu'on l'augmente. Il faudra qu'il sache encore tout ce que l'état dépense chaque année, afin que si quelqu'une
de ces dépenses est superflue, on la retranche, ou qu'on diminue celle qui sera trouvée trop grande, car non seulement on
devient plus riche quand on ajoute à ce qu'on possède déjà, mais même quand on retranche les dépenses inutiles. Or pour parler
pertinemment de toutes ces matières, il ne suffira pas simplement de les comprendre par sa propre expérience et par ce qui
sera arrivé dans l'état où l'on est, mais encore, il sera nécessaire de savoir tout ce qui aura été inventé là-dessus, et tout
ce qu'en disent les histoires.

Ainsi en doit-il être si nous avons à délibérer touchant les affaires de la guerre et de la paix. Il faudra qu'il sache la
puissance de l'état, combien il a de forces présentement, et jusqu'à quel point on les peut accroître; de plus, en quoi elles
consistent et celles qu'il y faudrait ajouter. Il sera bon encore de savoir les guerres que l'état a soutenues autrefois, et
comment il les a terminées. Et non seulement il les faudra savoir en particulier, mais encore celles de tous les autres états
voisins. Il ne faudra pas non plus ignorer quels sont les peuples à qui il sera glorieux de faire la guerre, afin que faisant
la paix avec ceux qui seront plus puissants que nous, il soit après en notre disposition de prendre les armes contre les autres
qui seront plus faibles. Il faudra aussi pouvoir faire comparaison de nos forces avec celles des ennemis, afin de connaître si
elles sont égales ou inégales, puisque en ce point consiste assez souvent le gain ou la perte des batailles. Or pour cela, il ne
suffira pas d'avoir fait réflexion sur toutes nos guerres en particulier, ni d'en avoir remarqué les évènements, mais encore il
sera nécessaire d'avoir fait la même chose sur toutes les guerres des autres peuples; vu que d'ordinaire les entreprises qui se
ressemblent ont des succès semblables.

Pour ce qui est des garnisons, non seulement il ne faudra pas ignorer comment une province est gardée, mais encore il faudra connaître
et la qualité de la garnison, et le nombre de ceux qui la composent, et la situation même de chaque place forte, afin que si quelque
garnison est faible, on la renforce, ou si quelqu'une es trop grosse, qu'on la diminue, et encore afin que les places les plus
importantes soient aussi les mieux gardées. Or est-il qu'il est impossible de savoir toutes ces choses, si l'on n'a une connaissance
particulière du pays.

Quant aux vivres, il faudra savoir, et la quantité qui sera nécessaire pour l'entretien de l'état, et la qualité de ceux qui croissent
dans le pays ou qu'on apporte d'ailleurs, et de plus, quelles sont les marchandises qui viennent de dehors, ou qui doivent être transportées.
Et le tout, afin que nous fassions alliance et amitié avec les peuples, ou qui emporteront ce que nous aurons de trop, ou qui nous fourniront
les choses nécessaires à la vie, car il se faut donner de garde principalement d'offenser deux sortes de personnes: ceux qui sont plus
puissants que nous et ceux qui nous sont absolument utiles. Voila pour ce qui regarde la sûreté d'un état, et qu'il faut qu'un orateur connaisse.

\bigbreak

Il nous reste à parler du dernier point, qui n'est pas moins considérable et que celui qui délibère ne doit pas non plus ignorer, savoir est,
de l'établissement des lois. Car c'est principalement de l'observation des lois et de leur établissement. que dépend. le salut d'un état. Il
faudra donc qu'il sache encore combien il y a de formes gouvernements, ce qui convient à chacun et ce qui les détruit, soit que ces choses là
leur soient propres et essentielles, soit que de leur nature elles leur soient contraires. Je dis que les états peuvent être détruits par les
choses mêmes qui leur font propres et qui les établissent, puisque si nous en exceptons l'état seul qui est véritablement parfait, on peut dire,
qu'il n'y en a pas un qu'on ne puisse détruire en lui donnant trop de ces, choses, ou en ne lui en donnant pas assez. Par exemple, si nous
donnons à l'état populaire plus ou moins de liberté qu'il ne faut, aussitôt, il s'affaiblit et dégénère en oligarchie. Car il en est de même que
des nez que nous appelions camus et aquilins: non seulement ajoutant aux uns et ôtant aux autres, on les ramène à la médiocrité, mais encore si
l'on s'efforce de les rendre toujours plus camus ou plus aquilins, on les met en tel état qu'à la fin, il ne leur reste pas même la moindre
apparence de nez. Or pour ce qui est de l'établissement des lois, ce ne sera pas assez à l'orateur de connaître par ce qui s'est passé dans l'état
où il parle, quelle façon de gouverner, est la meilleure, mais encore il faudra qu'il sache par une exacte lecture de tout ce qui s'est fait chez
les autres peuples, quelles sortes de lois sont plus propres a telles ou à telles sortes de personnes.

D'où il s'ensuit évidemment deux choses. La première: que pour être capable d'établir des lois, les voyages sont utiles,
puisque c'est principalement dans les voyages et en pratiquant plusieurs nations qu'on fait expérience des lois. La seconde:
que pour être en état de persuader dans les assemblées publiques, il faut être versé dans l'Histoire. Or tout cela est
l'ouvrage de la politique seulement, et n'appartient en aucune façon à la rhétorique.

Nous avons donc remarqué quels sont les points principaux que doit connaître l'orateur qui a à délibérer. Parlons, désormais
de ce qu'il doit employer, non seulement afin de pouvoir persuader sur ces mêmes points, ou dissuader, mais encore afin de
le pouvoir faire sur quelque matière qui se présente.

\section{Du souverain bien, et de ses parties}

Il n'y a presque personne, soit en commun soit en particulier, qui dans la vie ne se propose un certain
but. Et pour arriver à ce but, que sans cesse on a en vue, chacun de son côté fait tout ce qu'il peut afin
d'acquérir et d'éviter certaines choses. Or, ce but, en un mot, est ce que nous appelons \emph{souverain
bien}, \emph{félicité}, \emph{souverain bonheur}, et tout ce qui en dépend. Afin donc qu'on en ait quelque
idée, disons en gros ce que c'est que cette félicité ou ce souverain bien, et ce qui en fait partie, puisque
tout ce qu'on emploie, et a persuader, et à dissuader, regarde toujours ou la félicité elle même, ou ce qui
se rapporte à elle, ou qui lui est opposé. Et de fait tout ce qui est capable de nous rendre heureux absolument
ou en partie, ou qui, d'un petit bien, en peut faire naître un plus grand est toujours ce que nous devons nous
proposer de faire, comme nous devons toujours nous abstenir d'entreprendre des choses qui peuvent détruire
notre bonheur, ou l'empêcher, ou nous faire passer à un état contraire. 

Supposons donc que la félicité se rencontre:

\begin{emphpar}
     À mener une vie dont toutes les actions réussissent au contentement de celui qui les fait, sans pourtant
	 s'éloigner en rien de la vertu ni du devoir d'un honnête homme,
\end{emphpar}

Ou encore, 

\begin{emphpar}
     À se voir en tel état qu'on n'ait affaire de rien,
\end{emphpar}

Ou bien,

\begin{emphpar}
     À passer si agréablement ses jours que les plaisirs n'en puissent être troublés,
\end{emphpar}

Ou enfin,

\begin{emphpar}
     À jouir d'une possession si parfaite de toutes choses qu'on soit en puissance également de les conserver
	 dans le besoin et de les acquérir de nouveau si elles étaient perdues.
\end{emphpar}

Car sans doute tout le monde demeure d'accord que le souverain bien consiste: ou dans la possession de quelqu'une
de ces choses, ou de plusieurs ensemble,

Que si la félicité est véritablement ce que nous venons de dire, on doit mettre au nombre de ce qui en fait partie:
la naissance,le crédit et l'amitié des honnêtes gens, les richesses, l'avantage d'avoir des enfants parfaits et en
grand nombre, et enfin la jouissance d'une vieillesse exempte de toute sorte d'incommodités. De plus, il y faudra
ajouter toutes les qualités excellentes du corps, par exemple: la santé, la beauté, la force, la taille, l'adresse à
toutes sortes d'exercices, et encore la gloire et les honneurs, la bonne fortune; en un mot, la vertu et tout ce qui
en dépend, savoir la prudence, la valeur, la tempérance, la justice; car il est certain qu'un homme sera souverainement
content lors qu'il se verra possesseur, et des biens qui se trouvent dans nous-mêmes et que nous possédons en propre,
et de ceux qu'on emprunte d'ailleurs et qui sont hors de nous, puisque après ces deux sortes de biens, il n'en faut
point chercher d'autres. J'appelle biens qu'on trouve dans soi-même tout ce qui sert à l'embellissement de l'âme et à
perfectionner le corps. Et j'appelle biens étrangers et hors de nous la noblesse, les amis, les honneurs et les richesses.
Outre ces avantages, néanmoins, nous croyons encore que pour assurer entièrement le bonheur de notre vie, il est bon
d'avoir de la puissance et d'être favorisé de la fortune.

Examinons en particulier quelle est la nature de toutes ces choses, et premièrement en quoi consiste la noblesse.

\subsection{Les parties qui composent le souverain bien}

La noblesse se peut considérer en deux façons: ou à l'égard de tout un peuple ou d'un particulier seulement. Un peuple
sera remarquable par sa noblesse s'il est originaire du pays qu'il habite, ou du moins fort ancien; si ses fondateurs
ont été illustres, et s'il en est sorti quantité de grands hommes qui aient éclaté par leur sagesse, par leur valeur,
par leur justice et par tous les autres avantages qui donnent de l’émulation. La noblesse d'un particulier peut venir:
ou du côté des hommes, ou du côté des femmes, ou de tous les deux ensemble, surtout si sa naissance est légitime. Et
cette noblesse sera toujours d'autant plus considérable si, de même que nous venons de remarquer touchant les fondateurs
des états, les premiers de sa race ont été illustres pour leur vertu ou leurs grands biens, ou pour quelqu'une des autres
choses qui ont u crédit dans le monde; et non seulement si les premiers de sa race ont été illustres, mais encore si depuis,
on en peut compter beaucoup d'autres dans sa famille, aussi bien parmi les femmes que parmi les hommes; parmi les jeunes
gens et les vieillards qui aient ajouté à cette première gloire.

\bigbreak

Il n'est pas difficile de connaître en quoi consiste ce que nous appelons \emph{être heureux en enfants}. En général, donc,
ce bonheur se rencontre dans une ville ou dans un état s'il y a beaucoup de jeunesse, et qui ait de bonnes qualités, soit
que ces qualités regardent le corps, comme font la taille et la beauté, la force, et l'adresse à toutes sortes d'exercices,
soit qu'elles regardent l’âme, comme la tempérance et la valeur; car à proprement parler, ces deux vertus appartiennent aux
jeunes gens. En particulier, nous appellerons un homme heureux en enfants celui qui en aura un grand nombre, tant de l'un que
de l'autre sexe, et remarquables par toutes les qualités que nous venons de dire. Au reste les qualités qui rendent les femmes
recommandables, premièrement quant au corps, sont la beauté et la taille, en second lieu, pour l’âme et pour l'esprit, nous
recherchons aux femmes, particulièrement, la tempérance, et de plus cet amour du ménage qui ne tient point de la bassesse et
qui n'est pas indigne d'une femme de condition. De quelque façon, donc, que nous considérions la possession des enfants, tant
de l'un que de l'autre sexe, soit que nous la considérions en général ou en particulier, jamais elle ne pourra être heureuse
entièrement si ces enfants, autant les filles que les mâles, n'ont toutes les vertus et toutes les qualités que nous avons
remarquées. Et pour cela, on peut assurer de tous ceux qui ont des filles et des femmes aussi mal élevées que les Lacédémoniens
en ont qu'ils ne sont heureux en enfants qu'à demi.

\bigbreak

Quant aux richesses, ce qui en fait partie est l'argent comptant, la quantité des héritages et des belles terres; les meubles,
les troupeaux, les esclaves, surtout s'ils sont remarquables par la grandeur, par la beauté et par le nombre. Or, non seulement
pour être riche il faudra posséder toutes ces choses, mais encore il faudra que la possession en soit sûre, honnête et profitable
tout ensemble. Une chose est profitable lorsqu'elle est de rapport, et elle est honnête lorsqu'on ne s'en sert que pour le plaisir.
J'appelle possession de rapport celle dont nous tirons du revenu, et je nomme possession pour le plaisir simplement, celle qui n'a
rien de plus considérable que l'usage. Enfin nous possédons en assurance une chose, lorsque nous en jouissons en tel lieu et de
telle sorte que nous pouvons en user comme il nous plaît, et de plus quand la propriété nous en appartient. On possède en propre une
chose lorsqu'on la peut aliéner. J'appelle aliéner la vendre ou la donner. Après tout, il ne faut pas penser que la qualité de riche
dépende plus de la possession des richesses que de leur usage, car tant s'en faut que cela soit, que même se servir de son bien est
proprement ce que nous appelons être Riche.

\bigbreak

La gloire et la réputation consistent à passer pour homme de bien dans l'esprit de tous les hommes; et encore à être cru possesseur
d'un avantage ou que tout le monde souhaite passionnément, ou du moins les plus honnêtes gens, ou les personnes d'esprit. 

L'honneur est un témoignage d'estime qu'on rend à ceux qui sont bienfaisants. De là vient qu'on honore principalement les personnes
qui ont du bien; et quoi qu'il fût juste de ne porter de l'honneur qu'à ces gens-là, on ne laisse pas d'honorer encore ceux qui sont
en puissance de bien faire. Au reste, le bienfait regarde toujours ou la vie et tout ce qui peut être cause de sa conservation, ou les
richesses, ou enfin quelqu'un des autres avantages dont l'acquisition est difficile à faire, soit absolument, soit en certain lieu ou
en certain temps. Et c'est aussi pourquoi souvent nous voyons rendre beaucoup d'honneur et faire de grandes soumissions à des personnes
pour de très petites choses en apparence, seulement à cause que l'occasion ou la difficulté de les faire les avaient rendus considérables.
Les parties de l'honneur, ou les manières différentes d'honorer sont: les sacrifices, les inscriptions publiques soit en vers ou en prose,
les récompenses, les lieux consacrés, les préséances, les tombeaux, les statues, les pensions qu'on a du public; à quoi l'on peut ajouter
ce que pratiquent les nations étrangères quand elles veulent honorer quelqu'un, par exemple se prosterner contre terre ou se retirer d'un
chemin quand on passe. Il faut encore mettre les présents au nombre des choses qui sont en honneur. De vrai, le présent est de telle nature
qu'en même temps, il est: et la donation d'une chose et une marque d'estime; aussi les avares et les ambitieux en sont-ils grands amateurs,
à cause qu'ils y trouvent ce qu'ils cherchent. Les avares y rencontrent l'acquisition, et les ambitieux l'honneur, qui est ce que tous deux
demandent. 

\bigbreak

La santé est proprement la vertu du corps. Il faut néanmoins la posséder de manière que nous puissions faire toutes sortes
de fonctions sans en être malades; car il y en a beaucoup qui jouissent de la santé comme faisait Hérodicos, qu'on ne peut
pas dire être heureux en cet état, à cause qu'il faut qu'ils s'abstiennent de tout qui rend notre vie commode et agréable,
ou de la plus grande partie.

\bigbreak

Pour la beauté, elle est différente à raison des âges différents. La beauté d'un jeune homme est d'avoir le corps propre
à toutes sortes d'exercices, soit à la course et aux autres actions qui demandent de la force. Il faut encore qu'il soit
agréable à voir, et si agréable même, qu'on ne puisse se lasser de le regarder. Pour cette raison, les athlètes propres à
la course et à se battre sont très beaux. La beauté d'un homme fait est de pouvoir supporter toutes les fatigues de la guerre
et d'avoir je ne sais quoi dans le visage qui le rende agréable à voir et redoutable tout ensemble. Enfin, celle d'un
vieillard consiste à pouvoir faire toutes les fonctions nécessaires, et cela sans se plaindre, comme ne sentant aucune des
incommodités qui affligent d'ordinaire la vieillesse.

\bigbreak

La force consiste à tourner et manier quelqu'un comme on veut, ce qui se fait de cinq façons: ou en le tirant, ou en le
poussant, ou en l'élevant, ou en le terrassant, ou en l’étreignant. Car on ne peut pas dire qu'un homme soit fort, s'il
ne fait tout ceci, ou une partie.

\bigbreak

À l'égard de la belle taille, c'est quand on surpasse presque tous les autres, ou en hauteur, ou en largeur, ou en
grosseur, en sorte néanmoins que cet excès ne rende pas le corps plus pesant ni plus tardif dans tous ses mouvements.

\bigbreak

Pour réussir au métier d'athlète, qui comprend trois sortes d'exercices, savoir la lutte, la course et le combat des poings.
Le corps doit avoir ces trois avantages: la taille, la force et l'agilité. Car tout homme qui est agile est fort.

Au reste, quiconque peut jeter les jambes d'une certaine manière, les avancer loin et promptement, est propre à la course.

Celui qui peut étreindre son homme et le tenir ferme, est né pour la lutte.

Enfin, pouvoir, à force de poings, repousser un adversaire et le faire toujours reculer, c'est ce qu'il faut au combat des
poings.

Entre les athlètes, quelques-uns réussissent aux poings et à la lutte tout ensemble, et d'autres sont adroits à toutes ces
trois sortes d'exercices.

\bigbreak

La vieillesse commode est celle qui vient tard et qui ne fait rien souffrir. Pour en jouir, donc, il ne faudra pas vieillir
de trop bonne heure. Aussi ne suffira t'il pas de vieillir tard si en même temps on n'est exempt de toutes sortes
d'incommodités. Or, cet avantage ne dépendra pas seulement des qualités excellentes du corps, mais encore de la bonne
fortune. Car, qu'un homme soit sujet aux maladies et de faible complexion, le moyen qu'il ne souffre jamais? Et s'il doit
être incommodé, comment est-il possible que sans un grand bonheur, il puisse vivre longtemps en cet état. J'avoue
véritablement que, sans la santé et la bonne constitution, on ne laisse pas de vivre quelquefois assez longtemps puisque,
tous les jours, il se voit des gens privés de tous les avantages du corps arriver à de longues années. Mais ce n'est pas
ici qu'il faut donner une exacte connaissance de cette matière.

\bigbreak

Pour ce qui est du crédit et d'avoir l'amitié des honnêtes gens, ceci sera facile à connaître quand nous aurons déclaré ce
que nous entendons par le mot d'ami. Tout homme, donc, qui tâchera par toutes sortes de moyens de procurer à un autre ce
qu'il juge lui être avantageux, sans autre motif que de le vouloir obliger et seulement parce qu'il l'aime, c'est là
proprement ce que nous appelons être ami. Or, quiconque aura beaucoup de personnes disposées de cette sorte à son égard,
se pourra vanter d'avoir du crédit et beaucoup d'amis. Et si ces mêmes personnes ont du mérite et de la vertu, pour lors,
il aura l'amitié des honnêtes gens.

\bigbreak

On appelle bonne fortune quand il arrive à une personne: ou qu'il lui est arrivé tous les biens et les avantages dont la
fortune est la cause ordinairement, ou du moins quand de tous ces biens, il lui en est arrivé la meilleure partie et les
plus considérables. La fortune, au reste, peut être cause quelquefois des mêmes biens et nous procurer les mêmes avantages
que ceux que notre adresse et les arts nous procurent, quoi que d'ordinaire, la plupart de ceux qui viennent d'elle ne
soient nullement au pouvoir des arts, comme sont les biens de la nature. Quelquefois encore, elle est cause de certains
biens qui arrivent extraordinairement et en quelque façon contre le dessein de la nature même. Par exemple, la fortune est
quelquefois cause de la santé qui est un bien dépendant de la médecine, et cette même fortune, bien souvent, est cause de
la beauté et de la taille, qui sont des avantages purement dépendants de la nature. Mais en général, on peut nommer biens
de la fortune tous ceux qui sont sujets à l'envie. Outre ces biens, la fortune en donne encore d'autres quelquefois contre
toute sorte de raison et d'apparence, comme il arrive, quand entre plusieurs frères qui sont très laids, il s'en rencontre
un parfaitement beau; ou lorsque de plusieurs qui cherchent un trésor, il n'y en a qu'un qui le trouve; ou encore quand
une flèche qui a été tirée épargne celui-ci et en blesse un autre tout contre; ou enfin, lorsqu'une personne qui avait
accoutumé d'aller seule en certain lieu s'abstient d'y aller dans le temps que plusieurs qui étaient allés pour la première
fois y périssent. Car il semble que toutes ces choses-là soient de purs effets de la bonne fortune.

\bigbreak

Touchant la vertu, parce qu'elle regarde la louange particulièrement, nous remettons à faire savoir ce que c'est quand nous serons au genre démonstratif.


\section{De la fin du genre délibératif}
\subsection{Avec les lieux qui servent à prouver qu'une chose est bonne et utile}


Présentement, l'on voit quelles sont les choses à quoi il faut avoir égard lorsqu'on a à persuader, soit que ces choses-là
soient arrivées, ou aient à arriver. Et de même en est-il pour dissuader, puisqu'il n'y a qu'à prendre le contraire. 

Mais parce que celui qui délibère a toujours pour but ce qui est utile; d'ailleurs que jamais personne ne délibère de la
fin, mais seulement des moyens pour y arriver, et que ces moyens-là, c'est ce qui est utile touchant le dessein qu'on a;
enfin parce que ce qui est utile est toujours un bien et un avantage. Pour cela il faut que nous donnions ici quelques
notions du bien en général, et de ce qui est utile afin d'en tirer des propositions.

Supposons donc: 

\begin{emphpar}
	Que le bien est une chose souhaitable à cause d'elle-même;

	Ou, qui, pour l'avoir, oblige à en rechercher d'autres auxquelles on ne penserait jamais;

	Ou généralement, que c'est ce que souhaite tout ce qui est au monde, ou du moins tout ce qui a des sentiments ou de
	la raison; et même ce que souhaiterait tout ce qui est privé de raison s'il en avait.
\end{emphpar}

Disons encore:

\begin{emphpar}
	Que le bien est tout ce que la raison nous représente comme tel;
\end{emphpar}

Et encore:

\begin{emphpar}
	Que tout ce qu'elle nous représente comme un bien en chaque rencontre particulière, cela même nous est toujours
	avantageux\footnote{Chaque fois qu'il m'arrive une tuile, faut que je picole! Sans quoi je ne suis pas heureux.}.
\end{emphpar}

Ajoutons:

\begin{emphpar}
	Que le bien est ce qui, par sa présence, fait qu'on se trouve tout autre et si content qu'on ne souhaite rien au
	delà\footnote{Tout se passe comme si tu avais perdu toute forme de désir. Tu étais beau, avant, mais tu ressembles
	à un robot, programmé pour jouir.};

	Ou, ce qui tout seul nous suffit;
\end{emphpar}

Et même:

\begin{emphpar}
	Que c'est ce qui peut être cause de tous ces biens que nous venons de dire;

	Ou qui les peut conserver;

	Ou qui en est toujours suivi.
\end{emphpar}

\bigbreak

Supposons enfin:

\begin{emphpar}
	Que le bien est tout ce qui peut éloigner ou détruire ce qui est contraire aux avantages que nous avons remarqués.
\end{emphpar}

En passant, nous observerons qu'une chose peut être suivie d'une autre en deux manières: ou en même temps, ou quelque temps
après. Par exemple l'étude est suivie de la science quelque temps après, parce que pour être savant, il faut auparavant
avoir étudié. Et la vie suit toujours la santé en même temps, puisqu'il n'est pas possible qu'on jouisse de la santé et
qu'en même temps, on soit privé de la vie.

Nous observerons aussi qu'une chose peut être cause d'une autre en trois façons: Ou \emph{formellement et par elle-même},
ainsi la santé est toujours cause qu'une personne est saine; ou \emph{en réparant ce que cette chose perd}, ainsi les
aliments sont cause encore de la santé; ou bien enfin \emph{en la conservant}, et de cette sorte, l'exercice est cause de
la santé, parce que d'ordinaire, la santé en dépend.

\bigbreak

Suppose donc que le bien soit véritablement ce que nous venons de dire. Il sera nécessaire de tirer ces conséquences:
Premièrement, que l'acquisition d'un bien et la délivrance d'un mal seront des choses avantageuses puisque d'un côté,
acquérant un bien, on n'aura pas en même temps le mal qui lui est contraire; et d'un autre côté, étant délivré d'un
mal, on aura après le bien qui le suit. 

\bigbreak

En second lieu:

\begin{lieu}
	Que l'échange d'un petit bien pour un plus grand, ou l'échange d'un grand mal pour un plus petit seront encore de
	grands avantages\footnote{Notez qu'il m'est impossible de me plier à la règle du jeu avec un truisme. Et pourtant,
	beaucoup d'usages de ce lieu, ainsi que d'autres truismes, seront trompeurs. En effet, la première prémisse
	impliquera que la comparaison est certaine, ou presque, ce qui n'arrive pas souvent. Ce qui arrive souvent, en
	revanche, c'est qu'on croie à cette certitude sur la base des données passées. Pensez à l’œuvre de Nassim Taleb,
	ou plus simplement, aux vendeurs de miracles: <<~Si vous suivez ma formation à 1000 euros, votre épargne sera
	multipliée par dix en dix ans! Réfléchissez-y: il suffit que vous possédiez 1500 euros pour y gagner!~>>};
\end{lieu}

Puisque d'une part, il sera vrai d'assurer qu'autant que ce grand bien aura d'avantage sur le petit, autant de bien
aura-t-on acquis qu'on n'avait pas; et d'autre part, qu'autant que ce petit mal sera moindre que le plus grand,
d'autant de mal sera-t-on délivré qu'on n'aura plus.

\bigbreak

On pourra aussi inférer:

\begin{lieu}
	Que généralement, toutes les vertus seront des biens\footnote{Le Galérien: \begin{emphpar}\normalfont
	Hélas! Que ne me suis-je épargné ces travaux pénibles en présence des rats et de la peste?
	Eussé-je abjuré la religion de mes frères, quels honneurs n'eussé-je pas reçu?\end{emphpar}};
\end{lieu}

Puisque ceux qui les possèdent se trouvent contents en cet état et que, d'ailleurs, elles sont cause qu'il leur arrive
ensuite beaucoup d'autres avantages; et même qu'elles les rendent capables de faire du bien aux autres. Mais nous
parlerons de cette matière à part en un autre endroit, où il sera traité de chaque vertu en particulier, et de sa
différence.

\bigbreak

De plus on soutiendra:

\begin{lieu}
	Que le plaisir est un bien;
\end{lieu}

Parce que naturellement, tous les animaux le recherchent.

\bigbreak

Et par la même raison:

\begin{lieu}
	Toutes les belles choses, et qui sont agréables;
\end{lieu}

Car tout ce qui est agréable nous apporte du plaisir. Quant aux choses qui sont belles, il faut remarquer que les unes sont
agréables simplement, et les autres honnêtes et souhaitables pour l'amour d'elles-mêmes. 

\bigbreak

Enfin, pour ne rien oublier, et pour nommer tous les biens les uns après les autres, il faudra mettre encore au nombre des
biens: premièrement,

\begin{lieu}
	Le souverain bien;
\end{lieu}

Vu qu'il est souhaitable à cause de lui-même, qu'il peut satisfaire pleinement et que pour l'acquérir, nous n'épargnons rien
de tout ce qui est en notre pouvoir.

\bigbreak

Secondement, il y faudra mettre:

\begin{lieu}
	La justice, la valeur, la tempérance, la grandeur de courage, la magnificence et pareilles habitudes;
\end{lieu}

À cause que ce sont là les vertus de l'âme. 

\bigbreak

Il y faudra encore ajouter:

\begin{lieu}
	La santé la beauté, et telles choses semblables;
\end{lieu}

Puisque non seulement ce sont les vertus du corps et les qualités qui le perfectionnent, mais encore parce qu'elles sont capables
de nous faire entreprendre beaucoup de choses, et même de les exécuter. Par exemple, la santé est un bien, parce qu'elle est la
source de tous les plaisirs et de la vie même. Aussi est-ce principalement ce qui la fait passer pour un bien excellent, à cause
qu'elle est en même temps le principe des deux seules choses que le vulgaire estime le plus au monde, qui est de vivre, et de vivre
avec plaisir, 

\bigbreak

\begin{lieu}
	Les richesses encore sont à mettre au rang des biens;
\end{lieu}

Puisqu'il y a de la vertu à s'en bien servir et que par leur moyen, on peut faire bien des choses.

\bigbreak

Pareillement:

\begin{lieu}
	Les amis et l'amitié;
\end{lieu}

Car un ami est toujours souhaitable à cause de lui-même, joint qu'il peut beaucoup servir.

\bigbreak

\begin{lieu}
	L'honneur encore et la gloire sont des biens;
\end{lieu}

Car outre qu'il est agréable de les posséder et qu'ils nous peuvent servir beaucoup, c'est qu'il arrive d'ordinaire que les mêmes choses
qui nous font rendre de l'honneur se trouvent véritablement en nous.

\bigbreak

\begin{lieu}
	Savoir parler et agir sont encore des biens;
\end{lieu}

Puisque ces choses-là peuvent nous procurer de très grands avantages. 

\bigbreak

Il faut dire le même:

\begin{lieu}
	Du bel esprit, de la mémoire, de la docilité, de la vivacité et de telles autres qualités;
\end{lieu}

Car tout cela peut beaucoup contribuer à notre fortune et nous mettre en état de faire de grandes choses. 

\bigbreak

On doit aussi mettre au rang des biens:

\begin{lieu}
	Toutes les sciences et les arts, comme aussi la vie;
\end{lieu}

Puisque quand nous n'aurions autre avantage que de vivre, il ne faudrait pas laisser de souhaiter la vie à cause d'elle-même.

\bigbreak

Enfin, nous devons tenir pour bien:

\begin{lieu}
	Tout ce qui est juste;
\end{lieu}

Attendu qu'il regarde l'utilité publique.

\subsection{Biens douteux ou controversés et pour les faire valoir}

Quant aux autres choses à qui la qualité de bien est contestée, la preuve s'en pourra faire ainsi.

\bigbreak

Premièrement:

\begin{lieu}
	Que tout ce qui a pour son contraire un mal est un bien\footnote{La déflation est un appauvrissement, mais je
	connais un truc: Faites chauffer la planche à billets!};
\end{lieu}

En second lieu:

\begin{lieu}
	Tout ce qui a pour son contraire une chose dont les ennemis tirent de l'avantage;
\end{lieu}

Par exemple s'il est utile aux ennemis que nous soyons poltrons, sans doute la valeur nous sera fort avantageuse.

Et généralement enfin:

\begin{lieu}
	Tout ce qui sera contraire aux choses que les ennemis souhaitent, ou qui leur donnent de la joie, apparemment,
	nous doit être utile.
\end{lieu}

De là vient que Nestor, dans \textsc{Homère}, voulant réconcilier Achille et Agamemnon de qui la division allait ruiner
l'entreprise des grecs devant Troie, allègue d'abord comme un moyen très capable de les toucher:

\begin{emphpar}
	Quelle joie a Priam, s'il apprend ce désordre!
\end{emphpar}

Il faut pourtant remarquer que ceci n'est pas toujours vrai, mais seulement pour l'ordinaire, puisque enfin, rien
n'empêche que quelquefois une même chose ne soit utile à deux ennemis en même temps, d'où est venu le proverbe que:
\emph{souvent, les maux portent à la réconciliation et rendent les hommes amis}; ce qui se doit entendre lorsque la
même chose est dommageable également aux uns et aux autres.

\bigbreak

De plus, il y aura lieu de soutenir:

\begin{lieu}
	Que tout ce qui n'est point dans l’excès est un bien, puisque tout ce qui est excessif et plus grand qu'il ne
	faut est un mal;
\end{lieu}

\bigbreak

Comme encore:

\begin{lieu}
	Tout ce qui nous aura fait prendre beaucoup de peine et obligé à une grande dépense\footnote{25 décembre 2009,
	au repas de noël: <<~Garde tes actions de la Société Générale. Maintenant que Kerviel est en prison, ça ne peut
	que remonter!~>> Depuis, le même conseil revient à chaque noël et s'en trouve renforcé.}.
\end{lieu}

Et certainement pourrait-on dire que ces choses-là n'eussent pas toutes les apparences d'un véritable bien, puisque en
effet, elles seront le but et la fin de toute cette dépense et de tous ces grands travaux. Car ce qui tient lieu de fin
est toujours un bien. Aussi est-ce la raison qui oblige \textsc{Homère} de faire dire à Junon, lorsque les Grecs sont
prêts de s'en retourner et de lever honteusement le siège de devant Troie:

\begin{emphpar}
	Quoi donc? De leur retour, les Grecs trop désireux,
	
	Oublieront en fuyant tant d'exploits généreux?

	Les Troyens à leur honte auront donc la victoire,
	
	Et Priam pour jamais se verra plein de gloire?
\end{emphpar}

Il fait dire encore à Ulysse en un autre endroit, parlant à l'armée des Grecs pour les faire opiniâtrer à ce siège:

\begin{emphpar}
	Quelle honte, guerriers, à tant de combattants,

	De n'être pas vainqueurs après un si long temps,
	
	Et de s'en retourner sans honneur et sans gloire!
\end{emphpar}

C'est encore ce qui a donné lieu au Proverbe:

\begin{emphpar}
	Casser sa cruche à la porte. 
\end{emphpar}

\bigbreak

On pourra soutenir de même:

\begin{lieu}
	Que ce que quantité de personnes souhaitent passionnément, ou qui mérite en apparence qu'on en dispute la
	possession et qu'on se batte pour l'avoir, est un bien\footnote{D'où le truc très connu: écrire <<~plus d'un
	million d'exemplaires vendus!~>> en couverture\dots{} Et sinon, le désir mimétique de René \textsc{Girard}, on
	en parle?}.
\end{lieu}

Cette proposition doit passer pour certaine, suivant une des définitions du bien que nous avons données, vu qu'alors
il a été dit que le bien était \emph{une chose que généralement tous les hommes souhaitaient}. Il est vrai que cette
proposition est conçue en des termes moins universels, mais quand on dit un grand nombre, ou la plupart, il semble en
quelque façon qu'on veuille dire: tout le monde.

\bigbreak

Ce raisonnement encore sera plausible:

\begin{lieu}
	Que tout ce qui est louable est un bien;
\end{lieu}

À cause que personne ne se met en peine de louer une chose qui n'a rien de bon en soi. 

\bigbreak

Toute action encore passera pour bonne:

\begin{lieu}
	Qui tire des louanges de la bouche même des ennemis et des plus méchants;
\end{lieu}

Car qui pourrait dire alors que cette action ne fût pas dans une approbation générale, quand ceux qui ont le plus d'intérêt
d'en dire du mal pour leur avoir été préjudiciable eux-mêmes en disent du bien? Il est certain que jamais ils n'en auraient
fait cette estime si la vérité ne les y avait forcés. Ce fondement est si vrai, que c'est par cette raison qu'on tient pour
méchants ceux qui sont blâmés de leurs amis, et tout au contraire pour honnêtes gens et pour vertueux ceux qui obligent même
leurs propres ennemis à les louer. Et c'est de cette manière que \textsc{Simonide} loua un jour les Corinthiens, dont néanmoins
ils se tinrent fort offensés. C'est quand il dit:

\begin{emphpar}
	Et quoique tu sois Grecque, ô fameuse Corinthe,
	
	Ce n'est point contre toi qu'Illion fait sa plainte.
\end{emphpar}

\bigbreak

On pourra encore proposer comme excellent:

\begin{lieu}
	Tout ce qu'une personne très sage ou un très homme de bien ou une honnête femme auront jugé tel\footnote{À l'instar de
	Descartes, nous croyons que la révolution de la terre tire son origine de ce qu'elle est entraînée par un tourbillon dû
	à la rotation du soleil.};
\end{lieu}

Ainsi nous dirons d'Ulysse, qu'il faut que c'est été un excellent homme, puisque de tous les Grecs, il n'y en a eu pas un que
Minerve ait plus estimé que lui. Ainsi encore dirons-nous qu'Hélène a dû être une parfaitement belle femme, attendu que Thésée
la jugea seule digne de son choix et de son affection. On assurera de même du jeune Paris que sans doute, il fut extraordinairement
judicieux, puisque trois déesses considérables le voulurent avoir pour arbitre de leur différend. On maintiendra aussi qu'Achille
a été un très vaillant capitaine, à cause que le divin \textsc{Homère} l'a fait le premier héros de son poème.

\bigbreak

On mettra encore de ce rang:

\begin{lieu}
	Tout ce que d'ordinaire, on préfère aux autres choses.
\end{lieu}

Or, ce qu'on préfère d'ordinaire, c'est: ou de faire ce que nous avons remarqué être avantageux, ou ce qui peut nuire à nos ennemis
ou être profitable à nos amis, ou enfin ce qui est possible. Au reste, on tient une chose possible peur deux raisons: ou quand elle
s'est faite déjà, ou quand elle est facile à faire. Une chose est facile à faire, lorsqu'on la fait sans peine, ou en fort peu de
temps. Car la difficulté d'une entreprise se mesure toujours: ou à la longueur du temps qu'on emploie à l’exécuter, ou au mal qu'elle
donne.

\bigbreak

Il y aura lieu encore de soutenir:

\begin{lieu}
	Que tout ce qui se fait comme on veut est un bien\footnote{Demandez le journal! Justice est faite contre Batman! Le criminel
	a été tué par le Joker en légitime défense!}.
\end{lieu}

Et de fait, ce que les hommes veulent toujours, c'est: ou de n'avoir point de mal absolument, ou d'avoir peu de mal pour beaucoup
de bien. Ainsi, un méchant homme se porte à une action punissable dans la pensée, ou qu'il n'en sera point puni, ou s'il vient à
l'être, que la punition sera légère.

\bigbreak

Cet autre raisonnement encore pourra servir:

\begin{lieu}
	Que les choses que nous posséderons en propre ou que personne n'aura que nous, ou qui excelleront par-dessus toutes les autres
	seront bonnes;
\end{lieu}

À cause qu'il y aura plus d'honneur à les posséder.

\bigbreak

Comme aussi:

\begin{lieu}
	Tout ce qui nous conviendra particulièrement\footnote{De père en fils, nous sommes fossoyeurs depuis sept générations. Et tu
	voudrais être un clown?};
\end{lieu}

Par exemple, tout ce qui nous sera bienséant, ou à cause de notre naissance, ou à cause de nos grands emplois.

\bigbreak

Pareillement:

\begin{lieu}
	Toutes les choses que nous croirons nous manquer, pour petites qu'elles soient;
\end{lieu}

Puisqu'on ne se met pas moins en peine d'acquérir celles-là que les autres qui sont d'une plus grande importance.

\bigbreak

On fera passer encore pour de bonnes choses:

\begin{lieu}
	Celles dont on peut venir à bout aisément\footnote{Vous voulez vous débarrasser de ce pays d'emmerdeur, Monsieur le
	président? Vous avez les codes et le bouton est là.};
\end{lieu}

Car non seulement elles sont possibles, mais encore faciles à faire. Au reste, nous croyons pouvoir aisément, venir à
bout d'une chose, lorsque tout le monde l'a déjà faite, ou quantité de personnes, du moins nos pareils, ou ceux qui ne
nous valent pas.

\bigbreak

Une chose encore paraîtra avantageuse et à entreprendre:

\begin{lieu}
	Qui sera agréable à nos amis ou fera dépit à nos ennemis.
\end{lieu}

\bigbreak

Et encore: 

\begin{lieu}
	Tout ce que les personnes d'un haut mérite et qu'on estime infiniment au-dessus des autres d’ordinaire se
	proposent de faire.
\end{lieu}

\bigbreak

De plus:

\begin{lieu}
	Toutes les chose pour lesquelles il semble qu'on soit né, ou dont on a une très grande expérience;
\end{lieu}

Puisque c'est d'ordinaire en de telles rencontres que les hommes se promettent plus de succès. 

\bigbreak

Nous pourrons encore faire valoir:

\begin{lieu}
	Tout ce que les personnes de néant et de basse condition ne peuvent faire;
\end{lieu}

Vu qu'alors, il y aura d'autant plus de gloire à entreprendre ces choses qu'elles seront hors du commun et au-dessus
de la portée des hommes ordinaires. 

\bigbreak

Enfin, l'on fera passer pour bon:

\begin{lieu}
	Tout ce qu'ordinairement, on souhaite;
\end{lieu}

Car outre qu'on y trouve du plaisir, c'est que même on ne croit pas qu'il y ait rien de meilleur.

\bigbreak

Mais surtout, une chose sera aisée à proposer comme excellente à une personne:

\begin{lieu}
	Si c'est particulièrement sa passion et ce qu'elle souhaite le plus au monde;
\end{lieu}

Par exemple comme est la victoire a un ambitieux, l'argent à un avare, et ainsi des autres.

\bigbreak

Ce sont donc là les lieux qui doivent fournir des propositions quand on aura à montrer qu'une chose est bonne et utile.


\section{Lieux pour connaître quand un bien est plus grand ou plus petit qu'un autre}

Dein Syria per speciosam interpatet diffusa planitiem. hanc nobilitat Antiochia, mundo cognita civitas, cui non certaverit alia 
advecticiis ita adfluere copiis et internis, et Laodicia et Apamia itidemque Seleucia iam inde a primis auspiciis florentissimae.

Ipsam vero urbem Byzantiorum fuisse refertissimam atque ornatissimam signis quis ignorat? Quae illi, exhausti sumptibus bellisque 
maximis, cum omnis Mithridaticos impetus totumque Pontum armatum affervescentem in Asiam atque erumpentem, ore repulsum 
et cervicibus interclusum suis sustinerent, tum, inquam, Byzantii et postea signa illa et reliqua urbis ornanemta sanctissime 
custodita tenuerunt.

Qui cum venisset ob haec festinatis itineribus Antiochiam, praestrictis palatii ianuis, contempto Caesare, quem videri decuerat, ad 
praetorium cum pompa sollemni perrexit morbosque diu causatus nec regiam introiit nec processit in publicum, sed abditus multa in 
eius moliebatur exitium addens quaedam relationibus supervacua, quas subinde dimittebat ad principem.

Atque, ut Tullius ait, ut etiam ferae fame monitae plerumque ad eum locum ubi aliquando pastae sunt revertuntur, ita homines 
instar turbinis degressi montibus impeditis et arduis loca petivere mari confinia, per quae viis latebrosis sese convallibusque 
occultantes cum appeterent noctes luna etiam tum cornuta ideoque nondum solido splendore fulgente nauticos observabant quos 
cum in somnum sentirent effusos per ancoralia, quadrupedo gradu repentes seseque suspensis passibus iniectantes in scaphas 
eisdem sensim nihil opinantibus adsistebant et incendente aviditate saevitiam ne cedentium quidem ulli parcendo obtruncatis 
omnibus merces opimas velut viles nullis repugnantibus avertebant. haecque non diu sunt perpetrata.

Erat autem diritatis eius hoc quoque indicium nec obscurum nec latens, quod ludicris cruentis delectabatur et in circo sex vel 
septem aliquotiens vetitis certaminibus pugilum vicissim se concidentium perfusorumque sanguine specie ut lucratus ingentia 
laetabatur.

Superatis Tauri montis verticibus qui ad solis ortum sublimius attolluntur, Cilicia spatiis porrigitur late distentis dives bonis omnibus 
terra, eiusque lateri dextro adnexa Isauria, pari sorte uberi palmite viget et frugibus minutis, quam mediam navigabile flumen 
Calycadnus interscindit.

Quam quidem partem accusationis admiratus sum et moleste tuli potissimum esse Atratino datam. Neque enim decebat neque 
aetas illa postulabat neque, id quod animadvertere poteratis, pudor patiebatur optimi adulescentis in tali illum oratione versari. 
Vellem aliquis ex vobis robustioribus hunc male dicendi locum suscepisset; aliquanto liberius et fortius et magis more nostro 
refutaremus istam male dicendi licentiam. Tecum, Atratine, agam lenius, quod et pudor tuus moderatur orationi meae et meum 
erga te parentemque tuum beneficium tueri debeo.

Nemo quaeso miretur, si post exsudatos labores itinerum longos congestosque adfatim commeatus fiducia vestri ductante 
barbaricos pagos adventans velut mutato repente consilio ad placidiora deverti.

Ob haec et huius modi multa, quae cernebantur in paucis, omnibus timeri sunt coepta. et ne tot malis dissimulatis paulatimque 
serpentibus acervi crescerent aerumnarum, nobilitatis decreto legati mittuntur: Praetextatus ex urbi praefecto et ex vicario 
Venustus et ex consulari Minervius oraturi, ne delictis supplicia sint grandiora, neve senator quisquam inusitato et inlicito more 
tormentis exponeretur.

Eodem tempore etiam Hymetii praeclarae indolis viri negotium est actitatum, cuius hunc novimus esse textum. cum Africam pro 
consule regeret Carthaginiensibus victus inopia iam lassatis, ex horreis Romano populo destinatis frumentum dedit, pauloque 
postea cum provenisset segetum copia, integre sine ulla restituit mora.


\section{De l'autorité souveraine, et de chaque sorte d'état en particulier}


Après tout, le plus excellent moyen et le plus fort pour persuader et parler avec succès dans
une assemblée publique où l'on délibère, c'est de connaître toutes les formes de gouvernement
qu'il y a, les mœurs de chacune, et encore de savoir distinguer leurs lois, leurs coutumes et
tout ce qui est utile à un état. Car les hommes ont cela  qu'ils se laissent aller à leur intérêt
quand on leur propose des choses qui doivent leur apporter de l'utilité. Or est-il que dans un état,
tout ce qui sert à le maintenir est ce qui est le plus utile. Faisons donc connaître, en peu de mots,
quelle est la nature de chaque état en particulier, et premièrement en quoi consiste l'autorité
souveraine. 

\bigbreak

L'autorité, ou la puissance, souveraine n'est autre chose que ce qu’établissent et ordonnent, dans un
état, ceux qui y commandent et qui en ont la conduite. Cette autorité, au reste, se divise en autant
d'espèces qu'il y a de formes de gouvernement, car autant qu'il y a de gouvernements différents, autant
y a-t-il de souverainetés différentes.

\bigbreak

Touchant les formes du gouvernement, il s'en trouve quatre:

\begin{emphpar}
	La démocratie;

	L'oligarchie;

	L'aristocratie;

	La monarchie;
\end{emphpar}

De manière que ce qui commande dans ces états et qui a l'autorité en main y doit être considéré: ou comme
partie simplement, ou comme le tout. C'est à dire que cette autorité souveraine est: ou partagée en
plusieurs, ou renfermée toute entière en une seule personne.

La démocratie, ou l'état populaire, est une forme de gouvernement où les charges se donnent au sort.

L'oligarchie, ou le gouvernement de peu de personnes, est un état où celui qui possède davantage a le plus
d'autorité.

L'aristocratie est une forme de gouvernement où commandent ceux qui ont eu une meilleure éducation. Par
éducation, j'entends cette instruction et ces mœurs que les lois d'un tel état prescrivent. Car il faut
savoir que dans l'aristocratie, il n'y a que ceux qui ont été parfaits observateurs des lois qui montent
aux charges et qui prennent le maniement des affaires. Et parce que des personnes qui vivent ainsi paraissent
très honnêtes gens, cette forme de gouvernement a emprunté son nom de là, car le mot d'\emph{aristocratie}
proprement veut dire un état où les plus honnêtes gens ont l'autorité.

Il est aisé de connaître ce que c'est que la monarchie par le nom qu'elle porte, puisqu'il marque un état où
un seul homme commande à tous les autres. Il y a pourtant cette différence à faire que lorsque celui qui
gouverne observe quelque ordre, on l'appelle royauté, et tyrannie, au contraire, quand celui qui commande
gouverne à sa fantaisie sans observer ni règles ni lois.

Pour être capable encore de persuader dans uns assemblée publique, il ne faudra pas ignorer quelle fin se
propose en particulier chacune de ces formes de gouvernement; puisque tout ce qui se fait dans un état est
toujours rapporté au but et à la fin que cet état se propose.

La fin que se propose la démocratie, c'est la liberté. 

L'oligarchie se propose les richesses. 

L'aristocratie, la bonne éducation et l'exacte observation des lois. 

Et la tyrannie a pour but d’entretenir des gardes pour la sûreté de celui qui commande.

Pour persuader, donc, dans les Assemblées où l'on délibère, il faudra savoir distinguer les lois, les coutumes,
et tout ce qui est utile et qui se rapporte à la fin que se propose chaque état, puisqu'on n'entreprend jamais
rien dans quelque état que ce soit qu'on ne le rapporte toujours au but et à la fin particulière que cet état
se propose.

\bigbreak

Mais parce que l'orateur ne persuade pas seulement quand il démontre et prouve son sujet, mais encore lorsqu'il
parle de sorte qu'on peut juger de ses mœurs par son discours --- et de fait, souvent, il arrive que nous n'ajoutons
foi aux paroles d'un homme qu'a cause qu'il nous parait tel en particulier, je veux dire honnête homme, ou affectionné,
ou tous les deux ensemble --- pour cette raison, il sera encore à propos de savoir quelles mœurs conviennent à chaque
forme du gouvernement, vu qu'il n'est rien de plus puissant pour persuader que de faire paraître en sa personne des mœurs
conformes à celles de l'état où on parle. Ainsi, ce sont ces mœurs-là même qu'il faudra prendre pour modèle, et ne point
recourir ailleurs. Or ce qui les donnera à connaître, est la façon d'agir qu'affecte chaque état dans tout ce qu'il
entreprend, et ce choix particulier auquel il se détermine; car ceci se rapporte toujours au but et à la fin qu'ils se
proposent tous.

\bigbreak

A l’égard donc des choses que doit connaître l'orateur qui a à délibérer, soit qu'elles ne soient pas encore arrivées
ou le soient déjà; de plus, pour ce qui est des propositions dont, il se doit servir quand il aura à montrer que tel ou
tel moyen qu'il propose est utile et avantageux; enfin, pour ce qui regarde les mœurs, les lois, les coutumes de chaque
forme de gouvernement, nous en avons parlé autant qu'il était à propos de le faire à présent, puisque cette matière a
déjà été examinée ailleurs et traitée à fond dans nos livres de la \emph{Politique}.


\chapter{Le genre démonstratif}
\section{De la vertu en général et en particulier}
\subsection{Avec les lieux et les adresses qui regardent la louange et le blâme}

Parlons maintenant du vice et de la, vertu, de ce qui est honnête et déshonnête, puisque c'est le but que se
proposent ceux qui ont à louer ou à blâmer quelqu'un. Au reste, en traitant ces matières, il se rencontrera
qu'en même temps, nous ferons connaître les choses dont il se faut servir pour se mettre bien dans l'esprit
de l'auditeur et lui faire avoir bonne opinion de nos mœurs, qui est la seconde sorte de preuve artificielle
que nous avons remarquée. Car enfin, les mêmes moyens que doit employer l'orateur pour faire croire honnête
homme et vertueux celui qu'il a dessein de louer, ces mêmes moyens là lui serviront encore pour faire croire
qu'il est honnête homme lui-même.

Et parce qu'assez souvent il nous arrive de louer aussi bien par plaisir que sérieusement, non seulement un
homme, ou une divinité, mais même les choses qui n'ont point de vie, ou des animaux le premier venu, Il est
nécessaire encore que nous fassions ici comme nous avons déjà fait dans le genre délibératif, c'est à dire:
établir des propositions sur toutes les matières de la louange ou du blâme, afin qu'elles nous servent. Mais
auparavant, donnons quelque idée de ce que nous appelons honnête et de ce que nous appelons vertu.

\bigbreak

Par le mot d'honnête, on entend:

\begin{emphpar}
	Une chose qui, étant souhaitable à cause d'elle-même, mérite qu'on la loue.
\end{emphpar}

Ou si l'on veut encore:

\begin{emphpar}
	Une chose qui étant un bien en soi, outre cela, est agréable à cause que c'est un bien.
\end{emphpar}

Que si cette supposition est vraie, il s'ensuivra:

\begin{lieu}
	Que la vertu est une chose honnête;
\end{lieu}

Puisque étant un bien, elle mérite encore qu'on la loue. Au reste, à juger de la vertu seulement par ce qu'elle
nous parait, elle peut être définie:

\begin{emphpar}
	Une puissance capable de nous faire acquérir de très grands avantages et de nous les conserver;
\end{emphpar}

Ou encore:

\begin{emphpar}
	Une puissance capable d'obliger beaucoup et en des occasions importantes, et même à qui rien n'est difficile
	dans ce qu'elle entreprend, et qui peut tout en toutes choses.
\end{emphpar}

\bigbreak

Les parties de la vertu, ou les vertus en particulier, sont la justice, la valeur, la tempérance, la magnificence,
la magnanimité, la libéralité, la mansuétude ou la clémence, la prudence et la sagesse.

\bigbreak

Or, supposé que la vertu soit telle que nous venons de dire, il faudra mettre au nombre des vertus les plus hautes:

\begin{lieu}
	Celles qui sont très utiles aux autres.
\end{lieu}

Puisque le propre de la vertu, c'est d'obliger. Aussi est-ce pour cette raison que les peuples honorent
principalement les hommes justes et les vaillants, à cause que la justice leur est utile en temps de paix
et la valeur durant la guerre. Après ceux-ci, le libéral est la personne qu'on honore le plus puisque, loin
de quereller pour l'argent que tous les autres recherchent avec tant d'avidité, jamais au contraire il n'est
plus joyeux que lorsqu'il le donne et qu'il en fait des largesses.

\bigbreak

Quant aux définitions de chaque vertu en particulier, premièrement, la justice est définie une vertu qui
conserve à chacun ce qui lui appartient conformément aux lois et aux ordres établis dans chaque état.

L'injustice au contraire est un vice qui nous fait usurper et retenir le bien d'autrui contre l'ordonnance et
l'intention de ces mêmes lois.

La valeur, ou le courage, est une vertu qui, au milieu des plus grands périls, fait entreprendre de belles
actions, ce qui se doit entendre lorsque toutes les circonstances que les lois prescrivent sont exactement
observées. Le courage aussi parait à faire valoir les lois et à les maintenir dans leur vigueur.

La lâcheté est le vice contraire.

La tempérance est une vertu qui fait que nous nous réglons sur la loi touchant les plaisirs sensuels. 

L'intempérance, ou la débauche, est le vice opposé.

La libéralité est une vertu qui ne regarde ses richesses que pour en faire du bien et pour obliger. 

L'avarice est le contraire. 

La magnanimité est une vertu qui se plaît à obliger dans les grandes choses et aux occasions importantes.

La pusillanimité, ou bassesse d'âme, est un vice qui lui est opposé.

La magnificence est une vertu qui aime l'éclat et à faire de grandes dépenses.

La mesquinerie est tout le contraire.

La prudence, enfin, est une vertu de l'esprit qui, touchant les biens et les maux que nous avons dit
contribuer à nous rendre heureux ou malheureux, nous les fait distinguer afin de ne pas prendre l'un pour l'autre.

\subsection{Lieux communs pour la louange}

Après avoir considéré le vice et la vertu en général et en particulier autant qu'il était à propos de le faire pour
notre dessein, il ne sera pas difficile de passer au reste et de tirer des conséquences. Premièrement donc il sera
nécessaire de conclure:

\begin{lieu}
	Que tout ce qui contribue à nous rendre vertueux est honnête;
\end{lieu}

Puisqu'il se rapporte à la vertu.

\bigbreak

Secondement:

\begin{lieu}
	Tout ce qui vient de la vertu et qui en est une suite;
\end{lieu}

Comme sont tous les signes et les marques qu'on aide la vertu, et tout ce qu'elle produit.

\bigbreak

Que si tout ce qui sert de signe pour faire connaître la vertu en général, même si tous les effets qu'elle produit,
et tout ce qu'on peut souffrir à son occasion, est honnête, il sera vrai d'assurer encore de chaque vertu en particulier,
par exemple de la valeur:

\begin{lieu}
	Que tout ce qui sera un effet ou une marque de valeur et tout ce qui aura été souffert en se portant vaillamment
	fera honneur.
\end{lieu}

On en dira autant de la justice et de tous ses effets, à l'exception néanmoins d'une partie des choses qu'elle fait souffrir,
car c'est de la justice seulement qu'il n'est pas toujours vrai de dire que ce qu'elle fait souffrir soit honnête, attendu
qu'il est beaucoup plus honteux d'être justement puni, que de l'être injustement. Au reste, ce que nous remarquons ici de ces
deux vertus se doit entendre également de toutes les autres.

\bigbreak

Il faudra mettre encore au rang des choses honnêtes:

\begin{lieu}
	Toutes les actions à qui l'honneur sera proposée pour récompense;
\end{lieu}

Comme aussi:

\begin{lieu}
	Celles qui nous apporteront beaucoup plus d'honneur que de profit.
\end{lieu}

On en doit dire autant:

\begin{lieu}
	De toutes les choses qui sont à rechercher pour elles-mêmes, si celui qui les fait ne le fait point pour
	lui, mais pour d'autres.
\end{lieu}

Et ainsi en est-il:

\begin{lieu}
	De celles qui seront bonnes simplement en général.
\end{lieu}

Par exemple, tout ce qu'une personne entreprendra pour le salut ou la gloire de sa patrie, au préjudice de
son intérêt.

On assurera la même chose:

\begin{lieu}
	De tout ce qui est bon naturellement;
\end{lieu}

Et encore:

\begin{lieu}
	De tout ce qui ne sera pas bon pour soi et dont on ne tirera aucun usage;
\end{lieu}

Puisqu'on ne le possédera point par intérêt ni en sa propre considération.

\bigbreak

Il en sera de même:

\begin{lieu}
	De tout ce qui arrivera à une personne plutôt après sa mort que durant sa vie;
\end{lieu}

Attendu que tout ce qu'on fait pour un homme durant sa vie, et toutes les déférences qu'on lui rend, viennent
ordinairement de gens intéressés qui regardent moins le mérite et la vertu de celui qu'ils honorent que son crédit
et le pouvoir qu'il a de les obliger.

\bigbreak

Il faudra encore tenir pour honorable:

\begin{lieu}
	Tous les ouvrages publics et ce qui se fait pour les autres;
\end{lieu}

Puisque l'auteur de telles choses y aura le moins de part.

\bigbreak

Et pareillement:

\begin{lieu}
	Toutes les entreprises que nous aurons achevées heureusement, et toutes les affaires que nous aurons conduites
	avec succès, où il n'y allait nullement de notre intérêt, mais seulement de l’intérêt d'autrui.
\end{lieu}

\bigbreak

Comme encore:

\begin{lieu}
	Tout ce que nous aurons fait à l'avantage des personnes à qui nous serons obligés;
\end{lieu}

À cause que c'est une chose juste d'être reconnaissant.

\begin{lieu}
	Les bienfait encore seront de cette nature;
\end{lieu}

Puisque tout bienfait regarde autrui et qu'il n'en revient rien à celui qui le fait.

\bigbreak

On sera passer encore pour honnête:

\begin{lieu}
	Tout ce qui est contraire aux choses qui font rougir et donnent de la honte;
\end{lieu}

Car tout homme qui témoigne de la honte et qui rougit, c'est toujours pour des choses sales et déshonnêtes, soit qu'en
effet, il soit trouvé s'en entretenant ou qu'il les fasse, ou seulement qu'il soit prêt de les faire ou de les dire; ce
qu'a fort bien remarqué \textsc{Sapphô} à l'endroit où elle introduit Alcée qui lui parle en ces termes:

\begin{emphpar}
	Je voudrais bien, Sapphô, vous dire quelque chose,
	
	Mais un respect honteux à mon désir s'oppose\dots
\end{emphpar}

Sapphô répond:

\begin{emphpar}
	C'est trop me dire, Alcée: un si honteux respect

	Accuse ton désir et me le rend suspect.

	Si ce désir était un désir légitime,

	Si ta langue trop prompte à se charger d'un crime

	N'avait à mettre au jour un propos vicieux,

	Tu n'abaisserais pas honteusement les yeux,

	Et tu serais hardi dans une cause juste. 
\end{emphpar}

\bigbreak

Il faudra encore tenir pour honnête:

\begin{lieu}
	Tout ce qui nous donne de l'inquiétude et du soin, sans pourtant qu'en cet état nous nous trouvions saisis
	d'aucune appréhension;
\end{lieu}

Puisque cela ne pourra venir que d'émulation seulement, et de ce que nous nous serons proposé d'acquérir quelqu'un
de ces biens et de ces avantages éclatants qui regardent la réputation et la gloire. 

\bigbreak

On pourra assurer aussi:

\begin{lieu}
	Que les vertus et les œuvres des personnes plus parfaites seront aussi plus remarquables et plus dignes d'honneur.
\end{lieu}

Par exemple celles de l'homme, plus que celles de la femme. 

\bigbreak

Et encore:

\begin{lieu}
	Toutes celles qui feront plus pour la jouissance et le profit des autres que de celui en qui elles se trouveront.
\end{lieu}

D'où vient que la justice particulièrement est en honneur, et tout ce qui est juste.

\bigbreak

Il en sera de même:

\begin{lieu}
	Du dessein qu'on aura de se venger de ses ennemis plutôt que de faire accord avec eux.
\end{lieu}

Car il est juste de rendre la pareille à ses ennemis. Et si la chose est juste, il y a de l'honneur, joint que le
propre d'un homme généreux est de ne point céder à ses ennemis, et de ne souffrir jamais d'en être vaincu.

\bigbreak

Il faudra encore conclure:

\begin{lieu}
	Que la victoire et l'honneur seront fort à estimer.
\end{lieu}

Car puisque ce font des choses à souhaiter quand bien même elles seraient infructueuses et qu'il n'en reviendrait
rien, nous aurons encore cet avantage en les possédant qu'elles feront paraitre en nous un mérite extraordinaire
et un excès de vertu. 

\bigbreak

On fera encore passer pour honnête:

\begin{lieu}
	Tout se qui peut entretenir la mémoire d'un homme et faire parler de lui après sa mort.
\end{lieu}

D'où non seulement il s'en suivra:

\begin{lieu}
	Que plus une chose sera capable de produire un tel effet, plus elle sera honorable;
\end{lieu}

Mais encore:

\begin{lieu}
	Tout ce qui ne pourra arriver à une personne qu'après sa mort.
\end{lieu}

Comme aussi:

\begin{lieu}
	Tout ce qui sera suivi d'honneur, de gloire et de réputation.
\end{lieu}

Pareillement:

\begin{lieu}
	Toutes les choses extraordinaires et qui excellent.
\end{lieu}

Enfin:

\begin{lieu}
	Tout ce qui ne sera possédé d'aucun autre que de nom;
\end{lieu}

Car comme ces choses-là seront plus remarquables d'elles-mêmes, aussi seront-elles plus propres à faire
passer de nous et à nous mettre en estime.

\bigbreak

On pourra encore proposer comme honnête:

\begin{lieu}
	Toutes les acquisitions qui ne seront d'aucun rapport;
\end{lieu}

Puisqu'elles feront éclater davantage la libéralité de celui qui les aura en sa possession.

\bigbreak

Il en sera de même:

\begin{lieu}
	De tout ce qui sera particulier à chaque peuple et à chaque nation;
\end{lieu}

Comme aussi:

\begin{lieu}
	De tout ce qui servira de marque à chaque peuple des choses qui sont particulièrement en estime chez lui.
\end{lieu}

Par exemple c'est un honneur chez les Lacédémoniens de porter de grands cheveux, à cause qu'ils les prennent
pour une marque de liberté et d'indépendance. Et sans doute, il y a quelque raison à cela, puisque enfin, il
n'est pas aisé à un homme qui a les cheveux grands de faire rien de servile. Ainsi encore en est-il chez eux
de n'exercer aucun art mécanique, comme étant encore de la liberté de ne dépendre point d'autrui et de n'être
assujetti à personne.

\subsection{Adresses pour louer ce qui ne sera pas louable}

Outre les propositions et les conséquences que nous venons d'alléguer, on se pourra encore servir d'adresse.
Premièrement, donc, au vu des qualités véritables qui se trouveront en la personne qu'on voudra louer ou blâmer,
on se servira de celles qui leur ressemblent ou en approchent. Par exemple, si nous avons à parler contre un homme
vaillant en effet, mais qui, à la guerre, emploie plus ordinairement la ruse que la force, nous dirons que c'est
un poltron qui n'a du courage que quand il faut prendre en trahison et dresser des embuscades. Au contraire, si
nous avons à louer un sot et un niais, nous ferons passer sa niaiserie pour une bonté. Et encore, nous appellerons
doux et pacifique un homme insensible à toutes sortes d'injures. En un mot, nous tâcherons de faire prendre en bonne
part chaque défaut et chaque vice, en leur attribuant le nom des choses qui les accompagnent d'ordinaire et qui sont
de leur suite. Aussi, parlant à l'avantage d'un colère et d'un rébarbatif, nous dirons que c'est un homme ouvert et
qui ne peut dissimuler. De même, nous dirons d'un orgueilleux ou d'un arrogant que sa façon d'agir est noble et sent
sa personne de qualité.

\bigbreak

Une seconde adresse dont on se pourra servir, c'est d'attribuer la qualité de vertueux à des personnes qui pêchent
par excès, comme de nommer vaillant un téméraire; ou d'appeler libéral un prodigue. Car outre que bien des gens y
seront trompés, c'est qu'il y aura lieu même d'apporter un faux raisonnement pour le faire croire tout de bon. Et de
fait, on dira:

\begin{emphpar}
	Un homme qui court au danger sans nécessité, que ne fera-t-il point quand l'honneur l'y appellera?
\end{emphpar}

On en dira autant du prodigue en raisonnant de la même sorte:

\begin{emphpar}
	Celui qui donne a tout venant et qui ne peut refuser à personne, est-il croyable qu'il abandonne ses amis au
	besoin et qu'il ne soit avare que pour eux?
\end{emphpar}

Et véritablement il semble que faire ainsi du bien à tout le monde est l'effet d'une vertu extraordinaire, et
d'une bonté qui va jusqu'à l'excès.

\bigbreak

Une autre observation encore à faire pour la louange, c'est de prendre garde à qui sont ceux devant qui on doit parler.
Car ce n'est pas sans raison que Socrate disait qu'il n'était pas difficile de louer les Athéniens en parlant aux Athéniens.
Ainsi donc, selon les personnes devant qui on aura à paraître, il faudra voir quelles choses particulièrement seront chez
elles en estime, et alors en parler comme si véritablement elles étaient à estimer; par exemple, chez les Scythes, au cas
qu'on ait à parler devant eux, ou chez les Lacédémoniens, ou devant des philosophes. En un mot il faudra ramener à l'honneur
et faire passer pour tel, ce qui ne sera que simplement honorable et estimé de quelques personnes. Et de fait, il semble
qu'il n'y ait pas grande différence de l'un à l'autre.

\bigbreak

Outre ceci, quand on aura à louer quelqu'un, il sera bon encore d'examiner s'il a fait en sa vie des actions bienséantes et
qui conviennent à une personne de sa qualité. Par exemple, s'il a fait des choses dignes de sa naissance et de ses ancêtres,
ou si ce qu'il vient de faire répond à ses actions passées et à l'attente qu'on avait de lui, car non seulement il y a du
bonheur à augmenter toujours sa réputation et à entasser honneur sur honneur, mais encore c'est une chose glorieuse.

\bigbreak

Il sera encore bon d'examiner le contraire; car sans doute, c'est une très belle occasion de louer un homme que d'avoir
à montrer qu'il a été vertueux au delà même de ce qu'on en devait attendre et que ce qu'il a fait, il l'a toujours fait
de mieux en mieux; comme de dire qu'au milieu de la prospérité, il ne s'est point oublié et qu'il a été aussi modeste
que devant. Ou au contraire, que dans l'adversité ou le malheur de ses affaires, il s'est toujours soutenu et n'a pas moins
paru généreux. Ou enfin, qu'étant sorti de bas lieu, à mesure qu'il est monté aux charges et aux honneurs il n'en est devenu
que plus honnête homme et plus facile à aborder. Et de vrai, c'est là-dessus qu'est fondée la louange qu'Iphicrate se donnait
à lui-même, comme nous avons déjà remarqué, lorsqu'il dit: <<~Qui étais-je autrefois pour être maintenant ce que je suis?~>>
Et encore celle qu'on lit dans l'épigramme du Poissonnier d'Argos qui remporta le prix aux Jeux Olympiques:

\begin{emphpar}
	Aurait-on jamais cru qu'un jour j'eusse la gloire,

	Moi qu'on vit mille fois un panier sur le dos\dots
\end{emphpar}

Telle est encore la louange que \textsc{Simonide} donne à Archedicé qui se montra si bonne et si obligeante à tout le monde,
quoi qu'elle fut d'une très haute naissance, et comme il l'assure lui-même, 

\begin{emphpar}
	Et fille et femme et sœur de monarques puissants.
\end{emphpar}

Après tout, parce que la louange regarde principalement les actions de la vie, et que le propre d'un homme vertueux c'est
d'agir toujours de dessein, il faudra tâcher en louant une personne de montrer que toutes les actions qu'elle a faites,
elle ne les a point faites par hasard mais de dessein, de propos délibéré. Pour cela, donc, il sera nécessaire de faire voir
que souvent, elle a fait de même, et alors on ramassera tout ce qui lui sera arrivé en sa vie, ou fortuitement, ou par bonheur,
le faisant valoir comme des choses qu'elle avait résolues de faire et auxquelles elle s'était étudiée particulièrement, car
quand on peut alléguer d'une même personne plusieurs actions toutes semblables, c'est en quelque façon un préjugé et une preuve
certaine que cette personne est vertueuse effectivement, et qu'elle n'a rien fait que de dessein et après s'être proposé de le 
faire\footnote{Vous aimez ça, vous, les biopics? Parce que c'est exactement ça\dots}.

\subsection{Espèces différentes de louange}

Au reste, il y a plusieurs sortes de louanges. La première espèce regarde les vertus héroïques et confirmées par de
longues habitudes. Elle est définie \emph{un discours qui donne à connaître une très haute vertu}. Or, pour faire qu'on
puisse ajouter foi à une louange de cette qualité, il faudra montrer que toutes les actions de la personne qu'on loue
viennent d'habitude et sont des effets d'une vertu éminente.

\bigbreak

La seconde espèce de louange regarde les œuvres et chaque action louable en particulier. Pour tout le reste qui a
accoutume d'entrer dans la louange, comme sont les circonstances, cela sert seulement à rendre une chose croyable
et à la persuader plus aisément. Telles sont la naissance et l'éducation, pour ce qu'il est vraisemblable qu'un
homme qui est sorti d'honnêtes gens est honnête homme; et encore que celui qui a eu une telle éducation est tel
qu'on l'a élevé. Aussi, pour cela, toujours louons-nous bien d'avantage ces personnes quand elles font des actions
qui répondent à leur éducation ou à leur naissance, puisque alors il y a lieu de faire voir que de semblables
actions viennent d'une nature confirmée au bien, et procèdent d'habitude. Ce fondement est si véritable que, même,
nous ne laisserions pas de louer un homme quoi qu'il n'ait rien fait de remarquable en sa vie, si nous étions
assurés qu'il fût tel que nous venons de dire.

\bigbreak

La troisième espèce de louange à qui les Grecs donnent deux noms, quoi qu'en effet, ces deux noms n'aient que la
même signification, consiste à féliciter une personne et à la louer comme souverainement heureuse. Cette louange
est différente des deux autres en ce qu'elle se propose un sujet plus vaste et plus étendu. Car tout aussi que le
souverain bonheur et la félicité comprennent en soi la possession de toutes sortes de vertus, de même cette espèce
de louange renferme les deux autres, puisqu'elle n'a pas simplement pour objet une habitude vertueuse comme la
première, ou quelqu'un de louable en particulier comme la seconde, mais toutes les vertus généralement et les
riches qualités de l'âme.

\subsection{Ressemblance du genre démonstratif avec le délibératif}

Une observation à faire touchant la louange et le conseil, c'est que tous deux ont beaucoup de conformité. Car
enfin, ce qu'on propose en conseillant quelqu'un et tout ce que l'orateur alors met en avant comme des avis à
suivre, cela même peut servir de louange en changeant simplement la façon de parler. D'où il s'ensuit qu'ayant
la connaissance, comme nous avons, de tout ce qu'il faut qu'un homme fasse pour être loué et des qualités qu'il
doit posséder, il nous sera très facile de former des préceptes de toutes ces matières de louange, puisqu'il n'y
aura qu'à changer un peu la phrase. Donnons quelque exemple. Si, donc, on disait ainsi:

\begin{emphpar}
	Jamais il ne se faut prévaloir, ni tirer avantage, des biens que la fortune nous donne, mais seulement de
	notre vertu et des biens qui nous appartiennent en propre.
\end{emphpar}

Cela, sans doute, exprimé de 1a sorte, est un précepte et un conseil tout pur. Cependant, qu'on change un peu
la façon de parler, ce sera une louange, car il n'y aura qu'à dire:

\begin{emphpar}
	Jamais cet homme n'a tiré avantage des faveurs qu'il a reçues de la fortune, et quand il s'est voulu faire
	valoir, il ne s'est servi que de son mérite et de sa propre vertu,
\end{emphpar}

Toutes les fois, donc, que vous aurez à louer quelqu'un, prenez garde au conseil que vous lui donneriez si vous
aviez à lui faire entreprendre quelque belle action, et au contraire, quand vous aurez à donner un conseil, ou
quelque avis, jugez en vous même et examinez quelle action mériterait en effet d'être louée. À la vérité, l'expression
sera différente et doit être opposée nécessairement, car pour le conseil, il faut qu'elle soit prohibitive, et
pour la louange, il ne le faut pas.

\subsection{De l'amplification}

Ce ne sera pas encore une petite adresse, quand on voudra louer quelqu'un, d'user d'amplification et de se servir
des circonstances qui agrandissent une action et la font paraître plus considérable; comme de dire qu'il a été le
seul ou le premier qui ait osé faire une telle entreprise, ou de montrer qu'il l'a exécutée avec fort peu de monde,
ou qu'il n'y en a point qui s'y soit plus signalé que lui; car ces circonstances sont glorieuses à remarquer, et
méritent une louange particulière.

\bigbreak

Le temps encore et les occasions peuvent beaucoup faire valoir une action, parce qu'alors elle paraîtra extraordinaire
et sera regardée comme une chose qui a passé l'attente de tout le monde et l'espérance qu'on en avait conçue.

\bigbreak

On pourra aussi agrandir la louange d'un homme en faisant voir qu'il a souvent réussi dans les mêmes entreprises, car
outre que l'action en sera plus considérée et se fera davantage admirer, jamais on ne pourra croire qu'elle ait été
faite par hasard, et on l'attribuera toujours à l'adresse et à la vertu de celui qui y aura réussi.

\bigbreak

Il sera encore avantageux de remarquer si quelqu'une des choses qui sont faites pour donner de l'émulation et pour
porter les hommes aux belles actions ont été inventées et établies pour faire honneur à celui que nous aurons à louer,
ou s'il est le premier à qui on ait donné des éloges en public, comme il est arrivé a Hippoloque, enfin si sa gloire
peut être égalée à celle d'Harmodios et d'Aristogiton, qui furent les premiers à qui les Athéniens dressèrent des
statues dans la place publique. Or, non seulement ceci aura lieu pour embellir une action et la faire davantage valoir,
mais encore pour faire le contraire, pouvant servir également à enlaidir la vie et les actions de ceux que nous
voudrons blâmer. 

\subsection{Adresses pour louer un homme qui n'a rien fait de louable}

Mais s'il arrive que la personne que nous aurons à louer n'ait rien fait qui puisse fournir de matière pour en parler
glorieusement, en ce cas, il faudra avoir recours aux parallèles, et la comparer à d'autres, ce qu'\textsc{Isocrate} a
fait souvent pour n'avoir pas pratiqué le barreau, ni s'être étudié au judiciaire. Il y a ceci à observer touchant ces
comparaisons qu'il faut que les personnes qu'on choisit soient illustres et d'une haute réputation, à cause qu'il n'y
a rien qui agrandisse davantage la louange d'un homme, que de faire voir qu'il a des qualités plus éclatantes, et qu'il
a fait des actions plus vertueuses, que ceux-même qui passent pour être très vertueux.

Or, pour montrer que ce n'est pas sans sujet que l'amplification a lieu particulièrement dans la louange, c'est que la
louange aime l’excès et ne cherche que ce qui est excellent et qui passe l'ordinaire. Or est-il que nous avons déjà
remarqué que tout ce qui est excellent et qui passe à un excès louable est du nombre des choses honnêtes. Pour cela
donc, si celui que nous aurons à louer n'est pas assez considérable de lui-même pour être comparé à des personnes
illustres, il ne faudra pas laisser de le comparer à d'autres; car enfin, de quelque façon qu'on élève un homme au
dessus d'un autre, toujours cette élévation et ce degré d'éminence témoignent qu'il a du mérite.

\subsection{Les choses qui sont particulières à chaque genre}

En un mot, donc, et pour prononcer en général chaque partie de la rhétorique, nous pouvons dire que de tous ses
trois genres, il n'y en a pas un à qui l'amplification soit plus nécessaire et plus propre qu'au genre démonstratif.
La raison est qu'un orateur qui loue, prend toujours pour son sujet des actions véritables et reconnues telles de
tout le monde. De sorte que ce qu'il lui reste à faire, c'est d'embellir ces actions-là et de leur donner de l'éclat.

Pour les exemples, ils s'accommodent mieux avec le genre délibératif, puisque les jugements que nous formons dans
nos entreprises et dans tous nos desseins se fondent sur les conjectures que le passé donne de l'avenir, et sur
le rapport qui se remarque entre ce qui s'est déjà fait et ce qui se peut faire. 

Quant aux enthymèmes, ils sont plus propres au genre judiciaire, car comme, là, il s'agit de fait et de juger du
passé, qui est une chose qu'on ne connaît pas toujours et qui aisément peut être révoquée en doute, pour cela il
est besoin de rendre raison particulièrement pourquoi une chose a été faite, et d'en faire la preuve.

Voilà à peu près ce qui se peut dire sur le sujet du blâme et de la louange, et tout ce que l'orateur se doit
proposer quand il aura à louer ou à blâmer quelqu'un. En un mot, tous les lieux et toutes les adresses qui peuvent
servir à embellir ou à enlaidir quelque action que ce soit, car pour blâmer et parler au désavantage d'une personne,
il ne faut point d'autres préceptes que ceux que nous avons donnés pour louer. Tout contraire ayant cela de propre
de donner la connaissance de son contraire en même temps qu'il le fait connaître.

\bigbreak

Le blâme, donc, aura pour son sujet tout ce qui est contraire et opposé à la matière de la louange.

\chapter{Le genre judiciaire}
\section{Ce que c'est que faire tort ou injure}

Maintenant, il s'agit de l'accusation et de la défense et de donner à connaître le nombre et la qualité
des lieux dont le genre judiciaire se sert pour tirer ses arguments. Mais auparavant, il est important de
savoir ces trois points. 

\begin{emphpar}
	Quelles choses portent les hommes à se nuire et combien il y en a;

	Qui sont ceux qui les font et les dispositions qu'ils ont à ceci;

	Enfin, à quelles personnes ils s'attaquent et en quel état il faut qu'ils les trouvent.
\end{emphpar}

Ce que nous tâcherons d'éclaircir après avoir expliqué ce que c'est que faire injure à quelqu'un. 

On appelle faire injure \emph{quand volontairement, on nuit à un autre contre la défense de la loi}. Or,
il y a deux sortes de lois: les unes particulières et les autres communes. J'appelle lois particulières celles
qui sont écrites et qui servent de règle dans un état. Et j'appelle communes toutes celles qui ne sont point
écrites et qui semblent avoir été établies du commun consentement de tous les peuples.

Toute personne, au reste, agit volontairement lorsqu'elle sait bien ce qu'elle fait, et qu'elle n'y est point
forcée. Ce n'est pas que tout ce qui se fait volontairement se fasse toujours de dessein et de propos délibéré,
mais bien ce qui se fait de propos délibéré et de dessein se fait toujours volontairement et avec connaissance
de cause; puisque enfin, il n'est pas possible qu'un homme ignore la chose qu'il se propose de faire plutôt qu'une
autre et à laquelle il se détermine.

\bigbreak

Or, de savoir pourquoi on est porté à faire du mal et à nuire contre la défense des lois, cela vient de deux causes:
du vice ou de la passion. Car il se remarque que tous les vicieux généralement, soit qu'ils aient plusieurs vices ou
qu'ils  n'en aient qu'un, jamais, presque, ne sont injustes ni malfaisants qu'en ce qui touche le vice qui leur
commande. Ainsi, \emph{l'avare} n'est guère porté à mal faire qu'à cause de l'argent, ni le \emph{débauché} que parce
qu'il espère jouir de quelque plaisir, ni le \emph{fainéant} qu'afin d'avoir de quoi flatter sa paresse, ni le
\emph{poltron} que pour s'éviter quelque danger (car la crainte fait toujours abandonner aux lâches leurs compagnons
dans le danger), ni \emph{l'ambitieux}, que lorsqu'il y va de l'honneur, ni un \emph{homme prompt} que dans les
transports de sa colère, ni \emph{celui qui aime à vaincre} que lorsqu'il s'agit dé la victoire, ni un \emph{vindicatif}
qu'a cause de la vengeance, ni \emph{celui qui n'a point d'esprit} que par sa bêtise et sa stupidité pour ne pas discerner
ce qui est juste d'avec ce qui ne l'est pas et sans cesse s'y tromper, ni \emph{l'effronté}, enfin, et \emph{l'impudent},
que parce qu'il a toute honte perdue et qu'il se moque de la réputation. Et ainsi en est-il de tous les autres vicieux à
l'égard du vice auquel ils sont sujets. Tout ceci au reste nous est déjà connu en partie par ce qui a été dit de la vertu
et le sera pleinement quand nous aurons à traiter des passions. Il nous reste à faire voir pour quelle raison et de quel
esprit sont portés ceux qui font du tort, et à quelles personnes ils s'attaquent.

\subsection{Actions humaines et leurs causes}

Donnons donc à connaître en premier lieu les choses que, souhaitant d'avoir ou d'éviter, ensuite nous tâchons de nuire à
notre prochain et lui faire tort. Car, sans difficulté, tout avocat qui accuse doit regarder principalement combien de ces
choses-là qui tentent les hommes et les portent à faire du tort donnent prise sur sa partie adverse. Comme au contraire,
c'est à l'avocat qui défend, d'examiner en sa partie combien elle est éloignée de soupçon touchant ces choses-là, afin
de la justifier.

\bigbreak

Il faut donc savoir, qu'absolument, \emph{il n'y a rien que les hommes ne fassent}, et que de tout ce qu'ils font, il y en
a une partie \emph{qu'ils ne font point d'eux-mêmes} et une autre \emph{qu'ils font d'eux-mêmes} et de leur propre mouvement.

De plus, que ce qu'ils ne font point d'eux-mêmes, une partie se fait par hasard et l'autre par nécessité.

Et enfin, que ce qu'ils font par nécessité, il y en a encore une partie ou ils sont violentés, et l'autre qu'ils font par nature.

De manière que tout ce que les hommes ne font point d'eux-mêmes peut être rapporté à trois causes principales: au hasard, à la
nature et à la contrainte.

\bigbreak

Pour les choses que les hommes font de leur propre mouvement et dont ils sont eux-mêmes les auteurs, elles sont de quatre
sortes. Car il y en a une qu'ils font par coutume et l'autre par inclination parce que leur appétit les y porte, mais
comme il y a deux sortes d'appétits dans l'homme, l'un d'animal ou sensuel et l'autre raisonnable, il se trouve que tantôt
ils suivent l'appétit raisonnable et tantôt le sensuel.

L'appétit raisonnable, au reste, n'est autre chose que notre volonté, qui est définie, \emph{un appétit ou un désir du bien
conduit et réglé par la raison}, car il est certain que jamais personne ne veut une chose, que parce qu'il croit que c'est
son bien. 

L'appétit sensuel se partage en deux et reconnaît deux principes différents: l'un de la colère et l'autre de la convoitise.

\bigbreak

Si bien qu'à examiner les causes de toutes nos actions, il ne s'en trouve que sept, puisque enfin, tout ce que les hommes
font, en leur vie doit être rapporté:

\begin{emphpar}
	Ou au hasard,

	Ou à la contrainte,

	Ou à la nature,

	Ou à l'accoutumance,

	Ou au raisonnement,

	Ou à la colère,

	Ou à la convoitise,
\end{emphpar}

Car de faire une plus longue division et de vouloir encore distinguer les actions des hommes par les ages différents,
par les habitudes et par telles autres qualités, ce serait une chose superflue, étant certain que s'il arrive aux
jeunes gens d'être colères ou adonnés à leurs plaisirs, ce n'est point à leur jeunesse qu'il s'en faut prendre, mais
à leur passion bouillante et à leur convoitise déréglée. Ainsi en est-il des riches et des pauvres, car ce n'est point
leur pauvreté ni leurs richesses qu'il faut accuser. Et quoi qu'on puisse dire que, quelquefois, les pauvres ne souhaitent
avec passion d'avoir de l'argent qu'à cause qu'ils sont dans la nécessité, ni les riches que parce qu'étant dans l'abondance
et en puissance de faire tout ce qu'ils veulent, ils viennent alors à former une infinité de vains désirs et à rechercher
des plaisirs qui ne sont point nécessaires, ce n'est pas une conséquence pour cela que de tels désirs soient des effets
de leurs richesses simplement, ou de leur pauvreté, mais bien de leur passion et de leur convoitise. On doit assurer le
même des personnes justes et injustes, et généralement de tous ceux que nous disons agir par quelque habitude de cette
qualité, vu que tout ce que ces gens-là font ne peut être rapporté à d'autres causes qu'à celles qui ont été remarquées,
puisqu'il faut toujours que ce soit: ou parce qu'ils sont persuadés de quelque raison, ou parce qu'ils se laissent emporter
à leurs passions. Toute la différence qu'il y a, c'est que les uns ont des mœurs et des passions louables, les autres au
contraire en ont de mauvaises. À la vérité, je demeure d'accord que chaque habitude ayant toujours des accompagnements et
des suites conformes à sa nature ordinaire, il arrive que telles ou telles actions en particulier ne font faites que parce
ce qu'un homme a contracté telle ou telle habitude. Car, par exemple, il se peut faire que cet homme ici qui est tempérant
n'aura d'abord tels désirs honnêtes et ne sera persuadé des sentiments qu'il faut avoir touchant les plaisirs de la vie
que parce qu'il est tempérant. Et tout au contraire, que ce débauché n'aura tels désirs et tels sentiments déshonnêtes que
parce qu'il est attaché à la débauche. Néanmoins, comme ces distinctions ne sont pas considérables, on les peut laisser là.

\bigbreak

Ce que nous aurions à faire ici maintenant, ce serait d'examiner ce qui a accoutumé d'arriver à telles ou telles sortes de
personnes en suite de telle ou telle qualité. Et cela à cause que tout ce qui met de la différence dans les hommes n'apporte
pas toujours du changement dans leurs sentiments et leurs mœurs. Car, par exemple, qu'un homme soit blanc ou noir, grand ou
petit, tout cela de soi n'exige point qu'il ait telles mœurs en particulier, ni telles passions, mais bien s'il est vieux ou
jeune, s'il est homme de bien ou méchant. En un mot, nous aurions à examiner tout ce qui arrivant à une personne pour la fortune
fait que, d'ordinaire, elle vient à changer de mœurs et de sentiments; comme quand cette personne s'imagine qu'elle est riche ou
pauvre, heureuse ou malheureuse. Mais ce n'est pas encore ici le lieu de traiter cette matière. 

Achevons seulement d'expliquer ce qui nous reste à dire touchant les principes et les causes des actions humaines.

\bigbreak

Premièrement, donc, on attribue une chose à la fortune si l'on croit qu'elle a été faite par hasard, quand la cause qui l'a
produite est purement incertaine et indéterminée, ou qu'on ne voit pas ni à quel dessein ni pourquoi elle a été faite, ou quand
elle n'arrive ni toujours ni d'ordinaire de cette façon-là, ou enfin quand elle ne se fait point règlement et avec un certain
ordre: toutes lesquelles conditions ont été remarquées exactement dans la définition que nous avons données ailleurs du hasard
et de la fortune. 

\bigbreak

En second lieu, la nature est cause d'une chose lorsque cette chose a au-dedans d'elle-même le principe qui la produit et
que ce principe, en la produisant observe un certain ordre. Au reste, il n'importe que telle chose arrive nécessairement ou
pour l'ordinaire, puisque de quelque façon qu'elle arrive, toujours, elle se fera de la même sorte. Quant aux effets
extraordinaires de la nature et qui semblent violer ses règles ou n'arriver pas selon son dessein, ce n'est point ici le
lieu de les examiner particulièrement ni de voir si leur production se fait véritablement selon le dessein de la nature ou
s'il faut en rechercher une autre cause; quoi qu'après tout, on puisse dire qu'assez souvent, tels effets sont un pur ouvrage
de la fortune et du hasard.

\bigbreak

A l'égard de la force et de la contrainte, nous tenons qu'une chose est faite avec contrainte et par force quand celui qui la
fait agit contre sa propre inclination, ou contre son avis.

\bigbreak

Un homme agit par coutume quand il ne fait une chose que parce qu'il l'a déjà faite plusieurs fois.

\bigbreak

On agit par raisonnement lorsqu'on ne fait une chose qu'en intention d'acquérir quelqu'un des biens que nous avons remarqués
devoir apparemment apporter du profit et de l'avantage, soit qu'on borne les prétentions à posséder un tel bien et qu'on se le
propose pour but et pour fin, soit qu'on ne le considère que comme un moyen pour arriver à quelque autre chose qui tienne lieu
de fin et qui soit de plus grande importance; à condition, néanmoins, comme j'ai dit, que cela se fasse à dessein seulement
d'attirer de l'utilité. Car il ne faut pas ici confondre ce que nous voyons faire assez souvent aux débauchés qui, en apparence,
semblent faire beaucoup d'entreprises pour le gain seulement et pour le profits, puisqu'en effet, leur intention alors n'est
autre que de jouir après plus à leur aise des plaisirs qu'ils aiment, et pour donner lieu à une plus longue débauche, 

\bigbreak

Pour ce qui est de la colère et de l'animosité, elles regardent simplement la vengeance. Or il faut remarquer qu'il y a grande
différence entre la vengeance et le châtiment, puisque le châtiment est toujours pour le bien de celui qu'on châtie, et qu'au
contraire, la vengeance n'a pour but que la satisfaction et le contentement de celui qui se venge. De savoir maintenant ce que
c'est que la colère et ce qui la fait naître, nous en parlerons amplement quand nous serons au traité des passions.

\bigbreak

Enfin, on fait par convoitise tout ce qui en apparence doit donner du plaisir. Au reste, je mets au nombre des choses agréables
et qui donnent du plaisir toutes celles qu'on a accoutumé de faire, et auxquelles on s'est habitué. Car l'accoutumance a cela
qu'elle fait trouver du plaisir même dans ce qui n'est point plaisant à faire, si tôt qu'on y est accoutumé.

\bigbreak

Donc, pour trancher en un mot cette matière, il est certain:

\begin{emphpar}
	Que de toutes les choses que les hommes font d'eux-mêmes, il n'en pas une qui ne soit ou véritablement bonne et utile, ou
	telle en apparence, et encore qui ne soit, ou en effet ou apparemment, agréable.
\end{emphpar}

Et parce que tout ce que les hommes font d'eux-mêmes, ils le font toujours volontairement, et au contraire, que ce qu'ils ne
font point d'eux-mêmes, c'est toujours contre leur volonté et malgré eux qu'ils le font, il est certain encore, au regard de
ceux qui agissent volontairement:

\begin{emphpar}
	Que de tout ce qu'ils se proposent de faire, il n'y a rien qui ne soit, en effet ou apparemment, utile; et encore qui ne
	soit: ou agréable véritablement, ou du moins en apparence.
\end{emphpar}

\bigbreak

Après tout, je mets au nombre des biens et des avantages, non seulement la délivrance de quelque mal que ce soit, réel ou
apparent; mais aussi l'échange d'un grand mal pour un petit, parce que tout cela est à souhaiter. Et tout de même, je mets
au nombre de ce qui est agréable la délivrance de tout ce qui est fâcheux, soit qu'en effet il soit tel ou seulement en
apparence; et encore l'échange d'une chose très fâcheuse pour une qui sera moins. 

\bigbreak

Ainsi donc, puisque ce qui est utile et ce qui peut apporter du plaisir est toujours ce qui porte les hommes à faire ce qu'ils
font, pour cela, l'orateur doit s'étudier à connaître le nombre et la qualité des choses qui sont utiles et qui sont
agréables. Pour ce qui est de l'utile, nous en avons déjà parlé en traitant du genre délibératif. Il ne reste plus qu'à faire
voir celles qui sont agréables et qui apportent du plaisir. J'avertirai ici en passant qu'on ne doit pas prendre garde de si
près aux définitions que nous donnons, puisqu'il importe peu à la rhétorique qu'elles soient si exactes, pourvu qu'elles ne
paraissent pas obscures. 


\section{Des choses qui sont agréables et donnent du plaisir}

Posons pour fondement que le plaisir est \emph{une certaine émotion de l’âme  ou un changement qui arrive
tout à coup, qui se rend sensible et qui met la nature en l'état qu'elle demande}, et pour la douleur, que
c'est tout le contraire. Que si le plaisir est tel que nous venons de dire, il s'ensuit:

\begin{lieu}
	Que tout ce qui sera capable de nous mettre en l'état que nous venons de remarquer sera très agréable;
\end{lieu}

et au contraire, très fâcheux:

\begin{lieu}
	Tout ce qui détruira ce même état, ou qui sera cause que nous tomberons dans l'autre qui lui est opposé.
\end{lieu}

\bigbreak

Il s'ensuivra aussi:

\begin{lieu}
	Que pour l'ordinaire, c'est une chose agréable de se sentir arriver à cet état ou nous devons être
	naturellement, surtout quand ce qui se fera selon le désir de la nature aura atteint toute la perfection
	qu'il peut avoir. 
\end{lieu}

\bigbreak

Il faudra mettre encore au nombre des choses qui apportent du plaisir:

\begin{lieu}
	Toutes sortes d'accoutumances;
\end{lieu}

Puisque l'accoutumance est en quelque façon une chose qui a passé en nature. Aussi n'y a-t-il rien qui
ressemble davantage à la nature que l'accoutumance, par la même raison qu'il n'y a rien qui approche plus
de ce qui se fait toujours que ce qui se fait très souvent. Et de vrai, l'accoutumance est pour les choses
qui se font très souvent, et la nature, pour celles qui se font toujours.

\bigbreak

De plus, il s'ensuivra:

\begin{lieu}
	Que tout ce qui ne se fera point avec violence sera agréable;
\end{lieu}

Puisque la violence est ennemie de la nature. Et c'est pour cette raison que toutes les contraintes sont fâcheuses,
et toutes les occasions où il y a nécessité de faire quelque chose, ce qu'un poète a très bien remarqué lorsqu'il
a dit:

\begin{emphpar}
	Tout ce qu'on fait par force incommode toujours.
\end{emphpar}

Or, si cela est, il faudra encore tenir pour fâcheux les inquiétudes, les soins, l'étude, les fortes applications
d'esprit; en un mot toutes sortes d'efforts, à cause que de semblables actions tiennent toujours de la contrainte, 
si l'on n'y est accoutumé, vu qu'en ce cas l'accoutumance les adoucit et les rend agréables.

\bigbreak

D'où il faut conclure:

\begin{lieu}
	Que ce qui sera contraire à tout ce que nous venons de dire apportera du plaisir.
\end{lieu}

Par exemple, la paresse, l'oisiveté, la négligence, les divertissements, le repos, le sommeil; puisqu'il n'y a
rien de plus éloigné ni de plus affranchi de la contrainte que cela.

\bigbreak

Il faudra encore tenir pour agréable:

\begin{lieu}
	Toutes les choses où le désir et l'appétit nous portent;
\end{lieu}

Vu que le désir n'est qu'un appétit de jouir de ce qui est agréable et qui peut donner du plaisir. Or, comme nous
avons déjà remarqué, il y a deux sortes d'appétits dans l'homme, l'un sensuel ou d'animal, et l'autre raisonnable.
Par \emph{appétit sensuel}, j'entends tout ce que les hommes désirent sans faire de réflexion dessus ni l'examiner.
Ces sortes de désirs s'appellent proprement naturels et ne regardent que la satisfaction et les nécessités du corps.

Tels sont, premièrement, la faim et la soif, qui sont donnés au corps pour lui faire songer en général aux
aliments nécessaires à l'entretien de sa vie. Et tous les autres encore qui regardent chaque espèce de nourriture
en particulier.

Tels sont, en second lieu, les désirs qui tendent à l'amour et à la bonne chère. Bref, tous ceux qui flattent les
autres sens, et qui peuvent contenter le toucher, l'odorat, l'oreille et la vue.

J'appelle \emph{appétit raisonnable} ce qui fait désirer une chose seulement à cause qu'on est persuadé de sa bonté.
Car il se trouve beaucoup de choses dont on ne vient à désirer et la vue et la possession que parce qu'on en a ouï
faire de l'estime et que, véritablement, on croit qu'elles méritent d'être possédées. 

\bigbreak

Or, puisque le plaisir consiste à se sentir émouvoir en soi-même et à être touché de quelque passion, outre cela
que notre imagination, à la bien considérer, soit je ne sais quelle sorte de sentiment débile et imparfait, en un
mot, puisqu'il n'est pas possible d'espérer, ni de se souvenir de quoi que ce soit qu'en même temps on ne forme
dans son imagination l'idée et l'image de la chose qu'on espère ou dont ou se souvient, cela présupposé, il
s'ensuit:

\begin{lieu}
	Qu'il y aura du plaisir à se souvenir parfaitement d'une chose et à être dans une très grande espérance de
	l'avoir;
\end{lieu}

Puisque, selon ce que nous venons de dire, ce sera en jouir alors en quelque sorte et l'avoir présente à ses
sens. Tellement qu'il est nécessaire que rien ne nous puisse donner du plaisir qu'en l'une de ces trois façons:
ou quand il sera présent à nos sens et qu'en effet nous en jouissons, ou quand il nous souviendra qu'autrefois,
nous en avons joui, ou enfin, lorsque nous aurons espérance d'en jouir quelque jour. Car la jouissance regarde
toujours le présent, la mémoire le passé, et l'espérance l'avenir.

\bigbreak

La mémoire, donc, ne représente jamais rien qu'elle n'apporte du plaisir; car non seulement elle donne du plaisir
lorsqu'elle rappelle les images des choses qui étaient agréables dans le temps qu'on en jouissait, mais même encore
lorsqu'elle en représente d'autres d'une nature toute contraire et qui autrefois étaient très fâcheuses à supporter,
principalement quand les personnes ont changé d'état, et qu'à leurs travaux passés et à toutes leurs disgrâces a
succédé un grand repos ou beaucoup de gloire. C'est aussi ce qui a fait dire a \textsc{Euripide}:

\begin{emphpar}
	D'un péril évité, le souvenir est doux;
\end{emphpar}

Et encore à \textsc{Homère}:

\begin{emphpar}
	Quiconque a vu ses jours autrefois traversés

	Prend plaisir de songer à ses malheurs passés.

	Surtout quand son adresse et son propre courage,

	Après beaucoup d'efforts, ont surmonté l'orage.
\end{emphpar}

Et la raison de ceci est qu'il y a même du plaisir à n'avoir point de mal.

\bigbreak

Quant à l’espérance, il est certain encore qu'on ne saurait jamais rien espérer de tout ce qui semble
devoir réjouir par sa présence, apporter quelque grand avantage ou simplement être utile sans incommoder
qu'en même temps il n'en vienne du plaisir. En un mot, tout ce qui par sa présence cause de la joie, pour
l'ordinaire, apporte du plaisir à ceux qui s'en souviennent ou qui sont dans l'espérance de l'avoir. Et
de fait c'est pour cela encore:

\begin{lieu}
	Qu'il y a un très grand plaisir à se mettre en colère;
\end{lieu}

Comme \textsc{Homère} a fort bien remarqué quand, parlant de cette passion, il a dit:

\begin{emphpar}
	Lorsqu'en nous elle accroît son feu séditieux,

	Le miel n'est pas si doux, ni si délicieux.
\end{emphpar}

À cause que jamais on ne se met en colère contre les personnes de qui, en apparence, il est impossible de se
venger, non plus que contre ceux qui ont incomparablement plus de pouvoir et de crédit que nous; car s'il
arrive que nous nous mettions en colère contre eux, c'est toujours bien moins que contre d'autres.

\bigbreak

Il est certain encore:

\begin{lieu}
	Que la plupart de nos désirs seront accompagnés de plaisir;
\end{lieu}

Car soit qu'alors on se souvienne d'avoir joui autrefois de ce que l'on souhaite, ou qu'on espère d'en jouir
bientôt, toujours en cet état on vient à goûter je ne sais quel plaisir. Par exemple, ceux qui sont travaillés
de la soif pendant une fièvre, soit qu'alors il leur souvienne d'avoir bu autrefois à souhait étant extrêmement
altérés ou qu'ils espèrent de boire encore de même, toujours à cela ils trouvent je ne sais quelle joie. Le même
se remarque en ceux qui sont passionnés d'amour, car soit que dans l'entretien, ils viennent à parler de la
personne qu'ils aiment, soit qu'ils lui écrivent, qu'ils songent à elle ou qu'ils fassent quelque autre chose
qui la regarde, toujours alors ils sont joyeux. Et ce qui fait leur joie en toutes ces rencontres est qu'ayant
cette personne présente à la mémoire, il leur semble que, véritablement, ils sont avec elle. Aussi est-ce à cela
principalement qu'on reconnaît si l'amour commence à prendre empire sur l'esprit quand non seulement on se plaît
à demeurer avec la personne qu'on aime, mais encore lorsque l'affection persiste dans l'absence, et qu'on ne se
peut empêcher d'y songer, et tout de même lorsqu'en étant éloigné, ou s’attriste de ne la plus voir.

\bigbreak

Il faut dire encore:

\begin{lieu}
	Qu'il y aura je ne sais quel plaisir au milieu des plaintes et des soupirs;
\end{lieu}

Car si d'un côté la tristesse nous donne un déplaisir sensible d'avoir perdu pour jamais la personne que nous
pleurons, d'un autre côté, elle nous fait trouver du plaisir et de la consolation à nous la représenter telle
qu'elle était dans toutes ses actions et comme si nous l'avions encore devant nos yeux. Ce qu'\textsc{Homère}
justifie par ces vers:

\begin{emphpar}
	Il dit, et son discours fit lors trouver des charmes

	À pousser des soupirs et répandre des larmes.
\end{emphpar}

\bigbreak

On ne peut pas douter non plus:

\begin{lieu}
	Que la vengeance ne soit très douce;
\end{lieu}

Puisque autant qu'il est fâcheux de ne pouvoir venir à bout de ce qu'on souhaite, autant y a-t-il de douceur à
le voir réussir. Or est-il que ceux qui sont en colère se fâchent toujours extraordinairement lorsqu'ils perdent
l'occasion de se venger, et témoignent au contraire être très contents lorsqu'ils conçoivent le moindre espoir
de vengeance. 

\bigbreak

Il faudra encore conclure:

\begin{lieu}
	Que la victoire sera agréable;
\end{lieu}

Non seulement à ceux qui aiment à vaincre, mais encore à toutes sortes de personnes, puisque alors on s'imagine
qu'on est plus excellent qu'un autre, qui est une chose pour laquelle tous les hommes sont passionnés; à la
vérité, les uns plus et les autres moins.

\bigbreak

Que si, en effet, il se trouve du plaisir à vaincre, il s'ensuivra encore:

\begin{lieu}
	Que toutes sortes de jeux et de divertissements où il y aura défi et partie faite seront très agréables;
\end{lieu}

Et cela sans distinction, soit que la partie ait été faite entre musiciens, athlètes ou savants, puisqu'il arrive
toujours en ces rencontres de remporter la victoire.

\bigbreak

Il en sera de même:

\begin{lieu}
	Des dés, de la paume et des échecs;
\end{lieu}

Et encore:

\begin{lieu}
	Des jeux les plus sérieux et les plus graves;
\end{lieu}

Car quoi qu'ils ne soient pas tous divertissants d'abord, on ne laisse pas néanmoins d'y trouver du plaisir
sitôt qu'on y est accoutumé. Ceux qui, d'abord, apportent du plaisir sont la chasse et toute autre adresse à
prendre des animaux. Ce qui fait donc qu'on trouve du plaisir à toutes ces sortes d'occupations, c'est que
partout où il y a du combat, là, il y a de la victoire.

Et ainsi il se voit encore;

\begin{lieu}
	Que la profession du barreau et la dispute des écoles sont très agréables à ceux qui y réussissent et
	y sont accoutumés.
\end{lieu}

\bigbreak

De plus, il faudra mettre au nombre des choses qui apportent un très grand plaisir:

\begin{lieu}
	L'honneur et la réputation;
\end{lieu}

À cause de l'opinion qu'alors chacun a de soi-même que, véritablement, il est honnête homme et tel qu'on le
publie, à laquelle opinion on se laisse toujours aller d'autant plus aisément qu'on pense que ceux chez qui
on est en estime ne louent que parce qu'en effet c'est leur sentiment et qu'ils croient dire la vérité; tels
que sont des voisins plutôt que ceux qui sont éloignés, et plutôt encore les personnes avec qui on converse
familièrement ou qui sont de connaissance, ou de la même ville que des étrangers et des gens de dehors, et
encore, plutôt ceux qui sont vivants, que ceux qui ne sont pas encore au monde; bref, les personnes sages et
d'une haute prudence plutôt que des étourdis et des gens sans jugement, et enfin, quantité de personnes
plutôt qu'un petit nombre. Parce qu'en effet il y a toujours plus d'apparence que ces personnes-là disent la
vérité que non pas les autres. Et pour montrer que toute sorte d'estime n'est pas également considérable, c'est
qu'on ne se soucie point d'être estimé ni honoré de ceux que tout le monde méprise et dont on ne tient compte,
comme sont les enfants et les bêtes, au moins eu égard simplement à leur estime et aux honneurs qu'ils peuvent
rendre, puisque s'il arrive quelquefois qu'on témoigne faire cas de leur estime et de s'en mettre en peine,
c'est toujours par intérêt, ou pour quelqu autre raison.

\bigbreak

Il faut dire encore:

\begin{lieu}
	Que la possession d'un ami est une chose très douce;
\end{lieu}

À cause qu'il y a beaucoup de plaisir à aimer. Et de fait, qui est l'ivrogne et la personne aimant le vin qui
ne se plaise pas à voir du vin?

D'où il s'ensuit:

\begin{lieu}
	Qu'il y aura aussi du plaisir à être aimé;
\end{lieu}

À cause qu'on ne peut être aimé sans s'imaginer en même temps qu'on a en soi quelque bonne qualité dont tous
ceux qui ont la connaissance sont amateurs. Être aimé, au reste, proprement, veut dire être chéri pour sa
personne, et non point par intérêt.

\bigbreak

\begin{lieu}
	Être admiré encore doit être très agréable;
\end{lieu}

Puisqu'on ne peut pas être admiré sans être honoré en même temps.

\begin{lieu}
	Être flatté aussi, et avoir des flatteurs, plaît encore beaucoup;
\end{lieu}

Car tout flatteur parait en même temps admirateur et ami de celui qu'il flatte.

\bigbreak

On pourra soutenir encore:

\begin{lieu}
	Que faire les mêmes actions très souvent apporte du plaisir;
\end{lieu}

Puisque, comme nous avons déjà remarqué, l'accoutumance est agréable.

Et tout au contraire:

\begin{lieu}
	Qu'il y aura du plaisir a ne pas toujours faire la même chose et à changer parfois;
\end{lieu}

Vu que tout changement semble s'accommoder au dessein de la nature. Et de vrai, faire toujours la même chose
engendre un certain dégoût et témoigne je ne sais quel excès dans l'habitude qu'on a contractée; ce qui a fait
dire à un poète:

\begin{emphpar}
	Le changement nous plaît en toutes choses. 
\end{emphpar}

En effet, c'est par cette raison que tout ce qu'on est quelque temps sans voir, par exemple un homme ou quelque
autre chose, en parait plus agréable. Car outre que ce qu'on n'a pas vu il y a longtemps, apporte du changement
par sa présence, c'est que même il en parait plus rare, à raison qu'on ne le voit pas toujours.

\bigbreak

\begin{lieu}
	Apprendre encore, et avoir de l'admiration pour quelque chose, d'ordinaire, apporte du plaisir;
\end{lieu}

Puisque tout ce qu'on admire fait naître à l'instant le désir de savoir ce que c'est. De sorte qu'on peut
assurer que tout ce qui se fait admirer est souhaitable. En apprenant aussi, on a cet avantage que l'esprit
se perfectionne, et arrive à cet état excellent où il aspire de sa nature.

\bigbreak

Ce sont encore deux choses très agréables:

\begin{lieu}
	que d'obliger et d’être obligé;
\end{lieu}

Puisqu'on ne peut être obligé qu'en même temps on n'acquière ce qu'on désir; et de plus, qu'en obligeant,
on fait voir que non seulement on a de quoi obliger, mais même qu'en ce point on surpasse celui qu'on
oblige, qui sont deux avantages que tous les hommes souhaitent passionnément.

\bigbreak

Or, les mêmes raisons qui font dire qu'il y a du plaisir à obliger, les mêmes font dire encore:

\begin{lieu}
	Qu'il est très agréable de remontrer à son prochain et de le corriger de ses fautes;
\end{lieu}

Comme aussi:

\begin{lieu}
	D'achever quelque chose qui aura été commencé.
\end{lieu}

\bigbreak

Que s'il y a du plaisir à apprendre, à admirer, et autres choses semblables, il s'ensuivra encore:

\begin{lieu}
	Que tout ce qui sera imité parfaitement sera très agréable;
\end{lieu}

Comme sont les ouvrages de peinture, de sculpture, de poésie; en un mot, tout ce qui consiste en imitation,
quand bien même ce qui aurait été imité serait très désagréable en soi. Car enfin, le plaisir qu'on a de
voir une belle imitation ne vient point précisément de ce qui a été imité, mais bien de notre esprit qui
fait alors en lui-même cette réflexion et ce raisonnement \emph{qu'en effet, il n'est rien de plus ressemblant
et qu'on dirait que c'est la chose même,  et non pas une simple représentation}; de sorte qu'en telle rencontre,
il arrive qu'on apprend je ne sais quoi de nouveau.

\bigbreak

\begin{lieu}
	Les revers de fortune encore, et ces événements qui arrivent contre toute sorte d'attente, tels que
	d'ordinaire, représentent les tragédies et les théâtres, doivent apporter du plaisir.
\end{lieu}

Comme aussi:

\begin{lieu}
	De s'être vu en très grand danger et si près de périr que peu s'en soit fallu que cela ne soit arrivé;
\end{lieu}

Car tout ceci est surprenant et donne de l'admiration.

\bigbreak

Et parce que tout ce qui est selon la nature et qui a de la conformité avec elle est agréable, et de plus, que
toutes les choses qui sont de même genre et de même nature sont très conformes entre elles, il s'ensuit encore:

\begin{lieu}
	Que tout ce qui sera de même nature et de même genre, et aussi toutes les choses qui auront de la ressemblance,
	se plairont entre elles pour l'ordinaire. 
\end{lieu}

Par exemple un homme avec un autre homme, un cheval avec un cheval, un enfant avec un enfant, et ainsi du reste.
Et de fait, c'est de la que sont venus tous ces proverbes:

\begin{emphpar}
	Que chacun se plaît avec son pareil;

	Qu'un semblable aime son semblable;

	Qu'une bête connaît une autre bête et la cherche;

	Que la corneille est toujours avec la corneille;
\end{emphpar}

Et beaucoup d'autres.

\bigbreak

Davantage parce que toutes les choses qui se ressemblent, ou qui sont de même genre, se plaisent entre elles,
et qu'il n'y a rien qui nous soit plus semblable ou qui approche plus de notre nature que nous-mêmes, il sera
encore nécessaire de conclure:

\begin{lieu}
	Que tous les hommes généralement, plus ou moins, s'aimeront eux-mêmes;
\end{lieu}

Puisqu'il n'y a rien qui ait plus les qualités de conformité et de ressemblance qu'une personne comparée à
elle-même.

\bigbreak

Or, s'il est vrai que tous les hommes s'aiment eux-mêmes, il s'ensuivra encore:

\begin{lieu}
	Qu'ils aimeront tout ce qui viendra d'eux et y prendront du plaisir;
\end{lieu}

Comme sont leurs ouvrages, leurs discours, leurs raisonnements, ce qui doit encor servir de preuve pour montrer
que d'ordinaire ils aimeront les flatteurs et tous ceux qui auront pour eux de l'amour; enfin, qu'ils seront
jaloux d'honneur, et qu'ils auront une inclination particulière pour leurs enfants; car il n'y a rien qui soit
plus l'ouvrage d'un homme que ceux qu'ils a mis au monde.

\bigbreak

En un mot il s'ensuivra:

\begin{lieu}
	Que tous les hommes seront ravis et auront du plaisir d'achever un ouvrage qui aura été laissé imparfait;
\end{lieu}

Puisque l'ayant achevé, il semblera qu'il leur appartienne tout entier.

\bigbreak

Et parce que l'autorité et le commandement sont les choses du monde les plus agréables, il faut dire encore:

\begin{lieu}
	Qu'il y aura un très grand plaisir à passer pour un homme sage et prudent;
\end{lieu}

Puisque la prudence et la sagesse sont des vertus royales sans lesquelles on est incapable de commander.
La sagesse, au reste, est définie \emph{une science qui éclate par la diversité et le grand nombre des
connaissances, et qui peut rendre raison des effets les plus curieux et les plus propres à donner de
l'admiration}. 

\bigbreak

De plus, parce que d'ordinaire les hommes sont ambitieux et très aises de recevoir de l'honneur, il sera
encore nécessaire de conclure:

\begin{lieu}
	Que non seulement il y aura du plaisir à reprendre autrui et à le corriger de ses fautes, comme nous
	avons déjà remarqué, mais encore à s'occuper aux choses où l'on croit réussir et être plus excellent
	que les autres;
\end{lieu}

Comme a fort bien remarqué \textsc{Euripide} a l'endroit où il dit:

\begin{emphpar}
	Un artisan savant se plaît à son ouvrage,

	Il travaille sans cesse et ne perd point courage.

	Le désir de la gloire et de se surpasser

	Lui fait cent fois le jour son travail repasser.
\end{emphpar}

\bigbreak

D'ailleurs, parce que nous avons montré que le jeu est du nombre des choses qui plaisent, comme encore toutes
sortes de relâches, et aussi le rire, il sera nécessaire encore de tirer cette conséquence:

\begin{lieu}
	Que tout ce qui sera facétieux et ridicule, soit hommes, discours, actions, apportera du plaisir.
\end{lieu}

Quant au ridicule, nous en avons traité à part dans nos livres de la \emph{Poétique}.Voila pour ce qui regarde
les choses qui sont agréables et qui apportent ds plaisir. À l'égard de celles qui peuvent être fâcheuses et qui
attristent, il n'y a qu'à prendre le contraire.

\bigbreak

Nous avons donc fait voir que1s sont les motifs qui d'ordinaire portent les hommes à faire injure à leur prochain.


\begin{comment}

\section{Ceux qui font injure à autrui}

Faisons à présent connaître l'état et le raisonnement de ceux qui se proposent de faire injure à autrui, et de plus,
à quelles personnes ils s'attaquent ordinairement.

\bigbreak

Les hommes donc sont portés à faire injure, en quatre façons. 

\begin{emphpar}
	Ou quand ils croient que ce qu'ils veulent en reprendre est possible et qu'eux-mêmes en pourront venir à
	bout;

	Ou qu'ils pensent qu'après l'avoir fait, on n'en saura rien et qu'ils ne seront point découverts;

	Ou si l'on vient à découvrir que c'est eux, qu'ils n'en seront point punis;

	Enfin, s'ils en sont punis, que la punition n'égalera point le profit qui leur en reviendra, soit à eux en
	particulier, ou à ceux qui les touchent et pour qui ils s'intéressent.
\end{emphpar}

De savoir, maintenant, quelles choses sont possibles à faire ou impossibles, c'est une matière que nous ne
traiterons pas encore si tôt, à cause qu'elle regarde en commun toutes les parties de la rhétorique. 

\subsection{Ceux qui se promettent l'impunité}

Or, entre les personnes qui s'engagent à faire tort à autrui, ceux-là particulièrement croient le pouvoir faire
avec impunité qui sont éloquents ou entreprenants et gens d'exécution, ou qui ont acquis une grande expérience
dans le monde, et vu ou manié une infinité d'affaires; en un mot, ceux qui ont beaucoup d'amis, et qui sont
riches. Mais surtout, ils se promettront l'impunité, s'ils se voient fortifiés de tous les avantages que nous
venons de remarquer, ou du moins quelqu'un de leurs amis, ou de leurs associés, ou même des personnes qui dépendent
d'eux et qui sont à leur service. Car par le moyen de tous ces avantages, non seulement ils exécuteront leur mauvais
dessein, mais encore ils pourront ni être découverts, ni punis. 

Ceux-là encore se promettront l'impunité qui seront amis des personnes mêmes à qui ils voudront faire injure, ou
des juges devant qui ils auront à répondre. Car quant aux amis, il n'est rien de si aisé que de leur faire tort, à
cause qu'ils ne s'en défient pont, joint qu'ils sont plutôt d'accord et réconciliés qu'ils n'ont songé à plaider
ni à faire aucune poursuite en jugement. À l'égard des juges, il est certain encore qu'ils font toujours faveur à
leurs amis; car de deux choses l'une, ou ils les renvoient absous, ou ils ne les condamnent que légèrement.

\subsection{Ceux qui croient qu'ils ne seront point découverts}

Ceux-là aussi auront espérance de n'être point découvert de qui l'apparence sera si trompeuse qu'à juger d'eux par
l'extérieur, jamais on ne les prendrait pour avoir fait ce qu'ils auront fait effectivement, comme quand quelqu'un
en apparence très faible de corps en aura battu un autre outrageusement, qui paraîtra de beaucoup plus fort que lui,
ou quand un gueux aura couché avec une dame de condition, ou un homme très laid avec une fort belle femme. 

Les choses encore qui sont trop en jour, et exposées aux yeux de trop de monde, pourront faire croire à un
méchant homme qu'il ne sera point découvert s'il les prend; la raison est que personne ne s'en donne de garde,
et qu'ordinairement, on ne s'imagine pas qu'il y ait des gens assez hardis pour oser seulement y penser. 

De plus, on croira n'être point découvert si le crime est de telle nature et si énorme qu'on n'ait pas même
connaissance que jamais il ait été commis, puisque c'est une chose à laquelle on ne songe point et dont personne
ne se défie. Car les hommes n'ont point accoutumé de se préparer autrement contre les injures qu'ils font contre
les maladies; pas un ne craignant et ne tâchant d'éviter celles qu'il n'a pas encore éprouvées.

Tout homme encore qui n'aura point d'ennemis, ou au contraire qui en aura beaucoup, croira n'être pas découvert.
Car d'un coté, n'ayant point d'ennemis, il lui sera très facile de surprendre et de faire son coup, parce qu'on
ne se défiera point de lui. D'un autre côté aussi, ayant beaucoup d'ennemis, on aura de la peine à s'imaginer
qu'il ait osé s'attaquer à des personnes qui étaient sans cesse sur leurs gardes; outre que pour sa défense,
il aura cette raison à alléguer qu'il se fût bien empêché d'entreprendre une action de cette qualité, quand
bien même il en aurait eu envie, a cause qu'il devait être soupçonné plutôt qu'un autre.

Ceux-là enfin se pourront persuader de n'être point découverts, en faisant tort à autrui, qui auront moyen: ou
de cacher leur larcin, ou de le détourner, ou bien de lui faire changer de forme et de nature, ou de s'en défaire
promptement.

\subsection{Ceux qui ne craignent pas d'être punis}

D'autres, au contraires seront assurés d'être découverts et poursuivis en justice qui n'entreprendront pas moins
de faire du tort, par exemple s'ils espèrent: ou d'échapper aux juges et de décliner leur juridiction, ou de
faire durer le procès fort longtemps, ou enfin de gagner les juges et de les corrompre.

\bigbreak

D'autres encore verront leur condamnation inévitable, mais parce qu'au plus il n'ira que d'une amende, ils ne s'en
mettront pas en peine, à cause qu'ils sauront les moyens: ou de s'en défendre et de jamais n'en rien payer, ou de
se faire donner un long terme pour y satisfaire, ou bien même que leur pauvreté sera si grande qu'ils n'auront rien
à perdre.

Ceux-là encore ne craindront point d'être condamnés en faisant tort, à qui le larcin promettra présentement: ou
bientôt un profit assuré, ou quelque avantage important et cependant, si on vient à les condamner, qu'ils en
sortiront pour peu de choses, ou même qu'il ne leur en coûtera rien; en tout cas, s'il leur en doit coûter, qu'il
se passera bien du temps avant que d'y avoir satisfait. 

On mettra encore de ce nombre tous ceux qui se proposeront d'acquérir une chose si considérable que la punition,
pour grande quelle soit, supposé même qu'elle arrive, n'égalera jamais l'avantage ni le profit qu'ils en tireront.
Tel est l'avantage que semble promettre la tyrannie à ceux qui ont envie de se rendre maîtres d'un état.

Ceux-là aussi n'appréhenderont pas d'être condamnés pour leur injustice s'ils trouvent qu'il y ait à gagner pour
eux et, quant à la punition, qu'ils en seront quittes pour un affront et pour quelque peu d'injures. Ni tous ceux,
au contraire, dont le crime les fera estimer et leur tournera à honneur, comme quand un homme viendra à venger en
même temps la mort de son père ou de sa mère, ainsi qu'il arriva à Zénon; et cependant que la punition ne pourra
aller au plus qu'à une amende, à un simple bannissement ou à quelque autre peine semblable. Car il est certain
qu'en ces deux rencontres, ces personnes-là seront portées à exécuter leur mauvais dessein; quoi qu'entre elles,
il y ait cette différence que les dernières sont louables pour leurs mœurs et les autres dignes de punition.

Ceux-là encore volontiers se hasarderont à faire tort, qui jamais n'auront été pris sur le fait, ni découverts,
ni punis; et pareillement ceux qui auront manqué plusieurs fois leur coup. Car il en prend ici comme à la
guerre, où souvent, il arrive aux vaincus de tenter la fortune de nouveau et de retourner au combat.

Ceux-là encore seront hardis qui auront espérance de jouir présentement: ou de tel plaisir en particulier, ou
d'avoir un tel profit à cause que s'il y a quelque chose à souffrir en punition ou quelque perte à faire, ce
ne sera qu'après. Tels sont d'ordinaire les incontinents et les débauchés. L'incontinence, au reste est un vice
qui regarde les choses où nous portent toutes les passions agréables et le dérèglement de la convoitise.

Il s'en trouve d'autres qui font le contraire de ceux-ci: d'abord ils préfèrent d'endurer quelque chose, ou de
faire quelque perte, parce qu'ils espèrent à l'avenir ou d'avoir un établissement assuré, ou de jouir d'un plaisir
très durable. Et c'est ce que font ordinairement ceux qui ont de la prudence, et qui ne sont pas adonnés a leur
plaisir. 

D'autres encore ne se soucieront pas qu'on sache que c'est eux qui ont fait tort, à cause qu'ils ne paraîtront
l'avoir fait que par malheur: ou par nécessité, ou dans un transport et un premier mouvement, ou par accoutumance.
En un mot, parce qu'ils paraîtront avoir plutôt failli qu'offensé malicieusement.

Ceux-là encore seront de ce nombre qui espéreront en la bonté des juges et qu'on ne les traitera pas à là rigueur,
comme aussi ceux qui seront pauvres. Or il y a deux sortes de pauvres dans le monde: les uns le sont des choses
nécessaires à l'entretien de la vie, comme ceux qui mendient, et les autres des choses superflues, comme la plupart
des riches. 

Enfin, ceux-là ne craindront point de faire du tort qui seront en très bonne estime; ni ceux, au contraire qui
seront tout à fait perdus de réputation. Car quant à ceux qui auront de l'estime, jamais on ne voudra croire que
ce soit eux, et pour les autres, ils n'en seront pas plus décriés.

Voila ce que nous avions a dire touchant les personnes qui entreprennent de faire tort et injure à autrui. Voyons
maintenant ceux a qui on s'attaque ordinairement.

\subsection{Les personnes à qui d'ordinaire on fait tort}

Les méchants, donc, d'ordinaire, s'attaquent aux personnes qui possèdent les choses qu'ils n'ont pas et dont ils
ont besoin; soit qu'elles soient nécessaires à l'entretien de la vie, ou superflues, ou seulement pour la jouissance
et le plaisir.

Ils attaquent encore également leurs voisins et ceux qui sont d'un pays éloigné. Leurs voisins? Parce que leur coup
est bientôt fait. Les étrangers? À cause que, d'ordinaire, la vengeance en est tardive et qu'il leur faut beaucoup de
temps pour tirer raison du tort qu'ils ont reçu. Tels sont ceux par exemple qui attendent les Carthaginois au passage,
afin de les piller.

On fait tort encore ordinairement aux personnes négligentes et qui ne se tiennent point sur leurs gardes, ou qui sont
si simples qu'on leur peut faire accroire tout ce qu'on veut, pour ce qu'il y a lieu de s'imaginer qu'on ne sera point
découvert.

On s'adresse encore assez souvent à ceux qui sont d'un naturel lâche et qui aiment à vivre en repos; telles personnes
n'étant pas d'humeur à s'embarrasser d'un procès, à cause que la poursuite en est difficile, et qu'il faut être agissant
pour en venir à bout. 

Il en est de même de ceux qui ont beaucoup de pudeur, parce qu'ils ont l'honneur en recommandation et seraient
honteux de paraître en jugement pour un léger intérêt, et de plaider pour peu de chose.

On s'attaque encore, d'ordinaire, aux personnes que d'autres ont déjà attaquées ou offensées plusieurs fois, sans
que jamais elles en aient fait de poursuite, comme étant du nombre de ceux que le proverbe appelle \emph{la proie
des Mysiens}.

Tout ceux encore à qui une personne n'a jamais fait tort, et ceux au contraire à qui, plusieurs fois déjà, elle en
a fait sont en grand danger d'en être attaqués, à cause que ni les uns ni les autres ne se tiennent point sur leurs
gardes. Ceux-ci, parce qu'ils ne croient pas qu'elle leur en veuille plus faire; les autres, parce qu'elle ne leur
en a pas encore fait.

On se propose aussi de faire injure à ceux qui ont déjà été traduits en justice pour plusieurs crimes, ou à qui il
est très facile de faire faire le procès, à cause que telles gens n'oseront pas s'en plaindre, soit pour la crainte
qu'ils auront des juges, soit pour n'être pas en état d'être crus; ce qui se peut dire encore de tous ceux qui sont
haïs ou enviés de tout le monde.

D'ordinaire, encore, on s'attaque à ceux contre qui on a quelque prétexte et quelque raison spécieuse, soit qu'on
aille rechercher l'histoire de leurs ancêtres et qu'on déterre des querelles mortes et ensevelies, soit qu'on se
plaigne d'eux en particulier ou de quelqu'un de leurs amis; par exemple: ou pour être en état d'en recevoir
présentement du tort, ou pour en avoir déjà reçu plusieurs fois, soit en sa propre personne, soit en celle de ses
amis ou de ceux de qui on prend les intérêts. Car comme dit fort bien le proverbe: \emph{la malice n'a besoin que
de prétexte}.

On fait tort encore indifféremment à ses ennemis et à ses amis propres. À ses amis? Parce qu'il est très facile de
le faire. À ses ennemis? À cause qu'il y a du plaisir.

Il en est de même de ceux qui n'ont point d'amis, et des autre qui ne sont ni éloquents ni gens d'exécution, car:
ou ces personnes-là n'auront pas seulement la hardiesse de poursuivre en justice ceux qui leur auront fait tort,
ou, si elles le font, elles s'accorderont bientôt, ou même, ne gagneront rien à plaider.

Ceux-là aussi seront sujets à être attaqués, à qui il n'est pas avantageux de s'arrêter longtemps en un même lieu
dans l'attente qu'un procès soit terminé, ou qu'ils soient dédommagés et remboursés de leurs frais. Tels sont,
d'ordinaire, les personnes de dehors, et ceux qui n'ont d'autre revenu que le travail de leurs mains, car, pour
peu de chose, on compose avec eux, étant facile de les contenter.

On attaque encore volontiers ceux qui ont fait beaucoup de tort en leur vie, ou qui ont fait la même injure à
d'autres qu'on a dessein de leur faire; à cause qu'il ne semble pas que ce soit une injustice de traiter un méchant
homme de la même sorte qu'il a accoutumé de traiter les autres, comme quand quelqu'un qui est connu pour un
querelleur et pour battre ordinairement viendra lui-même à être très bien battu.

On tâche aussi de faire injure a ceux de qui, autrefois, on a reçu quelque déplaisir, ou qui ont eu dessein d'en
faire, ou qui ne manquent pas de volonté pour cela, ou même qui s'y préparent et qui font tout ce qu'ils peuvent
pour en venir à bout. Car non seulement on y trouvera du plaisir, mais encore cela fera honneur, outre qu'il ne
semblera pas qu'on ait fait une injustice.

C'est encore une occasion de faire injure à quelqu'un si en attaquant on est assuré de faire une chose agréable
et qui plaira extrêmement: ou à ses amis, ou à ceux qu'on estime beaucoup, ou aux personnes pour qui on a de
l'amour, ou à ses maîtres; en un mot, à tous ceux dont on dépend ou de qui on attend quelque faveur.

On cherche encore à nuire aux personnes qu'on a autrefois accusées de quelque crime, ou à l’amitié desquelles
on a renoncé, témoin ce que fit Calippe contre Dion. Et ce qui donne d'autant plus de hardiesse alors, c'est
que même il ne semble pas qu'on fasse une injustice.

On attaque encore les personnes qu'on sait que d'autres sont tout prêts d'attaquer si on ne les prévient, comme n'y
ayant plus lieu de délibérer si on le doit faire ou non. De là vient qu'Ænésidème envoya des présents à Gélon pour
l'avoir prévenu en la réduction de certains peuples qu'il avait dessein d'assujettir lui-même.

On s'adresse encore à ceux à qui on ne doit faire qu'une seule fois du tort pour être en état de leur faire après
beaucoup de bien, à cause qu'il sera facile alors de guérir le mal, et les récompenser de leur perte. C'était sur
ce fondement que Jason le Thessalien avait accoutumé de dire \emph{qu'il est bon quelquefois de faire un peu de mal,
pour être en état après de faire beaucoup de bien}.

\subsection{Les injustices qui se font d'ordinaire}

Pour ce qui est des injustices, d'ordinaire, on se laisse aller à celles que la plupart ou tout le monde fait, vu
qu'alors on se persuade qu'on aura sa grâce aisément.

On cherche encore a faire tort dans les choses qu'il est facile de cacher. Or ces choses-là sont de plusieurs sortes.
Les unes se consument en peu de temps, comme tout ce qui est bon à manger; d'autres sont aisées à déguiser, soit
qu'on leur donne une nouvelle figure, ou qu'on leur fasse changer de couleur, ou qu'on les mêle; d'autres peuvent
être détournées en divers lieux, comme tout ce qui est facile à transporter ou qui tient peu de place; et quelques-unes
enfin sont telles que, comme celui qui les veut dérober en a beaucoup chez lui toutes semblables, jamais on ne
pourra les reconnaître lorsqu'elles seront ensemble. 

On fait encore injure dans les choses qu'on sait être honteuses à dire aux personnes mêmes à qui l'injure est faite,
comme quand on a abusé de la femme de quelqu'un, ou que lui-même ou ses enfants ont été contraints de céder à la
brutalité d'un infâme. 

On fait tort enfin et injure dans les choses pour lesquelles on peut intenter des procès sans se décrier et passer
pour chicaneur: ou à cause qu'elles sont de peu d'importance, ou parce que ce sont des fautes pardonnables. 

C'est a peu prés ce qui se peut dire sur cette matière, soit à l'égard de ceux qui font tort, ou des choses qu'ils
recherchaient, ou des personnes qu'ils attaquent; soit à l'égard des motifs et des raisons qui d'ordinaire les portent
à exécuter leur mauvais dessein.

\input{chapitres/livre1chapitre13.tex}
\input{chapitres/livre1chapitre14.tex}
\input{chapitres/livre1chapitre15.tex}


\part{Le second livre}

\fancyhead[RO]{Le second livre}

\input{chapitres/livre2chapitre1.tex}
\input{chapitres/livre2chapitre2.tex}
\input{chapitres/livre2chapitre3.tex}
\input{chapitres/livre2chapitre4.tex}
\input{chapitres/livre2chapitre5.tex}
\input{chapitres/livre2chapitre6.tex}
\input{chapitres/livre2chapitre7.tex}
\input{chapitres/livre2chapitre8.tex}
\input{chapitres/livre2chapitre9.tex}
\input{chapitres/livre2chapitre10.tex}
\input{chapitres/livre2chapitre11.tex}
\input{chapitres/livre2chapitre12.tex}
\input{chapitres/livre2chapitre13.tex}
\input{chapitres/livre2chapitre14.tex}
\input{chapitres/livre2chapitre15.tex}
\input{chapitres/livre2chapitre16.tex}
\input{chapitres/livre2chapitre17.tex}
\input{chapitres/livre2chapitre18.tex}
\input{chapitres/livre2chapitre19.tex}
\input{chapitres/livre2chapitre20.tex}
\input{chapitres/livre2chapitre21.tex}
\input{chapitres/livre2chapitre22.tex}
\input{chapitres/livre2chapitre23.tex}
\input{chapitres/livre2chapitre24.tex}
\input{chapitres/livre2chapitre25.tex}
\input{chapitres/livre2chapitre26.tex}


\part{Le troisième livre}

\fancyhead[RO]{Le troisième livre}

\input{chapitres/livre3chapitre1.tex}
\input{chapitres/livre3chapitre2.tex}
\input{chapitres/livre3chapitre3.tex}
\input{chapitres/livre3chapitre4.tex}
\input{chapitres/livre3chapitre5.tex}
\input{chapitres/livre3chapitre6.tex}
\input{chapitres/livre3chapitre7.tex}
\input{chapitres/livre3chapitre8.tex}
\input{chapitres/livre3chapitre9.tex}
\input{chapitres/livre3chapitre10.tex}
\input{chapitres/livre3chapitre11.tex}
\input{chapitres/livre3chapitre12.tex}
\input{chapitres/livre3chapitre13.tex}
\input{chapitres/livre3chapitre14.tex}
\input{chapitres/livre3chapitre15.tex}
\input{chapitres/livre3chapitre16.tex}
\input{chapitres/livre3chapitre17.tex}
\input{chapitres/livre3chapitre18.tex}
\input{chapitres/livre3chapitre19.tex}

\end{comment}


\end{document}
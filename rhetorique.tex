
%\RequirePackage{luatex85}

\documentclass[11pt]{book}

\usepackage[utf8]{inputenc}
\usepackage[T1]{fontenc}

\usepackage{fontspec}

\usepackage[a5paper,
           inner=24mm,outer=15mm,top=24mm,bottom=24mm,
           includefoot,
            includehead
           ]{geometry}

\pagewidth=148mm
\pageheight=210mm

 \usepackage{comment}

\usepackage{polyglossia}
\setdefaultlanguage{french}

\usepackage{import}

\renewcommand{\thesubsection}{}
\renewcommand{\thesection}{\Roman{section}}
\renewcommand{\thechapter}{}
\renewcommand{\thepart}{}

\usepackage{titlesec}

\titleformat
{\subsection}
[hang]
{}
{}
{0pt}
{\Large\itshape}

\titleformat
{\section} % command
[display] % shape
{\bfseries\LARGE} % format
{\titlerule\vspace{1pc}%
\MakeUppercase{\chaptertitlename} \thesection} % label
{0pc} % sep
{} % before-code

\titleclass{\chapter}{top}
\titleformat
{\chapter}
[hang]
{}
{}
{0pt}
{\Huge\MakeUppercase}
\titlespacing*{\chapter}{0pt}{40pt}{40pt}

\titleclass{\part}{top}
\titleformat
{\part}
[hang]
{\Huge\bfseries}
{}
{0pt}
{\Huge\MakeUppercase}
\titlespacing*{\part}{0pt}{40pt}{40pt}

\newenvironment{emphpar}
    { 
    
    
    \itshape \leftskip=1cm
       }
    { 
    
    \leftskip=0cm
       }

\title{LA 
RHETORIQUE 
D'ARISTOTE}

\author{Traduite en français par M. \textsc{Cassandre}}

\begin{document}

\maketitle

Lettre de M. \textsc{D'Ablancourt} au traducteur, sur la première édition de cet ouvrage. 

Monsieur

Je ne vous ai pas donné mon \textsc{Lucien} comme une 
bonne chose, mais comme une chose due: Car après 
la faveur que vous m'aviez faite de me donner votre 
belle traduction, j'étais comme obligé de vous
présenter la mienne, quelque mauvaise qu'elle fût. Je ne 
pretends donc point qu'elle vous doive servir de modèle,
elle n'est pas assez exacte pour cela, et ne peut porter
le nom de traduction qu'improprement et parce 
qu'on ne lui en peut donner d'autre. Ce n'est pas que 
je croie avoir gâté ce que j'ai altéré, mais c'est afin 
qu'on ne croie pas que je veuille faire passer pour des 
règles de traduction les libertés que j'ai prises. C'est la 
vôtre, Monsieur, qui peut supporter la plus exacte censure, 
et qui n'a rien ôté à son auteur que l'obscurité. 
C'est pourquoi je vous conjure d'en entreprendre de 
nouvelles, et d'achever la traduction de ce grand 
homme qui est si peu intelligible dans les autres; car je 
soutiens qu'Aristote est beaucoup plus clair chez lui que 
chez les traducteurs latins, et que souvent il faut 
lire l'original pour entendre la version. Le public vous 
sera obligé du travail que vous entreprendrez, à cause 
du profit qui lui en reviendra, et moi je ferai gloire de 
vous en avoir donné l'envie, comme celui qui suis, etc.

\textsc{Perrot D'Ablancourt}

A Vitry le 9 novembre 1654

\tableofcontents


\part{Le premier livre}
\section{Servant de préface à tout l'ouvrage}

\subsection{Que la rhétorique et la dialectique se  ressemblent}

La rhétorique et la dialectique  ont beaucoup de rapport, car  toutes deux traitent de matières, qui, pour être communes, tombent
en quelque façon sous la connaissance de tout le monde, et ne sont renfermées dans les bornes d'aucune science particulière: d'où
vient aussi qu'il n'y a personne qui n'ait quelque usage de l'une et de l'autre, puisque chacun, selon sa portée et jusqu'à un
certain point, tâche d'examiner et de soutenir une raison, d'accuser et de défendre.

\subsection{Que la rhétorique est un art}

Parmi le peuple, quelques-uns réussissent à ces choses par hasard, d'autres parce qu'ils s'y sont habitués. Que si cela se fait
de toutes les deux façons, sans doute on peut avoir des règles là dessus, et trouver une méthode assurée pour y réussir toujours;
puisqu'enfin il y a lieu de découvrir la cause pourquoi, et ceux qui font ceci par un pur hasard, et ceux qui le font par habitude,
arrivent au but qu'ils se proposent; or on m'avouera que c'est à l'art à donner ces règles, et que c'est là proprement son ouvrage.

\subsection{Que l'adresse principale de la rhétorique consiste aux preuves.}

Tous ceux au reste qui jusques à présent ont écrit de la rhétorique, n'ont presque rien fait de ce qu'il fallait faire, parce que
toute l'adresse de cet art est renfermée dans la preuve, le reste n'en est que l'accessoire. Cependant ils ne  parlent point des
enthymèmes et des arguments, qui font tout le corps de la preuve, et se sont amusés à des choses éloignées de leur art, et purement
étrangères: Car l'invective, la compassion, la colère et les autres passions de cette nature dont ils traitent fort au long, ne
sont point du fait de l'orateur, mais regardent le juge; de sorte que si dans toutes les justices on faisait son devoir et que
partout on se gouvernât ainsi qu'en quelques républiques, et particulièrement les mieux policées, il se trouverait que ces gens-là,
lorsqu'ils voudraient parler en public, n'auraient rien a dire. Ce n'est pas que cela ne passe pour un abus, et qu'on ne croie
qu'il devrait y avoir des lois pour s'opposer à cette licence; mais peu de gens le mettent en pratique, et ce n'est qu'en certains
lieux qu'il est expressément défendu aux orateurs de sortir de leur sujet et de ne rien dire d'inutile, comme à Athènes, encore
n'est-ce que pour les jugements qui se rendent dans l'Aréopage. Et certainement ceux qui le font, ont grande raison d'en user ainsi,
puisque jamais il ne faut pervertir un juge, ni le porter ou à la compassion, ou à la colère, ou à l'envie; vu que c'est faire la même
chose que  celui qui courberait une règle dont il se voudrait servir. D'ailleurs, il n'y a personne qui ne voie que l'emploi de celui
qui plaide et à quoi il doit s'étudier, est de montrer simplement que la chose dont il s'agit est ou n'est pas, qu'elle a été faite ou
ne l'a pas été, car de savoir si elle  est de conséquence ou non, si véritablement elle est juste ou injuste, au cas que le législateur
ne s'en soit pas expliqué, c'est au juge à le connaître, sans l'apprendre de ceux qui parlent devant lui. 

\medbreak

On voit par là qu'il serait à souhaiter que les lois sagement établies fussent si exactes qu'elles remarquassent jusqu'aux moindres
circonstances, afin de laisser peu de choses à la discrétion des juges; et cela pour plusieurs raisons.

Premièrement, \emph{eu égard aux personnes}, attendu qu'il n'est pas si aisé de trouver d'habiles gens, et que pour un ou deux qu'on
rencontre capables de faire des lois et d'exercer la judicature, il y  en a cent qui ne le sont pas.

Secondement, \emph{à raison du temps}, vu que les lois dans leur établissement dépendent d'une longue observation et de l'expérience de
plusieurs siècles, au lieu que les jugements qui se rendent se font sur le champ; de sorte qu'en cet état, il est difficile aux juges et
à ceux qui délibèrent dans les grandes assemblées, de satisfaire  entièrement à l'intérêt public, et à celui des parties.

La dernière raison et la plus importante, est tirée \emph{des choses mêmes}; puisqu'enfin tout législateur n'a point à prononcer sur des
matières particulières, ni pour des personnes qui soient présentes; mais en général, et pour des personnes qui ne sont pas encore: Le juge
au contraire et ceux qui délibèrent ne connaissent que des faits  particuliers, où le plus souvent leur propre intérêt se rencontre; et ne
regardant que des personnes présentes, pour qui tantôt ils ont de l'amour, et tantôt de la haine. D'où vient qu'alors la passion les
aveugle, et les empêche de bien voir la vérité. 

Il est donc à propos, comme nous venons de remarquer, que le législateur laisse peu de chose au pouvoir des juges, afin, s'il est possible,
qu'ils n'aient qu'a examiner si ce qu'on leur dit est ou n'est pas, s'il arrivera ou s'il ne doit point arriver, qui sont des cas qu'un
législateur ne peut prévoir et que, nécessairement, il faut laisser à la connaissance des juges. 

\medbreak

Que si cela est ainsi, l'on voit manifestement que ceux-là sortent du sujet de la rhétorique qui enseignent, par exemple, comment il faut
faire un exorde, une narration, et ainsi de chacune des autres parties d'un discours; parce que tout ce qu'ils font, en telle rencontre, ne
tend qu'à altérer l'esprit du juge et ne montre point en quoi consiste l'artifice de la preuve, qui est de cultiver le raisonnement et
de rendre un homme fort en enthymèmes. Aussi est-ce pour cette considération qu'encore qu'il y ait deux parties principales dans la rhétorique,
dont l'une regarde les délibérations et l'autre les matières du Barreau --- et même que la partie qui sert à délibérer soit beaucoup plus noble
et plus politique que celle qui s'arrête seulement à examiner les clauses d'un simple contrat --- tous néanmoins ont abandonné la délibération,
sans en dire le moindre mot, et pour l'autre, c'est à qui en traitera plus au  long, à qui donnera plus de préceptes. Et la raison qui les
y a portés est qu'il est peu avantageux de sortir de son sujet en parlant dans un conseil, cette partie donnant beaucoup moins d'entrée à la
malice et à la finesse que ne fait le plaidoyer, à cause que l'intérêt qu'elle regarde est un intérêt commun, et que celui qui écoute est juge
en sa propre cause; de sorte qu'ici l'orateur n'a qu'à montrer simplement que ce qu'il dit est véritable. Il en va autrement du barreau, où il ne
suffit pas de prouver que la chose est, mais encore  qu'il est bon de gagner l'esprit de l'auditeur, et de le faire tourner de son côté; vu qu'il
s'agit là de l'intérêt d'autrui et qu'il n'a point à prononcer sur des choses qui le touchent. Ainsi, comme il ne regarde que sa propre
satisfaction, et qu'il n'écoute que pour faire faveur, il se laisse aisément emporter aux discours de ceux qui plaident et ne fait plus l'office
de juge. C'est aussi pour cela, comme nous disons auparavant, qu'en beaucoup de lieu, la loi défend aux orateurs de parler hors de leur sujet; ce
qu'il n'a point été nécessaire de faire dans la délibération, à cause que c'est une chose qui s'observe là assez d'elle-même.

\subsection{Que les plus fortes preuves de la rhétorique dépendent des enthymèmes}

Donc puisqu'il est certain:

\begin{itemize}
\item  Que tout l'artifice de la rhétorique consiste dans la preuve,
\end{itemize}

De plus:

\begin{itemize}
\item  Que la preuve est une sorte de démonstration,
\item  Que le plus puissant moyen qu'il y ait pour démontrer, c'est l'enthymème,
\end{itemize}

Enfin:

\begin{itemize}
\item  Que l'enthymème est une manière de syllogisme,
\end{itemize}

En un mot:

\begin{itemize}
\item  Puisque c'est ou à la dialectique toute entière, ou à l'une de ses parties, à traiter du syllogisme pleinement,
\end{itemize}

Il s'ensuit:

\begin{itemize}
\item  Que quiconque sera bon dialecticien, c'est à dire qui saura comment le syllogisme se fait et de quelles propositions il est composé,
             celui-là encore aisément pourra faire des enthymème; n'ayant plus qu'à observer sur quelles matières ils s'appliquent, et en quoi ils
             sont différents des syllogismes de la logique.
\end{itemize}

Et cela d'autant que c'est à la même faculté qui s'attache au \emph{vrai} qui est l'objet du syllogisme, de connaître encore le \emph{vraisemblable},
qui est l'objet de l'enthymème, joint que tous les hommes naturellement sont assez portés aux sciences et à la connaissance du vrai et qu'assez
souvent, hors ce qui regarde les sciences, ils découvrent la vérité en beaucoup de choses. Tellement que pour tirer de simples conjectures, et
découvrir la vraisemblance dans les matières douteuses, il ne faut point d'autre adresse ni d'autre lumière que celles qui dans les matières
certaines et infaillibles nous font raisonner régulièrement et trouver toujours la vérité.

\medbreak

Il paraît donc, évidemment, que ceux qui ont écrit de la rhétorique jusques à présent n'ont point traité son sujet; et nous avons dit pourquoi ils ont
quitté le \emph{genre délibératif} pour s'attacher au \emph{judiciaire}.

\subsection{Que la rhétorique est utile}

On ne peut pas douter que la rhétorique ne soit utile, puisqu'elle a pour but de faire rendre la justice, et de faire connaître la vérité, qui est
une chose avantageuse et toute autre que de faire le contraire. Aussi toutes les fois qu'on ne juge pas comme il faut, cela n'arrive que parce que
l'injustice et le mensonge ont prévalu sur la justice et sur la vérité, ce qui mérite punition. 

De plus, la rhétorique est de telle conséquence que, quand nous serions les plus savants du monde, néanmoins, il nous serait difficile en parlant à
certaines personnes de les persuader, à cause que les sciences ont une façon particulière de s'expliquer et certains termes, dont il est impossible de
se servir devant des ignorants. De sorte que, pour se faire entendre à eux et pour les persuader, il faut avoir recours à des notions générales, ou
\emph{lieux communs}, ainsi que nous avons remarqué dans nos Livres des \emph{Topiques}, en traitant \emph{de la manière de parler au peuple}.

Un troisième avantage de la rhétorique est qu'il faut être capable de persuader les deux partis contraires, de même que dans la dialectique on doit
savoir argumenter de part et d'autre; non pas à la vérité qu'il faille faire tous les deux, car jamais on ne doit persuader ce qui est mauvais, mais --- la
chose est importante --- afin qu'au moins on n'ignore pas comment cela se fait, et qu'en même temps on puisse répondre à ceux qui voudraient s'en servir
pour favoriser l'injustice. Or est-il que de tous les arts il n'y a que la dialectique et la rhétorique qui fassent profession de défendre les deux partis
contraires. Ce n'est pas pourtant qu'il faille croire que les matières qui se traitent en telles rencontres soient également probables, puisqu'absolument
parlant, tout ce qui est véritable et meilleur de soi est aussi, et plus aisé à être prouvé, et plus capable de persuader.

Après tout, il serait ridicule de s'imaginer qu'il y eut de la honte à ne se pouvoir aider de son corps, et qu'il n'y en eût point à être privé du secours
de la parole, dont l'usage, bien plus que celui du corps, appartient à l'homme naturellement. 

De dire que la rhétorique peut beaucoup nuire si l'on s'en veut mal servir, c'est une objection qui regarde en commun toutes les bonnes choses et les plus
utiles mêmes, excepté la vertu; par exemple, la force, la santé, les richesses, les armes; puisque selon l'usage qu'on en fera, bon ou mauvais, il en viendra
un grand mal ou un grand bien. 

De tout ce que nous avons dit jusques ici, il se voit en premier lieu que la rhétorique est utile, et qu'elle n'a point un sujet particulier ni déterminé non
plus que la dialectique.

En second lieu que l'ouvrage de la rhétorique ne consiste point à persuader absolument, mais à découvrir en chaque chose ce qui est capable de le faire, et en
cela convient-elle avec tous les autres arts. Par exemple, la médecine ne promet pas de guérir infailliblement, mais seulement de contribuer à la santé autant
qu'il est possible; puisqu'on ne laisse pas de bien traiter certains malades, encore que la santé ne leur puisse être rendue.

Enfin, il se voit que c'est à la rhétorique à considérer également, et ce qui est capable de persuader en effet, et ce qui ne le peut faire qu'en apparence;
comme c'est à la dialectique à traiter du syllogisme apparent et du véritable. Je dis que c'est à la dialectique à traiter du syllogisme apparent, afin qu'on
ne croie pas que cela soit réservé au sophiste; vu que ce qui donne la qualité de \emph{sophiste} à un homme n'est point cette connaissance et cette adresse de
pouvoir user de semblables arguments, mais bien le but qu'il se propose et le dessein qu'il a de n'argumenter que pour tromper. Véritablement, il y a cette
différence entre la rhétorique et la dialectique quant à ce point que dans la rhétorique, autant est \emph{orateur} celui qui n'emploie que de faux arguments, et
qui n'en veut point employer d'autres, que celui qui ne se sert que de bons, et qui ne tâche qu'à faire connaître la vérité. Pour la dialectique, il n'en va pas
ainsi, puisque là, le dessein de tromper et de ne s'attacher qu'à de vaines subtilités est proprement ce que nous appelions être \emph{sophistes}; au lieu que le
\emph{dialecticien} ne s'attache qu'à l'art et à la vérité.

Mais traitons tout de bon maintenant de la rhétorique, et voyons de quelle façon nous pourrons venir à bout des choses que nous avons proposées. Comme si donc nous
n'avions encore rien dit de cet art, commençons par sa définition, et ensuite, nous examinerons le reste.



\section{Éléments de la rhétorique}

\subsection{Ce que c'est que la rhétorique}

Posons que la rhétorique est \emph{un art ou une faculté qui considère en chaque sujet ce qui est capable de le persuader}; car il n'est point d'art
qui fasse la même chose, puisque tous les autres arts et toutes les autres facultés ne traitent que leur sujet et ne persuadent que là-dessus. Par
exemple, la médecine ne raisonne et ne persuade que sur ce qui regarde la santé et la maladie, la géométrie que sur les changements et les différences
remarquables qui arrivent aux grandeurs, et enfin, l'arithmétique, que sur ce qui touche le nombre; Ainsi en est-il des autres arts et des autres
sciences. Mais pour la rhétorique, quelque sujet qu'on lui propose, elle a droit, pour ainsi dire, d'y voir ce qui peut persuader. Aussi avons nous
remarqué qu'elle n'a point un sujet particulier ni déterminé sur lequel elle travaille. 

\subsection{Qualité des preuves de la rhétorique}

La rhétorique a deux sortes de preuves, les unes sont \emph{artificielles} et les autres \emph{sans artifice}. J'appelle preuves sans artifice celles
qui ne dépendent point de notre industrie, mais que nous trouvons toutes faites, comme sont les témoins, les réponses faites à la torture, les contrats
et autres choses semblables. Et je nomme artificielles toutes celles que nous pouvons trouver de nous-mêmes et par les règles de la rhétorique, de
sorte qu'il faut inventer celles-ci, au lieu qu'on se sert simplement des autres.

Pour les preuves \emph{artificielles}, il s'en trouve de trois espèces.

La première est fondée sur les mœurs et sur la bonne opinion qu'on a de celui qui parle.

La seconde vient de la disposition de l'auditeur, et d'avoir préparé son esprit d'une certaine façon.

Et la dernière enfin naît du discours, soit que véritablement, on ait démontré son sujet, ou seulement en apparence.

\bigbreak

L'orateur persuade à l'occasion de sa personne et de ses mœurs, lorsqu'il parle de manière qu'il se rend digne de foi; car la vertu est d'un tel crédit
qu'absolument, nous ajoutons toujours plus de foi et plutôt aux gens de bien qu'aux autres, et cela généralement en tout, mais particulièrement dans les
matières douteuses, et où l'esprit de part et d'autre ne voit point de raison qu'il puisse suivre avec sûreté; vu qu'alors nous nous abandonnons à eux
entièrement et croyons tout ce qu'ils disent. Or, il faut remarquer que ce crédit doit aussi venir de l'adresse de notre discours, et non pas simplement
de la préoccupation de l'auditeur, ni parce qu'il avait déjà cette bonne opinion de nous avant que de nous écouter; car enfin on ne doit point s'arrêter
à ce que disent quelques-uns de ceux qui ont traité de la rhétorique, qui, à propos des bonnes mœurs et de cette probité qui doit éclater dans le discours
de l'orateur, soutiennent qu'absolument, elle est inutile et ne contribue en rien à gagner l'esprit; mais tant s'en faut que cela soit, que même, c'est un
des plus forts et des plus puissants moyens qu'il y ait pour persuader.

On persuade à l'occasion de ses Auditeurs  lorsque par le discours, on les porte à quelque passion; aussi jugeons nous bien autrement quand nous sommes
tristes que quand nous sommes joyeux, et bien autrement quand nous aimons que quand nous avons de la haine; Or, comme il a déjà été dit, c'est la seule
chose que tous les rhétoriciens d'aujourd'hui se sont efforcés de traiter, mais il en sera parlé plus particulièrement quand nous serons au discours des
passions. 

Enfin, on persuade par la force du discours lorsque employant tout ce qui peut servir à prouver le sujet que l'on traite, on fait voir que la chose dont
il s'agit est véritable en effet, ou en apparence.

\bigbreak

Que si les preuves \emph{artificielles} dépendent de ces trois points, il est certain qu'il faudra s'étudier à trois choses. Premièrement, à savoir faire
des syllogismes. Secondement, à connaître les mœurs et les vertus de chacun. En dernier lieu, à connaître les passions. Par exemple, quelle est la nature
de chaque passion en particulier, sa différence, ce qui la fait naître, et comment on le peut faire: de sorte qu'il se voie  par là que la rhétorique est
comme un germe et un rejeton, non seulement, de la dialectique, mais encore de cette partie de la morale qu'on peut avec raison nommer \emph{politique}. Et
de fait, c'est pour cela que la rhétorique affecte de paraître sous un habit emprunté et de passer pour politique; aussi bien que ceux qui font profession
d'orateur, qui d'ordinaire se flattent de cette vanité en partie par présomption, en partie par ignorance, et en partie pour d'autres considérations
humaines.

Je dis que la rhétorique est comme un rejeton de la dialectique, parce qu'elle en est une partie et une image, ainsi que nous avons remarqué dès le
commencement; vu que ni l'une ni l'autre, ne sont point des sciences qui s'attachent à un sujet particulier, mais bien certaines facultés qui cherchent
à trouver des raisons dans toutes sortes de matières. Mais c'est assez parler de leur pouvoir, et du rapport qu'elles ont.

\subsection{De l'exemple et de l'enthymème, de leur rapport avec le syllogisme et l'induction}

Quant aux preuves qui en effet démontrent  une chose ou qui semblent la démontrer. Tout ainsi que pour démontrer dans la dialectique, l'on se sert toujours
de l'induction et du syllogisme, soit véritable ou apparent, de même, pour démontrer dans la rhétorique, l'on se sert  toujours de \emph{l'exemple}, qui est
la même chose que \emph{l'induction}, et encore de \emph{l'enthymème} qui répond au \emph{syllogisme}. Aussi est-ce pour cette  raison que je nomme l'enthymème
et l'exemple, l'un le syllogisme, et l'autre l'induction de la rhétorique.

Et de vrai pour montrer leur parfait rapport, c'est que tout orateur qui prouve une chose par démonstration apporte toujours: ou des exemples, ou des
enthymèmes, n'ayant point dans la rhétorique d'autres moyens pour démontrer que ceux-là. Et par conséquent, si la même nécessité se rencontre dans la
dialectique et que là, il soit impossible de rien prouver démonstrativement, quelque chose même que ce puisse être, sans se servir du syllogisme et de
l'induction --- comme nous l'avons fait voir dans nos Livres des \emph{Analytiques} --- il s'ensuit que chacun de ces deux moyens à l'égard des deux
autres, je veux dire, que le syllogisme à l'égard de l'enthymème, et que l'induction à l'égard de l'exemple, ne seront qu'une même chose.

\bigbreak

De savoir maintenant la différence qu'il y a entre l'exemple et l'enthymème, nous l'avons enseigné dans nos \emph{Topiques}. Car toutes les fois qu'on veut
montrer que quelque chose est d'une certaine façon, et qu'on apporte pour  preuve un grand nombre d'autres choses toutes semblables, dans la dialectique cela
s'appelle \emph{induction}, et dans la rhétorique, \emph{exemple}. Mais lorsqu'on établit certaines propositions, et que par une conséquence nécessaire, on
vient à tirer une autre proposition toute différente, à cause seulement que ces premières propositions ont été établies --- et cela indifféremment soit que
telles propositions soient vraies ou qu'elles ne soient que vraisemblables --- dans la dialectique, cela  s'appelle \emph{syllogisme} et dans la rhétorique,
\emph{enthymème}.

De là paraît, que l'un et l'autre de ces deux moyens, quand on sait bien s'en servir, sont de très grand usage et très considérables, chacun d'eux contenant
en soi comme une espèce de rhétorique à part. Car ce que nous avons remarqué de la dialectique, dans nos livres des \emph{Méthodes}, touchant sa façon de prouver,
est encore ici remarquable pour la rhétorique, attendu que la rhétorique aussi bien qu'elle a deux styles distingués ou deux manières différentes, dont l'une prouve
tout par les exemples et l'autre par les enthymèmes, comme il se trouve des orateurs qui ne se servent que d'enthymèmes et d'autres qui n'emploient que des exemples.
Et certainement, les discours qui prouvent par les exemples ne persuadent pas moins que les autres; toute la différence qu'il y a, c'est que ceux qui prouvent par
les enthymèmes font une plus forte impression sur l'esprit et troublent d'avantage. La raison en sera dite ailleurs, quand nous montrerons de quelle façon il se faut
servir de tous les deux. Pour maintenant, il suffit de nous expliquer et de débrouiller ces matières.

\subsection{Sur quelles matières s'appliquent les enthymèmes}

Donc,

\begin{emphpar}
     Puisque tout ce qui est propre à persuader est relatif aux personnes, c'est à dire propre à persuader quelqu'un,
\end{emphpar}

De plus,

\begin{emphpar}
     Puisque ces choses-là sont de deux sortes, les unes capables de persuader d'elles-mêmes et croyables d'abord, les
      autres simplement parce qu'elles semblent établies sur des preuves de la qualité de ces premières,
\end{emphpar}

Enfin,

\begin{emphpar}
     Puisqu'il n'est point d'art qui s'arrête à considérer la nature d'aucun particulier ni qui en fasse son objet; car
      la médecine, par exemple, ne se propose point de connaître en particulier ce qui est bon pour la santé de Socrate ou
      de Callias, mais seulement de tels en général, et de  tels, qui différent de tempérament, ou qui ont telles maladies;
      vu que c'est là proprement où l'art se fait voir, n'étant pas possible à quelque science ni à quelque art que ce soit
      de connaître tous les particuliers, le nombre en étant infini.
\end{emphpar}

De là, il s'ensuit:

\begin{emphpar}
     Que la rhétorique ne se proposera pas non plus, et ne considérera point ce qui est probable à l'égard d'un tel particulier
      et qui pourra le persuader --- par exemple ce qui est probable à l'égard de Socrate ou d'Hippias --- mais bien ce qui le
      sera à l'égard de telle ou de telle sorte d'esprits, qui ont des mœurs et des inclinations différentes.
\end{emphpar}

Et cela à l'imitation de la dialectique, car la dialectique ne s'amuse pas à argumenter ni à faire des syllogismes sur tout ce qui se présente
indifféremment, pour probable même qu'il puisse être à certaines personnes, vu qu'il y a des choses qui peuvent paraître probables à certains
particuliers, par exemple à des saouls et à des extravagants; mais seulement, elle argumente sur les matières qui ne sont pas assez établies
d'elles-mêmes, et qui ont besoin de preuve. 

Pour la rhétorique, elle s'attache seulement aux matières qui ont accoutumé de tomber en délibération, car c'est là proprement son ouvrage que
d'examiner les choses sur lesquelles ordinairement nous délibérons, et de qui nous n'avons aucun art, et même encore en présence de certains
auditeurs, qui pour être peu éclairés ne sont pas capables de comprendre ce qui embrasse plusieurs choses à la fois, ni suivre de l'esprit un
raisonnement de longue haleine.

Sur cela, il faut remarquer que jamais nous ne délibérons que sur ce qui nous paraît arriver diversement, n'y ayant point d'autre occasion de
délibérer que celle-là, puisque jamais on ne met en délibération ni le passé, quand il ne s'est pu faire autrement qu'il a été fait, ni l'avenir,
quand  il est impossible qu'il arrive d'une autre façon, ni le présent, quand on ne peut pas empêcher qu'il ne soit comme il est --- du moins
tandis qu'on demeure dans cette opinion et que la chose est crue ainsi.

\subsection{La manière d'argumenter en rhétorique}

Pour ce qui est d'argumenter et d'établir une chose par syllogismes et par conséquences, on s'y prend en deux manières. Car ou l'on tire des
conséquences de propositions qui ont déjà été prouvées par d'autres syllogismes et par d'autres arguments, ou bien de propositions qui ne l'ont
pas été, mais qui ont besoin de l'être parce qu'elles ne sont pas probables d'elles-mêmes. Or est-il que ni l'une ni l'autre de ces deux manières
n'est point propre à la rhétorique. La première, comme trop difficile à suivre à cause de sa longueur, vu qu'on suppose que l'auditeur est simple
et peu intelligent; et l'autre est incapable de persuader, parce qu'elle avance des choses qui ne sont pas avouées de tout le monde, et qui n'ont
aucune vraisemblance. 

\bigbreak

De ces observations, il s'ensuit premièrement, touchant la matière de l'exemple et de l'enthymème, que toujours ils seront employés sur les matières
incertaines et sur des choses qui pour ordinaire arrivent de différentes façons; l'exemple, qui comme il a été déjà remarqué est la même chose
que l'induction, et l'enthymème la même chose que le syllogisme.

De plus, il s'ensuit quant à la forme de l'enthymème que d'ordinaire, il ne pourra pas avancer tant de choses, ni être composé de tant de propositions
que le syllogisme parfait, attendu que si quelqu'une de ces propositions est connue, il faut l'omettre, puisque l'auditeur de lui-même la supplée alors.
Par exemple, on veut faire savoir que Doricus, ce fameux Athlète, a vaincu aux Jeux Olympiques et a été couronné; il suffit de dire que ce Doricus a
gagné le prix sans qu'il soit besoin d'ajouter cette proposition générale que ceux qui remportent la victoire à ces jeux y sont couronnés, parce qu'on
sait bien que cela se fait toujours. 

\subsection{De quelle sorte de propositions sont composés les enthymèmes}

Donc,

\begin{emphpar}
      Puisque entre les propositions, dont la rhétorique forme ses syllogismes, il s'en trouve peu de nécessaires, car la plupart des matières
      qui se jugent dans le barreau et qui se traitent dans les délibérations sont incertaines, et peuvent arriver de différentes façons, vu
       qu'on ne délibère jamais que sur les choses qu'on veut entreprendre et qu'on propose de faire, toutes les actions qui se font dans le
       monde étant de cette nature, et n'y en ayant pas une, pour ainsi dire, qui porte un effet nécessaire et dont l'événement soit certain,
\end{emphpar}

De plus,

\begin{emphpar}
      Puisque les propositions contingentes, et qui ne sont vraies que pour l'ordinaire, doivent toujours être prouvées par d'autres de même nature
      et incertaines comme elles, et tout au contraire les nécessaires, par des nécessaires, ainsi que nous avons fait voir dans nos Livres des
      \emph{Analytiques},
\end{emphpar}

Il s'ensuit:

\begin{emphpar}
      Que les matières d'où se tirent les enthymèmes seront pour la plupart incertaines ou contingentes, et qu'il y en aura fort peu de nécessaires.
\end{emphpar}

De vrai, tous les enthymèmes qui se font ont toujours leur preuve fondée ou sur le \emph{vraisemblable} ou sur les \emph{signes}; en sorte qu'il faut que
ces signes, et ce vraisemblable, eu égard au nécessaire et à l'incertain, ou contingent, ne soient entre eux qu'une même chose. Et de fait, proprement,
le vraisemblable est ce qui se fait d'ordinaire, non % "non pas à la vérité absolument": je coupe. Ce sera déjà un peu moins wtf... 
pas absolument, comme le prétendent quelques-uns dans la définition qu'ils en donnent, entendant par là indifféremment tout ce qui peut être compris sous
le mot de vraisemblable, de quelque nature que ces choses-là puissent être, soit que la qualité d'universel leur convienne ou ne leur convienne pas. Si
bien que dans la rhétorique, le \emph{vraisemblable} se doit seulement entendre des choses qui n'arrivent pas toujours de la même façon, et de plus le rapporter
à celles à l'égard desquelles il passe pour vraisemblable, de la même sorte que \emph{l'universel} se rapporte au \emph{particulier}.

\subsection{Des signes et de leur différence}

Pour les \emph{signes}, ils sont de deux sortes. Les uns se rapportent aux choses à qui ils servent de signes, comme le particulier se rapporte à l'universel,
c'est à dire que la preuve en est la même que si l'on prouvait une proposition générale par une proposition particulière; les autres, au contraire, ont
le rapport d'un universel à un particulier; et de ceux-ci, quelques-uns sont nécessaires à qui on donne le nom de \emph{tekmérion}; les autres ne sont
pas nécessaires, et sont simplement appelés signes, sans avoir d'autre nom qui les distinguent. J'appelle \emph{signes nécessaires}, ceux qui peuvent
servir de matière au syllogisme et dont la preuve est convaincante; c'est pourquoi le signe appelle \emph{tekmérion} est mis au nombre de ceux-là. Aussi
toutes les fois qu'un orateur allègue pour preuve des choses auxquelles il ne pense pas qu'on puisse répondre, alors il qualifie ces preuves du nom
\emph{tekmérion}, comme qui dirait une preuve démonstrative et qui \emph{termine} tout le différend. Car le mot de \emph{tekmar}, d'où est tiré celui de
\emph{tekmérion}, anciennement signifiait la même chose que le mot de \emph{terme}\footnote{300 ans après\dots{}
                                                                                        <<Le \emph{tekmérion} est l'indice sûr, le signe nécessaire
                                                                                        ou encore ``le signe indestructible'', celui qui est ce qu'il
                                                                                        est et qui ne peut pas être autrement>>. Roland \textsc{Barthes}, 
                                                                                        \emph{L'aventure sémiologique}, Seuil, 1985, p.\,134}.

\bigbreak

Mais donnons des exemples de ces signes et, premièrement, de celui que nous avons remarqué avoir le rapport du particulier à l'universel. Si donc on
raisonne ainsi:

\begin{emphpar}
      Un signe que tous les habiles gens sont gens de bien c'est que Socrate, qui était un habile homme, a été très homme de bien.
\end{emphpar}

Véritablement alors, ce serait apporter un signe pour sa preuve. Un tel signe, néanmoins, ne serait pas nécessaire ni convaincant, étant facile
d'y répondre. La raison est qu'on n'en peut pas faire un syllogisme, puisque le syllogisme ne tire jamais une conclusion universelle d'une simple
proposition particulière.

Mais si quelqu'un venait à raisonner de cette autre façon:

\begin{emphpar}
      Un signe que cet homme est malade, c'est qu'il a la fièvre,
\end{emphpar}

ou bien:

\begin{emphpar}
      Un signe que cette femme est mère, c'est qu'elle a du lait aux mamelles,
\end{emphpar}

Cette sorte de signe serait nécessaire, et le seul que nous appelons \emph{tekmérion}; car quand un signe est de telle qualité que lui seul suffit
pour faire connaître que ce qu'on dit est vrai; pour lors la preuve est convaincante et ne souffre point de réponse.

Quant aux autres signes qui ont le même rapport qu'a l'universel au particulier, mais qui ne sont pas nécessaires, c'est comme se quelqu'un disait:

\begin{emphpar}
      Un signe que cette personne a la fièvre, c'est  qu'elle respire comme si, elle était hors d'haleine.
\end{emphpar}

Certainement, ce signe serait véritable; il est aisé néanmoins d'y répondre, puisqu'il arrive quelque fois qu'un homme est hors d'haleine qui
pourtant n'a pas la fièvre.

\bigbreak

%\begin{flushleft}
\noindent
\begin{tabular}{|p{0.35\textwidth}|p{0.25\textwidth}|p{0.30\textwidth}|}
\cline{2-3}
\multicolumn{1}{c|}{} & Incertain & Nécessaire \\
\hline
Le particulier comme signe d'un universel & Socrate \rightarrow tous les habiles gens &  \\
\hline
L'universel comme signe du particulier & Je tousse \rightarrow coronavirus & N'importe quel \emph{tekmérion} \\
\hline
\end{tabular}
%\caption{Légende du tableau.}
%\End{flushleft}


\bigbreak

Nous avons donc enseigné ce que c'est que \emph{vraisemblable}, et ce que c'est que \emph{signe}; de plus, nous avons remarqué la différence qu'il y
a entre les signes nécessaires et ceux qui ne le sont pas. Mais ces choses-là ont été expliquées plus clairement et plus au long dans nos Livres des
\emph{Analytiques}, où nous avons touché les raisons pourquoi quelques-uns de ces signes peuvent servir de matière aux syllogismes, et pourquoi les
autres en sont incapables.

\subsection{De l'exemple, et comment il s'en faut servir}

Pour ce qui est de l'exemple, nous avons remarqué qu'il était la même chose que l'induction, et de plus nous avons fait voir en quoi consistait l'induction.
Au reste, il ne faut pas considérer l'exemple à l'égard des choses à qui il sert d'exemple comme le particulier est considéré à l'égard de l'universel, ou
comme l'universel est à l'égard du particulier, encore moins comme un universel le peut-être à l'égard d'un autre universel, mais bien toujours comme une chose
particulière est considérée à l'égard d'une autre particulière, et comme un semblable l'est à l'égard d'un autre semblable. Toutes les fois, donc, que deux
choses se trouvent sous un même genre, que l'une est plus connue que l'autre, celle qui est la plus connue est proprement ce que nous appelons \emph{exemple}.
Car si je voulais montrer que Denys de Syracuse a dessein de se faire tyran lorsqu'il demande des gardes, je dirais que Pisistrate, comme lui, demanda des gardes
d'abord, et que si tôt qu'il en eût, il se saisit du gouvernement d'Athènes. Je dirais que Théagène fit la même chose à Mégare, et alléguerais ensuite les autres,
qu'on saurait être venus à la tyrannie par telle voie, qui tous serviraient d'exemple à l'égard de Denys de Syracuse, dont il ne paraîtrait pas encore si véritablement
c'est à ce dessein qu'il demande des gardes. Or, tous ces exemples particuliers sont compris sous cette proposition générale, que \emph{quiconque pense à la tyrannie
et à se saisir du gouvernement demande des gardes}.

Nous avons donc montré en quoi consistent les preuves de la rhétorique qui paraissent démonstratives.

\subsection{De la différence des enthymèmes}

Quant aux enthymèmes, leur différence est si grande qu'il y a peu de personnes qui se puissent vanter de les bien connaître, puisque enfin cette
différence est la même que celle des  syllogismes de la dialectique --- attendu que quelques-uns sont particuliers à la rhétorique, ni plus ni moins
qu'entre les syllogismes, quelques-uns sont particuliers à la dialectique --- les autres  appartiennent aux autres arts, et aux autres facultés, tant
de celles qui sont à inventer, que de celles que nous connaissons et qui sont déjà inventées, ce qui fait qu'ils paraissent obscurs à l'auditeur, et
que ceux qui s'en servent autrement que la rhétorique ou la dialectique n'enseignent s'écartent de leur art, et ne raisonnent  plus alors ni comme un
dialecticien doit faire, ni en qualité d'orateurs.

Mais sans doute que ceci sera plus clair quand nous l'aurons davantage expliqué. Il faut donc savoir que les syllogismes que j'attribue à la dialectique
sont ceux à qui nous assignons des \emph{lieux}. Or, il y a deux sortes de lieux. Les uns sont \emph{communs} et les autres \emph{propres}. J'appelle
\emph{lieux communs} ceux qui servent à prouver diverses matières, comme de jurisprudence, de physique, de politique et de beaucoup d'autres qui diffèrent
d'espèces. Tel est le lieu commun qui traite du plus et du moins, parce que de ce lieu-là, nous pourrons aussitôt tirer des syllogismes et des enthymèmes
sur des matières de droit ou de physique que de quelque autre science que ce soit. Or est-il que toutes ces matières sont distingués d'espèces et
différentes entre elles. Pour les \emph{lieux propres}, ce sont ceux qui sont particuliers à chaque genre et à chaque espèce de propositions. Par exemple,
il y a des propositions tellement dépendantes de la physique qu'on n'en saurait faire d'enthymèmes ni de syllogismes pour prouver aucune proposition de la
morale, et d'autres, au contraire, tellement dépendantes de la morale qu'on ne s'en pourrait pas servir pour prouver aucune proposition de la physique; ce
qui se doit entendre également de toutes les autres propositions particulières et spécifiques.

\bigbreak

Il y a ceci a remarquer touchant les \emph{lieux communs} que jamais ils ne peuvent nous rendre savants sur aucune matière particulière, à cause qu'ils
sont vagues et ne traitent point un sujet déterminé. Il en est tout au contraire des \emph{lieux propres}, car plus les propositions que nous en tirerons
seront choisies et particulières au sujet que nous traitons, et plus insensiblement nous nous éloignerons de la dialectique et de la rhétorique pour nous
approcher d'une autre science; parce qu'enfin, si nous ramenons ces propositions jusqu'aux principes, alors notre raisonnement et notre preuve ne seront
plus l'ouvrage de la dialectique ni de la rhétorique, mais seulement de la science dont nous aurons touché les principes. 

Ici, nous observerons encore que la plupart des enthymèmes se tirent des lieux propres seulement, et qu'il y en a fort peu qui soient tirés des lieux
communs. Nous diviserons donc ici les enthymèmes de la même façon que nous avons déjà fait dans les \emph{Topiques}, savoir en autant de lieux propres
qu'il y a de sortes de  propositions d'où ils peuvent être tirés. Au reste, j'appelle \emph{lieux propres d'enthymèmes} les propositions qui sont
particulières à chaque genre de la rhétorique, et je nomme \emph{lieux communs} les propositions communes à tous les genres, et qui servent à prouver
toute sorte de matières.  Parlons donc, premièrement, des lieux propres des enthymèmes, mais auparavant des genres de la rhétorique, afin qu'ayant montré
combien il y en a, nous puissions voir en particulier quels sont les \emph{éléments} de chacun, et les propositions qui leur conviennent.



\begin{comment}

\section{Que la rhétorique a trois genres}

La rhétorique a sous soi trois genres, puisqu'il se trouve autant de sortes d'auditeurs. Car il faut savoir que
tout discours regarde trois choses: celui qui parle, le sujet que l'on traite et la personne à qui on parle, que
nous appelons \emph{l'auditeur}, et auquel se rapporte tout le discours. 

Tout auditeur, au reste, doit être nécessairement ou simple auditeur, ou juge. S'il est juge, il faut que ce soit
ou de choses qui aient été faites déjà, ou de choses qui ne le soient pas encore.

L'auditeur qui a son jugement à donner sur ce qui n'est pas encore arrivé mais qu'on propose de faire simplement
est, par exemple, le peuple d'Athènes, assemblé pour délibérer sur les affaires de la république.

Celui qui a à juger du passé et de ce qui a été fait est, proprement, le magistrat ou le juge.

Enfin, le simple auditeur est celui qui ne vient que pour contenter sa curiosité, et pour avoir le plaisir d'entendre
un excellent orateur.

De manière qu'il faut, par nécessité, qu'il y ait trois genres dans la rhétorique qui répondent à ces trois sortes
d'auditeurs:

\begin{emphpar}
Le genre délibératif,

Le genre judiciaire,

Le genre démonstratif.
\end{emphpar}

Le genre délibératif a deux parties: la \emph{persuasion} et la \emph{dissuasion}, car toujours ceux qui délibèrent font
l'une de ces deux choses, soit qu'ils délibèrent sur leurs affaires particulières ou sur les affaires publiques.

Le genre judiciaire a aussi deux parties sous soi: \emph{l'accusation} et la \emph{défense}, car il est nécessaire qu'en
plaidant, les avocats fassent l'un ou l'autre, qu'ils défendent ou qu'ils accusent.

Le genre démonstratif, pareillement, comprend deux parties: la \emph{louange} et le \emph{blâme}.

\bigbreak

Chacun de ces genres a aussi un \emph{temps} qui lui est particulièrement affecté.

\emph{L'avenir} appartient au délibératif, car tout homme qui délibère, soit qu'il conseille ou dissuade, délibère toujours
sur ce qui n'est pas encore arrivé. 

Le \emph{passé} convient au judiciaire, car on n'accuse et l'on ne défend jamais que les actions qu'une personne a faites. 

Enfin, le \emph{présent} est le plus propre au genre démonstratif, puisqu'on ne loue ou ne blâme que ce qui est effectivement.
Ce n'est pas, néanmoins, qu'assez souvent, en telle rencontre, les orateurs ne fassent aussi mention du passé afin d'en
renouveler la mémoire; et même par avance de ce qui n'est pas encore, comme par un préjugé de l'avenir. 

\bigbreak

De plus, chacun de ces genres se propose un but et une fin particulière; de sorte que, comme il y a trois genres, il se
trouve aussi trois fins différentes.

Celui qui délibère se propose pour but ce qui est \emph{utile}, ou \emph{nuisible}, car tout orateur qui entreprend de
persuader une chose la propose toujours comme la meilleure, et s'il veut la dissuader, il tâche de faire voir que c'est
la pire. Ce n'est pas qu'il ne se serve encore de tout le reste que les autres genres se proposent, afin d'en fortifier
sa preuve. Par exemple, il tâche de montrer que cette même chose est encore juste ou injuste, honnête ou contre l'honneur. 

Ceux qui plaident se proposent toujours de faire voir que la chose donc il s'agit est juste ou injuste et pareillement,
se servent de tout le reste pour ce dessein.

Enfin, ceux qui ont à louer ou à blâmer prétendent seulement démontrer que ce qu'ils louent ou blâment est honnête ou
honteux, et tout de même, y rapportent les autres choses que nous venons de dire.

Et une preuve certaine que chacun de ces genres ne se propose point une autre fin que celle dont nous venons de parler,
c'est que bien souvent, il n'y aurait point de contestation touchant les autres points. Par exemple, ceux qui plaident
demeurent souvent d'accord qu'une chose a été faite, et même qu'elle a porté préjudice, mais jamais ils n'avouent qu'ils
aient fait une injustice, autrement il serait inutile de plaider. Le même se peut dire de ceux qui délibèrent. Souvent,
ils accordent tout le reste, mais jamais ils n'avouent que ce qu'ils persuadent de faire soit inutile, ou que l'entreprise
dont ils veulent détourner soit avantageuse. De savoir maintenant si ce qu'ils conseillent est contre la justice ou non,
par exemple, d'assujettir des peuples voisins et qui n'ont jamais fait de tort, c'est bien souvent à quoi ils ne pensent
pas seulement, tant ils s'en mettent peu en peine. Il en est de même de ceux qui louent ou blâment quelqu'un, tant s'en
faut qu'ils examinent s'il a fait des choses qui lui aient apporté du profit ou de la perte que, bien souvent, ils le
louent davantage quand il a méprisé son propre intérêt pour entreprendre quelque action glorieuse. Par exemple, ils
donnent des louanges à Achille, de ce qu'étant assuré de perdre la vie en vengeant la mort de Patrocle, son meilleur ami,
il aima mieux mourir que de laisser cette mort impunie. Cependant il est certain, que d'une part cette mort lui fut
glorieuse, d'un autre côté la vie lui était utile.

\subsection{De la nécessité des lieux propres et des lieux communs}

On voit par ce qui a été dit qu'il faut avoir, premièrement, un certain fonds, ou amas de propositions sur toutes les matières
dont nous venons de parler qui appartiennent aux trois genres; et de plus, on se doit souvenir que les propositions dont la
rhétorique se sert sont toutes tirées des signes, tant simples que nécessaires, et du vraisemblable. La nécessité, au reste,
d'avoir ainsi des propositions toutes prêtes vient de ce qu'absolument, on ne saurait faire de syllogismes sans propositions.
Et ainsi, l'enthymème étant une espèce de syllogisme, il faut aussi qu'il soit composé de propositions, mais de propositions
de la qualité de celles que nous avons remarquées.

\bigbreak

Mais parce qu'on ne peut pas dire que ce qui est du tout impossible puisse jamais avoir été fait, ni qu'il le puisse être, vu
que cela n'appartient qu'aux choses qui sont possibles de leur nature; outre cela parce qu'il est encore impossible que ce qui
n'a point été, ou qui jamais ne doit être, ait été fait déjà, ou soit fait à l'avenir, il fera encore nécessaire à l'orateur,
soit dans une délibération, soit en plaidant, soit dans les sujets qui regardent le genre démonstratif, d'avoir un autre fonds
ou amas de propositions, tant sur la matière du \emph{possible} que sur celle de \emph{l'impossible}, afin de pouvoir connaître
si une chose aura été faite ou non, si elle arrivera ou n'arrivera pas.

\bigbreak

Et d'autant encore que tout orateur, soit qu'il loue ou blâme, qu'il accuse ou défende, qu'il persuade ou dissuade, ne tâche
pas seulement de prouver les matières que nous venons de dire, mais assez souvent même de faire voir qu'une chose qui est bonne
ou mauvaise, honnête ou déshonnête, juste ou injuste, est encore grande ou petite, de conséquence ou non. Et cela indifféremment,
soit qu'il considère ces choses là en elles-mêmes ou qu'il les compare entre elles; il est certain qu'il sera encore nécessaire
d'avoir des propositions et en général et en particulier, tant sur la \emph{grandeur} et la \emph{petitesse}, que sur ce que nous
appelons plus grand et plus petit, afin de savoir quel bien en particulier sera plus grand ou plus petit qu'un autre. Quelle
action sera plus juste, ou plus injuste, et ainsi du reste. 

Nous venons donc de montrer quelles sont les matières d'où se doivent tirer nécessairement les propositions dont il se faut servir.
Parlons ensuite de chacune en particulier, savoir de celles qui appartiennent au genre délibératif premièrement; en second lieu,
de celles qui appartiennent au genre démonstratif et enfin, des autres qui regardent le genre judiciaire. 




\chapter{Le genre déliberatif}
\section{Des matières qni tombent en délibération}

Dein Syria per speciosam interpatet diffusa planitiem. hanc nobilitat Antiochia, mundo cognita civitas, cui non certaverit alia 
advecticiis ita adfluere copiis et internis, et Laodicia et Apamia itidemque Seleucia iam inde a primis auspiciis florentissimae.

Ipsam vero urbem Byzantiorum fuisse refertissimam atque ornatissimam signis quis ignorat? Quae illi, exhausti sumptibus bellisque 
maximis, cum omnis Mithridaticos impetus totumque Pontum armatum affervescentem in Asiam atque erumpentem, ore repulsum 
et cervicibus interclusum suis sustinerent, tum, inquam, Byzantii et postea signa illa et reliqua urbis ornanemta sanctissime 
custodita tenuerunt.

Qui cum venisset ob haec festinatis itineribus Antiochiam, praestrictis palatii ianuis, contempto Caesare, quem videri decuerat, ad 
praetorium cum pompa sollemni perrexit morbosque diu causatus nec regiam introiit nec processit in publicum, sed abditus multa in 
eius moliebatur exitium addens quaedam relationibus supervacua, quas subinde dimittebat ad principem.

Atque, ut Tullius ait, ut etiam ferae fame monitae plerumque ad eum locum ubi aliquando pastae sunt revertuntur, ita homines 
instar turbinis degressi montibus impeditis et arduis loca petivere mari confinia, per quae viis latebrosis sese convallibusque 
occultantes cum appeterent noctes luna etiam tum cornuta ideoque nondum solido splendore fulgente nauticos observabant quos 
cum in somnum sentirent effusos per ancoralia, quadrupedo gradu repentes seseque suspensis passibus iniectantes in scaphas 
eisdem sensim nihil opinantibus adsistebant et incendente aviditate saevitiam ne cedentium quidem ulli parcendo obtruncatis 
omnibus merces opimas velut viles nullis repugnantibus avertebant. haecque non diu sunt perpetrata.

Erat autem diritatis eius hoc quoque indicium nec obscurum nec latens, quod ludicris cruentis delectabatur et in circo sex vel 
septem aliquotiens vetitis certaminibus pugilum vicissim se concidentium perfusorumque sanguine specie ut lucratus ingentia 
laetabatur.

Superatis Tauri montis verticibus qui ad solis ortum sublimius attolluntur, Cilicia spatiis porrigitur late distentis dives bonis omnibus 
terra, eiusque lateri dextro adnexa Isauria, pari sorte uberi palmite viget et frugibus minutis, quam mediam navigabile flumen 
Calycadnus interscindit.

Quam quidem partem accusationis admiratus sum et moleste tuli potissimum esse Atratino datam. Neque enim decebat neque 
aetas illa postulabat neque, id quod animadvertere poteratis, pudor patiebatur optimi adulescentis in tali illum oratione versari. 
Vellem aliquis ex vobis robustioribus hunc male dicendi locum suscepisset; aliquanto liberius et fortius et magis more nostro 
refutaremus istam male dicendi licentiam. Tecum, Atratine, agam lenius, quod et pudor tuus moderatur orationi meae et meum 
erga te parentemque tuum beneficium tueri debeo.

Nemo quaeso miretur, si post exsudatos labores itinerum longos congestosque adfatim commeatus fiducia vestri ductante 
barbaricos pagos adventans velut mutato repente consilio ad placidiora deverti.

Ob haec et huius modi multa, quae cernebantur in paucis, omnibus timeri sunt coepta. et ne tot malis dissimulatis paulatimque 
serpentibus acervi crescerent aerumnarum, nobilitatis decreto legati mittuntur: Praetextatus ex urbi praefecto et ex vicario 
Venustus et ex consulari Minervius oraturi, ne delictis supplicia sint grandiora, neve senator quisquam inusitato et inlicito more 
tormentis exponeretur.

Eodem tempore etiam Hymetii praeclarae indolis viri negotium est actitatum, cuius hunc novimus esse textum. cum Africam pro 
consule regeret Carthaginiensibus victus inopia iam lassatis, ex horreis Romano populo destinatis frumentum dedit, pauloque 
postea cum provenisset segetum copia, integre sine ulla restituit mora.


\section{Du souverain bien, et de ses parties}

Il n'y a presque personne, soit en commun soit en particulier, qui dans la vie ne se propose un certain
but. Et pour arriver à ce but, que sans cesse on a en vue, chacun de son côté fait tout ce qu'il peut afin
d'acquérir et d'éviter certaines choses. Or, ce but, en un mot, est ce que nous appelons \emph{souverain
bien}, \emph{félicité}, \emph{souverain bonheur}, et tout ce qui en dépend. Afin donc qu'on en ait quelque
idée, disons en gros ce que c'est que cette félicité ou ce souverain bien, et ce qui en fait partie, puisque
tout ce qu'on emploie, et a persuader, et à dissuader, regarde toujours ou la félicité elle-même, ou ce qui
se rapporte à elle, ou qui lui est opposé. Et de fait tout ce qui est capable de nous rendre heureux absolument
ou en partie, ou qui, d'un petit bien, en peut faire naître un plus grand est toujours ce que nous devons nous
proposer de faire, comme nous devons toujours nous abstenir d'entreprendre des choses qui peuvent détruire
notre bonheur, ou l'empêcher, ou nous faire passer à un état contraire. 

Supposons donc que la félicité se rencontre:

\begin{emphpar}
	À mener une vie dont toutes les actions réussissent au contentement de celui qui les fait, sans pourtant
	s'éloigner en rien de la vertu ni du devoir d'un honnête homme;
\end{emphpar}

Ou encore, 

\begin{emphpar}
	À se voir en tel état qu'on n'ait affaire de rien;
\end{emphpar}

Ou bien,

\begin{emphpar}
	À passer si agréablement ses jours que les plaisirs n'en puissent être troublés;
\end{emphpar}

Ou enfin,

\begin{emphpar}
	À jouir d'une possession si parfaite de toutes choses qu'on soit en puissance également de les conserver
	dans le besoin et de les acquérir de nouveau si elles étaient perdues.
\end{emphpar}

Car sans doute tout le monde demeure d'accord que le souverain bien consiste: ou dans la possession de quelqu'une
de ces choses, ou de plusieurs ensemble,

Que si la félicité est véritablement ce que nous venons de dire, on doit mettre au nombre de ce qui en fait partie:
la naissance, le crédit et l'amitié des honnêtes gens, les richesses, l'avantage d'avoir des enfants parfaits et en
grand nombre, et enfin, la jouissance d'une vieillesse exempte de toute sorte d'incommodités. De plus, il y faudra
ajouter toutes les qualités excellentes du corps, par exemple: la santé, la beauté, la force, la taille, l'adresse à
toutes sortes d'exercices, et encore la gloire et les honneurs, la bonne fortune; en un mot, la vertu et tout ce qui
en dépend, savoir la prudence, la valeur, la tempérance, la justice; car il est certain qu'un homme sera souverainement
content lorsqu'il se verra possesseur, et des biens qui se trouvent dans nous-mêmes et que nous possédons en propre,
et de ceux qu'on emprunte d'ailleurs et qui sont hors de nous, puisque après ces deux sortes de biens, il n'en faut
point chercher d'autres. J'appelle biens qu'on trouve dans soi-même tout ce qui sert à l'embellissement de l'âme et à
perfectionner le corps. Et j'appelle biens étrangers et hors de nous la noblesse, les amis, les honneurs et les richesses.
Outre ces avantages, néanmoins, nous croyons encore que pour assurer entièrement le bonheur de notre vie, il est bon
d'avoir de la puissance et d'être favorisé de la fortune.

Examinons en particulier quelle est la nature de toutes ces choses, et premièrement en quoi consiste la noblesse.

\subsection{Les parties qui composent le souverain bien}

La noblesse se peut considérer en deux façons: ou à l'égard de tout un peuple ou d'un particulier seulement. Un peuple
sera remarquable par sa noblesse s'il est originaire du pays qu'il habite, ou du moins fort ancien; si ses fondateurs
ont été illustres, et s'il en est sorti quantité de grands hommes qui aient éclaté par leur sagesse, par leur valeur,
par leur justice et par tous les autres avantages qui donnent de l’émulation. La noblesse d'un particulier peut venir:
ou du côté des hommes, ou du côté des femmes, ou de tous les deux ensemble, surtout si sa naissance est légitime. Et
cette noblesse sera toujours d'autant plus considérable si, de même que nous venons de remarquer touchant les fondateurs
des états, les premiers de sa race ont été illustres pour leur vertu ou leurs grands biens, ou pour quelqu'une des autres
choses qui ont du crédit dans le monde; et non seulement si les premiers de sa race ont été illustres, mais encore si depuis,
on en peut compter beaucoup d'autres dans sa famille, aussi bien parmi les femmes que parmi les hommes; parmi les jeunes
gens et les vieillards qui aient ajouté à cette première gloire.

\bigbreak

Il n'est pas difficile de connaître en quoi consiste ce que nous appelons \emph{être heureux en enfants}. En général, donc,
ce bonheur se rencontre dans une ville ou dans un état s'il y a beaucoup de jeunesse, et qui ait de bonnes qualités, soit
que ces qualités regardent le corps, comme font la taille et la beauté, la force, et l'adresse à toutes sortes d'exercices,
soit qu'elles regardent l’âme, comme la tempérance et la valeur; car à proprement parler, ces deux vertus appartiennent aux
jeunes gens. En particulier, nous appellerons un homme heureux en enfants celui qui en aura un grand nombre, tant de l'un que
de l'autre sexe, et remarquables par toutes les qualités que nous venons de dire. Au reste les qualités qui rendent les femmes
recommandables, premièrement quant au corps, sont la beauté et la taille; en second lieu, pour l’âme et pour l'esprit, nous
recherchons aux femmes, particulièrement, la tempérance, et de plus cet amour du ménage qui ne tient point de la bassesse et
qui n'est pas indigne d'une femme de condition. De quelque façon, donc, que nous considérions la possession des enfants, tant
de l'un que de l'autre sexe, soit que nous la considérions en général ou en particulier, jamais elle ne pourra être heureuse
entièrement si ces enfants, autant les filles que les mâles, n'ont toutes les vertus et toutes les qualités que nous avons
remarquées. Et pour cela, on peut assurer de tous ceux qui ont des filles et des femmes aussi mal élevées que les Lacédémoniens
en ont qu'ils ne sont heureux en enfants qu'à demi.

\bigbreak

Quant aux richesses, ce qui en fait partie est l'argent comptant, la quantité des héritages et des belles terres; les meubles,
les troupeaux, les esclaves, surtout s'ils sont remarquables par la grandeur, par la beauté et par le nombre. Or, non seulement
pour être riche il faudra posséder toutes ces choses, mais encore il faudra que la possession en soit sûre, honnête et profitable
tout ensemble. Une chose est profitable lorsqu'elle est de rapport, et elle est honnête lorsqu'on ne s'en sert que pour le plaisir.
J'appelle possession de rapport celle dont nous tirons du revenu, et je nomme possession pour le plaisir simplement, celle qui n'a
rien de plus considérable que l'usage. Enfin nous possédons en assurance une chose, lorsque nous en jouissons en tel lieu et de
telle sorte que nous pouvons en user comme il nous plaît, et de plus, quand la propriété nous en appartient. On possède en propre une
chose lorsqu'on la peut aliéner. J'appelle aliéner la vendre ou la donner. Après tout, il ne faut pas penser que la qualité de riche
dépende plus de la possession des richesses que de leur usage, car tant s'en faut que cela soit, que même se servir de son bien est
proprement ce que nous appelons être riche.

\bigbreak

La gloire et la réputation consistent à passer pour homme de bien dans l'esprit de tous les hommes; et encore à être cru possesseur
d'un avantage ou que tout le monde souhaite passionnément, ou du moins les plus honnêtes gens, ou les personnes d'esprit. 

\bigbreak

L'honneur est un témoignage d'estime qu'on rend à ceux qui sont bienfaisants. De là vient qu'on honore principalement les personnes
qui ont du bien; et quoi qu'il fût juste de ne porter de l'honneur qu'à ces gens-là, on ne laisse pas d'honorer encore ceux qui sont
en puissance de bien faire. Au reste, le bienfait regarde toujours ou la vie et tout ce qui peut être cause de sa conservation, ou les
richesses, ou enfin quelqu'un des autres avantages dont l'acquisition est difficile à faire, soit absolument, soit en certain lieu ou
en certain temps. Et c'est aussi pourquoi souvent nous voyons rendre beaucoup d'honneur et faire de grandes soumissions à des personnes
pour de très petites choses en apparence, seulement à cause que l'occasion ou la difficulté de les faire les avaient rendus considérables.
Les parties de l'honneur, ou les manières différentes d'honorer sont: les sacrifices, les inscriptions publiques soit en vers ou en prose,
les récompenses, les lieux consacrés, les préséances, les tombeaux, les statues, les pensions qu'on a du public; à quoi l'on peut ajouter
ce que pratiquent les nations étrangères quand elles veulent honorer quelqu'un, par exemple se prosterner contre terre ou se retirer d'un
chemin quand on passe. Il faut encore mettre les présents au nombre des choses qui sont en honneur. De vrai, le présent est de telle nature
qu'en même temps, il est: et la donation d'une chose et une marque d'estime; aussi les avares et les ambitieux en sont-ils grands amateurs,
à cause qu'ils y trouvent ce qu'ils cherchent. Les avares y rencontrent l'acquisition, et les ambitieux l'honneur, qui est ce que tous deux
demandent. 

\bigbreak

La santé est proprement la vertu du corps. Il faut néanmoins la posséder de manière que nous puissions faire toutes sortes
de fonctions sans en être malades; car il y en a beaucoup qui jouissent de la santé comme faisait Hérodicos, qu'on ne peut
pas dire être heureux en cet état, à cause qu'il faut qu'ils s'abstiennent de tout qui rend notre vie commode et agréable,
ou de la plus grande partie.

\bigbreak

Pour la beauté, elle est différente à raison des âges différents. La beauté d'un jeune homme est d'avoir le corps propre
à toutes sortes d'exercices, soit à la course et aux autres actions qui demandent de la force. Il faut encore qu'il soit
agréable à voir, et si agréable même, qu'on ne puisse se lasser de le regarder. Pour cette raison, les athlètes propres à
la course et à se battre sont très beaux. La beauté d'un homme fait est de pouvoir supporter toutes les fatigues de la guerre
et d'avoir je ne sais quoi dans le visage qui le rende agréable à voir et redoutable tout ensemble. Enfin, celle d'un
vieillard consiste à pouvoir faire toutes les fonctions nécessaires, et cela sans se plaindre, comme ne sentant aucune des
incommodités qui affligent d'ordinaire la vieillesse.

\bigbreak

La force consiste à tourner et manier quelqu'un comme on veut, ce qui se fait de cinq façons: ou en le tirant, ou en le
poussant, ou en l'élevant, ou en le terrassant, ou en l’étreignant. Car on ne peut pas dire qu'un homme soit fort, s'il
ne fait tout ceci, ou une partie.

\bigbreak

À l'égard de la belle taille, c'est quand on surpasse presque tous les autres, ou en hauteur, ou en largeur, ou en
grosseur, en sorte néanmoins que cet excès ne rende pas le corps plus pesant ni plus tardif dans tous ses mouvements.

\bigbreak

Pour réussir au métier d'athlète, qui comprend trois sortes d'exercices --- savoir la lutte, la course et le combat des poings
--- le corps doit avoir ces trois avantages: la taille, la force et l'agilité. Car tout homme qui est agile est fort.

Au reste, quiconque peut jeter les jambes d'une certaine manière, les avancer loin et promptement, est propre à la course.

Celui qui peut étreindre son homme et le tenir ferme, est né pour la lutte.

Enfin, pouvoir, à force de poings, repousser un adversaire et le faire toujours reculer, c'est ce qu'il faut au combat des
poings.

Entre les athlètes, quelques-uns réussissent aux poings et à la lutte tout ensemble, et d'autres sont adroits à toutes ces
trois sortes d'exercices.

\bigbreak

La vieillesse commode est celle qui vient tard et qui ne fait rien souffrir. Pour en jouir, donc, il ne faudra pas vieillir
de trop bonne heure. Aussi ne suffira-t-il pas de vieillir tard si en même temps on n'est exempt de toutes sortes
d'incommodités. Or, cet avantage ne dépendra pas seulement des qualités excellentes du corps, mais encore de la bonne
fortune. Car, qu'un homme soit sujet aux maladies et de faible complexion, le moyen qu'il ne souffre jamais? Et s'il doit
être incommodé, comment est-il possible que sans un grand bonheur, il puisse vivre longtemps en cet état. J'avoue
véritablement que, sans la santé et la bonne constitution, on ne laisse pas de vivre quelquefois assez longtemps puisque,
tous les jours, il se voit des gens privés de tous les avantages du corps arriver à de longues années. Mais ce n'est pas
ici qu'il faut donner une exacte connaissance de cette matière.

\bigbreak

Pour ce qui est du crédit et d'avoir l'amitié des honnêtes gens, ceci sera facile à connaître quand nous aurons déclaré ce
que nous entendons par le mot d'ami. Tout homme, donc, qui tâchera par toutes sortes de moyens de procurer à un autre ce
qu'il juge lui être avantageux, sans autre motif que de le vouloir obliger et seulement parce qu'il l'aime, c'est là
proprement ce que nous appelons être ami. Or, quiconque aura beaucoup de personnes disposées de cette sorte à son égard,
se pourra vanter d'avoir du crédit et beaucoup d'amis. Et si ces mêmes personnes ont du mérite et de la vertu, pour lors,
il aura l'amitié des honnêtes gens.

\bigbreak

On appelle bonne fortune quand il arrive à une personne: ou qu'il lui est arrivé tous les biens et les avantages dont la
fortune est la cause ordinairement, ou du moins quand de tous ces biens, il lui en est arrivé la meilleure partie et les
plus considérables. La fortune, au reste, peut être cause quelquefois des mêmes biens et nous procurer les mêmes avantages
que ceux que notre adresse et les arts nous procurent, quoi que d'ordinaire, la plupart de ceux qui viennent d'elle ne
soient nullement au pouvoir des arts, comme sont les biens de la nature. Quelquefois encore, elle est cause de certains
biens qui arrivent extraordinairement et en quelque façon contre le dessein de la nature même. Par exemple, la fortune est
quelquefois cause de la santé qui est un bien dépendant de la médecine, et cette même fortune, bien souvent, est cause de
la beauté et de la taille, qui sont des avantages purement dépendants de la nature. Mais en général, on peut nommer biens
de la fortune tous ceux qui sont sujets à l'envie. Outre ces biens, la fortune en donne encore d'autres quelquefois contre
toute sorte de raison et d'apparence, comme il arrive, quand entre plusieurs frères qui sont très laids, il s'en rencontre
un parfaitement beau; ou lorsque de plusieurs qui cherchent un trésor, il n'y en a qu'un qui le trouve; ou encore quand
une flèche qui a été tirée épargne celui-ci et en blesse un autre tout contre; ou enfin, lorsqu'une personne qui avait
accoutumé d'aller seule en certain lieu s'abstient d'y aller dans le temps que plusieurs qui étaient allés pour la première
fois y périssent. Car il semble que toutes ces choses-là soient de purs effets de la bonne fortune.

\bigbreak

Touchant la vertu, parce qu'elle regarde la louange particulièrement, nous remettons à faire savoir ce que c'est quand nous
serons au genre démonstratif.


\section{De la fin du genre délibératif, avec les lieux qui servent à prouver qu'une choses est bonne et utile}

Dein Syria per speciosam interpatet diffusa planitiem. hanc nobilitat Antiochia, mundo cognita civitas, cui non certaverit alia 
advecticiis ita adfluere copiis et internis, et Laodicia et Apamia itidemque Seleucia iam inde a primis auspiciis florentissimae.

Ipsam vero urbem Byzantiorum fuisse refertissimam atque ornatissimam signis quis ignorat? Quae illi, exhausti sumptibus bellisque 
maximis, cum omnis Mithridaticos impetus totumque Pontum armatum affervescentem in Asiam atque erumpentem, ore repulsum 
et cervicibus interclusum suis sustinerent, tum, inquam, Byzantii et postea signa illa et reliqua urbis ornanemta sanctissime 
custodita tenuerunt.

Qui cum venisset ob haec festinatis itineribus Antiochiam, praestrictis palatii ianuis, contempto Caesare, quem videri decuerat, ad 
praetorium cum pompa sollemni perrexit morbosque diu causatus nec regiam introiit nec processit in publicum, sed abditus multa in 
eius moliebatur exitium addens quaedam relationibus supervacua, quas subinde dimittebat ad principem.

Atque, ut Tullius ait, ut etiam ferae fame monitae plerumque ad eum locum ubi aliquando pastae sunt revertuntur, ita homines 
instar turbinis degressi montibus impeditis et arduis loca petivere mari confinia, per quae viis latebrosis sese convallibusque 
occultantes cum appeterent noctes luna etiam tum cornuta ideoque nondum solido splendore fulgente nauticos observabant quos 
cum in somnum sentirent effusos per ancoralia, quadrupedo gradu repentes seseque suspensis passibus iniectantes in scaphas 
eisdem sensim nihil opinantibus adsistebant et incendente aviditate saevitiam ne cedentium quidem ulli parcendo obtruncatis 
omnibus merces opimas velut viles nullis repugnantibus avertebant. haecque non diu sunt perpetrata.

Erat autem diritatis eius hoc quoque indicium nec obscurum nec latens, quod ludicris cruentis delectabatur et in circo sex vel 
septem aliquotiens vetitis certaminibus pugilum vicissim se concidentium perfusorumque sanguine specie ut lucratus ingentia 
laetabatur.

Superatis Tauri montis verticibus qui ad solis ortum sublimius attolluntur, Cilicia spatiis porrigitur late distentis dives bonis omnibus 
terra, eiusque lateri dextro adnexa Isauria, pari sorte uberi palmite viget et frugibus minutis, quam mediam navigabile flumen 
Calycadnus interscindit.

Quam quidem partem accusationis admiratus sum et moleste tuli potissimum esse Atratino datam. Neque enim decebat neque 
aetas illa postulabat neque, id quod animadvertere poteratis, pudor patiebatur optimi adulescentis in tali illum oratione versari. 
Vellem aliquis ex vobis robustioribus hunc male dicendi locum suscepisset; aliquanto liberius et fortius et magis more nostro 
refutaremus istam male dicendi licentiam. Tecum, Atratine, agam lenius, quod et pudor tuus moderatur orationi meae et meum 
erga te parentemque tuum beneficium tueri debeo.

Nemo quaeso miretur, si post exsudatos labores itinerum longos congestosque adfatim commeatus fiducia vestri ductante 
barbaricos pagos adventans velut mutato repente consilio ad placidiora deverti.

Ob haec et huius modi multa, quae cernebantur in paucis, omnibus timeri sunt coepta. et ne tot malis dissimulatis paulatimque 
serpentibus acervi crescerent aerumnarum, nobilitatis decreto legati mittuntur: Praetextatus ex urbi praefecto et ex vicario 
Venustus et ex consulari Minervius oraturi, ne delictis supplicia sint grandiora, neve senator quisquam inusitato et inlicito more 
tormentis exponeretur.

Eodem tempore etiam Hymetii praeclarae indolis viri negotium est actitatum, cuius hunc novimus esse textum. cum Africam pro 
consule regeret Carthaginiensibus victus inopia iam lassatis, ex horreis Romano populo destinatis frumentum dedit, pauloque 
postea cum provenisset segetum copia, integre sine ulla restituit mora.


\section{Lieux pour connaître quand un bien est plus grand ou plus petit qu'un autre}

Mais parce qu'assez souvent, il arrive que les mêmes personnes qui demeurent d'accord que deux choses,
véritablement, sont utiles ne laissent pas d'être en contestation sur le plus et le moins, il faut encore
que nous enseignions à connaître quand un bien est plus grand qu'un autre, et quand une chose sera plus
utile. 

Supposons donc, premièrement, que tout ce qui surpasse une chose en quoi que ce soit est ce qui déjà contient
en soi tout autant que cette chose là contient, et qui a encore quelque chose de plus.

Et, au contraire, que tout ce qui est surpassé et moindre est ce qui est renfermé et compris dans la chose
qui le surpasse.

Supposons en second lieu que tout ce qu'on dit être plus grand ou en plus grand nombre n'est tel qu'à cause
qu'on en fait comparaison avec quelque chose qui est plus petite. Et tout de même, qu'autant de fois qu'on
se sert des termes de grand et petit, de peu et de beaucoup, est toujours à l'occasion de choses qu'on fait
rapporter à d'autres dont on veut faire savoir la grandeur et en quelle quantité elles sont.

Supposons enfin que tout ce qui surpasse une autre chose est proprement ce que nous appelons grand; et, au
contraire, que tout ce qui en est surpassé est proprement ce que nous appelons petit. Ainsi du peu et du
beaucoup.

\bigbreak

Donc, puisqu'il a été remarqué:

\begin{emphpar}
	Que le bien est une chose qu'on doit souhaiter à cause d'elle-même, non pas à cause d'une autre,
\end{emphpar}

Et encore:

\begin{emphpar}
	Que c'est généralement ce que désire tout ce qui est au monde, ou qui a de la raison, ou que rechercherait
	tout ce qui est privé de raison s'il avait de la connaissance et du jugement,
\end{emphpar}

En un mot:

\begin{emphpar}
	Que c'est tout ce qui est capable de nous procurer de pareils avantages ou de nous les conserver, ou qui
	en est suivi,
\end{emphpar}

Supposé encore ce que nous avons dit: 

\begin{emphpar}
	Qu'une chose à laquelle nous en rapportons d'autres tient toujours lieu de fin,
\end{emphpar}

Puisque c'est à la fin seule à quoi on rapporte tout ce qu'on fait, et pour laquelle tout le reste est recherché.

Supposé, en dernier lieu:

\begin{emphpar}
	Que tout ce que chaque particulier se propose comme un bien, jamais a son égard il ne peut être tel qu'il n'ait
	en lui quelqu'une des bonnes qualités que nous venons de dire.
\end{emphpar}

Tout cela présupposé, on pourra tirer les conséquences qui suivent.

\bigbreak 

Premièrement:

\begin{lieu}
	Que plus de choses qu'une seule prise à part, ou qu'un petit nombre, et cela comparé de sorte l'un avec l'autre
	que dans ce plus grand nombre se trouve aussi compris ce même petit nombre ou cette seule chose, sans doute le plus
	grand nombre en cet état l'emportera et sera à préférer\footnote{Jeff Bezos peut profiter à peu près mille fois
	plus du bonheur d’être riche que Will Smith.}.
\end{lieu}

En effet, les deux conditions que nous avons remarquées, pour avoir l'avantage et être considéré comme le meilleur s'y
rencontrent. Car déjà, de soi et en qualité de plus grand nombre, on ne peut pas douter qu'il ne les surpasse. Et
d'ailleurs, ces autres choses ici, pour être comprises en lui, en sont surpassées.

\bigbreak 

Secondement cette conséquence sera bonne:

\begin{lieu}
	Que si une chose, qui est la plus excellente dans son genre, l'emporte sur une autre qui soit aussi la plus excellente
	dans le sien, sans difficulté, le genre de la plus excellente l'emportera sur le genre de l'autre\footnote{En affirmant
	d'Hitler qu'il est le pire chef d'état de tous les temps, n'ai-je pas prouvé que les chefs d'états allemands sont pires
	que les français?}.
\end{lieu}

Et réciproquement, 
\begin{lieu}
	Que si un genre est plus excellent qu'un autre genre, ce qu'il y aura de plus excellent dans ce plus parfait genre
	l'emportera sur tout ce qu'il y aura de plus excellent dans l'autre.
\end{lieu}

Par exemple, s'il est vrai de dire en particulier que le plus excellent de tous les hommes est plus parfait que la plus
excellente de toutes les femmes, en général, il sera vrai de dire encore que tous les hommes seront plus parfaits que
toutes les femmes généralement. 

Et au contraire, par la même raison, si l'on peut dire que tous les hommes généralement sont plus excellents et plus parfaits
que toutes les femmes en général, on pourra dire en particulier aussi que le plus excellent de tous les hommes sera plus
parfait que la plus excellente de toutes les femmes; puisque les degrés d'excellence de chaque genre, et des choses qui sous
eux tiennent le premier rang, ont toujours un parfait rapport entre eux, et sont dans une juste proportion\footnote{Sans parler
de l'image de la femme au temps d'Aristote, puisque je ne veux pas répondre aux prémisses explicites mais seulement aux lieux,
notez cette insistance d'Aristote à justifier ses lieux et à les prendre comme argent comptant.}.

\bigbreak 

De plus on pourra inférer:

\begin{lieu}
	Qu'un bien qui en aura un second à sa suite vaudra mieux qu'un autre qui n'en aura point,
\end{lieu}

À cause que, jouissant de l'un, on jouira aussi de celui qui le suit. Au reste, nous avons déjà fait
savoir qu'une chose peut venir en suite d'une autre en deux façons: ou \emph{en même temps} ou \emph{
quelque temps après}. Il se remarque encore une troisième façon que nous appelons \emph{suivre en 
puissance}. Donnons des exemples:

La vie suit la santé en même temps. Il n'en est pas toujours de même de la santé à l'égard de la vie.

La science encore suit l’étude, mais ce n'est que quelque temps après.

Enfin, le larcin suit en puissance le sacrilège, puisque quiconque a la hardiesse de voler sur les autels
et de piller les temples, celui-là ne fera pas difficulté de dérober ailleurs.

\bigbreak

Cette conséquence aussi aura lieu:

\begin{lieu}
 Que de deux choses qui en surpasseront une troisième, celle qui la surpassera de davantage sera la
 meilleure,
\end{lieu}

Attendu que, pour être en cet état, il est nécessaire qu'elle surpasse aussi l'autre qui était plus grande.

\bigbreak

On pourra dire encore:

\begin{lieu}
	Que tout ce qui produira un plus grand bien vaudra mieux et sera plus digne de notre choix\footnote{La
	guerre émerge de la paix. La paix est la fin de la guerre. Si, donc, la paix est préférable à la guerre,
	alors la guerre vaut mieux que la paix.},
\end{lieu}
 
Puisque c'est à cela principalement qu'on connaît quand un bien est plus grand qu'un autre.

Et réciproquement:

\begin{lieu}
	Tout ce qui sera produit par une plus excellente chose,
\end{lieu}

Car si tout ce qui est bon pour la santé est plus souhaitable, et est un plus grand bien que ce qui apporte
simplement du plaisir, la conséquence est nette que le plaisir est bien moins considérable que la santé.

\bigbreak

Il y aura lieu encore d'inférer:

\begin{lieu}
	Que tout ce qui sera souhaitable de soi-même vaudra mieux que ce qui ne sera souhaitable qu'à cause d'une
	autre chose.
\end{lieu}
 
Par exemple, la force doit être tenue pour un plus grand bien que tout ce qui regarde la santé seulement, puisque,
sans la santé, on ne souhaiterait jamais pas une de ces chose. Au lieu que la force est toujours désirable d'elle-même,
en quoi nous avons dit que consistait principalement la nature du bien, entre les définitions que nous en avons données.

\bigbreak

On pourra aussi prétendre:

\begin{lieu}
	Que tout ce qui tient lieu de fin est meilleur que ce qui n'est point considérable en cette qualité,
\end{lieu}

Vu que celui-ci n'est recherché qu'à cause d'une autre chose et que l'autre est recherché pour l'amour de lui-mème. Par
exemple, l'exercice le doit céder à la santé, à cause qu'on n'aime à faire de l'exercice, qu'afin de se bien porter.

\bigbreak

Cette conséquence encore sera bonne:

\begin{lieu}
	Que ce qui n'aura pas tant de besoin d'une chose ou de plusieurs qu'une autre sera meilleur\footnote{Non, on va plutôt
	jouer au Kem's, parce que Mario Kart, faut du matériel, donc c'est moins bien.},
\end{lieu}

Attendu qu'il sera beaucoup plus parfait et plus capable de satisfaire tout seul. Au reste, une chose a moins de besoin
qu'une autre en deux manières: ou quand elle n'a pas affaire de tant, ou que ce qui lui manque est plus aisé à trouver.

\bigbreak

On aura aussi raison d'assurer:

\begin{lieu}
	Que de deux biens dont l'un sera tellement dépendant de l'autre que sans lui, il ne serait pas ou ne pourrait être,
	l'autre au contraire ne dépendra point de ce bien en aucune façon, l'indépendant vaudra beaucoup mieux,
\end{lieu}

Car, comme il n'aura que faire de rien, c'est une marque qu'il sera: et plus capable de satisfaire tout seul, et plus
parfait de lui-même.

La même conséquence encore aura lieu à l'égard de deux choses comparées ensemble:

\begin{lieu}
	Si l'une a la qualité de principe et que l'autre ne l'ait pas.
\end{lieu}

Et tout de même de deux autres:

\begin{lieu}
	Si l'une est cause et que l'autre non,
\end{lieu}

Vu qu'on sera obligé d'en faire d'autant plus d'état qu'absolument, il est impossible que sans aucune cause et sans aucun
principe, quelque chose que ce soit puisse être jamais, ni être faite.

\bigbreak

Cette conséquence aussi sera nécessaire:

\begin{lieu}
	Que de deux biens qui reconnaîtront chacun un principe différent, celui qui sera produit par le plus excellent principe
	sera aussi le plus excellent.
\end{lieu}

Et encore celle-ci:

\begin{lieu}
	Que de deux biens qui reconnaîtront chacun une cause différente, celui qui sera l'effet de la plus noble cause sera aussi
	le plus noble.
\end{lieu}

\bigbreak

Et réciproquement, il sera vrai de dire en renversant ces deux mêmes conséquences:

\begin{lieu}
	Que de deux principes différents, celui qui produira un plus grand bien sera aussi le meilleur,
\end{lieu}

Comme aussi:

\begin{lieu}
	De deux causes, celles qui produira un plus grand effet.
\end{lieu}

Par ce que nous venons de dire, il se voit que, de quelque façon qu'on puisse raisonner en ce sens, toujours de part et d'autre
il sera aisé de faire paraître une chose plus considérable, car non seulement un bien en paraîtra plus considérable \emph{si
étant reconnu pour principe, on le compare avec un autre qui ne soit pas tel}, mais encore \emph{Si n'étant point principe, on
en fait comparaison avec un autre qui soit principe véritablement}. Et, de fait, dans tout ce qu'on se propose, la fin est
toujours la chose la plus considérable et ce qui tient le premier rang et cependant, ce n'est point un principe. Donnons
quelque exemple. Léodamas, accusant Callistratos, soutenait que \emph{celui qui avait conseillé de faire une mauvaise action 
était plus coupable que celui qui l'avait commise, parce que cette action n'aurait jamais été faite si premièrement elle n'avait
été conseillée}. Ici le conseil est considéré comme le principe de l'action. Et tout au contraire, une autre fois, le même,
accusant Chabrias, soutînt que \emph{celui qui avait commis une injustice était beaucoup plus coupable que celui qui l'avait
conseillée, puisque tout conseil demeure inutile si un autre ne l’exécute, et que que ceux qui conseillent de faire une chose
la conseillent toujours à dessein que d'autres la mettent à exécution}. Ici l’exécution est considérée comme la fin.

\bigbreak

On pourra encore tirer cette conséquence:

\begin{lieu}
	Que ce qui se trouve rarement est plus excellent que ce qui se trouve communément et en abondance.
\end{lieu}

Ainsi, l'or est plus excellent que le fer, car quoi qu'on n'en tire pas tant d'usage que du fer, il semble néanmoins plus
précieux, à cause que l'acquisition en est plus difficile à faire.

Dans un autre sens, aussi, on pourra soutenir:

\begin{lieu}
	Qu'une chose qu'on aura en abondance sera meilleure qu'une autre qui sera plus rare,
\end{lieu}

Puisque en effet, on se servira beaucoup plus de l'une que de l'autre, et que tout ce qui sert très souvent vaut mieux que
ce qui ne sert que quelques fois et très peu. C'est ce qui a fait dire à \textsc{Pindare} dans une de ses Odes:

\begin{emphpar}
	Il n'est rien de si bon que l'eau.
\end{emphpar}

\bigbreak

Et tout de même on pourra prétendre:

\begin{lieu}
	Que ce qui est plus difficile à acquérir est préférable à tout ce qui s'acquiert aisément,
\end{lieu}

Parce qu'il sera plus rare que l'autre, et tout au contraire:

\begin{lieu}
	Que ce qui est plus facile à acquérir vaut mieux que ce qui ne peut être acquis qu'avec difficulté,
\end{lieu}

Puisque nous avons ces choses-là comme nous voulons et quand bon nous semble.

\bigbreak

Pareillement:

\begin{lieu}
	Tout ce qui aura pour son contraire un plus grand mal\footnote{S'il fallait choisir entre se faire larguer
	ou tomber dans la misère, qui prendrait la misère? C'est pourquoi il est toujours plus judicieux de s'étudier
	à augmenter son patrimoine que de chercher l'amour.}.
\end{lieu}

\bigbreak

Et encore:

\begin{lieu}
	Toutes les choses dont la privation nous apportera plus de dommage ou d'incommodité.
\end{lieu}

\bigbreak

On prendra aussi:

\begin{lieu}
	Que tout ce qui est vertueux vaut mieux que ce qui n'est point une vertu.
\end{lieu}

Et au contraire:

\begin{lieu}
	Que ce qui est vicieux est pire que ce qui n'est point un vice et qui n'y a aucune disposition,
\end{lieu}

Attendu que ces choses sont arrivées à leur terme et à leur fin, et que les autres ne sont pas en cet état-là.

\bigbreak

Cette conséquence encore aura lieu:

\begin{lieu}
	Que ce qui produira des effets plus louables ou plus blâmables sera aussi plus blâmable ou plus louable
	lui-même.
\end{lieu}

Et par la même raison:

\begin{lieu}
	Que les plus hautes vertus et les plus grands vices produiront aussi des actions plus vicieuses et plus
	vertueuses,
\end{lieu}

Puisque ce qu'un principe et une cause sont à l’égard de leurs effets, tels sont toujours effets à l'égard de
leurs principes.

\bigbreak

De plus on pourra raisonner ainsi:

\begin{lieu}
	Que toutes les choses dont l'excès sera plus souhaitable ou plus honnête, ces choses seront elles-mêmes
	plus honnêtes et plus souhaitables.
\end{lieu}

Par exemple, à cause qu'il est plus souhaitable d'avoir une excellente vue que d'avoir l'odorat excellent, il
s'ensuit que la bonne vue est quelque chose de plus souhaitable que le parfait odorat.

Et pareillement, que s'il est beaucoup plus honnête d'aimer à faire des amis que d'aimer à acquérir des richesses,
l'amour des richesses sera moins honnête que l'amour des amis.

Et réciproquement, on pourra dire en renversant les deux propositions précédentes:

\begin{lieu}
	Que plus une chose sera excellente et honnête, et plus l'excès en sera honnête et excellent,
\end{lieu}

Et tout de même:

\begin{lieu}
	Que plus le désir d'un bien sera honnête et raisonnable, et plus ce bien-là aussi sera honnête,
\end{lieu}

Car il est certain que plus les choses que nous souhaitons sont grandes en elles-mêmes, et plus à proportion nos désirs
croissent et sont grands pour l'ordinaire.

\bigbreak

Tout au contraire on dira:

\begin{lieu}
	Que d'autant plus qu'une chose sera honnête et bonne, d'autant plus aussi le désir en sera bon et honnête.
\end{lieu}

Et encore:

\begin{lieu}
	Que plus une science sera honnête et belle, et plus les matières qu'elle traitera seront belles aussi et honnêtes,
\end{lieu}

Attendu que telle qu'est la nature d'une science, telle est la doctrine; puisque chaque science n'enseigne rien que ce qui
est de son sujet.

Et par la même raison, à cause de l'analogie et du parfait rapport:

\begin{lieu}
	Que plus une chose sera belle et honnête, et plus la science qui en traitera sera telle.
\end{lieu}

\bigbreak

Ce raisonnement encore aura lieu:

\begin{lieu}
	Que tout ce que des hommes prudents et très judicieux, ou tous les hommes ensemble en un fort grand nombre de personnes,
	ou la plupart, ou les plus habiles gens d'une profession jugeraient sans doute ou auront déjà jugé être un bien ou un plus
	grand bien, assurément ce doit passer pour tel.
\end{lieu}

Ah reste, il n'importe que ç'ait été simplement leur avis ou qu'ils aient rendu ce jugement en qualité de maîtres et d'experts.

Or, non seulement on pourra se servir de cette proposition quand il sera question de juger si un bien sera plus grand qu'un autre,
mais encore de quelque matière que ce soit. Car on s'en pourra servir également: et en raisonnant sur la nature d'une chose, et en
traitant de sa quantité, de sa qualité, et ainsi du reste; puisque, toujours, il y aura lieu d’assurer qu'une chose ne sera jamais
autre que ce que la prudence, ou la science qui en doit juger, en aura déterminé. Ce que nous avons déjà remarqué être vrai entre
les définitions du bien que nous avons données, vu qu'il a été dit que \emph{le bien était une chose que tout ce qui est au monde
rechercherait sil avait du sens et de la prudence}. D'où il s'ensuit qu'un bien sera toujours d'autant plus grand et à souhaiter
que celui qui le jugera tel aura de prudence et de jugement.

\bigbreak

Sur ce fondement, on pourra dire encore:

\begin{lieu}
	Que tout ce qui se rencontre dans les honnêtes gens, soit absolument, soit en qualité d’honnêtes gens, est plus à rechercher.
\end{lieu}

Par exemple, à cause que la valeur se rencontre plus ordinairement dans un honnête homme que la force du corps, il sera vrai de
dire que la valeur est quelque chose de plus considérable que la force.

\bigbreak

\begin{lieu}
	Que ce que le plus homme de bien choisirait et préférerait à toute autre chose, soit absolument, soit en qualité de plus homme de
	bien, on le doit croire meilleur.
\end{lieu}

Ainsi, nous pourrons assurer qu'il vaut mieux souffrir l'injustice que de la faire, à cause que le plus homme de bien qui soit au monde
sera de ce sentiment.

On pourra dire encore:

\begin{lieu}
	Que ce qui donnera plus de plaisir sera préférable à tout ce qui en donnera moins,
\end{lieu}

Car ce raisonnement paraîtra d'autant plus vrai qu'il n'y a rien dans le monde qui ne recherche le plaisir, et qu'on souhaite
toujours le plaisir à cause de lui-même, qui sont deux qualités essentielles que nous avons attribuées au bien et à la fin en
apportant leurs définitions. Au reste une chose apporte plus de plaisir qu'une autre en deux façons: et quand elle est
accompagnée de moins de douleur, et quand le plaisir qu'elle donne est d'une plus longue durée.

Il y aura lieu encore de conclure:

\begin{lieu}
	Que ce qui sera plus honnête vaudra mieux que ce qui le sera moins
\end{lieu}

Vu que tout ce qui est honnête ou apporte du plaisir est souhaitable à cause de lui-même.

\bigbreak

Et pareillement:

\begin{lieu}
	Que tout ce que nous aimerions mieux nous procurer à nous-mêmes ou à nos amis sera un plus grand avantage,
\end{lieu}

Comme au contraire un plus grand mal:

\begin{lieu}
	Tout ce que nous aimerions mieux éviter ou faire éviter à nos amis.
\end{lieu}

\bigbreak

Il y aura lieu encore de soutenir:

\begin{lieu}
	Que ce qui sera d'une plus longue durée doit être préféré à ce qui ne durera pas tant,
\end{lieu}

Et tout de même:

\begin{lieu}
	Que ce qui sera moins sujet au changement vaudra mieux que ce qui sera d'une naturel plus changeant,
\end{lieu}

Attendu que l'usage de ces deux choses l'emportera sur celui des deux autres; puisque ce qui sera d'une plus
longue durée apportera plus d'utilité, à cause qu'on s'en servira plus longtemps. Et par la même raison, tout
ce qui sera d'une nature moins changeante, vu que nous aurons la liberté de nous en servir toutes et quantes
fois qu'il nous plaira. Car c'est seulement de ce qui ne change point qu'on se peut servir quand on veut, parce
qu'on le trouve toujours en même état.

\bigbreak

On pourra dire encore:

\begin{lieu}
	Que telles que seront entre elles deux chose comprises sous quelqu'un des termes que nous appelons \emph{
	conjugués et de cas semblables}, telles seront entre elles aussi toutes les autre qui seront de leur suite
	et de leur dépendance.
\end{lieu}

Par exemple, s'il est vrai de dire que ce que signifie le mot \emph{vaillamment}, qui est un terme conjugué, est
quelque chose de plus honnête et plus à souhaiter que ce qui est signifié par ce mot \emph{tempéramment}\footnote{
Avec deux m. Paléonéologisme de François \textsc{Cassandre}. Exemple: <<~Pour votre santé, buvez tempéramment.~>>},
qui est un autre terme conjugué, il faudra conclure que la valeur sera plus souhaitable que la tempérance, et qu'être
vaillant sera une vertu beaucoup plus considérable que d'être tempérant.

\bigbreak

Ce raisonnement encore pourra servir:

\begin{lieu}
	Qu'une chose que tout le monde souhaitera, ou beaucoup de personnes, vaudra mieux qu'une autre que tout le monde ne
	souhaitera pas, ou peu de personnes seulement.
\end{lieu}

Et cela conformément à la définition du bien que nous avons donnée. Car s'il est vrai que le bien soit \emph{une chose
que tout le monde souhaite généralement}, la conséquence est nécessaire, que tout ce qui sera davantage souhaité sera
aussi un plus grand bien.

\bigbreak

On pourra encore faire valoir ce raisonnement:

\begin{lieu}
	Que ce que nos parties adverses, ou nos ennemis, ou nos juges, ou des experts ayant commission d'eux de nous juger,
	auront déclaré être un plus grand bien, sans doute il doit passer pour tel.
\end{lieu}

Car, quant à l'approbation de nos parties adverses et de nos ennemis, on pourra soutenir qu'elle doit tenir lieu d'une
approbation générale et que c'est autant que si tout le monde en était demeuré d'accord. A l'égard des juges, leur jugement
encore sera très considérable, tant parce qu'il n'y aura qu'eux qui aient autorité de prononcer sur de semblables matières
que parce qu'ils y seront très intelligents.

\bigbreak

Quelquefois encore on pourra soutenir:

\begin{lieu}
	Qu'une chose à laquelle tout le monde participe est digne d'une plus grande estime,
\end{lieu}

Puisque en quelque façon, il y a de la honte à n'y pas participer comme les autres.

Quelquefois, le contraire aura lieu, par exemple, si une chose est de telle qualité:

\begin{lieu}
	Qu'aucun autre ne la possède que nous, ou fort peu de personnes,
\end{lieu}

Attendu qu'elle en sera beaucoup plus rare.

\bigbreak

On pourra dire encore:

\begin{lieu}
	Que ce qui est plus digne de louange est aussi plus considérable,
\end{lieu}

Puisque pour être tel, il faut qu'il soit plus honnête.

\bigbreak

Et tout de même:

\begin{lieu}
	Qu'une chose à qui on rend plus d'honneur doit être plus estimée,
\end{lieu}

À cause que l'honneur qu'on lui rend fait comme voir ce qu'elle vaut.

Et par la même raison au contraire:

\begin{lieu}
	Que tout ce qui est suivi d'un plus grand châtiment est un plus grand mal.
\end{lieu}

\bigbreak

Et encore il sera aisé de représenter comme meilleur:

\begin{lieu}
	Ce qui surpassera une chose reconnue généralement pour un très grand avantage, ou du moins qui paraîtra telle.
\end{lieu}

\bigbreak

Pour rendre une chose plus grande qu'une autre en apparence, on se pourra servir d'adresse qui est de la diviser en
plusieurs parties, parce que toutes ces parties la feront paraître comme multipliée et surpasser par un plus grand
nombre d'effets. De cette adresse s'est servi le poète à l'endroit où la femme de Méléagre veut persuader son mari de
prendre les armes pour la défense de son pays, car faisant la peinture du malheur d'une ville prise par force, c'est
ainsi qu'elle parle:

\begin{emphpar}
  Hélas! combien répand et de sang et de larmes

  Une ville exposée à la fureur des armes?

  Partout ce n'est que meurtre et que feux allumés

  Maisons, temples, palais brûlent, sont consumés.

  On voit traîner captifs par des troupes barbares, femmes, filles, enfants\dots
\end{emphpar}

\bigbreak

On se pourra aussi servir de l'adresse contraire, en assemblant plusieurs choses en une et en les entassant, comme fait
\textsc{Épicharme}, et cela pour la même raison que nous venons d'alléguer touchant la division d'une chose en ses
parties; car d'assembler ainsi plusieurs choses, non seulement l'objet grossit à la vue et parait beaucoup plus, mais
encore on se figure qu'il cause de très grands effets. 

\bigbreak

Outre ces adresses, parce que nous avons remarqué qu'un bien qui est plus difficile à acquérir, ou plus rare, est aussi
ordinairement plus considérable, on pourra encore faire paraître une chose plus grande en faisant valoir toutes les circonstances
qui l’accompagnent, comme sont les occasions, les lieux, le temps, l'âge et les forces des personnes qui l'auront faite. Car
si quelqu'un, par exemple, a réussi dans une entreprise qui passait de beaucoup ses forces et son âge, ou à laquelle pas un
de ses pareils n'eût jamais osé penser, ou encore s'il l'a exécutée d'une certaine façon ou en certain lieu, sans doute
qu'alors cette entreprise doit être tenue pour bien plus glorieuse, que si elle était sans toutes ces circonstances. Or, non
seulement cette adresse pourra servir à faire estimer davantage une action juste, utile, ou qui aura été faite pour acquérir
de l'honneur, mais encore, elle servira à rendre plus blâmable tout ce qui aura été fait au contraire. Il y a un exemple de
ceci dans l'épigramme composée à la louange du poissonnier d'Argos qui remporta le prix aux jeux olympiques. C'est lui-même
qui parle:

\begin{emphpar}
  Aurait-on jamais cru qu'un jour j'eusse la gloire

  D'obtenir en ce lieu cette illustre victoire?

  Moi qu'on vit mille fois, un panier sur le dos et l'épaule chargée,

  Apporter du poisson de la ville d'Argos pour le vendre à Tégée.
\end{emphpar}

C'est de cette façon que se louait Iphicrate lui-même, ce fameux général d'armée des Athéniens, qui, de fils de savetier
qu'il était, monta enfin a ce haut degré d'honneur: \emph{Qui étais-je autrefois}, disait-il, \emph{pour être maintenant
ce que je suis?}

\bigbreak

On pourra encore tirer cette conséquence:

\begin{lieu}
  Que ce qui nous vient naturellement et qui naît avec nous vaudra beaucoup mieux que tout ce que nous empruntons d'ailleurs
  et que nous pouvons acquérir, 
\end{lieu}

À cause que l'acquisition en est bien plus difficile. D'où vient qu'\textsc{Homère} fait dire à Phémios:

\begin{emphpar}
  Ce que je sais, je le sais de moi-même.
\end{emphpar}

\bigbreak

Cette conséquence encore sera bonne:

\begin{lieu}
  Que la partie la plus considérable d'une chose qui de soi est considérable doit être plus estimée que pas une des autres
  parties.
\end{lieu}

C'est aussi sur ce fondement que \textsc{Périclès}, dans l'oraison funèbre qu'il fit à l'honneur de ceux qui étaient morts
au service de l’état dit \emph{que la perte d'une jeunesse si vaillante n'était pas moins considérable à la république
d'Athènes que ne le serait a l'année le retranchement du printemps}.

\bigbreak

Il faudra mettre encore au rang des plus grands biens:

\begin{lieu}
  Les choses qui nous serviront davantage dans notre plus grand besoin.
\end{lieu}

Par exemple dans notre vieillesse, ou lorsque nous serons malades.

\bigbreak

On pourra aussi prétendre 

\begin{lieu}
  Que de deux choses qui se rapportent à une même fin, celle qui touche de plus prés cette fin est la meilleure.
\end{lieu}

\bigbreak

Et encore, il y aura lieu d'assurer:

\begin{lieu}
  Qu'un bien qui nous regardera particulièrement vaut mieux qu'un autre qui sera simplement un bien en général.
\end{lieu}

Et tout de même:

\begin{lieu}
  Qu'un bien qui sera en notre puissance d'acquérir si nous voulons sera préférable à un autre que de toute
  impossibilité, nous ne saurions jamais avoir,
\end{lieu}

Attendu que l'un nous regarde et que nous en pouvons jouir, mais non pas de l'autre.

\bigbreak

De plus ce raisonnement pourra servir:

\begin{lieu}
  Que les biens qu'on ne pourra obtenir que sur la fin de ses jours seront beaucoup plus à estimer,
\end{lieu}

À cause qu'étant plus proches de la fin, ils sembleront plus participer à sa nature.

On dira aussi:

\begin{lieu}
  Que ce qui tient plus de la vérité vaut mieux que tout ce qui ne dépend que de l’opinion.
\end{lieu}

Or, pour savoir quand une chose dépendra seulement de l'opinion, il faut examiner si celui qui la fait voudrait la
faire au cas qu'elle ne vint à la connaissance de personnes. Ainsi, on pourra dire que recevoir du bien de quelqu'un
est plus à souhaiter que d'en faire, à cause que volontiers on recevrait d'autrui, si l'on était assuré que cela fût
toujours secret. Il n'en est pas de même de donner, vu qu'il semble que jamais on ne voudrait rien donner, si en donnant
l'on croyait que cette libéralité demeurât toujours cachée.

\bigbreak

On pourra encore faire passer pour meilleur et plus avantageux:

\begin{lieu}
  Tout ce que l'on aimerait mieux avoir en effet qu'en apparence,
\end{lieu}

Puisque ces choses-là tiendront davantage de la vérité. D'où vient que quelques-uns soutiennent que la justice est une vertu
dont on ne doit pas faire grand état, à cause, disent-ils, qu'on aimerait beaucoup mieux paraître juste que de l'être. Il
n'en est pas ainsi de la santé, puisqu'il vaut mieux se bien porter en effet que de n'avoir la santé qu'en apparence.

\bigbreak

On pourra encore avancer:

\begin{lieu}
  Que ce qui sera utile à plus de choses doit être davantage estimé.
\end{lieu}

Par exemple, ce qui en même temps contribuera non seulement à nous faire vivre, mais encore à nous faire vivre agréablement,
à nous donner la jouissance de toutes sortes de plaisirs, et à nous faire entreprendre de grandes choses. Aussi est-ce pour
cela que les richesses et la santé sont si estimées dans le monde, parce qu'en elles se trouvent tous ces grands avantages.

\bigbreak

On dira le même:

\begin{lieu}
  De tout ce qui est à la fois exempt de douleur et accompagné de plaisir,
\end{lieu}

Car, sans doute, deux biens ensemble valent mieux qu'un. Or le plaisir est un bien, et aussi l'indolence. Par indolence,
j'entends la privation de toute sorte d'incommodité et de douleur.

\bigbreak

Il y aura encore lieu d'inférer:

\begin{lieu}
  Que de deux biens, dont l'un ajouté à une certaine chose sera un tout plus considérable que si on y ajoutait l'autre,
  celui qui sera un tout plus considérable sera beaucoup meilleur.
\end{lieu}

Et tout de même on dira:

\begin{lieu}
  Qu'un bien qui se fait sentir et apercevoir aussitôt qu'on l'a sera préférable à un autre qui ne se fait pas sentir ni
  apercevoir davantage quand on l'a que quand on ne l'a pas,
\end{lieu}

Car l'un, sans doute, tient beaucoup plus de la vérité que l'autre. Aussi estime t-on un bien plus grand avantage d'être
riche en effet, que de le paraître simplement.

\bigbreak

Enfin on soutiendra:

\begin{lieu}
  Que les choses qu'on tient plus chères que d'autres seront aussi beaucoup plus estimables; et plus, sans doute, à ceux à
  qui il n'en restera plus qu'une, de plusieurs qu'ils avaient auparavant, qu'aux autres qui n'en ont pas pour une seule.
\end{lieu}

D'où vient aussi que la loi est beaucoup plus sévère à un homme qui a crevé l'œil à un autre qui n'avait que celui-là que s'il
le crevait à un qui eût encore ses deux yeux; parce qu'en effet, il le prive alors d'une chose qui lui était d'autant plus chère,
que c'était la seule qui lui restait.

\bigbreak

Voilà à peu près les preuves dont il se faut servir quand on aura à persuader ou à dissuader.


\section{De l'autorité souveraine, et de chaque sorte d'état en particulier}

Dein Syria per speciosam interpatet diffusa planitiem. hanc nobilitat Antiochia, mundo cognita civitas, cui non certaverit alia 
advecticiis ita adfluere copiis et internis, et Laodicia et Apamia itidemque Seleucia iam inde a primis auspiciis florentissimae.

Ipsam vero urbem Byzantiorum fuisse refertissimam atque ornatissimam signis quis ignorat? Quae illi, exhausti sumptibus bellisque 
maximis, cum omnis Mithridaticos impetus totumque Pontum armatum affervescentem in Asiam atque erumpentem, ore repulsum 
et cervicibus interclusum suis sustinerent, tum, inquam, Byzantii et postea signa illa et reliqua urbis ornanemta sanctissime 
custodita tenuerunt.

Qui cum venisset ob haec festinatis itineribus Antiochiam, praestrictis palatii ianuis, contempto Caesare, quem videri decuerat, ad 
praetorium cum pompa sollemni perrexit morbosque diu causatus nec regiam introiit nec processit in publicum, sed abditus multa in 
eius moliebatur exitium addens quaedam relationibus supervacua, quas subinde dimittebat ad principem.

Atque, ut Tullius ait, ut etiam ferae fame monitae plerumque ad eum locum ubi aliquando pastae sunt revertuntur, ita homines 
instar turbinis degressi montibus impeditis et arduis loca petivere mari confinia, per quae viis latebrosis sese convallibusque 
occultantes cum appeterent noctes luna etiam tum cornuta ideoque nondum solido splendore fulgente nauticos observabant quos 
cum in somnum sentirent effusos per ancoralia, quadrupedo gradu repentes seseque suspensis passibus iniectantes in scaphas 
eisdem sensim nihil opinantibus adsistebant et incendente aviditate saevitiam ne cedentium quidem ulli parcendo obtruncatis 
omnibus merces opimas velut viles nullis repugnantibus avertebant. haecque non diu sunt perpetrata.

Erat autem diritatis eius hoc quoque indicium nec obscurum nec latens, quod ludicris cruentis delectabatur et in circo sex vel 
septem aliquotiens vetitis certaminibus pugilum vicissim se concidentium perfusorumque sanguine specie ut lucratus ingentia 
laetabatur.

Superatis Tauri montis verticibus qui ad solis ortum sublimius attolluntur, Cilicia spatiis porrigitur late distentis dives bonis omnibus 
terra, eiusque lateri dextro adnexa Isauria, pari sorte uberi palmite viget et frugibus minutis, quam mediam navigabile flumen 
Calycadnus interscindit.

Quam quidem partem accusationis admiratus sum et moleste tuli potissimum esse Atratino datam. Neque enim decebat neque 
aetas illa postulabat neque, id quod animadvertere poteratis, pudor patiebatur optimi adulescentis in tali illum oratione versari. 
Vellem aliquis ex vobis robustioribus hunc male dicendi locum suscepisset; aliquanto liberius et fortius et magis more nostro 
refutaremus istam male dicendi licentiam. Tecum, Atratine, agam lenius, quod et pudor tuus moderatur orationi meae et meum 
erga te parentemque tuum beneficium tueri debeo.

Nemo quaeso miretur, si post exsudatos labores itinerum longos congestosque adfatim commeatus fiducia vestri ductante 
barbaricos pagos adventans velut mutato repente consilio ad placidiora deverti.

Ob haec et huius modi multa, quae cernebantur in paucis, omnibus timeri sunt coepta. et ne tot malis dissimulatis paulatimque 
serpentibus acervi crescerent aerumnarum, nobilitatis decreto legati mittuntur: Praetextatus ex urbi praefecto et ex vicario 
Venustus et ex consulari Minervius oraturi, ne delictis supplicia sint grandiora, neve senator quisquam inusitato et inlicito more 
tormentis exponeretur.

Eodem tempore etiam Hymetii praeclarae indolis viri negotium est actitatum, cuius hunc novimus esse textum. cum Africam pro 
consule regeret Carthaginiensibus victus inopia iam lassatis, ex horreis Romano populo destinatis frumentum dedit, pauloque 
postea cum provenisset segetum copia, integre sine ulla restituit mora.


\chapter{Le genre démonstratif}
\section{De la vertu en général et en particulier, avec les lieux et les adresses qui regardent la louange et le blâme}

Dein Syria per speciosam interpatet diffusa planitiem. hanc nobilitat Antiochia, mundo cognita civitas, cui non certaverit alia 
advecticiis ita adfluere copiis et internis, et Laodicia et Apamia itidemque Seleucia iam inde a primis auspiciis florentissimae.

Ipsam vero urbem Byzantiorum fuisse refertissimam atque ornatissimam signis quis ignorat? Quae illi, exhausti sumptibus bellisque 
maximis, cum omnis Mithridaticos impetus totumque Pontum armatum affervescentem in Asiam atque erumpentem, ore repulsum 
et cervicibus interclusum suis sustinerent, tum, inquam, Byzantii et postea signa illa et reliqua urbis ornanemta sanctissime 
custodita tenuerunt.

Qui cum venisset ob haec festinatis itineribus Antiochiam, praestrictis palatii ianuis, contempto Caesare, quem videri decuerat, ad 
praetorium cum pompa sollemni perrexit morbosque diu causatus nec regiam introiit nec processit in publicum, sed abditus multa in 
eius moliebatur exitium addens quaedam relationibus supervacua, quas subinde dimittebat ad principem.

Atque, ut Tullius ait, ut etiam ferae fame monitae plerumque ad eum locum ubi aliquando pastae sunt revertuntur, ita homines 
instar turbinis degressi montibus impeditis et arduis loca petivere mari confinia, per quae viis latebrosis sese convallibusque 
occultantes cum appeterent noctes luna etiam tum cornuta ideoque nondum solido splendore fulgente nauticos observabant quos 
cum in somnum sentirent effusos per ancoralia, quadrupedo gradu repentes seseque suspensis passibus iniectantes in scaphas 
eisdem sensim nihil opinantibus adsistebant et incendente aviditate saevitiam ne cedentium quidem ulli parcendo obtruncatis 
omnibus merces opimas velut viles nullis repugnantibus avertebant. haecque non diu sunt perpetrata.

Erat autem diritatis eius hoc quoque indicium nec obscurum nec latens, quod ludicris cruentis delectabatur et in circo sex vel 
septem aliquotiens vetitis certaminibus pugilum vicissim se concidentium perfusorumque sanguine specie ut lucratus ingentia 
laetabatur.

Superatis Tauri montis verticibus qui ad solis ortum sublimius attolluntur, Cilicia spatiis porrigitur late distentis dives bonis omnibus 
terra, eiusque lateri dextro adnexa Isauria, pari sorte uberi palmite viget et frugibus minutis, quam mediam navigabile flumen 
Calycadnus interscindit.

Quam quidem partem accusationis admiratus sum et moleste tuli potissimum esse Atratino datam. Neque enim decebat neque 
aetas illa postulabat neque, id quod animadvertere poteratis, pudor patiebatur optimi adulescentis in tali illum oratione versari. 
Vellem aliquis ex vobis robustioribus hunc male dicendi locum suscepisset; aliquanto liberius et fortius et magis more nostro 
refutaremus istam male dicendi licentiam. Tecum, Atratine, agam lenius, quod et pudor tuus moderatur orationi meae et meum 
erga te parentemque tuum beneficium tueri debeo.

Nemo quaeso miretur, si post exsudatos labores itinerum longos congestosque adfatim commeatus fiducia vestri ductante 
barbaricos pagos adventans velut mutato repente consilio ad placidiora deverti.

Ob haec et huius modi multa, quae cernebantur in paucis, omnibus timeri sunt coepta. et ne tot malis dissimulatis paulatimque 
serpentibus acervi crescerent aerumnarum, nobilitatis decreto legati mittuntur: Praetextatus ex urbi praefecto et ex vicario 
Venustus et ex consulari Minervius oraturi, ne delictis supplicia sint grandiora, neve senator quisquam inusitato et inlicito more 
tormentis exponeretur.

Eodem tempore etiam Hymetii praeclarae indolis viri negotium est actitatum, cuius hunc novimus esse textum. cum Africam pro 
consule regeret Carthaginiensibus victus inopia iam lassatis, ex horreis Romano populo destinatis frumentum dedit, pauloque 
postea cum provenisset segetum copia, integre sine ulla restituit mora.


\chapter{Le genre judiciaire}
\section{Ce que c'est que faire tort ou injure}

Dein Syria per speciosam interpatet diffusa planitiem. hanc nobilitat Antiochia, mundo cognita civitas, cui non certaverit alia 
advecticiis ita adfluere copiis et internis, et Laodicia et Apamia itidemque Seleucia iam inde a primis auspiciis florentissimae.

Ipsam vero urbem Byzantiorum fuisse refertissimam atque ornatissimam signis quis ignorat? Quae illi, exhausti sumptibus bellisque 
maximis, cum omnis Mithridaticos impetus totumque Pontum armatum affervescentem in Asiam atque erumpentem, ore repulsum 
et cervicibus interclusum suis sustinerent, tum, inquam, Byzantii et postea signa illa et reliqua urbis ornanemta sanctissime 
custodita tenuerunt.

Qui cum venisset ob haec festinatis itineribus Antiochiam, praestrictis palatii ianuis, contempto Caesare, quem videri decuerat, ad 
praetorium cum pompa sollemni perrexit morbosque diu causatus nec regiam introiit nec processit in publicum, sed abditus multa in 
eius moliebatur exitium addens quaedam relationibus supervacua, quas subinde dimittebat ad principem.

Atque, ut Tullius ait, ut etiam ferae fame monitae plerumque ad eum locum ubi aliquando pastae sunt revertuntur, ita homines 
instar turbinis degressi montibus impeditis et arduis loca petivere mari confinia, per quae viis latebrosis sese convallibusque 
occultantes cum appeterent noctes luna etiam tum cornuta ideoque nondum solido splendore fulgente nauticos observabant quos 
cum in somnum sentirent effusos per ancoralia, quadrupedo gradu repentes seseque suspensis passibus iniectantes in scaphas 
eisdem sensim nihil opinantibus adsistebant et incendente aviditate saevitiam ne cedentium quidem ulli parcendo obtruncatis 
omnibus merces opimas velut viles nullis repugnantibus avertebant. haecque non diu sunt perpetrata.

Erat autem diritatis eius hoc quoque indicium nec obscurum nec latens, quod ludicris cruentis delectabatur et in circo sex vel 
septem aliquotiens vetitis certaminibus pugilum vicissim se concidentium perfusorumque sanguine specie ut lucratus ingentia 
laetabatur.

Superatis Tauri montis verticibus qui ad solis ortum sublimius attolluntur, Cilicia spatiis porrigitur late distentis dives bonis omnibus 
terra, eiusque lateri dextro adnexa Isauria, pari sorte uberi palmite viget et frugibus minutis, quam mediam navigabile flumen 
Calycadnus interscindit.

Quam quidem partem accusationis admiratus sum et moleste tuli potissimum esse Atratino datam. Neque enim decebat neque 
aetas illa postulabat neque, id quod animadvertere poteratis, pudor patiebatur optimi adulescentis in tali illum oratione versari. 
Vellem aliquis ex vobis robustioribus hunc male dicendi locum suscepisset; aliquanto liberius et fortius et magis more nostro 
refutaremus istam male dicendi licentiam. Tecum, Atratine, agam lenius, quod et pudor tuus moderatur orationi meae et meum 
erga te parentemque tuum beneficium tueri debeo.

Nemo quaeso miretur, si post exsudatos labores itinerum longos congestosque adfatim commeatus fiducia vestri ductante 
barbaricos pagos adventans velut mutato repente consilio ad placidiora deverti.

Ob haec et huius modi multa, quae cernebantur in paucis, omnibus timeri sunt coepta. et ne tot malis dissimulatis paulatimque 
serpentibus acervi crescerent aerumnarum, nobilitatis decreto legati mittuntur: Praetextatus ex urbi praefecto et ex vicario 
Venustus et ex consulari Minervius oraturi, ne delictis supplicia sint grandiora, neve senator quisquam inusitato et inlicito more 
tormentis exponeretur.

Eodem tempore etiam Hymetii praeclarae indolis viri negotium est actitatum, cuius hunc novimus esse textum. cum Africam pro 
consule regeret Carthaginiensibus victus inopia iam lassatis, ex horreis Romano populo destinatis frumentum dedit, pauloque 
postea cum provenisset segetum copia, integre sine ulla restituit mora.


\section{Des choses qui sont agréables et donnent du plaisir}

Dein Syria per speciosam interpatet diffusa planitiem. hanc nobilitat Antiochia, mundo cognita civitas, cui non certaverit alia 
advecticiis ita adfluere copiis et internis, et Laodicia et Apamia itidemque Seleucia iam inde a primis auspiciis florentissimae.

Ipsam vero urbem Byzantiorum fuisse refertissimam atque ornatissimam signis quis ignorat? Quae illi, exhausti sumptibus bellisque 
maximis, cum omnis Mithridaticos impetus totumque Pontum armatum affervescentem in Asiam atque erumpentem, ore repulsum 
et cervicibus interclusum suis sustinerent, tum, inquam, Byzantii et postea signa illa et reliqua urbis ornanemta sanctissime 
custodita tenuerunt.

Qui cum venisset ob haec festinatis itineribus Antiochiam, praestrictis palatii ianuis, contempto Caesare, quem videri decuerat, ad 
praetorium cum pompa sollemni perrexit morbosque diu causatus nec regiam introiit nec processit in publicum, sed abditus multa in 
eius moliebatur exitium addens quaedam relationibus supervacua, quas subinde dimittebat ad principem.

Atque, ut Tullius ait, ut etiam ferae fame monitae plerumque ad eum locum ubi aliquando pastae sunt revertuntur, ita homines 
instar turbinis degressi montibus impeditis et arduis loca petivere mari confinia, per quae viis latebrosis sese convallibusque 
occultantes cum appeterent noctes luna etiam tum cornuta ideoque nondum solido splendore fulgente nauticos observabant quos 
cum in somnum sentirent effusos per ancoralia, quadrupedo gradu repentes seseque suspensis passibus iniectantes in scaphas 
eisdem sensim nihil opinantibus adsistebant et incendente aviditate saevitiam ne cedentium quidem ulli parcendo obtruncatis 
omnibus merces opimas velut viles nullis repugnantibus avertebant. haecque non diu sunt perpetrata.

Erat autem diritatis eius hoc quoque indicium nec obscurum nec latens, quod ludicris cruentis delectabatur et in circo sex vel 
septem aliquotiens vetitis certaminibus pugilum vicissim se concidentium perfusorumque sanguine specie ut lucratus ingentia 
laetabatur.

Superatis Tauri montis verticibus qui ad solis ortum sublimius attolluntur, Cilicia spatiis porrigitur late distentis dives bonis omnibus 
terra, eiusque lateri dextro adnexa Isauria, pari sorte uberi palmite viget et frugibus minutis, quam mediam navigabile flumen 
Calycadnus interscindit.

Quam quidem partem accusationis admiratus sum et moleste tuli potissimum esse Atratino datam. Neque enim decebat neque 
aetas illa postulabat neque, id quod animadvertere poteratis, pudor patiebatur optimi adulescentis in tali illum oratione versari. 
Vellem aliquis ex vobis robustioribus hunc male dicendi locum suscepisset; aliquanto liberius et fortius et magis more nostro 
refutaremus istam male dicendi licentiam. Tecum, Atratine, agam lenius, quod et pudor tuus moderatur orationi meae et meum 
erga te parentemque tuum beneficium tueri debeo.

Nemo quaeso miretur, si post exsudatos labores itinerum longos congestosque adfatim commeatus fiducia vestri ductante 
barbaricos pagos adventans velut mutato repente consilio ad placidiora deverti.

Ob haec et huius modi multa, quae cernebantur in paucis, omnibus timeri sunt coepta. et ne tot malis dissimulatis paulatimque 
serpentibus acervi crescerent aerumnarum, nobilitatis decreto legati mittuntur: Praetextatus ex urbi praefecto et ex vicario 
Venustus et ex consulari Minervius oraturi, ne delictis supplicia sint grandiora, neve senator quisquam inusitato et inlicito more 
tormentis exponeretur.

Eodem tempore etiam Hymetii praeclarae indolis viri negotium est actitatum, cuius hunc novimus esse textum. cum Africam pro 
consule regeret Carthaginiensibus victus inopia iam lassatis, ex horreis Romano populo destinatis frumentum dedit, pauloque 
postea cum provenisset segetum copia, integre sine ulla restituit mora.


\section{Ceux qui font injure à autrui}

Faisons à présent connaître l'état et le raisonnement de ceux qui proposent de faire injure à autrui, et de plus,
à quelles personnes ils s'attaquent ordinairement.

\bigbreak

Les hommes donc sont portés à faire injure, en quatre façons. 

\begin{emphpar}
	Ou quand ils croient que ce qu'ils veulent en reprendre est possible et qu'eux-mêmes en pourront venir à
	bout;

	Ou qu'ils pensent qu'après l'avoir fait, on n'en saura rien et qu'ils ne seront point découverts;

	Ou si l'on vient à découvrir que c'est eux, qu'ils n'en seront point punis;

	Enfin, s'ils en sont punis, que la punition n'égalera point le profit qui leur en reviendra, soit à eux en
	particulier, ou à ceux qui les touchent et pour qui ils s'intéressent.
\end{emphpar}

De savoir, maintenant, quelles choses sont possibles à faire ou impossibles, c'est une matière que nous ne
traiterons pas encore si tôt, à cause qu'elle regarde en commun toutes les parties de la rhétorique. 

\subsection{Ceux qui se promettent l'impunité}

Or, entre les personnes qui s'engagent à faire tort à autrui, ceux-là particulièrement croient le pouvoir faire
avec impunité qui sont éloquents ou entreprenants et gens d'exécution, ou qui ont acquis une grande expérience
dans le monde, et vu ou manié une infinité d'affaires; en un mot, ceux qui ont beaucoup d'amis, et qui sont
riches. Mais surtout, ils se promettront l'impunité, s'ils se voient fortifiés de tous les avantages que nous
venons de remarquer, ou du moins quelqu'un de leurs amis, ou de leurs Associés, ou même des personnes qui dépendent
d'eux et qui sont à leur service. Car par le moyen de tous ces avantages, non seulement ils exécuteront leur mauvais
dessein, mais encore ils pourront ni être découverts, ni punis. 

Ceux-là encore se promettront l'impunité qui seront amis des personnes mêmes à qui ils voudront faire injure, ou
des juges devant qui ils auront à répondre. Car quant aux amis, il n'est rien de si aisé que de leur faire tort, à
cause qu'ils ne s'en défient pont, joint qu'ils sont plutôt d'accord et réconciliés qu'ils n'ont songé à plaider
ni à faire aucune poursuite en jugement. À l'égard des juges, il est certain encore qu'ils font toujours faveur à
leurs amis; car de deux choses l'une, ou ils les renvoient absous, ou ils ne les condamnent que légèrement.

\subsection{Ceux qui croient qu'ils ne seront point découverts}

Ceux-là aussi auront espérance de n'être point découvert de qui l'apparence sera si trompeuse qu'à juger d'eux par
l'extérieur, jamais on ne les prendrait pour avoir fait ce qu'ils auront fait effectivement, comme quand quelqu'un
en apparence très faible de corps en aura battu un autre outrageusement, qui paraîtra de beaucoup plus fort que lui,
ou quand un gueux aura couché avec une dame de condition, ou un homme très laid avec une fort belle femme. 

Les choses encore qui sont trop en jour, et exposées aux yeux de trop de monde, pourront faire croire à un
méchant homme qu'il ne sera point découvert s'il les prend; la raison est que personne ne s'en donne de garde,
et qu'ordinairement, on ne s'imagine pas qu'il y ait des gens assez hardis pour oser seulement y penser. 

De plus, on croira n'être point découvert si le crime est de telle nature et si énorme qu'on n'ait pas même
connaissance que jamais il ait été commis, puisque c'est une chose à laquelle on ne songe point et dont personne
ne se défie. Car les hommes n'ont point accoutumé de se préparer autrement contre les injures qu'ils font contre
les maladies; pas un ne craignant et ne tâchant d'éviter celles qu'il n'a pas encore éprouvées.

Tout homme encore qui n'aura point d'ennemis, ou au contraire qui en aura beaucoup, croira n'être pas découvert.
Car d'un coté, n'ayant point d'ennemis, il lui sera très facile de surprendre et de faire son coup, parce qu'on
ne se défiera point de lui. D'un autre côté aussi, ayant beaucoup d'ennemis, on aura de la peine à s'imaginer
qu'il ait osé s'attaquer à des personnes qui étaient sans cesse sur leurs gardes; outre que pour sa défense,
il aura cette raison à alléguer qu'il se fût bien empêché d'entreprendre une action de cette qualité, quand
bien même il en aurait eu envie, a cause qu'il devait être soupçonné plutôt qu'un autre.

Ceux-là enfin se pourront persuader de n'être point découverts, en faisant tort à autrui, qui auront moyen: ou
de cacher leur larcin, ou de le détourner, ou bien de lui faire changer de forme et de nature, ou de s'en défaire
promptement.

\subsection{Ceux qui ne craignent pas d'être punis}

D'autres, au contraires seront assurés d'être découverts et poursuivis en justice qui n'entreprendront pas moins
de faire du torts par exemple s'ils espèrent: ou d'échapper aux juges et de décliner leur juridiction, ou de
faire durer le procès fort longtemps, ou enfin de gagner les juges et de les corrompre.

\bigbreak

D'autres encore verront leur condamnation inévitable, mais parce qu'au plus il n'ira que d'une amende, ils ne s'en
mettront pas en peine, à cause qu'ils sauront les moyens: ou de s'en défendre et de jamais n'en rien payer, ou de
se faire donner un long terme pour y satisfaire, ou bien même que leur pauvreté sera si grande qu'ils n'auront rien
à perdre.

Ceux-là encore ne craindront point d'être condamnés en faisant tort, à qui le larcin promettra présentement: ou
bientôt un profit assuré, ou quelque avantage important et cependant, si on vient à les condamner, qu'ils en
sortiront pour peu de choses, ou même qu'il ne leur en coûtera rien; en tout cas, s'il leur en doit coûter, qu'il
se passera bien du temps avant que d'y avoir satisfait. 

On mettra encore de ce nombre tous ceux qui se proposeront d'acquérir une chose si considérable que la punition,
pour grande quelle soit, supposé même qu'elle arrive, n'égalera jamais l'avantage ni le profit qu'ils en tireront.
Tel est l'avantage que semble promettre la tyrannie à ceux qui ont envie de se rendre maîtres d'un état.

Ceux-là aussi n'appréhenderont pas d'être condamnés pour leur injustice s'ils trouvent qu'il y ait à gagner pour
eux et, quant à la punition, qu'ils en seront quittes pour un affront et pour quelque peu d'injures. Ni tous ceux,
au contraire, dont le crime les fera estimer et leur tournera à honneur, comme quand un homme viendra à venger en
même temps la mort de son père ou de sa mère, ainsi qu'il arriva à Zénon; et cependant que la punition ne pourra
aller au plus qu'à une amende, à un simple bannissement ou à quelque autre peine semblable. Car il est certain
qu'en ces deux rencontres, ces personnes-là seront portées à exécuter leur mauvais dessein; quoi qu'entre elles,
il y ait cette différence que les dernières sont louables pour leurs mœurs et les autres dignes de punition.

Ceux-là encore volontiers se hasarderont à faire tort, qui jamais n'auront été pris sur le fait, ni découverts,
ni punis; et pareillement ceux qui auront manqué plusieurs fois leur coup. Car il en prend ici comme à la
guerre, où souvent, il arrive aux vaincus de tenter la fortune de nouveau et de retourner au combat.

Ceux-là encore seront hardis qui auront espérance de jouir présentement: ou de tel plaisir en particulier, ou
d'avoir un tel profit à cause que s'il y a quelque chose à souffrir en punition ou quelque perte à faire, ce
ne sera qu'après. Tels sont d'ordinaire les incontinents et les débauchés. L'incontinence, au reste est un vice
qui regarde les choses où nous portent toutes les passions agréables et le dérèglement de la convoitise.

Il s'en trouve d'autres qui font le contraire de ceux-ci: d'abord ils préfèrent d'endurer quelque chose, ou de
faire quelque perte, parce qu'ils espèrent à l'avenir ou d'avoir un établissement assuré, ou de jouir d'un plaisir
très durable. Et c'est ce que font ordinairement ceux qui ont de la prudence, et qui ne sont pas adonnés a leur
plaisir. 

D'autres encore ne se soucieront pas qu'on sache que c'est eux qui ont fait tort, à cause qu'ils ne paraîtront
l'avoir fait que par malheur: ou par nécessité, ou dans un transport et un premier mouvement, ou par accoutumance.
En un mot, parce qu'ils paraîtront avoir plutôt failli qu'offensé malicieusement.

Ceux-là encore seront de ce nombre qui espéreront en la bonté des juges et qu'on ne les traitera pas à là rigueur,
comme aussi ceux qui seront pauvres. Or il y a deux sortes de pauvres dans le monde: les uns le sont des choses
nécessaires à l'entretien de la vie, comme ceux qui mendient, et les autres des choses superflues, comme la plupart
des riches. 

Enfin, ceux-là ne craindront point de faire du tort qui seront en très bonne estime; ni ceux, au contraire qui
seront tout à fait perdus de réputation. Car quant à ceux qui auront de l'estime, jamais on ne voudra croire que
ce soit eux, et pour les autres, ils n en seront pas plus décriés.

Voila ce que nous avions a dire touchant les personnes qui entreprennent de faire tort et injure à autrui. Voyons
maintenant ceux a qui on s'attaque ordinairement.

\subsection{Les personnes à qui d'ordinaire on fait tort}

Les méchants, donc, d'ordinaire, s'attaquent aux personnes qui possèdent les choses qu'ils n'ont pas et dont ils
ont besoin; soit qu'elles soient nécessaires à l'entretien de la vie, ou superflues, ou seulement pour la jouissance
et le plaisir.

Ils attaquent encore également leurs voisins et ceux qui sont d'un pays éloigné. Leurs voisins? Parce que leur coup
est bientôt fait. Les étrangers? À cause que, d'ordinaire, la vengeance en est tardive et qu'il leur faut beaucoup de
temps pour tirer raison du tort qu'ils ont reçu. Tels sont ceux par exemple qui attendent les Carthaginois au passage,
afin de les piller.

On fait tort encore ordinairement aux personnes négligentes et qui ne se tiennent point sur leurs gardes, ou qui sont
si simples qu'on leur peut faire accroire tout ce qu'on veut, pour ce qu'il y a lieu de s'imaginer qu'on ne sera point
découvert.

On s'adresse encore assez souvent à ceux qui sont d'un naturel lâche et qui aiment à vivre en repos; telles personnes
n'étant pas d'humeur à s'embarrasser d'un procès, à cause que la poursuite en est difficile, et qu'il faut être agissant
pour en venir à bout. 

Il en est de même de ceux qui ont beaucoup de pudeur, parce qu'ils ont l'honneur en recommandation et seraient
honteux de paraître en jugement pour un léger intérêt, et de plaider pour peu de chose.

On s'attaque encore, d'ordinaire, aux personnes que d'autres ont déjà attaquées ou offensées plusieurs fois, sans
que jamais elles en aient fait de poursuite, comme étant du nombre de ceux que le proverbe appelle \emph{la proie
des Mysiens}.

Tout ceux encore à qui une personne n'a jamais fait tort, et ceux au contraire à qui, plusieurs fois déjà, elle en
a fait sont en grand danger d'en être attaqués, à cause que ni les uns ni les autres ne se tiennent point sur leurs
gardes. Ceux-ci, parce qu'ils ne croient pas qu'elle leur en veuille plus faire; les autres, parce qu'elle ne leur
en a pas encore fait.

On se propose aussi de faire injure à ceux qui ont déjà été traduits en justice pour plusieurs crimes, ou à qui il
est très facile de faire faire le procès, à cause que telles gens n'oseront pas s'en plaindre, soit pour la crainte
qu'ils auront des juges, soit pour n'être pas en état d'être crus; ce qui se peut dire encore de tous ceux qui sont
haïs ou enviés de tout le monde.

D'ordinaire, encore, on s'attaque à ceux contre qui on a quelque prétexte et quelque raison spécieuse, soit qu'on
aille rechercher l'histoire de leurs ancêtres et qu'on déterre des querelles mortes et ensevelies, soit qu'on se
plaigne d'eux en particulier ou de quelqu'un de leurs amis; par exemple: ou pour être en état d'en recevoir
présentement du tort, ou pour en avoir déjà reçu plusieurs fois, soit en sa propre personne, soit en celle de ses
amis ou de ceux de qui on prend les intérêts. Car comme dit fort bien le proverbe: \emph{la malice n'a besoin que
de prétexte}.

On fait tort encore indifféremment à les ennemis et à ses amis propres. À ses amis? parce qu'il est très facile de
le faire. À ses ennemis? À cause qu'il y a du plaisir.

Il en est de même de ceux qui n'ont point d'amis, et des autre qui ne sont ni éloquents ni gens d'exécution, car:
ou ces personnes-là n'auront pas seulement la hardiesse de poursuivre en justice ceux qui leur auront fait tort,
ou, si elles le font, elles s'accorderont bientôt, ou même, ne gagneront rien à plaider.

Ceux-là aussi seront sujets à être attaqués, à qui il n'est pas avantageux de s'arrêter longtemps en un même lieu
dans l'attente qu'un procès soit terminé, ou qu'ils soient dédommagés et remboursés de leurs frais. Tels sont,
d'ordinaire, les personnes de dehors, et ceux qui n'ont d'autre revenu que le travail de leurs mains, car, pour
peu de chose, on compose avec eux, étant facile de les contenter.

On attaque encore volontiers ceux qui ont fait beaucoup de tort en leur vie, ou qui ont fait la même injure à
d'autres qu'on a dessein de leur faire; à cause qu'il ne semble pas que ce soit une injustice de traiter un méchant
homme de la même sorte qu'il a accoutumé de traiter les autres, comme quand quelqu'un qui est connu pour un
querelleur et pour battre ordinairement viendra lui-même à être très bien battu.

On tâche aussi de faire injure a ceux de qui, autrefois, on a reçu quelque déplaisir, ou qui ont eu dessein d'en
faire, ou qui ne manquent pas de volonté pour cela, ou même qui s'y préparent et qui font tout ce qu'ils peuvent
pour en venir à bout. Car non seulement on y trouvera du plaisir, mais encore cela fera honneur, outre qu'il ne
semblera pas qu'on ait fait une injustice.

C'est encore une occasion de faire injure à quelqu'un si en attaquant on est assuré de faire une chose agréable
et qui plaira extrêmement: ou à ses amis, ou à ceux qu'on estime beaucoup, ou aux personnes pour qui on a de
l'amour, ou à ses maîtres; en un mot, à tous ceux donc on dépend ou de qui on attend quelque faveur.

On cherche encore à nuire aux personnes qu'on a autrefois accusées de quelque crime, ou à l’amitié desquelles
on a renonce, témoin ce que fit Calippe contre Dion. Et ce qui donne d'autant plus de hardiesse alors, c'est
que même il ne semble pas qu'on fasse une injustice.

On attaque encore les personnes qu'on sait que d'autres sont tout prêts d'attaquer si on ne les prévient, comme n'y
ayant plus lieu de délibérer si on le doit faire ou non. De là vient qu'Ænésidème envoya des présents à Gélon pour
l'avoir prévenu en la réduction de certains peuples qu'il avait dessein d'assujettir lui-même.

On s'adresse encore à ceux à qui on ne doit faire qu'une seule fois du tort pour être en état de leur faire après
beaucoup de bien, à cause qu'il sera facile alors de guérir le mal, et les récompenser de leur perte. C'était sur
ce fondement que Jason le Thessalien avait accoutumé de dire \emph{qu'il est bon quelquefois de faire un peu de mal,
pour être en état après de faire beaucoup de bien}.

\subsection{Les injustices qui se font d'ordinaire}

Pour ce qui est des injustices, d'ordinaire on se laisse aller à celles que la plupart ou tout le monde fait, vu
qu'alors on se persuade qu'on aura sa grâce aisément.

On cherche encore a faire tort dans les choses qu'il est facile de cacher. Or ces choses-là sont de plusieurs sortes.
Les unes se consument en peu de temps, comme tout ce qui est bon à manger; d'autres sont aisées à déguiser, soit
qu'on leur donne une nouvelle figure, ou qu'on leur fasse changer de couleur, ou qu'on les mêle; d'autres peuvent
être détournées en divers lieux, comme tout ce qui est facile à transporter, ou qui tient peu de place; et quelques-unes
enfin sont telles, que comme celui qui les veut dérober en a beaucoup chez lui toutes semblables, jamais on ne
pourra les reconnaître lorsqu'elles seront ensemble. 

On fait encore injure dans les choses qu'on sait être honteuses à dire aux personnes mêmes à qui l'injure est faite,
comme quand on a abusé de la femme de quelqu'un, ou que lui-même ou ses enfants ont été contraints de céder à la
brutalité d'un infâme. 

On fait tort enfin et injure dans les choses pour lesquelles on peut intenter des procès sans se décrier et passer
pour chicaneur: ou à cause qu'elles sont de peu d'importance, ou parce que ce sont des fautes pardonnables. 

C'est a peu prés ce qui se peut dire sur cette matière, soit à l'égard de ceux qui font tort, ou des choses qu'ils
recherchaient, ou des personnes qu'ils attaquent; soit à l'égard des motifs et des raisons qui d'ordinaire les portent
à exécuter leur mauvais dessein.


\section{Des actions justes et injustes, et de l'équité}

Dein Syria per speciosam interpatet diffusa planitiem. hanc nobilitat Antiochia, mundo cognita civitas, cui non certaverit alia 
advecticiis ita adfluere copiis et internis, et Laodicia et Apamia itidemque Seleucia iam inde a primis auspiciis florentissimae.

Ipsam vero urbem Byzantiorum fuisse refertissimam atque ornatissimam signis quis ignorat? Quae illi, exhausti sumptibus bellisque 
maximis, cum omnis Mithridaticos impetus totumque Pontum armatum affervescentem in Asiam atque erumpentem, ore repulsum 
et cervicibus interclusum suis sustinerent, tum, inquam, Byzantii et postea signa illa et reliqua urbis ornanemta sanctissime 
custodita tenuerunt.

Qui cum venisset ob haec festinatis itineribus Antiochiam, praestrictis palatii ianuis, contempto Caesare, quem videri decuerat, ad 
praetorium cum pompa sollemni perrexit morbosque diu causatus nec regiam introiit nec processit in publicum, sed abditus multa in 
eius moliebatur exitium addens quaedam relationibus supervacua, quas subinde dimittebat ad principem.

Atque, ut Tullius ait, ut etiam ferae fame monitae plerumque ad eum locum ubi aliquando pastae sunt revertuntur, ita homines 
instar turbinis degressi montibus impeditis et arduis loca petivere mari confinia, per quae viis latebrosis sese convallibusque 
occultantes cum appeterent noctes luna etiam tum cornuta ideoque nondum solido splendore fulgente nauticos observabant quos 
cum in somnum sentirent effusos per ancoralia, quadrupedo gradu repentes seseque suspensis passibus iniectantes in scaphas 
eisdem sensim nihil opinantibus adsistebant et incendente aviditate saevitiam ne cedentium quidem ulli parcendo obtruncatis 
omnibus merces opimas velut viles nullis repugnantibus avertebant. haecque non diu sunt perpetrata.

Erat autem diritatis eius hoc quoque indicium nec obscurum nec latens, quod ludicris cruentis delectabatur et in circo sex vel 
septem aliquotiens vetitis certaminibus pugilum vicissim se concidentium perfusorumque sanguine specie ut lucratus ingentia 
laetabatur.

Superatis Tauri montis verticibus qui ad solis ortum sublimius attolluntur, Cilicia spatiis porrigitur late distentis dives bonis omnibus 
terra, eiusque lateri dextro adnexa Isauria, pari sorte uberi palmite viget et frugibus minutis, quam mediam navigabile flumen 
Calycadnus interscindit.

Quam quidem partem accusationis admiratus sum et moleste tuli potissimum esse Atratino datam. Neque enim decebat neque 
aetas illa postulabat neque, id quod animadvertere poteratis, pudor patiebatur optimi adulescentis in tali illum oratione versari. 
Vellem aliquis ex vobis robustioribus hunc male dicendi locum suscepisset; aliquanto liberius et fortius et magis more nostro 
refutaremus istam male dicendi licentiam. Tecum, Atratine, agam lenius, quod et pudor tuus moderatur orationi meae et meum 
erga te parentemque tuum beneficium tueri debeo.

Nemo quaeso miretur, si post exsudatos labores itinerum longos congestosque adfatim commeatus fiducia vestri ductante 
barbaricos pagos adventans velut mutato repente consilio ad placidiora deverti.

Ob haec et huius modi multa, quae cernebantur in paucis, omnibus timeri sunt coepta. et ne tot malis dissimulatis paulatimque 
serpentibus acervi crescerent aerumnarum, nobilitatis decreto legati mittuntur: Praetextatus ex urbi praefecto et ex vicario 
Venustus et ex consulari Minervius oraturi, ne delictis supplicia sint grandiora, neve senator quisquam inusitato et inlicito more 
tormentis exponeretur.

Eodem tempore etiam Hymetii praeclarae indolis viri negotium est actitatum, cuius hunc novimus esse textum. cum Africam pro 
consule regeret Carthaginiensibus victus inopia iam lassatis, ex horreis Romano populo destinatis frumentum dedit, pauloque 
postea cum provenisset segetum copia, integre sine ulla restituit mora.


\section{Pour connaître quand une action est plus injuste et plus criminelle qu'une autre}

Dein Syria per speciosam interpatet diffusa planitiem. hanc nobilitat Antiochia, mundo cognita civitas, cui non certaverit alia 
advecticiis ita adfluere copiis et internis, et Laodicia et Apamia itidemque Seleucia iam inde a primis auspiciis florentissimae.

Ipsam vero urbem Byzantiorum fuisse refertissimam atque ornatissimam signis quis ignorat? Quae illi, exhausti sumptibus bellisque 
maximis, cum omnis Mithridaticos impetus totumque Pontum armatum affervescentem in Asiam atque erumpentem, ore repulsum 
et cervicibus interclusum suis sustinerent, tum, inquam, Byzantii et postea signa illa et reliqua urbis ornanemta sanctissime 
custodita tenuerunt.

Qui cum venisset ob haec festinatis itineribus Antiochiam, praestrictis palatii ianuis, contempto Caesare, quem videri decuerat, ad 
praetorium cum pompa sollemni perrexit morbosque diu causatus nec regiam introiit nec processit in publicum, sed abditus multa in 
eius moliebatur exitium addens quaedam relationibus supervacua, quas subinde dimittebat ad principem.

Atque, ut Tullius ait, ut etiam ferae fame monitae plerumque ad eum locum ubi aliquando pastae sunt revertuntur, ita homines 
instar turbinis degressi montibus impeditis et arduis loca petivere mari confinia, per quae viis latebrosis sese convallibusque 
occultantes cum appeterent noctes luna etiam tum cornuta ideoque nondum solido splendore fulgente nauticos observabant quos 
cum in somnum sentirent effusos per ancoralia, quadrupedo gradu repentes seseque suspensis passibus iniectantes in scaphas 
eisdem sensim nihil opinantibus adsistebant et incendente aviditate saevitiam ne cedentium quidem ulli parcendo obtruncatis 
omnibus merces opimas velut viles nullis repugnantibus avertebant. haecque non diu sunt perpetrata.

Erat autem diritatis eius hoc quoque indicium nec obscurum nec latens, quod ludicris cruentis delectabatur et in circo sex vel 
septem aliquotiens vetitis certaminibus pugilum vicissim se concidentium perfusorumque sanguine specie ut lucratus ingentia 
laetabatur.

Superatis Tauri montis verticibus qui ad solis ortum sublimius attolluntur, Cilicia spatiis porrigitur late distentis dives bonis omnibus 
terra, eiusque lateri dextro adnexa Isauria, pari sorte uberi palmite viget et frugibus minutis, quam mediam navigabile flumen 
Calycadnus interscindit.

Quam quidem partem accusationis admiratus sum et moleste tuli potissimum esse Atratino datam. Neque enim decebat neque 
aetas illa postulabat neque, id quod animadvertere poteratis, pudor patiebatur optimi adulescentis in tali illum oratione versari. 
Vellem aliquis ex vobis robustioribus hunc male dicendi locum suscepisset; aliquanto liberius et fortius et magis more nostro 
refutaremus istam male dicendi licentiam. Tecum, Atratine, agam lenius, quod et pudor tuus moderatur orationi meae et meum 
erga te parentemque tuum beneficium tueri debeo.

Nemo quaeso miretur, si post exsudatos labores itinerum longos congestosque adfatim commeatus fiducia vestri ductante 
barbaricos pagos adventans velut mutato repente consilio ad placidiora deverti.

Ob haec et huius modi multa, quae cernebantur in paucis, omnibus timeri sunt coepta. et ne tot malis dissimulatis paulatimque 
serpentibus acervi crescerent aerumnarum, nobilitatis decreto legati mittuntur: Praetextatus ex urbi praefecto et ex vicario 
Venustus et ex consulari Minervius oraturi, ne delictis supplicia sint grandiora, neve senator quisquam inusitato et inlicito more 
tormentis exponeretur.

Eodem tempore etiam Hymetii praeclarae indolis viri negotium est actitatum, cuius hunc novimus esse textum. cum Africam pro 
consule regeret Carthaginiensibus victus inopia iam lassatis, ex horreis Romano populo destinatis frumentum dedit, pauloque 
postea cum provenisset segetum copia, integre sine ulla restituit mora.


\section{Des preuves qui ne dépendent point de l'adresse de l'orateur}

Dein Syria per speciosam interpatet diffusa planitiem. hanc nobilitat Antiochia, mundo cognita civitas, cui non certaverit alia 
advecticiis ita adfluere copiis et internis, et Laodicia et Apamia itidemque Seleucia iam inde a primis auspiciis florentissimae.

Ipsam vero urbem Byzantiorum fuisse refertissimam atque ornatissimam signis quis ignorat? Quae illi, exhausti sumptibus bellisque 
maximis, cum omnis Mithridaticos impetus totumque Pontum armatum affervescentem in Asiam atque erumpentem, ore repulsum 
et cervicibus interclusum suis sustinerent, tum, inquam, Byzantii et postea signa illa et reliqua urbis ornanemta sanctissime 
custodita tenuerunt.

Qui cum venisset ob haec festinatis itineribus Antiochiam, praestrictis palatii ianuis, contempto Caesare, quem videri decuerat, ad 
praetorium cum pompa sollemni perrexit morbosque diu causatus nec regiam introiit nec processit in publicum, sed abditus multa in 
eius moliebatur exitium addens quaedam relationibus supervacua, quas subinde dimittebat ad principem.

Atque, ut Tullius ait, ut etiam ferae fame monitae plerumque ad eum locum ubi aliquando pastae sunt revertuntur, ita homines 
instar turbinis degressi montibus impeditis et arduis loca petivere mari confinia, per quae viis latebrosis sese convallibusque 
occultantes cum appeterent noctes luna etiam tum cornuta ideoque nondum solido splendore fulgente nauticos observabant quos 
cum in somnum sentirent effusos per ancoralia, quadrupedo gradu repentes seseque suspensis passibus iniectantes in scaphas 
eisdem sensim nihil opinantibus adsistebant et incendente aviditate saevitiam ne cedentium quidem ulli parcendo obtruncatis 
omnibus merces opimas velut viles nullis repugnantibus avertebant. haecque non diu sunt perpetrata.

Erat autem diritatis eius hoc quoque indicium nec obscurum nec latens, quod ludicris cruentis delectabatur et in circo sex vel 
septem aliquotiens vetitis certaminibus pugilum vicissim se concidentium perfusorumque sanguine specie ut lucratus ingentia 
laetabatur.

Superatis Tauri montis verticibus qui ad solis ortum sublimius attolluntur, Cilicia spatiis porrigitur late distentis dives bonis omnibus 
terra, eiusque lateri dextro adnexa Isauria, pari sorte uberi palmite viget et frugibus minutis, quam mediam navigabile flumen 
Calycadnus interscindit.

Quam quidem partem accusationis admiratus sum et moleste tuli potissimum esse Atratino datam. Neque enim decebat neque 
aetas illa postulabat neque, id quod animadvertere poteratis, pudor patiebatur optimi adulescentis in tali illum oratione versari. 
Vellem aliquis ex vobis robustioribus hunc male dicendi locum suscepisset; aliquanto liberius et fortius et magis more nostro 
refutaremus istam male dicendi licentiam. Tecum, Atratine, agam lenius, quod et pudor tuus moderatur orationi meae et meum 
erga te parentemque tuum beneficium tueri debeo.

Nemo quaeso miretur, si post exsudatos labores itinerum longos congestosque adfatim commeatus fiducia vestri ductante 
barbaricos pagos adventans velut mutato repente consilio ad placidiora deverti.

Ob haec et huius modi multa, quae cernebantur in paucis, omnibus timeri sunt coepta. et ne tot malis dissimulatis paulatimque 
serpentibus acervi crescerent aerumnarum, nobilitatis decreto legati mittuntur: Praetextatus ex urbi praefecto et ex vicario 
Venustus et ex consulari Minervius oraturi, ne delictis supplicia sint grandiora, neve senator quisquam inusitato et inlicito more 
tormentis exponeretur.

Eodem tempore etiam Hymetii praeclarae indolis viri negotium est actitatum, cuius hunc novimus esse textum. cum Africam pro 
consule regeret Carthaginiensibus victus inopia iam lassatis, ex horreis Romano populo destinatis frumentum dedit, pauloque 
postea cum provenisset segetum copia, integre sine ulla restituit mora.



\section{Que l'orateur doit avoir une connaissance particulière des moeurs et des passions}

Dein Syria per speciosam interpatet diffusa planitiem. hanc nobilitat Antiochia, mundo cognita civitas, cui non certaverit alia 
advecticiis ita adfluere copiis et internis, et Laodicia et Apamia itidemque Seleucia iam inde a primis auspiciis florentissimae.

Ipsam vero urbem Byzantiorum fuisse refertissimam atque ornatissimam signis quis ignorat? Quae illi, exhausti sumptibus bellisque 
maximis, cum omnis Mithridaticos impetus totumque Pontum armatum affervescentem in Asiam atque erumpentem, ore repulsum 
et cervicibus interclusum suis sustinerent, tum, inquam, Byzantii et postea signa illa et reliqua urbis ornanemta sanctissime 
custodita tenuerunt.

Qui cum venisset ob haec festinatis itineribus Antiochiam, praestrictis palatii ianuis, contempto Caesare, quem videri decuerat, ad 
praetorium cum pompa sollemni perrexit morbosque diu causatus nec regiam introiit nec processit in publicum, sed abditus multa in 
eius moliebatur exitium addens quaedam relationibus supervacua, quas subinde dimittebat ad principem.

Atque, ut Tullius ait, ut etiam ferae fame monitae plerumque ad eum locum ubi aliquando pastae sunt revertuntur, ita homines 
instar turbinis degressi montibus impeditis et arduis loca petivere mari confinia, per quae viis latebrosis sese convallibusque 
occultantes cum appeterent noctes luna etiam tum cornuta ideoque nondum solido splendore fulgente nauticos observabant quos 
cum in somnum sentirent effusos per ancoralia, quadrupedo gradu repentes seseque suspensis passibus iniectantes in scaphas 
eisdem sensim nihil opinantibus adsistebant et incendente aviditate saevitiam ne cedentium quidem ulli parcendo obtruncatis 
omnibus merces opimas velut viles nullis repugnantibus avertebant. haecque non diu sunt perpetrata.

Erat autem diritatis eius hoc quoque indicium nec obscurum nec latens, quod ludicris cruentis delectabatur et in circo sex vel 
septem aliquotiens vetitis certaminibus pugilum vicissim se concidentium perfusorumque sanguine specie ut lucratus ingentia 
laetabatur.

Superatis Tauri montis verticibus qui ad solis ortum sublimius attolluntur, Cilicia spatiis porrigitur late distentis dives bonis omnibus 
terra, eiusque lateri dextro adnexa Isauria, pari sorte uberi palmite viget et frugibus minutis, quam mediam navigabile flumen 
Calycadnus interscindit.

Quam quidem partem accusationis admiratus sum et moleste tuli potissimum esse Atratino datam. Neque enim decebat neque 
aetas illa postulabat neque, id quod animadvertere poteratis, pudor patiebatur optimi adulescentis in tali illum oratione versari. 
Vellem aliquis ex vobis robustioribus hunc male dicendi locum suscepisset; aliquanto liberius et fortius et magis more nostro 
refutaremus istam male dicendi licentiam. Tecum, Atratine, agam lenius, quod et pudor tuus moderatur orationi meae et meum 
erga te parentemque tuum beneficium tueri debeo.

Nemo quaeso miretur, si post exsudatos labores itinerum longos congestosque adfatim commeatus fiducia vestri ductante 
barbaricos pagos adventans velut mutato repente consilio ad placidiora deverti.

Ob haec et huius modi multa, quae cernebantur in paucis, omnibus timeri sunt coepta. et ne tot malis dissimulatis paulatimque 
serpentibus acervi crescerent aerumnarum, nobilitatis decreto legati mittuntur: Praetextatus ex urbi praefecto et ex vicario 
Venustus et ex consulari Minervius oraturi, ne delictis supplicia sint grandiora, neve senator quisquam inusitato et inlicito more 
tormentis exponeretur.

Eodem tempore etiam Hymetii praeclarae indolis viri negotium est actitatum, cuius hunc novimus esse textum. cum Africam pro 
consule regeret Carthaginiensibus victus inopia iam lassatis, ex horreis Romano populo destinatis frumentum dedit, pauloque 
postea cum provenisset segetum copia, integre sine ulla restituit mora.


\chapter{Des passions}
\section{De la colère}

Dein Syria per speciosam interpatet diffusa planitiem. hanc nobilitat Antiochia, mundo cognita civitas, cui non certaverit alia 
advecticiis ita adfluere copiis et internis, et Laodicia et Apamia itidemque Seleucia iam inde a primis auspiciis florentissimae.

Ipsam vero urbem Byzantiorum fuisse refertissimam atque ornatissimam signis quis ignorat? Quae illi, exhausti sumptibus bellisque 
maximis, cum omnis Mithridaticos impetus totumque Pontum armatum affervescentem in Asiam atque erumpentem, ore repulsum 
et cervicibus interclusum suis sustinerent, tum, inquam, Byzantii et postea signa illa et reliqua urbis ornanemta sanctissime 
custodita tenuerunt.

Qui cum venisset ob haec festinatis itineribus Antiochiam, praestrictis palatii ianuis, contempto Caesare, quem videri decuerat, ad 
praetorium cum pompa sollemni perrexit morbosque diu causatus nec regiam introiit nec processit in publicum, sed abditus multa in 
eius moliebatur exitium addens quaedam relationibus supervacua, quas subinde dimittebat ad principem.

Atque, ut Tullius ait, ut etiam ferae fame monitae plerumque ad eum locum ubi aliquando pastae sunt revertuntur, ita homines 
instar turbinis degressi montibus impeditis et arduis loca petivere mari confinia, per quae viis latebrosis sese convallibusque 
occultantes cum appeterent noctes luna etiam tum cornuta ideoque nondum solido splendore fulgente nauticos observabant quos 
cum in somnum sentirent effusos per ancoralia, quadrupedo gradu repentes seseque suspensis passibus iniectantes in scaphas 
eisdem sensim nihil opinantibus adsistebant et incendente aviditate saevitiam ne cedentium quidem ulli parcendo obtruncatis 
omnibus merces opimas velut viles nullis repugnantibus avertebant. haecque non diu sunt perpetrata.

Erat autem diritatis eius hoc quoque indicium nec obscurum nec latens, quod ludicris cruentis delectabatur et in circo sex vel 
septem aliquotiens vetitis certaminibus pugilum vicissim se concidentium perfusorumque sanguine specie ut lucratus ingentia 
laetabatur.

Superatis Tauri montis verticibus qui ad solis ortum sublimius attolluntur, Cilicia spatiis porrigitur late distentis dives bonis omnibus 
terra, eiusque lateri dextro adnexa Isauria, pari sorte uberi palmite viget et frugibus minutis, quam mediam navigabile flumen 
Calycadnus interscindit.

Quam quidem partem accusationis admiratus sum et moleste tuli potissimum esse Atratino datam. Neque enim decebat neque 
aetas illa postulabat neque, id quod animadvertere poteratis, pudor patiebatur optimi adulescentis in tali illum oratione versari. 
Vellem aliquis ex vobis robustioribus hunc male dicendi locum suscepisset; aliquanto liberius et fortius et magis more nostro 
refutaremus istam male dicendi licentiam. Tecum, Atratine, agam lenius, quod et pudor tuus moderatur orationi meae et meum 
erga te parentemque tuum beneficium tueri debeo.

Nemo quaeso miretur, si post exsudatos labores itinerum longos congestosque adfatim commeatus fiducia vestri ductante 
barbaricos pagos adventans velut mutato repente consilio ad placidiora deverti.

Ob haec et huius modi multa, quae cernebantur in paucis, omnibus timeri sunt coepta. et ne tot malis dissimulatis paulatimque 
serpentibus acervi crescerent aerumnarum, nobilitatis decreto legati mittuntur: Praetextatus ex urbi praefecto et ex vicario 
Venustus et ex consulari Minervius oraturi, ne delictis supplicia sint grandiora, neve senator quisquam inusitato et inlicito more 
tormentis exponeretur.

Eodem tempore etiam Hymetii praeclarae indolis viri negotium est actitatum, cuius hunc novimus esse textum. cum Africam pro 
consule regeret Carthaginiensibus victus inopia iam lassatis, ex horreis Romano populo destinatis frumentum dedit, pauloque 
postea cum provenisset segetum copia, integre sine ulla restituit mora.


\section{De la douceur d'esprit, et pour apaiser la colère}

Dein Syria per speciosam interpatet diffusa planitiem. hanc nobilitat Antiochia, mundo cognita civitas, cui non certaverit alia 
advecticiis ita adfluere copiis et internis, et Laodicia et Apamia itidemque Seleucia iam inde a primis auspiciis florentissimae.

Ipsam vero urbem Byzantiorum fuisse refertissimam atque ornatissimam signis quis ignorat? Quae illi, exhausti sumptibus bellisque 
maximis, cum omnis Mithridaticos impetus totumque Pontum armatum affervescentem in Asiam atque erumpentem, ore repulsum 
et cervicibus interclusum suis sustinerent, tum, inquam, Byzantii et postea signa illa et reliqua urbis ornanemta sanctissime 
custodita tenuerunt.

Qui cum venisset ob haec festinatis itineribus Antiochiam, praestrictis palatii ianuis, contempto Caesare, quem videri decuerat, ad 
praetorium cum pompa sollemni perrexit morbosque diu causatus nec regiam introiit nec processit in publicum, sed abditus multa in 
eius moliebatur exitium addens quaedam relationibus supervacua, quas subinde dimittebat ad principem.

Atque, ut Tullius ait, ut etiam ferae fame monitae plerumque ad eum locum ubi aliquando pastae sunt revertuntur, ita homines 
instar turbinis degressi montibus impeditis et arduis loca petivere mari confinia, per quae viis latebrosis sese convallibusque 
occultantes cum appeterent noctes luna etiam tum cornuta ideoque nondum solido splendore fulgente nauticos observabant quos 
cum in somnum sentirent effusos per ancoralia, quadrupedo gradu repentes seseque suspensis passibus iniectantes in scaphas 
eisdem sensim nihil opinantibus adsistebant et incendente aviditate saevitiam ne cedentium quidem ulli parcendo obtruncatis 
omnibus merces opimas velut viles nullis repugnantibus avertebant. haecque non diu sunt perpetrata.

Erat autem diritatis eius hoc quoque indicium nec obscurum nec latens, quod ludicris cruentis delectabatur et in circo sex vel 
septem aliquotiens vetitis certaminibus pugilum vicissim se concidentium perfusorumque sanguine specie ut lucratus ingentia 
laetabatur.

Superatis Tauri montis verticibus qui ad solis ortum sublimius attolluntur, Cilicia spatiis porrigitur late distentis dives bonis omnibus 
terra, eiusque lateri dextro adnexa Isauria, pari sorte uberi palmite viget et frugibus minutis, quam mediam navigabile flumen 
Calycadnus interscindit.

Quam quidem partem accusationis admiratus sum et moleste tuli potissimum esse Atratino datam. Neque enim decebat neque 
aetas illa postulabat neque, id quod animadvertere poteratis, pudor patiebatur optimi adulescentis in tali illum oratione versari. 
Vellem aliquis ex vobis robustioribus hunc male dicendi locum suscepisset; aliquanto liberius et fortius et magis more nostro 
refutaremus istam male dicendi licentiam. Tecum, Atratine, agam lenius, quod et pudor tuus moderatur orationi meae et meum 
erga te parentemque tuum beneficium tueri debeo.

Nemo quaeso miretur, si post exsudatos labores itinerum longos congestosque adfatim commeatus fiducia vestri ductante 
barbaricos pagos adventans velut mutato repente consilio ad placidiora deverti.

Ob haec et huius modi multa, quae cernebantur in paucis, omnibus timeri sunt coepta. et ne tot malis dissimulatis paulatimque 
serpentibus acervi crescerent aerumnarum, nobilitatis decreto legati mittuntur: Praetextatus ex urbi praefecto et ex vicario 
Venustus et ex consulari Minervius oraturi, ne delictis supplicia sint grandiora, neve senator quisquam inusitato et inlicito more 
tormentis exponeretur.

Eodem tempore etiam Hymetii praeclarae indolis viri negotium est actitatum, cuius hunc novimus esse textum. cum Africam pro 
consule regeret Carthaginiensibus victus inopia iam lassatis, ex horreis Romano populo destinatis frumentum dedit, pauloque 
postea cum provenisset segetum copia, integre sine ulla restituit mora.


\section{De l'amour et de la haine}

Dein Syria per speciosam interpatet diffusa planitiem. hanc nobilitat Antiochia, mundo cognita civitas, cui non certaverit alia 
advecticiis ita adfluere copiis et internis, et Laodicia et Apamia itidemque Seleucia iam inde a primis auspiciis florentissimae.

Ipsam vero urbem Byzantiorum fuisse refertissimam atque ornatissimam signis quis ignorat? Quae illi, exhausti sumptibus bellisque 
maximis, cum omnis Mithridaticos impetus totumque Pontum armatum affervescentem in Asiam atque erumpentem, ore repulsum 
et cervicibus interclusum suis sustinerent, tum, inquam, Byzantii et postea signa illa et reliqua urbis ornanemta sanctissime 
custodita tenuerunt.

Qui cum venisset ob haec festinatis itineribus Antiochiam, praestrictis palatii ianuis, contempto Caesare, quem videri decuerat, ad 
praetorium cum pompa sollemni perrexit morbosque diu causatus nec regiam introiit nec processit in publicum, sed abditus multa in 
eius moliebatur exitium addens quaedam relationibus supervacua, quas subinde dimittebat ad principem.

Atque, ut Tullius ait, ut etiam ferae fame monitae plerumque ad eum locum ubi aliquando pastae sunt revertuntur, ita homines 
instar turbinis degressi montibus impeditis et arduis loca petivere mari confinia, per quae viis latebrosis sese convallibusque 
occultantes cum appeterent noctes luna etiam tum cornuta ideoque nondum solido splendore fulgente nauticos observabant quos 
cum in somnum sentirent effusos per ancoralia, quadrupedo gradu repentes seseque suspensis passibus iniectantes in scaphas 
eisdem sensim nihil opinantibus adsistebant et incendente aviditate saevitiam ne cedentium quidem ulli parcendo obtruncatis 
omnibus merces opimas velut viles nullis repugnantibus avertebant. haecque non diu sunt perpetrata.

Erat autem diritatis eius hoc quoque indicium nec obscurum nec latens, quod ludicris cruentis delectabatur et in circo sex vel 
septem aliquotiens vetitis certaminibus pugilum vicissim se concidentium perfusorumque sanguine specie ut lucratus ingentia 
laetabatur.

Superatis Tauri montis verticibus qui ad solis ortum sublimius attolluntur, Cilicia spatiis porrigitur late distentis dives bonis omnibus 
terra, eiusque lateri dextro adnexa Isauria, pari sorte uberi palmite viget et frugibus minutis, quam mediam navigabile flumen 
Calycadnus interscindit.

Quam quidem partem accusationis admiratus sum et moleste tuli potissimum esse Atratino datam. Neque enim decebat neque 
aetas illa postulabat neque, id quod animadvertere poteratis, pudor patiebatur optimi adulescentis in tali illum oratione versari. 
Vellem aliquis ex vobis robustioribus hunc male dicendi locum suscepisset; aliquanto liberius et fortius et magis more nostro 
refutaremus istam male dicendi licentiam. Tecum, Atratine, agam lenius, quod et pudor tuus moderatur orationi meae et meum 
erga te parentemque tuum beneficium tueri debeo.

Nemo quaeso miretur, si post exsudatos labores itinerum longos congestosque adfatim commeatus fiducia vestri ductante 
barbaricos pagos adventans velut mutato repente consilio ad placidiora deverti.

Ob haec et huius modi multa, quae cernebantur in paucis, omnibus timeri sunt coepta. et ne tot malis dissimulatis paulatimque 
serpentibus acervi crescerent aerumnarum, nobilitatis decreto legati mittuntur: Praetextatus ex urbi praefecto et ex vicario 
Venustus et ex consulari Minervius oraturi, ne delictis supplicia sint grandiora, neve senator quisquam inusitato et inlicito more 
tormentis exponeretur.

Eodem tempore etiam Hymetii praeclarae indolis viri negotium est actitatum, cuius hunc novimus esse textum. cum Africam pro 
consule regeret Carthaginiensibus victus inopia iam lassatis, ex horreis Romano populo destinatis frumentum dedit, pauloque 
postea cum provenisset segetum copia, integre sine ulla restituit mora.


\section{De la crainte et de l'assurance}

Dein Syria per speciosam interpatet diffusa planitiem. hanc nobilitat Antiochia, mundo cognita civitas, cui non certaverit alia 
advecticiis ita adfluere copiis et internis, et Laodicia et Apamia itidemque Seleucia iam inde a primis auspiciis florentissimae.

Ipsam vero urbem Byzantiorum fuisse refertissimam atque ornatissimam signis quis ignorat? Quae illi, exhausti sumptibus bellisque 
maximis, cum omnis Mithridaticos impetus totumque Pontum armatum affervescentem in Asiam atque erumpentem, ore repulsum 
et cervicibus interclusum suis sustinerent, tum, inquam, Byzantii et postea signa illa et reliqua urbis ornanemta sanctissime 
custodita tenuerunt.

Qui cum venisset ob haec festinatis itineribus Antiochiam, praestrictis palatii ianuis, contempto Caesare, quem videri decuerat, ad 
praetorium cum pompa sollemni perrexit morbosque diu causatus nec regiam introiit nec processit in publicum, sed abditus multa in 
eius moliebatur exitium addens quaedam relationibus supervacua, quas subinde dimittebat ad principem.

Atque, ut Tullius ait, ut etiam ferae fame monitae plerumque ad eum locum ubi aliquando pastae sunt revertuntur, ita homines 
instar turbinis degressi montibus impeditis et arduis loca petivere mari confinia, per quae viis latebrosis sese convallibusque 
occultantes cum appeterent noctes luna etiam tum cornuta ideoque nondum solido splendore fulgente nauticos observabant quos 
cum in somnum sentirent effusos per ancoralia, quadrupedo gradu repentes seseque suspensis passibus iniectantes in scaphas 
eisdem sensim nihil opinantibus adsistebant et incendente aviditate saevitiam ne cedentium quidem ulli parcendo obtruncatis 
omnibus merces opimas velut viles nullis repugnantibus avertebant. haecque non diu sunt perpetrata.

Erat autem diritatis eius hoc quoque indicium nec obscurum nec latens, quod ludicris cruentis delectabatur et in circo sex vel 
septem aliquotiens vetitis certaminibus pugilum vicissim se concidentium perfusorumque sanguine specie ut lucratus ingentia 
laetabatur.

Superatis Tauri montis verticibus qui ad solis ortum sublimius attolluntur, Cilicia spatiis porrigitur late distentis dives bonis omnibus 
terra, eiusque lateri dextro adnexa Isauria, pari sorte uberi palmite viget et frugibus minutis, quam mediam navigabile flumen 
Calycadnus interscindit.

Quam quidem partem accusationis admiratus sum et moleste tuli potissimum esse Atratino datam. Neque enim decebat neque 
aetas illa postulabat neque, id quod animadvertere poteratis, pudor patiebatur optimi adulescentis in tali illum oratione versari. 
Vellem aliquis ex vobis robustioribus hunc male dicendi locum suscepisset; aliquanto liberius et fortius et magis more nostro 
refutaremus istam male dicendi licentiam. Tecum, Atratine, agam lenius, quod et pudor tuus moderatur orationi meae et meum 
erga te parentemque tuum beneficium tueri debeo.

Nemo quaeso miretur, si post exsudatos labores itinerum longos congestosque adfatim commeatus fiducia vestri ductante 
barbaricos pagos adventans velut mutato repente consilio ad placidiora deverti.

Ob haec et huius modi multa, quae cernebantur in paucis, omnibus timeri sunt coepta. et ne tot malis dissimulatis paulatimque 
serpentibus acervi crescerent aerumnarum, nobilitatis decreto legati mittuntur: Praetextatus ex urbi praefecto et ex vicario 
Venustus et ex consulari Minervius oraturi, ne delictis supplicia sint grandiora, neve senator quisquam inusitato et inlicito more 
tormentis exponeretur.

Eodem tempore etiam Hymetii praeclarae indolis viri negotium est actitatum, cuius hunc novimus esse textum. cum Africam pro 
consule regeret Carthaginiensibus victus inopia iam lassatis, ex horreis Romano populo destinatis frumentum dedit, pauloque 
postea cum provenisset segetum copia, integre sine ulla restituit mora.


\section{De la honte et de l'impudence}

Dein Syria per speciosam interpatet diffusa planitiem. hanc nobilitat Antiochia, mundo cognita civitas, cui non certaverit alia 
advecticiis ita adfluere copiis et internis, et Laodicia et Apamia itidemque Seleucia iam inde a primis auspiciis florentissimae.

Ipsam vero urbem Byzantiorum fuisse refertissimam atque ornatissimam signis quis ignorat? Quae illi, exhausti sumptibus bellisque 
maximis, cum omnis Mithridaticos impetus totumque Pontum armatum affervescentem in Asiam atque erumpentem, ore repulsum 
et cervicibus interclusum suis sustinerent, tum, inquam, Byzantii et postea signa illa et reliqua urbis ornanemta sanctissime 
custodita tenuerunt.

Qui cum venisset ob haec festinatis itineribus Antiochiam, praestrictis palatii ianuis, contempto Caesare, quem videri decuerat, ad 
praetorium cum pompa sollemni perrexit morbosque diu causatus nec regiam introiit nec processit in publicum, sed abditus multa in 
eius moliebatur exitium addens quaedam relationibus supervacua, quas subinde dimittebat ad principem.

Atque, ut Tullius ait, ut etiam ferae fame monitae plerumque ad eum locum ubi aliquando pastae sunt revertuntur, ita homines 
instar turbinis degressi montibus impeditis et arduis loca petivere mari confinia, per quae viis latebrosis sese convallibusque 
occultantes cum appeterent noctes luna etiam tum cornuta ideoque nondum solido splendore fulgente nauticos observabant quos 
cum in somnum sentirent effusos per ancoralia, quadrupedo gradu repentes seseque suspensis passibus iniectantes in scaphas 
eisdem sensim nihil opinantibus adsistebant et incendente aviditate saevitiam ne cedentium quidem ulli parcendo obtruncatis 
omnibus merces opimas velut viles nullis repugnantibus avertebant. haecque non diu sunt perpetrata.

Erat autem diritatis eius hoc quoque indicium nec obscurum nec latens, quod ludicris cruentis delectabatur et in circo sex vel 
septem aliquotiens vetitis certaminibus pugilum vicissim se concidentium perfusorumque sanguine specie ut lucratus ingentia 
laetabatur.

Superatis Tauri montis verticibus qui ad solis ortum sublimius attolluntur, Cilicia spatiis porrigitur late distentis dives bonis omnibus 
terra, eiusque lateri dextro adnexa Isauria, pari sorte uberi palmite viget et frugibus minutis, quam mediam navigabile flumen 
Calycadnus interscindit.

Quam quidem partem accusationis admiratus sum et moleste tuli potissimum esse Atratino datam. Neque enim decebat neque 
aetas illa postulabat neque, id quod animadvertere poteratis, pudor patiebatur optimi adulescentis in tali illum oratione versari. 
Vellem aliquis ex vobis robustioribus hunc male dicendi locum suscepisset; aliquanto liberius et fortius et magis more nostro 
refutaremus istam male dicendi licentiam. Tecum, Atratine, agam lenius, quod et pudor tuus moderatur orationi meae et meum 
erga te parentemque tuum beneficium tueri debeo.

Nemo quaeso miretur, si post exsudatos labores itinerum longos congestosque adfatim commeatus fiducia vestri ductante 
barbaricos pagos adventans velut mutato repente consilio ad placidiora deverti.

Ob haec et huius modi multa, quae cernebantur in paucis, omnibus timeri sunt coepta. et ne tot malis dissimulatis paulatimque 
serpentibus acervi crescerent aerumnarum, nobilitatis decreto legati mittuntur: Praetextatus ex urbi praefecto et ex vicario 
Venustus et ex consulari Minervius oraturi, ne delictis supplicia sint grandiora, neve senator quisquam inusitato et inlicito more 
tormentis exponeretur.

Eodem tempore etiam Hymetii praeclarae indolis viri negotium est actitatum, cuius hunc novimus esse textum. cum Africam pro 
consule regeret Carthaginiensibus victus inopia iam lassatis, ex horreis Romano populo destinatis frumentum dedit, pauloque 
postea cum provenisset segetum copia, integre sine ulla restituit mora.


\section{Du bienfait}

Dein Syria per speciosam interpatet diffusa planitiem. hanc nobilitat Antiochia, mundo cognita civitas, cui non certaverit alia 
advecticiis ita adfluere copiis et internis, et Laodicia et Apamia itidemque Seleucia iam inde a primis auspiciis florentissimae.

Ipsam vero urbem Byzantiorum fuisse refertissimam atque ornatissimam signis quis ignorat? Quae illi, exhausti sumptibus bellisque 
maximis, cum omnis Mithridaticos impetus totumque Pontum armatum affervescentem in Asiam atque erumpentem, ore repulsum 
et cervicibus interclusum suis sustinerent, tum, inquam, Byzantii et postea signa illa et reliqua urbis ornanemta sanctissime 
custodita tenuerunt.

Qui cum venisset ob haec festinatis itineribus Antiochiam, praestrictis palatii ianuis, contempto Caesare, quem videri decuerat, ad 
praetorium cum pompa sollemni perrexit morbosque diu causatus nec regiam introiit nec processit in publicum, sed abditus multa in 
eius moliebatur exitium addens quaedam relationibus supervacua, quas subinde dimittebat ad principem.

Atque, ut Tullius ait, ut etiam ferae fame monitae plerumque ad eum locum ubi aliquando pastae sunt revertuntur, ita homines 
instar turbinis degressi montibus impeditis et arduis loca petivere mari confinia, per quae viis latebrosis sese convallibusque 
occultantes cum appeterent noctes luna etiam tum cornuta ideoque nondum solido splendore fulgente nauticos observabant quos 
cum in somnum sentirent effusos per ancoralia, quadrupedo gradu repentes seseque suspensis passibus iniectantes in scaphas 
eisdem sensim nihil opinantibus adsistebant et incendente aviditate saevitiam ne cedentium quidem ulli parcendo obtruncatis 
omnibus merces opimas velut viles nullis repugnantibus avertebant. haecque non diu sunt perpetrata.

Erat autem diritatis eius hoc quoque indicium nec obscurum nec latens, quod ludicris cruentis delectabatur et in circo sex vel 
septem aliquotiens vetitis certaminibus pugilum vicissim se concidentium perfusorumque sanguine specie ut lucratus ingentia 
laetabatur.

Superatis Tauri montis verticibus qui ad solis ortum sublimius attolluntur, Cilicia spatiis porrigitur late distentis dives bonis omnibus 
terra, eiusque lateri dextro adnexa Isauria, pari sorte uberi palmite viget et frugibus minutis, quam mediam navigabile flumen 
Calycadnus interscindit.

Quam quidem partem accusationis admiratus sum et moleste tuli potissimum esse Atratino datam. Neque enim decebat neque 
aetas illa postulabat neque, id quod animadvertere poteratis, pudor patiebatur optimi adulescentis in tali illum oratione versari. 
Vellem aliquis ex vobis robustioribus hunc male dicendi locum suscepisset; aliquanto liberius et fortius et magis more nostro 
refutaremus istam male dicendi licentiam. Tecum, Atratine, agam lenius, quod et pudor tuus moderatur orationi meae et meum 
erga te parentemque tuum beneficium tueri debeo.

Nemo quaeso miretur, si post exsudatos labores itinerum longos congestosque adfatim commeatus fiducia vestri ductante 
barbaricos pagos adventans velut mutato repente consilio ad placidiora deverti.

Ob haec et huius modi multa, quae cernebantur in paucis, omnibus timeri sunt coepta. et ne tot malis dissimulatis paulatimque 
serpentibus acervi crescerent aerumnarum, nobilitatis decreto legati mittuntur: Praetextatus ex urbi praefecto et ex vicario 
Venustus et ex consulari Minervius oraturi, ne delictis supplicia sint grandiora, neve senator quisquam inusitato et inlicito more 
tormentis exponeretur.

Eodem tempore etiam Hymetii praeclarae indolis viri negotium est actitatum, cuius hunc novimus esse textum. cum Africam pro 
consule regeret Carthaginiensibus victus inopia iam lassatis, ex horreis Romano populo destinatis frumentum dedit, pauloque 
postea cum provenisset segetum copia, integre sine ulla restituit mora.


\section{De la compassion}

Dein Syria per speciosam interpatet diffusa planitiem. hanc nobilitat Antiochia, mundo cognita civitas, cui non certaverit alia 
advecticiis ita adfluere copiis et internis, et Laodicia et Apamia itidemque Seleucia iam inde a primis auspiciis florentissimae.

Ipsam vero urbem Byzantiorum fuisse refertissimam atque ornatissimam signis quis ignorat? Quae illi, exhausti sumptibus bellisque 
maximis, cum omnis Mithridaticos impetus totumque Pontum armatum affervescentem in Asiam atque erumpentem, ore repulsum 
et cervicibus interclusum suis sustinerent, tum, inquam, Byzantii et postea signa illa et reliqua urbis ornanemta sanctissime 
custodita tenuerunt.

Qui cum venisset ob haec festinatis itineribus Antiochiam, praestrictis palatii ianuis, contempto Caesare, quem videri decuerat, ad 
praetorium cum pompa sollemni perrexit morbosque diu causatus nec regiam introiit nec processit in publicum, sed abditus multa in 
eius moliebatur exitium addens quaedam relationibus supervacua, quas subinde dimittebat ad principem.

Atque, ut Tullius ait, ut etiam ferae fame monitae plerumque ad eum locum ubi aliquando pastae sunt revertuntur, ita homines 
instar turbinis degressi montibus impeditis et arduis loca petivere mari confinia, per quae viis latebrosis sese convallibusque 
occultantes cum appeterent noctes luna etiam tum cornuta ideoque nondum solido splendore fulgente nauticos observabant quos 
cum in somnum sentirent effusos per ancoralia, quadrupedo gradu repentes seseque suspensis passibus iniectantes in scaphas 
eisdem sensim nihil opinantibus adsistebant et incendente aviditate saevitiam ne cedentium quidem ulli parcendo obtruncatis 
omnibus merces opimas velut viles nullis repugnantibus avertebant. haecque non diu sunt perpetrata.

Erat autem diritatis eius hoc quoque indicium nec obscurum nec latens, quod ludicris cruentis delectabatur et in circo sex vel 
septem aliquotiens vetitis certaminibus pugilum vicissim se concidentium perfusorumque sanguine specie ut lucratus ingentia 
laetabatur.

Superatis Tauri montis verticibus qui ad solis ortum sublimius attolluntur, Cilicia spatiis porrigitur late distentis dives bonis omnibus 
terra, eiusque lateri dextro adnexa Isauria, pari sorte uberi palmite viget et frugibus minutis, quam mediam navigabile flumen 
Calycadnus interscindit.

Quam quidem partem accusationis admiratus sum et moleste tuli potissimum esse Atratino datam. Neque enim decebat neque 
aetas illa postulabat neque, id quod animadvertere poteratis, pudor patiebatur optimi adulescentis in tali illum oratione versari. 
Vellem aliquis ex vobis robustioribus hunc male dicendi locum suscepisset; aliquanto liberius et fortius et magis more nostro 
refutaremus istam male dicendi licentiam. Tecum, Atratine, agam lenius, quod et pudor tuus moderatur orationi meae et meum 
erga te parentemque tuum beneficium tueri debeo.

Nemo quaeso miretur, si post exsudatos labores itinerum longos congestosque adfatim commeatus fiducia vestri ductante 
barbaricos pagos adventans velut mutato repente consilio ad placidiora deverti.

Ob haec et huius modi multa, quae cernebantur in paucis, omnibus timeri sunt coepta. et ne tot malis dissimulatis paulatimque 
serpentibus acervi crescerent aerumnarum, nobilitatis decreto legati mittuntur: Praetextatus ex urbi praefecto et ex vicario 
Venustus et ex consulari Minervius oraturi, ne delictis supplicia sint grandiora, neve senator quisquam inusitato et inlicito more 
tormentis exponeretur.

Eodem tempore etiam Hymetii praeclarae indolis viri negotium est actitatum, cuius hunc novimus esse textum. cum Africam pro 
consule regeret Carthaginiensibus victus inopia iam lassatis, ex horreis Romano populo destinatis frumentum dedit, pauloque 
postea cum provenisset segetum copia, integre sine ulla restituit mora.


\section{De l'indignation}

Dein Syria per speciosam interpatet diffusa planitiem. hanc nobilitat Antiochia, mundo cognita civitas, cui non certaverit alia 
advecticiis ita adfluere copiis et internis, et Laodicia et Apamia itidemque Seleucia iam inde a primis auspiciis florentissimae.

Ipsam vero urbem Byzantiorum fuisse refertissimam atque ornatissimam signis quis ignorat? Quae illi, exhausti sumptibus bellisque 
maximis, cum omnis Mithridaticos impetus totumque Pontum armatum affervescentem in Asiam atque erumpentem, ore repulsum 
et cervicibus interclusum suis sustinerent, tum, inquam, Byzantii et postea signa illa et reliqua urbis ornanemta sanctissime 
custodita tenuerunt.

Qui cum venisset ob haec festinatis itineribus Antiochiam, praestrictis palatii ianuis, contempto Caesare, quem videri decuerat, ad 
praetorium cum pompa sollemni perrexit morbosque diu causatus nec regiam introiit nec processit in publicum, sed abditus multa in 
eius moliebatur exitium addens quaedam relationibus supervacua, quas subinde dimittebat ad principem.

Atque, ut Tullius ait, ut etiam ferae fame monitae plerumque ad eum locum ubi aliquando pastae sunt revertuntur, ita homines 
instar turbinis degressi montibus impeditis et arduis loca petivere mari confinia, per quae viis latebrosis sese convallibusque 
occultantes cum appeterent noctes luna etiam tum cornuta ideoque nondum solido splendore fulgente nauticos observabant quos 
cum in somnum sentirent effusos per ancoralia, quadrupedo gradu repentes seseque suspensis passibus iniectantes in scaphas 
eisdem sensim nihil opinantibus adsistebant et incendente aviditate saevitiam ne cedentium quidem ulli parcendo obtruncatis 
omnibus merces opimas velut viles nullis repugnantibus avertebant. haecque non diu sunt perpetrata.

Erat autem diritatis eius hoc quoque indicium nec obscurum nec latens, quod ludicris cruentis delectabatur et in circo sex vel 
septem aliquotiens vetitis certaminibus pugilum vicissim se concidentium perfusorumque sanguine specie ut lucratus ingentia 
laetabatur.

Superatis Tauri montis verticibus qui ad solis ortum sublimius attolluntur, Cilicia spatiis porrigitur late distentis dives bonis omnibus 
terra, eiusque lateri dextro adnexa Isauria, pari sorte uberi palmite viget et frugibus minutis, quam mediam navigabile flumen 
Calycadnus interscindit.

Quam quidem partem accusationis admiratus sum et moleste tuli potissimum esse Atratino datam. Neque enim decebat neque 
aetas illa postulabat neque, id quod animadvertere poteratis, pudor patiebatur optimi adulescentis in tali illum oratione versari. 
Vellem aliquis ex vobis robustioribus hunc male dicendi locum suscepisset; aliquanto liberius et fortius et magis more nostro 
refutaremus istam male dicendi licentiam. Tecum, Atratine, agam lenius, quod et pudor tuus moderatur orationi meae et meum 
erga te parentemque tuum beneficium tueri debeo.

Nemo quaeso miretur, si post exsudatos labores itinerum longos congestosque adfatim commeatus fiducia vestri ductante 
barbaricos pagos adventans velut mutato repente consilio ad placidiora deverti.

Ob haec et huius modi multa, quae cernebantur in paucis, omnibus timeri sunt coepta. et ne tot malis dissimulatis paulatimque 
serpentibus acervi crescerent aerumnarum, nobilitatis decreto legati mittuntur: Praetextatus ex urbi praefecto et ex vicario 
Venustus et ex consulari Minervius oraturi, ne delictis supplicia sint grandiora, neve senator quisquam inusitato et inlicito more 
tormentis exponeretur.

Eodem tempore etiam Hymetii praeclarae indolis viri negotium est actitatum, cuius hunc novimus esse textum. cum Africam pro 
consule regeret Carthaginiensibus victus inopia iam lassatis, ex horreis Romano populo destinatis frumentum dedit, pauloque 
postea cum provenisset segetum copia, integre sine ulla restituit mora.


\section{De l'envie}

Dein Syria per speciosam interpatet diffusa planitiem. hanc nobilitat Antiochia, mundo cognita civitas, cui non certaverit alia 
advecticiis ita adfluere copiis et internis, et Laodicia et Apamia itidemque Seleucia iam inde a primis auspiciis florentissimae.

Ipsam vero urbem Byzantiorum fuisse refertissimam atque ornatissimam signis quis ignorat? Quae illi, exhausti sumptibus bellisque 
maximis, cum omnis Mithridaticos impetus totumque Pontum armatum affervescentem in Asiam atque erumpentem, ore repulsum 
et cervicibus interclusum suis sustinerent, tum, inquam, Byzantii et postea signa illa et reliqua urbis ornanemta sanctissime 
custodita tenuerunt.

Qui cum venisset ob haec festinatis itineribus Antiochiam, praestrictis palatii ianuis, contempto Caesare, quem videri decuerat, ad 
praetorium cum pompa sollemni perrexit morbosque diu causatus nec regiam introiit nec processit in publicum, sed abditus multa in 
eius moliebatur exitium addens quaedam relationibus supervacua, quas subinde dimittebat ad principem.

Atque, ut Tullius ait, ut etiam ferae fame monitae plerumque ad eum locum ubi aliquando pastae sunt revertuntur, ita homines 
instar turbinis degressi montibus impeditis et arduis loca petivere mari confinia, per quae viis latebrosis sese convallibusque 
occultantes cum appeterent noctes luna etiam tum cornuta ideoque nondum solido splendore fulgente nauticos observabant quos 
cum in somnum sentirent effusos per ancoralia, quadrupedo gradu repentes seseque suspensis passibus iniectantes in scaphas 
eisdem sensim nihil opinantibus adsistebant et incendente aviditate saevitiam ne cedentium quidem ulli parcendo obtruncatis 
omnibus merces opimas velut viles nullis repugnantibus avertebant. haecque non diu sunt perpetrata.

Erat autem diritatis eius hoc quoque indicium nec obscurum nec latens, quod ludicris cruentis delectabatur et in circo sex vel 
septem aliquotiens vetitis certaminibus pugilum vicissim se concidentium perfusorumque sanguine specie ut lucratus ingentia 
laetabatur.

Superatis Tauri montis verticibus qui ad solis ortum sublimius attolluntur, Cilicia spatiis porrigitur late distentis dives bonis omnibus 
terra, eiusque lateri dextro adnexa Isauria, pari sorte uberi palmite viget et frugibus minutis, quam mediam navigabile flumen 
Calycadnus interscindit.

Quam quidem partem accusationis admiratus sum et moleste tuli potissimum esse Atratino datam. Neque enim decebat neque 
aetas illa postulabat neque, id quod animadvertere poteratis, pudor patiebatur optimi adulescentis in tali illum oratione versari. 
Vellem aliquis ex vobis robustioribus hunc male dicendi locum suscepisset; aliquanto liberius et fortius et magis more nostro 
refutaremus istam male dicendi licentiam. Tecum, Atratine, agam lenius, quod et pudor tuus moderatur orationi meae et meum 
erga te parentemque tuum beneficium tueri debeo.

Nemo quaeso miretur, si post exsudatos labores itinerum longos congestosque adfatim commeatus fiducia vestri ductante 
barbaricos pagos adventans velut mutato repente consilio ad placidiora deverti.

Ob haec et huius modi multa, quae cernebantur in paucis, omnibus timeri sunt coepta. et ne tot malis dissimulatis paulatimque 
serpentibus acervi crescerent aerumnarum, nobilitatis decreto legati mittuntur: Praetextatus ex urbi praefecto et ex vicario 
Venustus et ex consulari Minervius oraturi, ne delictis supplicia sint grandiora, neve senator quisquam inusitato et inlicito more 
tormentis exponeretur.

Eodem tempore etiam Hymetii praeclarae indolis viri negotium est actitatum, cuius hunc novimus esse textum. cum Africam pro 
consule regeret Carthaginiensibus victus inopia iam lassatis, ex horreis Romano populo destinatis frumentum dedit, pauloque 
postea cum provenisset segetum copia, integre sine ulla restituit mora.


\section{De l'émulation}

Dein Syria per speciosam interpatet diffusa planitiem. hanc nobilitat Antiochia, mundo cognita civitas, cui non certaverit alia 
advecticiis ita adfluere copiis et internis, et Laodicia et Apamia itidemque Seleucia iam inde a primis auspiciis florentissimae.

Ipsam vero urbem Byzantiorum fuisse refertissimam atque ornatissimam signis quis ignorat? Quae illi, exhausti sumptibus bellisque 
maximis, cum omnis Mithridaticos impetus totumque Pontum armatum affervescentem in Asiam atque erumpentem, ore repulsum 
et cervicibus interclusum suis sustinerent, tum, inquam, Byzantii et postea signa illa et reliqua urbis ornanemta sanctissime 
custodita tenuerunt.

Qui cum venisset ob haec festinatis itineribus Antiochiam, praestrictis palatii ianuis, contempto Caesare, quem videri decuerat, ad 
praetorium cum pompa sollemni perrexit morbosque diu causatus nec regiam introiit nec processit in publicum, sed abditus multa in 
eius moliebatur exitium addens quaedam relationibus supervacua, quas subinde dimittebat ad principem.

Atque, ut Tullius ait, ut etiam ferae fame monitae plerumque ad eum locum ubi aliquando pastae sunt revertuntur, ita homines 
instar turbinis degressi montibus impeditis et arduis loca petivere mari confinia, per quae viis latebrosis sese convallibusque 
occultantes cum appeterent noctes luna etiam tum cornuta ideoque nondum solido splendore fulgente nauticos observabant quos 
cum in somnum sentirent effusos per ancoralia, quadrupedo gradu repentes seseque suspensis passibus iniectantes in scaphas 
eisdem sensim nihil opinantibus adsistebant et incendente aviditate saevitiam ne cedentium quidem ulli parcendo obtruncatis 
omnibus merces opimas velut viles nullis repugnantibus avertebant. haecque non diu sunt perpetrata.

Erat autem diritatis eius hoc quoque indicium nec obscurum nec latens, quod ludicris cruentis delectabatur et in circo sex vel 
septem aliquotiens vetitis certaminibus pugilum vicissim se concidentium perfusorumque sanguine specie ut lucratus ingentia 
laetabatur.

Superatis Tauri montis verticibus qui ad solis ortum sublimius attolluntur, Cilicia spatiis porrigitur late distentis dives bonis omnibus 
terra, eiusque lateri dextro adnexa Isauria, pari sorte uberi palmite viget et frugibus minutis, quam mediam navigabile flumen 
Calycadnus interscindit.

Quam quidem partem accusationis admiratus sum et moleste tuli potissimum esse Atratino datam. Neque enim decebat neque 
aetas illa postulabat neque, id quod animadvertere poteratis, pudor patiebatur optimi adulescentis in tali illum oratione versari. 
Vellem aliquis ex vobis robustioribus hunc male dicendi locum suscepisset; aliquanto liberius et fortius et magis more nostro 
refutaremus istam male dicendi licentiam. Tecum, Atratine, agam lenius, quod et pudor tuus moderatur orationi meae et meum 
erga te parentemque tuum beneficium tueri debeo.

Nemo quaeso miretur, si post exsudatos labores itinerum longos congestosque adfatim commeatus fiducia vestri ductante 
barbaricos pagos adventans velut mutato repente consilio ad placidiora deverti.

Ob haec et huius modi multa, quae cernebantur in paucis, omnibus timeri sunt coepta. et ne tot malis dissimulatis paulatimque 
serpentibus acervi crescerent aerumnarum, nobilitatis decreto legati mittuntur: Praetextatus ex urbi praefecto et ex vicario 
Venustus et ex consulari Minervius oraturi, ne delictis supplicia sint grandiora, neve senator quisquam inusitato et inlicito more 
tormentis exponeretur.

Eodem tempore etiam Hymetii praeclarae indolis viri negotium est actitatum, cuius hunc novimus esse textum. cum Africam pro 
consule regeret Carthaginiensibus victus inopia iam lassatis, ex horreis Romano populo destinatis frumentum dedit, pauloque 
postea cum provenisset segetum copia, integre sine ulla restituit mora.


\chapter{Les moeurs}
\section{Le naturel des jeunes gens, et leur humeur}

Dein Syria per speciosam interpatet diffusa planitiem. hanc nobilitat Antiochia, mundo cognita civitas, cui non certaverit alia 
advecticiis ita adfluere copiis et internis, et Laodicia et Apamia itidemque Seleucia iam inde a primis auspiciis florentissimae.

Ipsam vero urbem Byzantiorum fuisse refertissimam atque ornatissimam signis quis ignorat? Quae illi, exhausti sumptibus bellisque 
maximis, cum omnis Mithridaticos impetus totumque Pontum armatum affervescentem in Asiam atque erumpentem, ore repulsum 
et cervicibus interclusum suis sustinerent, tum, inquam, Byzantii et postea signa illa et reliqua urbis ornanemta sanctissime 
custodita tenuerunt.

Qui cum venisset ob haec festinatis itineribus Antiochiam, praestrictis palatii ianuis, contempto Caesare, quem videri decuerat, ad 
praetorium cum pompa sollemni perrexit morbosque diu causatus nec regiam introiit nec processit in publicum, sed abditus multa in 
eius moliebatur exitium addens quaedam relationibus supervacua, quas subinde dimittebat ad principem.

Atque, ut Tullius ait, ut etiam ferae fame monitae plerumque ad eum locum ubi aliquando pastae sunt revertuntur, ita homines 
instar turbinis degressi montibus impeditis et arduis loca petivere mari confinia, per quae viis latebrosis sese convallibusque 
occultantes cum appeterent noctes luna etiam tum cornuta ideoque nondum solido splendore fulgente nauticos observabant quos 
cum in somnum sentirent effusos per ancoralia, quadrupedo gradu repentes seseque suspensis passibus iniectantes in scaphas 
eisdem sensim nihil opinantibus adsistebant et incendente aviditate saevitiam ne cedentium quidem ulli parcendo obtruncatis 
omnibus merces opimas velut viles nullis repugnantibus avertebant. haecque non diu sunt perpetrata.

Erat autem diritatis eius hoc quoque indicium nec obscurum nec latens, quod ludicris cruentis delectabatur et in circo sex vel 
septem aliquotiens vetitis certaminibus pugilum vicissim se concidentium perfusorumque sanguine specie ut lucratus ingentia 
laetabatur.

Superatis Tauri montis verticibus qui ad solis ortum sublimius attolluntur, Cilicia spatiis porrigitur late distentis dives bonis omnibus 
terra, eiusque lateri dextro adnexa Isauria, pari sorte uberi palmite viget et frugibus minutis, quam mediam navigabile flumen 
Calycadnus interscindit.

Quam quidem partem accusationis admiratus sum et moleste tuli potissimum esse Atratino datam. Neque enim decebat neque 
aetas illa postulabat neque, id quod animadvertere poteratis, pudor patiebatur optimi adulescentis in tali illum oratione versari. 
Vellem aliquis ex vobis robustioribus hunc male dicendi locum suscepisset; aliquanto liberius et fortius et magis more nostro 
refutaremus istam male dicendi licentiam. Tecum, Atratine, agam lenius, quod et pudor tuus moderatur orationi meae et meum 
erga te parentemque tuum beneficium tueri debeo.

Nemo quaeso miretur, si post exsudatos labores itinerum longos congestosque adfatim commeatus fiducia vestri ductante 
barbaricos pagos adventans velut mutato repente consilio ad placidiora deverti.

Ob haec et huius modi multa, quae cernebantur in paucis, omnibus timeri sunt coepta. et ne tot malis dissimulatis paulatimque 
serpentibus acervi crescerent aerumnarum, nobilitatis decreto legati mittuntur: Praetextatus ex urbi praefecto et ex vicario 
Venustus et ex consulari Minervius oraturi, ne delictis supplicia sint grandiora, neve senator quisquam inusitato et inlicito more 
tormentis exponeretur.

Eodem tempore etiam Hymetii praeclarae indolis viri negotium est actitatum, cuius hunc novimus esse textum. cum Africam pro 
consule regeret Carthaginiensibus victus inopia iam lassatis, ex horreis Romano populo destinatis frumentum dedit, pauloque 
postea cum provenisset segetum copia, integre sine ulla restituit mora.


\section{L'humeur des vieillards}

Dein Syria per speciosam interpatet diffusa planitiem. hanc nobilitat Antiochia, mundo cognita civitas, cui non certaverit alia 
advecticiis ita adfluere copiis et internis, et Laodicia et Apamia itidemque Seleucia iam inde a primis auspiciis florentissimae.

Ipsam vero urbem Byzantiorum fuisse refertissimam atque ornatissimam signis quis ignorat? Quae illi, exhausti sumptibus bellisque 
maximis, cum omnis Mithridaticos impetus totumque Pontum armatum affervescentem in Asiam atque erumpentem, ore repulsum 
et cervicibus interclusum suis sustinerent, tum, inquam, Byzantii et postea signa illa et reliqua urbis ornanemta sanctissime 
custodita tenuerunt.

Qui cum venisset ob haec festinatis itineribus Antiochiam, praestrictis palatii ianuis, contempto Caesare, quem videri decuerat, ad 
praetorium cum pompa sollemni perrexit morbosque diu causatus nec regiam introiit nec processit in publicum, sed abditus multa in 
eius moliebatur exitium addens quaedam relationibus supervacua, quas subinde dimittebat ad principem.

Atque, ut Tullius ait, ut etiam ferae fame monitae plerumque ad eum locum ubi aliquando pastae sunt revertuntur, ita homines 
instar turbinis degressi montibus impeditis et arduis loca petivere mari confinia, per quae viis latebrosis sese convallibusque 
occultantes cum appeterent noctes luna etiam tum cornuta ideoque nondum solido splendore fulgente nauticos observabant quos 
cum in somnum sentirent effusos per ancoralia, quadrupedo gradu repentes seseque suspensis passibus iniectantes in scaphas 
eisdem sensim nihil opinantibus adsistebant et incendente aviditate saevitiam ne cedentium quidem ulli parcendo obtruncatis 
omnibus merces opimas velut viles nullis repugnantibus avertebant. haecque non diu sunt perpetrata.

Erat autem diritatis eius hoc quoque indicium nec obscurum nec latens, quod ludicris cruentis delectabatur et in circo sex vel 
septem aliquotiens vetitis certaminibus pugilum vicissim se concidentium perfusorumque sanguine specie ut lucratus ingentia 
laetabatur.

Superatis Tauri montis verticibus qui ad solis ortum sublimius attolluntur, Cilicia spatiis porrigitur late distentis dives bonis omnibus 
terra, eiusque lateri dextro adnexa Isauria, pari sorte uberi palmite viget et frugibus minutis, quam mediam navigabile flumen 
Calycadnus interscindit.

Quam quidem partem accusationis admiratus sum et moleste tuli potissimum esse Atratino datam. Neque enim decebat neque 
aetas illa postulabat neque, id quod animadvertere poteratis, pudor patiebatur optimi adulescentis in tali illum oratione versari. 
Vellem aliquis ex vobis robustioribus hunc male dicendi locum suscepisset; aliquanto liberius et fortius et magis more nostro 
refutaremus istam male dicendi licentiam. Tecum, Atratine, agam lenius, quod et pudor tuus moderatur orationi meae et meum 
erga te parentemque tuum beneficium tueri debeo.

Nemo quaeso miretur, si post exsudatos labores itinerum longos congestosque adfatim commeatus fiducia vestri ductante 
barbaricos pagos adventans velut mutato repente consilio ad placidiora deverti.

Ob haec et huius modi multa, quae cernebantur in paucis, omnibus timeri sunt coepta. et ne tot malis dissimulatis paulatimque 
serpentibus acervi crescerent aerumnarum, nobilitatis decreto legati mittuntur: Praetextatus ex urbi praefecto et ex vicario 
Venustus et ex consulari Minervius oraturi, ne delictis supplicia sint grandiora, neve senator quisquam inusitato et inlicito more 
tormentis exponeretur.

Eodem tempore etiam Hymetii praeclarae indolis viri negotium est actitatum, cuius hunc novimus esse textum. cum Africam pro 
consule regeret Carthaginiensibus victus inopia iam lassatis, ex horreis Romano populo destinatis frumentum dedit, pauloque 
postea cum provenisset segetum copia, integre sine ulla restituit mora.


\section{Les moeurs de l'homme fait}

Dein Syria per speciosam interpatet diffusa planitiem. hanc nobilitat Antiochia, mundo cognita civitas, cui non certaverit alia 
advecticiis ita adfluere copiis et internis, et Laodicia et Apamia itidemque Seleucia iam inde a primis auspiciis florentissimae.

Ipsam vero urbem Byzantiorum fuisse refertissimam atque ornatissimam signis quis ignorat? Quae illi, exhausti sumptibus bellisque 
maximis, cum omnis Mithridaticos impetus totumque Pontum armatum affervescentem in Asiam atque erumpentem, ore repulsum 
et cervicibus interclusum suis sustinerent, tum, inquam, Byzantii et postea signa illa et reliqua urbis ornanemta sanctissime 
custodita tenuerunt.

Qui cum venisset ob haec festinatis itineribus Antiochiam, praestrictis palatii ianuis, contempto Caesare, quem videri decuerat, ad 
praetorium cum pompa sollemni perrexit morbosque diu causatus nec regiam introiit nec processit in publicum, sed abditus multa in 
eius moliebatur exitium addens quaedam relationibus supervacua, quas subinde dimittebat ad principem.

Atque, ut Tullius ait, ut etiam ferae fame monitae plerumque ad eum locum ubi aliquando pastae sunt revertuntur, ita homines 
instar turbinis degressi montibus impeditis et arduis loca petivere mari confinia, per quae viis latebrosis sese convallibusque 
occultantes cum appeterent noctes luna etiam tum cornuta ideoque nondum solido splendore fulgente nauticos observabant quos 
cum in somnum sentirent effusos per ancoralia, quadrupedo gradu repentes seseque suspensis passibus iniectantes in scaphas 
eisdem sensim nihil opinantibus adsistebant et incendente aviditate saevitiam ne cedentium quidem ulli parcendo obtruncatis 
omnibus merces opimas velut viles nullis repugnantibus avertebant. haecque non diu sunt perpetrata.

Erat autem diritatis eius hoc quoque indicium nec obscurum nec latens, quod ludicris cruentis delectabatur et in circo sex vel 
septem aliquotiens vetitis certaminibus pugilum vicissim se concidentium perfusorumque sanguine specie ut lucratus ingentia 
laetabatur.

Superatis Tauri montis verticibus qui ad solis ortum sublimius attolluntur, Cilicia spatiis porrigitur late distentis dives bonis omnibus 
terra, eiusque lateri dextro adnexa Isauria, pari sorte uberi palmite viget et frugibus minutis, quam mediam navigabile flumen 
Calycadnus interscindit.

Quam quidem partem accusationis admiratus sum et moleste tuli potissimum esse Atratino datam. Neque enim decebat neque 
aetas illa postulabat neque, id quod animadvertere poteratis, pudor patiebatur optimi adulescentis in tali illum oratione versari. 
Vellem aliquis ex vobis robustioribus hunc male dicendi locum suscepisset; aliquanto liberius et fortius et magis more nostro 
refutaremus istam male dicendi licentiam. Tecum, Atratine, agam lenius, quod et pudor tuus moderatur orationi meae et meum 
erga te parentemque tuum beneficium tueri debeo.

Nemo quaeso miretur, si post exsudatos labores itinerum longos congestosque adfatim commeatus fiducia vestri ductante 
barbaricos pagos adventans velut mutato repente consilio ad placidiora deverti.

Ob haec et huius modi multa, quae cernebantur in paucis, omnibus timeri sunt coepta. et ne tot malis dissimulatis paulatimque 
serpentibus acervi crescerent aerumnarum, nobilitatis decreto legati mittuntur: Praetextatus ex urbi praefecto et ex vicario 
Venustus et ex consulari Minervius oraturi, ne delictis supplicia sint grandiora, neve senator quisquam inusitato et inlicito more 
tormentis exponeretur.

Eodem tempore etiam Hymetii praeclarae indolis viri negotium est actitatum, cuius hunc novimus esse textum. cum Africam pro 
consule regeret Carthaginiensibus victus inopia iam lassatis, ex horreis Romano populo destinatis frumentum dedit, pauloque 
postea cum provenisset segetum copia, integre sine ulla restituit mora.


\section{L'humeur des nobles}

Dein Syria per speciosam interpatet diffusa planitiem. hanc nobilitat Antiochia, mundo cognita civitas, cui non certaverit alia 
advecticiis ita adfluere copiis et internis, et Laodicia et Apamia itidemque Seleucia iam inde a primis auspiciis florentissimae.

Ipsam vero urbem Byzantiorum fuisse refertissimam atque ornatissimam signis quis ignorat? Quae illi, exhausti sumptibus bellisque 
maximis, cum omnis Mithridaticos impetus totumque Pontum armatum affervescentem in Asiam atque erumpentem, ore repulsum 
et cervicibus interclusum suis sustinerent, tum, inquam, Byzantii et postea signa illa et reliqua urbis ornanemta sanctissime 
custodita tenuerunt.

Qui cum venisset ob haec festinatis itineribus Antiochiam, praestrictis palatii ianuis, contempto Caesare, quem videri decuerat, ad 
praetorium cum pompa sollemni perrexit morbosque diu causatus nec regiam introiit nec processit in publicum, sed abditus multa in 
eius moliebatur exitium addens quaedam relationibus supervacua, quas subinde dimittebat ad principem.

Atque, ut Tullius ait, ut etiam ferae fame monitae plerumque ad eum locum ubi aliquando pastae sunt revertuntur, ita homines 
instar turbinis degressi montibus impeditis et arduis loca petivere mari confinia, per quae viis latebrosis sese convallibusque 
occultantes cum appeterent noctes luna etiam tum cornuta ideoque nondum solido splendore fulgente nauticos observabant quos 
cum in somnum sentirent effusos per ancoralia, quadrupedo gradu repentes seseque suspensis passibus iniectantes in scaphas 
eisdem sensim nihil opinantibus adsistebant et incendente aviditate saevitiam ne cedentium quidem ulli parcendo obtruncatis 
omnibus merces opimas velut viles nullis repugnantibus avertebant. haecque non diu sunt perpetrata.

Erat autem diritatis eius hoc quoque indicium nec obscurum nec latens, quod ludicris cruentis delectabatur et in circo sex vel 
septem aliquotiens vetitis certaminibus pugilum vicissim se concidentium perfusorumque sanguine specie ut lucratus ingentia 
laetabatur.

Superatis Tauri montis verticibus qui ad solis ortum sublimius attolluntur, Cilicia spatiis porrigitur late distentis dives bonis omnibus 
terra, eiusque lateri dextro adnexa Isauria, pari sorte uberi palmite viget et frugibus minutis, quam mediam navigabile flumen 
Calycadnus interscindit.

Quam quidem partem accusationis admiratus sum et moleste tuli potissimum esse Atratino datam. Neque enim decebat neque 
aetas illa postulabat neque, id quod animadvertere poteratis, pudor patiebatur optimi adulescentis in tali illum oratione versari. 
Vellem aliquis ex vobis robustioribus hunc male dicendi locum suscepisset; aliquanto liberius et fortius et magis more nostro 
refutaremus istam male dicendi licentiam. Tecum, Atratine, agam lenius, quod et pudor tuus moderatur orationi meae et meum 
erga te parentemque tuum beneficium tueri debeo.

Nemo quaeso miretur, si post exsudatos labores itinerum longos congestosque adfatim commeatus fiducia vestri ductante 
barbaricos pagos adventans velut mutato repente consilio ad placidiora deverti.

Ob haec et huius modi multa, quae cernebantur in paucis, omnibus timeri sunt coepta. et ne tot malis dissimulatis paulatimque 
serpentibus acervi crescerent aerumnarum, nobilitatis decreto legati mittuntur: Praetextatus ex urbi praefecto et ex vicario 
Venustus et ex consulari Minervius oraturi, ne delictis supplicia sint grandiora, neve senator quisquam inusitato et inlicito more 
tormentis exponeretur.

Eodem tempore etiam Hymetii praeclarae indolis viri negotium est actitatum, cuius hunc novimus esse textum. cum Africam pro 
consule regeret Carthaginiensibus victus inopia iam lassatis, ex horreis Romano populo destinatis frumentum dedit, pauloque 
postea cum provenisset segetum copia, integre sine ulla restituit mora.


\section{L'humeur des riches}

Dein Syria per speciosam interpatet diffusa planitiem. hanc nobilitat Antiochia, mundo cognita civitas, cui non certaverit alia 
advecticiis ita adfluere copiis et internis, et Laodicia et Apamia itidemque Seleucia iam inde a primis auspiciis florentissimae.

Ipsam vero urbem Byzantiorum fuisse refertissimam atque ornatissimam signis quis ignorat? Quae illi, exhausti sumptibus bellisque 
maximis, cum omnis Mithridaticos impetus totumque Pontum armatum affervescentem in Asiam atque erumpentem, ore repulsum 
et cervicibus interclusum suis sustinerent, tum, inquam, Byzantii et postea signa illa et reliqua urbis ornanemta sanctissime 
custodita tenuerunt.

Qui cum venisset ob haec festinatis itineribus Antiochiam, praestrictis palatii ianuis, contempto Caesare, quem videri decuerat, ad 
praetorium cum pompa sollemni perrexit morbosque diu causatus nec regiam introiit nec processit in publicum, sed abditus multa in 
eius moliebatur exitium addens quaedam relationibus supervacua, quas subinde dimittebat ad principem.

Atque, ut Tullius ait, ut etiam ferae fame monitae plerumque ad eum locum ubi aliquando pastae sunt revertuntur, ita homines 
instar turbinis degressi montibus impeditis et arduis loca petivere mari confinia, per quae viis latebrosis sese convallibusque 
occultantes cum appeterent noctes luna etiam tum cornuta ideoque nondum solido splendore fulgente nauticos observabant quos 
cum in somnum sentirent effusos per ancoralia, quadrupedo gradu repentes seseque suspensis passibus iniectantes in scaphas 
eisdem sensim nihil opinantibus adsistebant et incendente aviditate saevitiam ne cedentium quidem ulli parcendo obtruncatis 
omnibus merces opimas velut viles nullis repugnantibus avertebant. haecque non diu sunt perpetrata.

Erat autem diritatis eius hoc quoque indicium nec obscurum nec latens, quod ludicris cruentis delectabatur et in circo sex vel 
septem aliquotiens vetitis certaminibus pugilum vicissim se concidentium perfusorumque sanguine specie ut lucratus ingentia 
laetabatur.

Superatis Tauri montis verticibus qui ad solis ortum sublimius attolluntur, Cilicia spatiis porrigitur late distentis dives bonis omnibus 
terra, eiusque lateri dextro adnexa Isauria, pari sorte uberi palmite viget et frugibus minutis, quam mediam navigabile flumen 
Calycadnus interscindit.

Quam quidem partem accusationis admiratus sum et moleste tuli potissimum esse Atratino datam. Neque enim decebat neque 
aetas illa postulabat neque, id quod animadvertere poteratis, pudor patiebatur optimi adulescentis in tali illum oratione versari. 
Vellem aliquis ex vobis robustioribus hunc male dicendi locum suscepisset; aliquanto liberius et fortius et magis more nostro 
refutaremus istam male dicendi licentiam. Tecum, Atratine, agam lenius, quod et pudor tuus moderatur orationi meae et meum 
erga te parentemque tuum beneficium tueri debeo.

Nemo quaeso miretur, si post exsudatos labores itinerum longos congestosque adfatim commeatus fiducia vestri ductante 
barbaricos pagos adventans velut mutato repente consilio ad placidiora deverti.

Ob haec et huius modi multa, quae cernebantur in paucis, omnibus timeri sunt coepta. et ne tot malis dissimulatis paulatimque 
serpentibus acervi crescerent aerumnarum, nobilitatis decreto legati mittuntur: Praetextatus ex urbi praefecto et ex vicario 
Venustus et ex consulari Minervius oraturi, ne delictis supplicia sint grandiora, neve senator quisquam inusitato et inlicito more 
tormentis exponeretur.

Eodem tempore etiam Hymetii praeclarae indolis viri negotium est actitatum, cuius hunc novimus esse textum. cum Africam pro 
consule regeret Carthaginiensibus victus inopia iam lassatis, ex horreis Romano populo destinatis frumentum dedit, pauloque 
postea cum provenisset segetum copia, integre sine ulla restituit mora.


\section{L'humeur des grands seigneurs, et de ceux qui sont dans une haute prospérité}

Dein Syria per speciosam interpatet diffusa planitiem. hanc nobilitat Antiochia, mundo cognita civitas, cui non certaverit alia 
advecticiis ita adfluere copiis et internis, et Laodicia et Apamia itidemque Seleucia iam inde a primis auspiciis florentissimae.

Ipsam vero urbem Byzantiorum fuisse refertissimam atque ornatissimam signis quis ignorat? Quae illi, exhausti sumptibus bellisque 
maximis, cum omnis Mithridaticos impetus totumque Pontum armatum affervescentem in Asiam atque erumpentem, ore repulsum 
et cervicibus interclusum suis sustinerent, tum, inquam, Byzantii et postea signa illa et reliqua urbis ornanemta sanctissime 
custodita tenuerunt.

Qui cum venisset ob haec festinatis itineribus Antiochiam, praestrictis palatii ianuis, contempto Caesare, quem videri decuerat, ad 
praetorium cum pompa sollemni perrexit morbosque diu causatus nec regiam introiit nec processit in publicum, sed abditus multa in 
eius moliebatur exitium addens quaedam relationibus supervacua, quas subinde dimittebat ad principem.

Atque, ut Tullius ait, ut etiam ferae fame monitae plerumque ad eum locum ubi aliquando pastae sunt revertuntur, ita homines 
instar turbinis degressi montibus impeditis et arduis loca petivere mari confinia, per quae viis latebrosis sese convallibusque 
occultantes cum appeterent noctes luna etiam tum cornuta ideoque nondum solido splendore fulgente nauticos observabant quos 
cum in somnum sentirent effusos per ancoralia, quadrupedo gradu repentes seseque suspensis passibus iniectantes in scaphas 
eisdem sensim nihil opinantibus adsistebant et incendente aviditate saevitiam ne cedentium quidem ulli parcendo obtruncatis 
omnibus merces opimas velut viles nullis repugnantibus avertebant. haecque non diu sunt perpetrata.

Erat autem diritatis eius hoc quoque indicium nec obscurum nec latens, quod ludicris cruentis delectabatur et in circo sex vel 
septem aliquotiens vetitis certaminibus pugilum vicissim se concidentium perfusorumque sanguine specie ut lucratus ingentia 
laetabatur.

Superatis Tauri montis verticibus qui ad solis ortum sublimius attolluntur, Cilicia spatiis porrigitur late distentis dives bonis omnibus 
terra, eiusque lateri dextro adnexa Isauria, pari sorte uberi palmite viget et frugibus minutis, quam mediam navigabile flumen 
Calycadnus interscindit.

Quam quidem partem accusationis admiratus sum et moleste tuli potissimum esse Atratino datam. Neque enim decebat neque 
aetas illa postulabat neque, id quod animadvertere poteratis, pudor patiebatur optimi adulescentis in tali illum oratione versari. 
Vellem aliquis ex vobis robustioribus hunc male dicendi locum suscepisset; aliquanto liberius et fortius et magis more nostro 
refutaremus istam male dicendi licentiam. Tecum, Atratine, agam lenius, quod et pudor tuus moderatur orationi meae et meum 
erga te parentemque tuum beneficium tueri debeo.

Nemo quaeso miretur, si post exsudatos labores itinerum longos congestosque adfatim commeatus fiducia vestri ductante 
barbaricos pagos adventans velut mutato repente consilio ad placidiora deverti.

Ob haec et huius modi multa, quae cernebantur in paucis, omnibus timeri sunt coepta. et ne tot malis dissimulatis paulatimque 
serpentibus acervi crescerent aerumnarum, nobilitatis decreto legati mittuntur: Praetextatus ex urbi praefecto et ex vicario 
Venustus et ex consulari Minervius oraturi, ne delictis supplicia sint grandiora, neve senator quisquam inusitato et inlicito more 
tormentis exponeretur.

Eodem tempore etiam Hymetii praeclarae indolis viri negotium est actitatum, cuius hunc novimus esse textum. cum Africam pro 
consule regeret Carthaginiensibus victus inopia iam lassatis, ex horreis Romano populo destinatis frumentum dedit, pauloque 
postea cum provenisset segetum copia, integre sine ulla restituit mora.


\chapter{Lieux et preuves pour les trois genres en commun}
\section{De la nécessité de ces lieux}

Dein Syria per speciosam interpatet diffusa planitiem. hanc nobilitat Antiochia, mundo cognita civitas, cui non certaverit alia 
advecticiis ita adfluere copiis et internis, et Laodicia et Apamia itidemque Seleucia iam inde a primis auspiciis florentissimae.

Ipsam vero urbem Byzantiorum fuisse refertissimam atque ornatissimam signis quis ignorat? Quae illi, exhausti sumptibus bellisque 
maximis, cum omnis Mithridaticos impetus totumque Pontum armatum affervescentem in Asiam atque erumpentem, ore repulsum 
et cervicibus interclusum suis sustinerent, tum, inquam, Byzantii et postea signa illa et reliqua urbis ornanemta sanctissime 
custodita tenuerunt.

Qui cum venisset ob haec festinatis itineribus Antiochiam, praestrictis palatii ianuis, contempto Caesare, quem videri decuerat, ad 
praetorium cum pompa sollemni perrexit morbosque diu causatus nec regiam introiit nec processit in publicum, sed abditus multa in 
eius moliebatur exitium addens quaedam relationibus supervacua, quas subinde dimittebat ad principem.

Atque, ut Tullius ait, ut etiam ferae fame monitae plerumque ad eum locum ubi aliquando pastae sunt revertuntur, ita homines 
instar turbinis degressi montibus impeditis et arduis loca petivere mari confinia, per quae viis latebrosis sese convallibusque 
occultantes cum appeterent noctes luna etiam tum cornuta ideoque nondum solido splendore fulgente nauticos observabant quos 
cum in somnum sentirent effusos per ancoralia, quadrupedo gradu repentes seseque suspensis passibus iniectantes in scaphas 
eisdem sensim nihil opinantibus adsistebant et incendente aviditate saevitiam ne cedentium quidem ulli parcendo obtruncatis 
omnibus merces opimas velut viles nullis repugnantibus avertebant. haecque non diu sunt perpetrata.

Erat autem diritatis eius hoc quoque indicium nec obscurum nec latens, quod ludicris cruentis delectabatur et in circo sex vel 
septem aliquotiens vetitis certaminibus pugilum vicissim se concidentium perfusorumque sanguine specie ut lucratus ingentia 
laetabatur.

Superatis Tauri montis verticibus qui ad solis ortum sublimius attolluntur, Cilicia spatiis porrigitur late distentis dives bonis omnibus 
terra, eiusque lateri dextro adnexa Isauria, pari sorte uberi palmite viget et frugibus minutis, quam mediam navigabile flumen 
Calycadnus interscindit.

Quam quidem partem accusationis admiratus sum et moleste tuli potissimum esse Atratino datam. Neque enim decebat neque 
aetas illa postulabat neque, id quod animadvertere poteratis, pudor patiebatur optimi adulescentis in tali illum oratione versari. 
Vellem aliquis ex vobis robustioribus hunc male dicendi locum suscepisset; aliquanto liberius et fortius et magis more nostro 
refutaremus istam male dicendi licentiam. Tecum, Atratine, agam lenius, quod et pudor tuus moderatur orationi meae et meum 
erga te parentemque tuum beneficium tueri debeo.

Nemo quaeso miretur, si post exsudatos labores itinerum longos congestosque adfatim commeatus fiducia vestri ductante 
barbaricos pagos adventans velut mutato repente consilio ad placidiora deverti.

Ob haec et huius modi multa, quae cernebantur in paucis, omnibus timeri sunt coepta. et ne tot malis dissimulatis paulatimque 
serpentibus acervi crescerent aerumnarum, nobilitatis decreto legati mittuntur: Praetextatus ex urbi praefecto et ex vicario 
Venustus et ex consulari Minervius oraturi, ne delictis supplicia sint grandiora, neve senator quisquam inusitato et inlicito more 
tormentis exponeretur.

Eodem tempore etiam Hymetii praeclarae indolis viri negotium est actitatum, cuius hunc novimus esse textum. cum Africam pro 
consule regeret Carthaginiensibus victus inopia iam lassatis, ex horreis Romano populo destinatis frumentum dedit, pauloque 
postea cum provenisset segetum copia, integre sine ulla restituit mora.


\section{Pour connaître si une chose est possible ou impossible}

Dein Syria per speciosam interpatet diffusa planitiem. hanc nobilitat Antiochia, mundo cognita civitas, cui non certaverit alia 
advecticiis ita adfluere copiis et internis, et Laodicia et Apamia itidemque Seleucia iam inde a primis auspiciis florentissimae.

Ipsam vero urbem Byzantiorum fuisse refertissimam atque ornatissimam signis quis ignorat? Quae illi, exhausti sumptibus bellisque 
maximis, cum omnis Mithridaticos impetus totumque Pontum armatum affervescentem in Asiam atque erumpentem, ore repulsum 
et cervicibus interclusum suis sustinerent, tum, inquam, Byzantii et postea signa illa et reliqua urbis ornanemta sanctissime 
custodita tenuerunt.

Qui cum venisset ob haec festinatis itineribus Antiochiam, praestrictis palatii ianuis, contempto Caesare, quem videri decuerat, ad 
praetorium cum pompa sollemni perrexit morbosque diu causatus nec regiam introiit nec processit in publicum, sed abditus multa in 
eius moliebatur exitium addens quaedam relationibus supervacua, quas subinde dimittebat ad principem.

Atque, ut Tullius ait, ut etiam ferae fame monitae plerumque ad eum locum ubi aliquando pastae sunt revertuntur, ita homines 
instar turbinis degressi montibus impeditis et arduis loca petivere mari confinia, per quae viis latebrosis sese convallibusque 
occultantes cum appeterent noctes luna etiam tum cornuta ideoque nondum solido splendore fulgente nauticos observabant quos 
cum in somnum sentirent effusos per ancoralia, quadrupedo gradu repentes seseque suspensis passibus iniectantes in scaphas 
eisdem sensim nihil opinantibus adsistebant et incendente aviditate saevitiam ne cedentium quidem ulli parcendo obtruncatis 
omnibus merces opimas velut viles nullis repugnantibus avertebant. haecque non diu sunt perpetrata.

Erat autem diritatis eius hoc quoque indicium nec obscurum nec latens, quod ludicris cruentis delectabatur et in circo sex vel 
septem aliquotiens vetitis certaminibus pugilum vicissim se concidentium perfusorumque sanguine specie ut lucratus ingentia 
laetabatur.

Superatis Tauri montis verticibus qui ad solis ortum sublimius attolluntur, Cilicia spatiis porrigitur late distentis dives bonis omnibus 
terra, eiusque lateri dextro adnexa Isauria, pari sorte uberi palmite viget et frugibus minutis, quam mediam navigabile flumen 
Calycadnus interscindit.

Quam quidem partem accusationis admiratus sum et moleste tuli potissimum esse Atratino datam. Neque enim decebat neque 
aetas illa postulabat neque, id quod animadvertere poteratis, pudor patiebatur optimi adulescentis in tali illum oratione versari. 
Vellem aliquis ex vobis robustioribus hunc male dicendi locum suscepisset; aliquanto liberius et fortius et magis more nostro 
refutaremus istam male dicendi licentiam. Tecum, Atratine, agam lenius, quod et pudor tuus moderatur orationi meae et meum 
erga te parentemque tuum beneficium tueri debeo.

Nemo quaeso miretur, si post exsudatos labores itinerum longos congestosque adfatim commeatus fiducia vestri ductante 
barbaricos pagos adventans velut mutato repente consilio ad placidiora deverti.

Ob haec et huius modi multa, quae cernebantur in paucis, omnibus timeri sunt coepta. et ne tot malis dissimulatis paulatimque 
serpentibus acervi crescerent aerumnarum, nobilitatis decreto legati mittuntur: Praetextatus ex urbi praefecto et ex vicario 
Venustus et ex consulari Minervius oraturi, ne delictis supplicia sint grandiora, neve senator quisquam inusitato et inlicito more 
tormentis exponeretur.

Eodem tempore etiam Hymetii praeclarae indolis viri negotium est actitatum, cuius hunc novimus esse textum. cum Africam pro 
consule regeret Carthaginiensibus victus inopia iam lassatis, ex horreis Romano populo destinatis frumentum dedit, pauloque 
postea cum provenisset segetum copia, integre sine ulla restituit mora.


\section{De l'exemple}

Dein Syria per speciosam interpatet diffusa planitiem. hanc nobilitat Antiochia, mundo cognita civitas, cui non certaverit alia 
advecticiis ita adfluere copiis et internis, et Laodicia et Apamia itidemque Seleucia iam inde a primis auspiciis florentissimae.

Ipsam vero urbem Byzantiorum fuisse refertissimam atque ornatissimam signis quis ignorat? Quae illi, exhausti sumptibus bellisque 
maximis, cum omnis Mithridaticos impetus totumque Pontum armatum affervescentem in Asiam atque erumpentem, ore repulsum 
et cervicibus interclusum suis sustinerent, tum, inquam, Byzantii et postea signa illa et reliqua urbis ornanemta sanctissime 
custodita tenuerunt.

Qui cum venisset ob haec festinatis itineribus Antiochiam, praestrictis palatii ianuis, contempto Caesare, quem videri decuerat, ad 
praetorium cum pompa sollemni perrexit morbosque diu causatus nec regiam introiit nec processit in publicum, sed abditus multa in 
eius moliebatur exitium addens quaedam relationibus supervacua, quas subinde dimittebat ad principem.

Atque, ut Tullius ait, ut etiam ferae fame monitae plerumque ad eum locum ubi aliquando pastae sunt revertuntur, ita homines 
instar turbinis degressi montibus impeditis et arduis loca petivere mari confinia, per quae viis latebrosis sese convallibusque 
occultantes cum appeterent noctes luna etiam tum cornuta ideoque nondum solido splendore fulgente nauticos observabant quos 
cum in somnum sentirent effusos per ancoralia, quadrupedo gradu repentes seseque suspensis passibus iniectantes in scaphas 
eisdem sensim nihil opinantibus adsistebant et incendente aviditate saevitiam ne cedentium quidem ulli parcendo obtruncatis 
omnibus merces opimas velut viles nullis repugnantibus avertebant. haecque non diu sunt perpetrata.

Erat autem diritatis eius hoc quoque indicium nec obscurum nec latens, quod ludicris cruentis delectabatur et in circo sex vel 
septem aliquotiens vetitis certaminibus pugilum vicissim se concidentium perfusorumque sanguine specie ut lucratus ingentia 
laetabatur.

Superatis Tauri montis verticibus qui ad solis ortum sublimius attolluntur, Cilicia spatiis porrigitur late distentis dives bonis omnibus 
terra, eiusque lateri dextro adnexa Isauria, pari sorte uberi palmite viget et frugibus minutis, quam mediam navigabile flumen 
Calycadnus interscindit.

Quam quidem partem accusationis admiratus sum et moleste tuli potissimum esse Atratino datam. Neque enim decebat neque 
aetas illa postulabat neque, id quod animadvertere poteratis, pudor patiebatur optimi adulescentis in tali illum oratione versari. 
Vellem aliquis ex vobis robustioribus hunc male dicendi locum suscepisset; aliquanto liberius et fortius et magis more nostro 
refutaremus istam male dicendi licentiam. Tecum, Atratine, agam lenius, quod et pudor tuus moderatur orationi meae et meum 
erga te parentemque tuum beneficium tueri debeo.

Nemo quaeso miretur, si post exsudatos labores itinerum longos congestosque adfatim commeatus fiducia vestri ductante 
barbaricos pagos adventans velut mutato repente consilio ad placidiora deverti.

Ob haec et huius modi multa, quae cernebantur in paucis, omnibus timeri sunt coepta. et ne tot malis dissimulatis paulatimque 
serpentibus acervi crescerent aerumnarum, nobilitatis decreto legati mittuntur: Praetextatus ex urbi praefecto et ex vicario 
Venustus et ex consulari Minervius oraturi, ne delictis supplicia sint grandiora, neve senator quisquam inusitato et inlicito more 
tormentis exponeretur.

Eodem tempore etiam Hymetii praeclarae indolis viri negotium est actitatum, cuius hunc novimus esse textum. cum Africam pro 
consule regeret Carthaginiensibus victus inopia iam lassatis, ex horreis Romano populo destinatis frumentum dedit, pauloque 
postea cum provenisset segetum copia, integre sine ulla restituit mora.


\section{Des sentences}

Dein Syria per speciosam interpatet diffusa planitiem. hanc nobilitat Antiochia, mundo cognita civitas, cui non certaverit alia 
advecticiis ita adfluere copiis et internis, et Laodicia et Apamia itidemque Seleucia iam inde a primis auspiciis florentissimae.

Ipsam vero urbem Byzantiorum fuisse refertissimam atque ornatissimam signis quis ignorat? Quae illi, exhausti sumptibus bellisque 
maximis, cum omnis Mithridaticos impetus totumque Pontum armatum affervescentem in Asiam atque erumpentem, ore repulsum 
et cervicibus interclusum suis sustinerent, tum, inquam, Byzantii et postea signa illa et reliqua urbis ornanemta sanctissime 
custodita tenuerunt.

Qui cum venisset ob haec festinatis itineribus Antiochiam, praestrictis palatii ianuis, contempto Caesare, quem videri decuerat, ad 
praetorium cum pompa sollemni perrexit morbosque diu causatus nec regiam introiit nec processit in publicum, sed abditus multa in 
eius moliebatur exitium addens quaedam relationibus supervacua, quas subinde dimittebat ad principem.

Atque, ut Tullius ait, ut etiam ferae fame monitae plerumque ad eum locum ubi aliquando pastae sunt revertuntur, ita homines 
instar turbinis degressi montibus impeditis et arduis loca petivere mari confinia, per quae viis latebrosis sese convallibusque 
occultantes cum appeterent noctes luna etiam tum cornuta ideoque nondum solido splendore fulgente nauticos observabant quos 
cum in somnum sentirent effusos per ancoralia, quadrupedo gradu repentes seseque suspensis passibus iniectantes in scaphas 
eisdem sensim nihil opinantibus adsistebant et incendente aviditate saevitiam ne cedentium quidem ulli parcendo obtruncatis 
omnibus merces opimas velut viles nullis repugnantibus avertebant. haecque non diu sunt perpetrata.

Erat autem diritatis eius hoc quoque indicium nec obscurum nec latens, quod ludicris cruentis delectabatur et in circo sex vel 
septem aliquotiens vetitis certaminibus pugilum vicissim se concidentium perfusorumque sanguine specie ut lucratus ingentia 
laetabatur.

Superatis Tauri montis verticibus qui ad solis ortum sublimius attolluntur, Cilicia spatiis porrigitur late distentis dives bonis omnibus 
terra, eiusque lateri dextro adnexa Isauria, pari sorte uberi palmite viget et frugibus minutis, quam mediam navigabile flumen 
Calycadnus interscindit.

Quam quidem partem accusationis admiratus sum et moleste tuli potissimum esse Atratino datam. Neque enim decebat neque 
aetas illa postulabat neque, id quod animadvertere poteratis, pudor patiebatur optimi adulescentis in tali illum oratione versari. 
Vellem aliquis ex vobis robustioribus hunc male dicendi locum suscepisset; aliquanto liberius et fortius et magis more nostro 
refutaremus istam male dicendi licentiam. Tecum, Atratine, agam lenius, quod et pudor tuus moderatur orationi meae et meum 
erga te parentemque tuum beneficium tueri debeo.

Nemo quaeso miretur, si post exsudatos labores itinerum longos congestosque adfatim commeatus fiducia vestri ductante 
barbaricos pagos adventans velut mutato repente consilio ad placidiora deverti.

Ob haec et huius modi multa, quae cernebantur in paucis, omnibus timeri sunt coepta. et ne tot malis dissimulatis paulatimque 
serpentibus acervi crescerent aerumnarum, nobilitatis decreto legati mittuntur: Praetextatus ex urbi praefecto et ex vicario 
Venustus et ex consulari Minervius oraturi, ne delictis supplicia sint grandiora, neve senator quisquam inusitato et inlicito more 
tormentis exponeretur.

Eodem tempore etiam Hymetii praeclarae indolis viri negotium est actitatum, cuius hunc novimus esse textum. cum Africam pro 
consule regeret Carthaginiensibus victus inopia iam lassatis, ex horreis Romano populo destinatis frumentum dedit, pauloque 
postea cum provenisset segetum copia, integre sine ulla restituit mora.


\section{Des enthymèmes en général}

Dein Syria per speciosam interpatet diffusa planitiem. hanc nobilitat Antiochia, mundo cognita civitas, cui non certaverit alia 
advecticiis ita adfluere copiis et internis, et Laodicia et Apamia itidemque Seleucia iam inde a primis auspiciis florentissimae.

Ipsam vero urbem Byzantiorum fuisse refertissimam atque ornatissimam signis quis ignorat? Quae illi, exhausti sumptibus bellisque 
maximis, cum omnis Mithridaticos impetus totumque Pontum armatum affervescentem in Asiam atque erumpentem, ore repulsum 
et cervicibus interclusum suis sustinerent, tum, inquam, Byzantii et postea signa illa et reliqua urbis ornanemta sanctissime 
custodita tenuerunt.

Qui cum venisset ob haec festinatis itineribus Antiochiam, praestrictis palatii ianuis, contempto Caesare, quem videri decuerat, ad 
praetorium cum pompa sollemni perrexit morbosque diu causatus nec regiam introiit nec processit in publicum, sed abditus multa in 
eius moliebatur exitium addens quaedam relationibus supervacua, quas subinde dimittebat ad principem.

Atque, ut Tullius ait, ut etiam ferae fame monitae plerumque ad eum locum ubi aliquando pastae sunt revertuntur, ita homines 
instar turbinis degressi montibus impeditis et arduis loca petivere mari confinia, per quae viis latebrosis sese convallibusque 
occultantes cum appeterent noctes luna etiam tum cornuta ideoque nondum solido splendore fulgente nauticos observabant quos 
cum in somnum sentirent effusos per ancoralia, quadrupedo gradu repentes seseque suspensis passibus iniectantes in scaphas 
eisdem sensim nihil opinantibus adsistebant et incendente aviditate saevitiam ne cedentium quidem ulli parcendo obtruncatis 
omnibus merces opimas velut viles nullis repugnantibus avertebant. haecque non diu sunt perpetrata.

Erat autem diritatis eius hoc quoque indicium nec obscurum nec latens, quod ludicris cruentis delectabatur et in circo sex vel 
septem aliquotiens vetitis certaminibus pugilum vicissim se concidentium perfusorumque sanguine specie ut lucratus ingentia 
laetabatur.

Superatis Tauri montis verticibus qui ad solis ortum sublimius attolluntur, Cilicia spatiis porrigitur late distentis dives bonis omnibus 
terra, eiusque lateri dextro adnexa Isauria, pari sorte uberi palmite viget et frugibus minutis, quam mediam navigabile flumen 
Calycadnus interscindit.

Quam quidem partem accusationis admiratus sum et moleste tuli potissimum esse Atratino datam. Neque enim decebat neque 
aetas illa postulabat neque, id quod animadvertere poteratis, pudor patiebatur optimi adulescentis in tali illum oratione versari. 
Vellem aliquis ex vobis robustioribus hunc male dicendi locum suscepisset; aliquanto liberius et fortius et magis more nostro 
refutaremus istam male dicendi licentiam. Tecum, Atratine, agam lenius, quod et pudor tuus moderatur orationi meae et meum 
erga te parentemque tuum beneficium tueri debeo.

Nemo quaeso miretur, si post exsudatos labores itinerum longos congestosque adfatim commeatus fiducia vestri ductante 
barbaricos pagos adventans velut mutato repente consilio ad placidiora deverti.

Ob haec et huius modi multa, quae cernebantur in paucis, omnibus timeri sunt coepta. et ne tot malis dissimulatis paulatimque 
serpentibus acervi crescerent aerumnarum, nobilitatis decreto legati mittuntur: Praetextatus ex urbi praefecto et ex vicario 
Venustus et ex consulari Minervius oraturi, ne delictis supplicia sint grandiora, neve senator quisquam inusitato et inlicito more 
tormentis exponeretur.

Eodem tempore etiam Hymetii praeclarae indolis viri negotium est actitatum, cuius hunc novimus esse textum. cum Africam pro 
consule regeret Carthaginiensibus victus inopia iam lassatis, ex horreis Romano populo destinatis frumentum dedit, pauloque 
postea cum provenisset segetum copia, integre sine ulla restituit mora.


\section{Lieux pour les enthymèmes véritables}

Dein Syria per speciosam interpatet diffusa planitiem. hanc nobilitat Antiochia, mundo cognita civitas, cui non certaverit alia 
advecticiis ita adfluere copiis et internis, et Laodicia et Apamia itidemque Seleucia iam inde a primis auspiciis florentissimae.

Ipsam vero urbem Byzantiorum fuisse refertissimam atque ornatissimam signis quis ignorat? Quae illi, exhausti sumptibus bellisque 
maximis, cum omnis Mithridaticos impetus totumque Pontum armatum affervescentem in Asiam atque erumpentem, ore repulsum 
et cervicibus interclusum suis sustinerent, tum, inquam, Byzantii et postea signa illa et reliqua urbis ornanemta sanctissime 
custodita tenuerunt.

Qui cum venisset ob haec festinatis itineribus Antiochiam, praestrictis palatii ianuis, contempto Caesare, quem videri decuerat, ad 
praetorium cum pompa sollemni perrexit morbosque diu causatus nec regiam introiit nec processit in publicum, sed abditus multa in 
eius moliebatur exitium addens quaedam relationibus supervacua, quas subinde dimittebat ad principem.

Atque, ut Tullius ait, ut etiam ferae fame monitae plerumque ad eum locum ubi aliquando pastae sunt revertuntur, ita homines 
instar turbinis degressi montibus impeditis et arduis loca petivere mari confinia, per quae viis latebrosis sese convallibusque 
occultantes cum appeterent noctes luna etiam tum cornuta ideoque nondum solido splendore fulgente nauticos observabant quos 
cum in somnum sentirent effusos per ancoralia, quadrupedo gradu repentes seseque suspensis passibus iniectantes in scaphas 
eisdem sensim nihil opinantibus adsistebant et incendente aviditate saevitiam ne cedentium quidem ulli parcendo obtruncatis 
omnibus merces opimas velut viles nullis repugnantibus avertebant. haecque non diu sunt perpetrata.

Erat autem diritatis eius hoc quoque indicium nec obscurum nec latens, quod ludicris cruentis delectabatur et in circo sex vel 
septem aliquotiens vetitis certaminibus pugilum vicissim se concidentium perfusorumque sanguine specie ut lucratus ingentia 
laetabatur.

Superatis Tauri montis verticibus qui ad solis ortum sublimius attolluntur, Cilicia spatiis porrigitur late distentis dives bonis omnibus 
terra, eiusque lateri dextro adnexa Isauria, pari sorte uberi palmite viget et frugibus minutis, quam mediam navigabile flumen 
Calycadnus interscindit.

Quam quidem partem accusationis admiratus sum et moleste tuli potissimum esse Atratino datam. Neque enim decebat neque 
aetas illa postulabat neque, id quod animadvertere poteratis, pudor patiebatur optimi adulescentis in tali illum oratione versari. 
Vellem aliquis ex vobis robustioribus hunc male dicendi locum suscepisset; aliquanto liberius et fortius et magis more nostro 
refutaremus istam male dicendi licentiam. Tecum, Atratine, agam lenius, quod et pudor tuus moderatur orationi meae et meum 
erga te parentemque tuum beneficium tueri debeo.

Nemo quaeso miretur, si post exsudatos labores itinerum longos congestosque adfatim commeatus fiducia vestri ductante 
barbaricos pagos adventans velut mutato repente consilio ad placidiora deverti.

Ob haec et huius modi multa, quae cernebantur in paucis, omnibus timeri sunt coepta. et ne tot malis dissimulatis paulatimque 
serpentibus acervi crescerent aerumnarum, nobilitatis decreto legati mittuntur: Praetextatus ex urbi praefecto et ex vicario 
Venustus et ex consulari Minervius oraturi, ne delictis supplicia sint grandiora, neve senator quisquam inusitato et inlicito more 
tormentis exponeretur.

Eodem tempore etiam Hymetii praeclarae indolis viri negotium est actitatum, cuius hunc novimus esse textum. cum Africam pro 
consule regeret Carthaginiensibus victus inopia iam lassatis, ex horreis Romano populo destinatis frumentum dedit, pauloque 
postea cum provenisset segetum copia, integre sine ulla restituit mora.


\section{Lieux pour les enthymèmes faux}

Dein Syria per speciosam interpatet diffusa planitiem. hanc nobilitat Antiochia, mundo cognita civitas, cui non certaverit alia 
advecticiis ita adfluere copiis et internis, et Laodicia et Apamia itidemque Seleucia iam inde a primis auspiciis florentissimae.

Ipsam vero urbem Byzantiorum fuisse refertissimam atque ornatissimam signis quis ignorat? Quae illi, exhausti sumptibus bellisque 
maximis, cum omnis Mithridaticos impetus totumque Pontum armatum affervescentem in Asiam atque erumpentem, ore repulsum 
et cervicibus interclusum suis sustinerent, tum, inquam, Byzantii et postea signa illa et reliqua urbis ornanemta sanctissime 
custodita tenuerunt.

Qui cum venisset ob haec festinatis itineribus Antiochiam, praestrictis palatii ianuis, contempto Caesare, quem videri decuerat, ad 
praetorium cum pompa sollemni perrexit morbosque diu causatus nec regiam introiit nec processit in publicum, sed abditus multa in 
eius moliebatur exitium addens quaedam relationibus supervacua, quas subinde dimittebat ad principem.

Atque, ut Tullius ait, ut etiam ferae fame monitae plerumque ad eum locum ubi aliquando pastae sunt revertuntur, ita homines 
instar turbinis degressi montibus impeditis et arduis loca petivere mari confinia, per quae viis latebrosis sese convallibusque 
occultantes cum appeterent noctes luna etiam tum cornuta ideoque nondum solido splendore fulgente nauticos observabant quos 
cum in somnum sentirent effusos per ancoralia, quadrupedo gradu repentes seseque suspensis passibus iniectantes in scaphas 
eisdem sensim nihil opinantibus adsistebant et incendente aviditate saevitiam ne cedentium quidem ulli parcendo obtruncatis 
omnibus merces opimas velut viles nullis repugnantibus avertebant. haecque non diu sunt perpetrata.

Erat autem diritatis eius hoc quoque indicium nec obscurum nec latens, quod ludicris cruentis delectabatur et in circo sex vel 
septem aliquotiens vetitis certaminibus pugilum vicissim se concidentium perfusorumque sanguine specie ut lucratus ingentia 
laetabatur.

Superatis Tauri montis verticibus qui ad solis ortum sublimius attolluntur, Cilicia spatiis porrigitur late distentis dives bonis omnibus 
terra, eiusque lateri dextro adnexa Isauria, pari sorte uberi palmite viget et frugibus minutis, quam mediam navigabile flumen 
Calycadnus interscindit.

Quam quidem partem accusationis admiratus sum et moleste tuli potissimum esse Atratino datam. Neque enim decebat neque 
aetas illa postulabat neque, id quod animadvertere poteratis, pudor patiebatur optimi adulescentis in tali illum oratione versari. 
Vellem aliquis ex vobis robustioribus hunc male dicendi locum suscepisset; aliquanto liberius et fortius et magis more nostro 
refutaremus istam male dicendi licentiam. Tecum, Atratine, agam lenius, quod et pudor tuus moderatur orationi meae et meum 
erga te parentemque tuum beneficium tueri debeo.

Nemo quaeso miretur, si post exsudatos labores itinerum longos congestosque adfatim commeatus fiducia vestri ductante 
barbaricos pagos adventans velut mutato repente consilio ad placidiora deverti.

Ob haec et huius modi multa, quae cernebantur in paucis, omnibus timeri sunt coepta. et ne tot malis dissimulatis paulatimque 
serpentibus acervi crescerent aerumnarum, nobilitatis decreto legati mittuntur: Praetextatus ex urbi praefecto et ex vicario 
Venustus et ex consulari Minervius oraturi, ne delictis supplicia sint grandiora, neve senator quisquam inusitato et inlicito more 
tormentis exponeretur.

Eodem tempore etiam Hymetii praeclarae indolis viri negotium est actitatum, cuius hunc novimus esse textum. cum Africam pro 
consule regeret Carthaginiensibus victus inopia iam lassatis, ex horreis Romano populo destinatis frumentum dedit, pauloque 
postea cum provenisset segetum copia, integre sine ulla restituit mora.


\section{Des solutions}

Dein Syria per speciosam interpatet diffusa planitiem. hanc nobilitat Antiochia, mundo cognita civitas, cui non certaverit alia 
advecticiis ita adfluere copiis et internis, et Laodicia et Apamia itidemque Seleucia iam inde a primis auspiciis florentissimae.

Ipsam vero urbem Byzantiorum fuisse refertissimam atque ornatissimam signis quis ignorat? Quae illi, exhausti sumptibus bellisque 
maximis, cum omnis Mithridaticos impetus totumque Pontum armatum affervescentem in Asiam atque erumpentem, ore repulsum 
et cervicibus interclusum suis sustinerent, tum, inquam, Byzantii et postea signa illa et reliqua urbis ornanemta sanctissime 
custodita tenuerunt.

Qui cum venisset ob haec festinatis itineribus Antiochiam, praestrictis palatii ianuis, contempto Caesare, quem videri decuerat, ad 
praetorium cum pompa sollemni perrexit morbosque diu causatus nec regiam introiit nec processit in publicum, sed abditus multa in 
eius moliebatur exitium addens quaedam relationibus supervacua, quas subinde dimittebat ad principem.

Atque, ut Tullius ait, ut etiam ferae fame monitae plerumque ad eum locum ubi aliquando pastae sunt revertuntur, ita homines 
instar turbinis degressi montibus impeditis et arduis loca petivere mari confinia, per quae viis latebrosis sese convallibusque 
occultantes cum appeterent noctes luna etiam tum cornuta ideoque nondum solido splendore fulgente nauticos observabant quos 
cum in somnum sentirent effusos per ancoralia, quadrupedo gradu repentes seseque suspensis passibus iniectantes in scaphas 
eisdem sensim nihil opinantibus adsistebant et incendente aviditate saevitiam ne cedentium quidem ulli parcendo obtruncatis 
omnibus merces opimas velut viles nullis repugnantibus avertebant. haecque non diu sunt perpetrata.

Erat autem diritatis eius hoc quoque indicium nec obscurum nec latens, quod ludicris cruentis delectabatur et in circo sex vel 
septem aliquotiens vetitis certaminibus pugilum vicissim se concidentium perfusorumque sanguine specie ut lucratus ingentia 
laetabatur.

Superatis Tauri montis verticibus qui ad solis ortum sublimius attolluntur, Cilicia spatiis porrigitur late distentis dives bonis omnibus 
terra, eiusque lateri dextro adnexa Isauria, pari sorte uberi palmite viget et frugibus minutis, quam mediam navigabile flumen 
Calycadnus interscindit.

Quam quidem partem accusationis admiratus sum et moleste tuli potissimum esse Atratino datam. Neque enim decebat neque 
aetas illa postulabat neque, id quod animadvertere poteratis, pudor patiebatur optimi adulescentis in tali illum oratione versari. 
Vellem aliquis ex vobis robustioribus hunc male dicendi locum suscepisset; aliquanto liberius et fortius et magis more nostro 
refutaremus istam male dicendi licentiam. Tecum, Atratine, agam lenius, quod et pudor tuus moderatur orationi meae et meum 
erga te parentemque tuum beneficium tueri debeo.

Nemo quaeso miretur, si post exsudatos labores itinerum longos congestosque adfatim commeatus fiducia vestri ductante 
barbaricos pagos adventans velut mutato repente consilio ad placidiora deverti.

Ob haec et huius modi multa, quae cernebantur in paucis, omnibus timeri sunt coepta. et ne tot malis dissimulatis paulatimque 
serpentibus acervi crescerent aerumnarum, nobilitatis decreto legati mittuntur: Praetextatus ex urbi praefecto et ex vicario 
Venustus et ex consulari Minervius oraturi, ne delictis supplicia sint grandiora, neve senator quisquam inusitato et inlicito more 
tormentis exponeretur.

Eodem tempore etiam Hymetii praeclarae indolis viri negotium est actitatum, cuius hunc novimus esse textum. cum Africam pro 
consule regeret Carthaginiensibus victus inopia iam lassatis, ex horreis Romano populo destinatis frumentum dedit, pauloque 
postea cum provenisset segetum copia, integre sine ulla restituit mora.


\section{De l'amplification}

Dein Syria per speciosam interpatet diffusa planitiem. hanc nobilitat Antiochia, mundo cognita civitas, cui non certaverit alia 
advecticiis ita adfluere copiis et internis, et Laodicia et Apamia itidemque Seleucia iam inde a primis auspiciis florentissimae.

Ipsam vero urbem Byzantiorum fuisse refertissimam atque ornatissimam signis quis ignorat? Quae illi, exhausti sumptibus bellisque 
maximis, cum omnis Mithridaticos impetus totumque Pontum armatum affervescentem in Asiam atque erumpentem, ore repulsum 
et cervicibus interclusum suis sustinerent, tum, inquam, Byzantii et postea signa illa et reliqua urbis ornanemta sanctissime 
custodita tenuerunt.

Qui cum venisset ob haec festinatis itineribus Antiochiam, praestrictis palatii ianuis, contempto Caesare, quem videri decuerat, ad 
praetorium cum pompa sollemni perrexit morbosque diu causatus nec regiam introiit nec processit in publicum, sed abditus multa in 
eius moliebatur exitium addens quaedam relationibus supervacua, quas subinde dimittebat ad principem.

Atque, ut Tullius ait, ut etiam ferae fame monitae plerumque ad eum locum ubi aliquando pastae sunt revertuntur, ita homines 
instar turbinis degressi montibus impeditis et arduis loca petivere mari confinia, per quae viis latebrosis sese convallibusque 
occultantes cum appeterent noctes luna etiam tum cornuta ideoque nondum solido splendore fulgente nauticos observabant quos 
cum in somnum sentirent effusos per ancoralia, quadrupedo gradu repentes seseque suspensis passibus iniectantes in scaphas 
eisdem sensim nihil opinantibus adsistebant et incendente aviditate saevitiam ne cedentium quidem ulli parcendo obtruncatis 
omnibus merces opimas velut viles nullis repugnantibus avertebant. haecque non diu sunt perpetrata.

Erat autem diritatis eius hoc quoque indicium nec obscurum nec latens, quod ludicris cruentis delectabatur et in circo sex vel 
septem aliquotiens vetitis certaminibus pugilum vicissim se concidentium perfusorumque sanguine specie ut lucratus ingentia 
laetabatur.

Superatis Tauri montis verticibus qui ad solis ortum sublimius attolluntur, Cilicia spatiis porrigitur late distentis dives bonis omnibus 
terra, eiusque lateri dextro adnexa Isauria, pari sorte uberi palmite viget et frugibus minutis, quam mediam navigabile flumen 
Calycadnus interscindit.

Quam quidem partem accusationis admiratus sum et moleste tuli potissimum esse Atratino datam. Neque enim decebat neque 
aetas illa postulabat neque, id quod animadvertere poteratis, pudor patiebatur optimi adulescentis in tali illum oratione versari. 
Vellem aliquis ex vobis robustioribus hunc male dicendi locum suscepisset; aliquanto liberius et fortius et magis more nostro 
refutaremus istam male dicendi licentiam. Tecum, Atratine, agam lenius, quod et pudor tuus moderatur orationi meae et meum 
erga te parentemque tuum beneficium tueri debeo.

Nemo quaeso miretur, si post exsudatos labores itinerum longos congestosque adfatim commeatus fiducia vestri ductante 
barbaricos pagos adventans velut mutato repente consilio ad placidiora deverti.

Ob haec et huius modi multa, quae cernebantur in paucis, omnibus timeri sunt coepta. et ne tot malis dissimulatis paulatimque 
serpentibus acervi crescerent aerumnarum, nobilitatis decreto legati mittuntur: Praetextatus ex urbi praefecto et ex vicario 
Venustus et ex consulari Minervius oraturi, ne delictis supplicia sint grandiora, neve senator quisquam inusitato et inlicito more 
tormentis exponeretur.

Eodem tempore etiam Hymetii praeclarae indolis viri negotium est actitatum, cuius hunc novimus esse textum. cum Africam pro 
consule regeret Carthaginiensibus victus inopia iam lassatis, ex horreis Romano populo destinatis frumentum dedit, pauloque 
postea cum provenisset segetum copia, integre sine ulla restituit mora.



\part{Le troisième livre}
\section{Servant de préface}

Dein Syria per speciosam interpatet diffusa planitiem. hanc nobilitat Antiochia, mundo cognita civitas, cui non certaverit alia 
advecticiis ita adfluere copiis et internis, et Laodicia et Apamia itidemque Seleucia iam inde a primis auspiciis florentissimae.

Ipsam vero urbem Byzantiorum fuisse refertissimam atque ornatissimam signis quis ignorat? Quae illi, exhausti sumptibus bellisque 
maximis, cum omnis Mithridaticos impetus totumque Pontum armatum affervescentem in Asiam atque erumpentem, ore repulsum 
et cervicibus interclusum suis sustinerent, tum, inquam, Byzantii et postea signa illa et reliqua urbis ornanemta sanctissime 
custodita tenuerunt.

Qui cum venisset ob haec festinatis itineribus Antiochiam, praestrictis palatii ianuis, contempto Caesare, quem videri decuerat, ad 
praetorium cum pompa sollemni perrexit morbosque diu causatus nec regiam introiit nec processit in publicum, sed abditus multa in 
eius moliebatur exitium addens quaedam relationibus supervacua, quas subinde dimittebat ad principem.

Atque, ut Tullius ait, ut etiam ferae fame monitae plerumque ad eum locum ubi aliquando pastae sunt revertuntur, ita homines 
instar turbinis degressi montibus impeditis et arduis loca petivere mari confinia, per quae viis latebrosis sese convallibusque 
occultantes cum appeterent noctes luna etiam tum cornuta ideoque nondum solido splendore fulgente nauticos observabant quos 
cum in somnum sentirent effusos per ancoralia, quadrupedo gradu repentes seseque suspensis passibus iniectantes in scaphas 
eisdem sensim nihil opinantibus adsistebant et incendente aviditate saevitiam ne cedentium quidem ulli parcendo obtruncatis 
omnibus merces opimas velut viles nullis repugnantibus avertebant. haecque non diu sunt perpetrata.

Erat autem diritatis eius hoc quoque indicium nec obscurum nec latens, quod ludicris cruentis delectabatur et in circo sex vel 
septem aliquotiens vetitis certaminibus pugilum vicissim se concidentium perfusorumque sanguine specie ut lucratus ingentia 
laetabatur.

Superatis Tauri montis verticibus qui ad solis ortum sublimius attolluntur, Cilicia spatiis porrigitur late distentis dives bonis omnibus 
terra, eiusque lateri dextro adnexa Isauria, pari sorte uberi palmite viget et frugibus minutis, quam mediam navigabile flumen 
Calycadnus interscindit.

Quam quidem partem accusationis admiratus sum et moleste tuli potissimum esse Atratino datam. Neque enim decebat neque 
aetas illa postulabat neque, id quod animadvertere poteratis, pudor patiebatur optimi adulescentis in tali illum oratione versari. 
Vellem aliquis ex vobis robustioribus hunc male dicendi locum suscepisset; aliquanto liberius et fortius et magis more nostro 
refutaremus istam male dicendi licentiam. Tecum, Atratine, agam lenius, quod et pudor tuus moderatur orationi meae et meum 
erga te parentemque tuum beneficium tueri debeo.

Nemo quaeso miretur, si post exsudatos labores itinerum longos congestosque adfatim commeatus fiducia vestri ductante 
barbaricos pagos adventans velut mutato repente consilio ad placidiora deverti.

Ob haec et huius modi multa, quae cernebantur in paucis, omnibus timeri sunt coepta. et ne tot malis dissimulatis paulatimque 
serpentibus acervi crescerent aerumnarum, nobilitatis decreto legati mittuntur: Praetextatus ex urbi praefecto et ex vicario 
Venustus et ex consulari Minervius oraturi, ne delictis supplicia sint grandiora, neve senator quisquam inusitato et inlicito more 
tormentis exponeretur.

Eodem tempore etiam Hymetii praeclarae indolis viri negotium est actitatum, cuius hunc novimus esse textum. cum Africam pro 
consule regeret Carthaginiensibus victus inopia iam lassatis, ex horreis Romano populo destinatis frumentum dedit, pauloque 
postea cum provenisset segetum copia, integre sine ulla restituit mora.


\chapter{La diction}
\section{De la belle élocution}

Dein Syria per speciosam interpatet diffusa planitiem. hanc nobilitat Antiochia, mundo cognita civitas, cui non certaverit alia 
advecticiis ita adfluere copiis et internis, et Laodicia et Apamia itidemque Seleucia iam inde a primis auspiciis florentissimae.

Ipsam vero urbem Byzantiorum fuisse refertissimam atque ornatissimam signis quis ignorat? Quae illi, exhausti sumptibus bellisque 
maximis, cum omnis Mithridaticos impetus totumque Pontum armatum affervescentem in Asiam atque erumpentem, ore repulsum 
et cervicibus interclusum suis sustinerent, tum, inquam, Byzantii et postea signa illa et reliqua urbis ornanemta sanctissime 
custodita tenuerunt.

Qui cum venisset ob haec festinatis itineribus Antiochiam, praestrictis palatii ianuis, contempto Caesare, quem videri decuerat, ad 
praetorium cum pompa sollemni perrexit morbosque diu causatus nec regiam introiit nec processit in publicum, sed abditus multa in 
eius moliebatur exitium addens quaedam relationibus supervacua, quas subinde dimittebat ad principem.

Atque, ut Tullius ait, ut etiam ferae fame monitae plerumque ad eum locum ubi aliquando pastae sunt revertuntur, ita homines 
instar turbinis degressi montibus impeditis et arduis loca petivere mari confinia, per quae viis latebrosis sese convallibusque 
occultantes cum appeterent noctes luna etiam tum cornuta ideoque nondum solido splendore fulgente nauticos observabant quos 
cum in somnum sentirent effusos per ancoralia, quadrupedo gradu repentes seseque suspensis passibus iniectantes in scaphas 
eisdem sensim nihil opinantibus adsistebant et incendente aviditate saevitiam ne cedentium quidem ulli parcendo obtruncatis 
omnibus merces opimas velut viles nullis repugnantibus avertebant. haecque non diu sunt perpetrata.

Erat autem diritatis eius hoc quoque indicium nec obscurum nec latens, quod ludicris cruentis delectabatur et in circo sex vel 
septem aliquotiens vetitis certaminibus pugilum vicissim se concidentium perfusorumque sanguine specie ut lucratus ingentia 
laetabatur.

Superatis Tauri montis verticibus qui ad solis ortum sublimius attolluntur, Cilicia spatiis porrigitur late distentis dives bonis omnibus 
terra, eiusque lateri dextro adnexa Isauria, pari sorte uberi palmite viget et frugibus minutis, quam mediam navigabile flumen 
Calycadnus interscindit.

Quam quidem partem accusationis admiratus sum et moleste tuli potissimum esse Atratino datam. Neque enim decebat neque 
aetas illa postulabat neque, id quod animadvertere poteratis, pudor patiebatur optimi adulescentis in tali illum oratione versari. 
Vellem aliquis ex vobis robustioribus hunc male dicendi locum suscepisset; aliquanto liberius et fortius et magis more nostro 
refutaremus istam male dicendi licentiam. Tecum, Atratine, agam lenius, quod et pudor tuus moderatur orationi meae et meum 
erga te parentemque tuum beneficium tueri debeo.

Nemo quaeso miretur, si post exsudatos labores itinerum longos congestosque adfatim commeatus fiducia vestri ductante 
barbaricos pagos adventans velut mutato repente consilio ad placidiora deverti.

Ob haec et huius modi multa, quae cernebantur in paucis, omnibus timeri sunt coepta. et ne tot malis dissimulatis paulatimque 
serpentibus acervi crescerent aerumnarum, nobilitatis decreto legati mittuntur: Praetextatus ex urbi praefecto et ex vicario 
Venustus et ex consulari Minervius oraturi, ne delictis supplicia sint grandiora, neve senator quisquam inusitato et inlicito more 
tormentis exponeretur.

Eodem tempore etiam Hymetii praeclarae indolis viri negotium est actitatum, cuius hunc novimus esse textum. cum Africam pro 
consule regeret Carthaginiensibus victus inopia iam lassatis, ex horreis Romano populo destinatis frumentum dedit, pauloque 
postea cum provenisset segetum copia, integre sine ulla restituit mora.


\section{De l'élocution froide}

Dein Syria per speciosam interpatet diffusa planitiem. hanc nobilitat Antiochia, mundo cognita civitas, cui non certaverit alia 
advecticiis ita adfluere copiis et internis, et Laodicia et Apamia itidemque Seleucia iam inde a primis auspiciis florentissimae.

Ipsam vero urbem Byzantiorum fuisse refertissimam atque ornatissimam signis quis ignorat? Quae illi, exhausti sumptibus bellisque 
maximis, cum omnis Mithridaticos impetus totumque Pontum armatum affervescentem in Asiam atque erumpentem, ore repulsum 
et cervicibus interclusum suis sustinerent, tum, inquam, Byzantii et postea signa illa et reliqua urbis ornanemta sanctissime 
custodita tenuerunt.

Qui cum venisset ob haec festinatis itineribus Antiochiam, praestrictis palatii ianuis, contempto Caesare, quem videri decuerat, ad 
praetorium cum pompa sollemni perrexit morbosque diu causatus nec regiam introiit nec processit in publicum, sed abditus multa in 
eius moliebatur exitium addens quaedam relationibus supervacua, quas subinde dimittebat ad principem.

Atque, ut Tullius ait, ut etiam ferae fame monitae plerumque ad eum locum ubi aliquando pastae sunt revertuntur, ita homines 
instar turbinis degressi montibus impeditis et arduis loca petivere mari confinia, per quae viis latebrosis sese convallibusque 
occultantes cum appeterent noctes luna etiam tum cornuta ideoque nondum solido splendore fulgente nauticos observabant quos 
cum in somnum sentirent effusos per ancoralia, quadrupedo gradu repentes seseque suspensis passibus iniectantes in scaphas 
eisdem sensim nihil opinantibus adsistebant et incendente aviditate saevitiam ne cedentium quidem ulli parcendo obtruncatis 
omnibus merces opimas velut viles nullis repugnantibus avertebant. haecque non diu sunt perpetrata.

Erat autem diritatis eius hoc quoque indicium nec obscurum nec latens, quod ludicris cruentis delectabatur et in circo sex vel 
septem aliquotiens vetitis certaminibus pugilum vicissim se concidentium perfusorumque sanguine specie ut lucratus ingentia 
laetabatur.

Superatis Tauri montis verticibus qui ad solis ortum sublimius attolluntur, Cilicia spatiis porrigitur late distentis dives bonis omnibus 
terra, eiusque lateri dextro adnexa Isauria, pari sorte uberi palmite viget et frugibus minutis, quam mediam navigabile flumen 
Calycadnus interscindit.

Quam quidem partem accusationis admiratus sum et moleste tuli potissimum esse Atratino datam. Neque enim decebat neque 
aetas illa postulabat neque, id quod animadvertere poteratis, pudor patiebatur optimi adulescentis in tali illum oratione versari. 
Vellem aliquis ex vobis robustioribus hunc male dicendi locum suscepisset; aliquanto liberius et fortius et magis more nostro 
refutaremus istam male dicendi licentiam. Tecum, Atratine, agam lenius, quod et pudor tuus moderatur orationi meae et meum 
erga te parentemque tuum beneficium tueri debeo.

Nemo quaeso miretur, si post exsudatos labores itinerum longos congestosque adfatim commeatus fiducia vestri ductante 
barbaricos pagos adventans velut mutato repente consilio ad placidiora deverti.

Ob haec et huius modi multa, quae cernebantur in paucis, omnibus timeri sunt coepta. et ne tot malis dissimulatis paulatimque 
serpentibus acervi crescerent aerumnarum, nobilitatis decreto legati mittuntur: Praetextatus ex urbi praefecto et ex vicario 
Venustus et ex consulari Minervius oraturi, ne delictis supplicia sint grandiora, neve senator quisquam inusitato et inlicito more 
tormentis exponeretur.

Eodem tempore etiam Hymetii praeclarae indolis viri negotium est actitatum, cuius hunc novimus esse textum. cum Africam pro 
consule regeret Carthaginiensibus victus inopia iam lassatis, ex horreis Romano populo destinatis frumentum dedit, pauloque 
postea cum provenisset segetum copia, integre sine ulla restituit mora.


\section{De l'image}

Dein Syria per speciosam interpatet diffusa planitiem. hanc nobilitat Antiochia, mundo cognita civitas, cui non certaverit alia 
advecticiis ita adfluere copiis et internis, et Laodicia et Apamia itidemque Seleucia iam inde a primis auspiciis florentissimae.

Ipsam vero urbem Byzantiorum fuisse refertissimam atque ornatissimam signis quis ignorat? Quae illi, exhausti sumptibus bellisque 
maximis, cum omnis Mithridaticos impetus totumque Pontum armatum affervescentem in Asiam atque erumpentem, ore repulsum 
et cervicibus interclusum suis sustinerent, tum, inquam, Byzantii et postea signa illa et reliqua urbis ornanemta sanctissime 
custodita tenuerunt.

Qui cum venisset ob haec festinatis itineribus Antiochiam, praestrictis palatii ianuis, contempto Caesare, quem videri decuerat, ad 
praetorium cum pompa sollemni perrexit morbosque diu causatus nec regiam introiit nec processit in publicum, sed abditus multa in 
eius moliebatur exitium addens quaedam relationibus supervacua, quas subinde dimittebat ad principem.

Atque, ut Tullius ait, ut etiam ferae fame monitae plerumque ad eum locum ubi aliquando pastae sunt revertuntur, ita homines 
instar turbinis degressi montibus impeditis et arduis loca petivere mari confinia, per quae viis latebrosis sese convallibusque 
occultantes cum appeterent noctes luna etiam tum cornuta ideoque nondum solido splendore fulgente nauticos observabant quos 
cum in somnum sentirent effusos per ancoralia, quadrupedo gradu repentes seseque suspensis passibus iniectantes in scaphas 
eisdem sensim nihil opinantibus adsistebant et incendente aviditate saevitiam ne cedentium quidem ulli parcendo obtruncatis 
omnibus merces opimas velut viles nullis repugnantibus avertebant. haecque non diu sunt perpetrata.

Erat autem diritatis eius hoc quoque indicium nec obscurum nec latens, quod ludicris cruentis delectabatur et in circo sex vel 
septem aliquotiens vetitis certaminibus pugilum vicissim se concidentium perfusorumque sanguine specie ut lucratus ingentia 
laetabatur.

Superatis Tauri montis verticibus qui ad solis ortum sublimius attolluntur, Cilicia spatiis porrigitur late distentis dives bonis omnibus 
terra, eiusque lateri dextro adnexa Isauria, pari sorte uberi palmite viget et frugibus minutis, quam mediam navigabile flumen 
Calycadnus interscindit.

Quam quidem partem accusationis admiratus sum et moleste tuli potissimum esse Atratino datam. Neque enim decebat neque 
aetas illa postulabat neque, id quod animadvertere poteratis, pudor patiebatur optimi adulescentis in tali illum oratione versari. 
Vellem aliquis ex vobis robustioribus hunc male dicendi locum suscepisset; aliquanto liberius et fortius et magis more nostro 
refutaremus istam male dicendi licentiam. Tecum, Atratine, agam lenius, quod et pudor tuus moderatur orationi meae et meum 
erga te parentemque tuum beneficium tueri debeo.

Nemo quaeso miretur, si post exsudatos labores itinerum longos congestosque adfatim commeatus fiducia vestri ductante 
barbaricos pagos adventans velut mutato repente consilio ad placidiora deverti.

Ob haec et huius modi multa, quae cernebantur in paucis, omnibus timeri sunt coepta. et ne tot malis dissimulatis paulatimque 
serpentibus acervi crescerent aerumnarum, nobilitatis decreto legati mittuntur: Praetextatus ex urbi praefecto et ex vicario 
Venustus et ex consulari Minervius oraturi, ne delictis supplicia sint grandiora, neve senator quisquam inusitato et inlicito more 
tormentis exponeretur.

Eodem tempore etiam Hymetii praeclarae indolis viri negotium est actitatum, cuius hunc novimus esse textum. cum Africam pro 
consule regeret Carthaginiensibus victus inopia iam lassatis, ex horreis Romano populo destinatis frumentum dedit, pauloque 
postea cum provenisset segetum copia, integre sine ulla restituit mora.


\section{De la pureté de l'élocution}

Dein Syria per speciosam interpatet diffusa planitiem. hanc nobilitat Antiochia, mundo cognita civitas, cui non certaverit alia 
advecticiis ita adfluere copiis et internis, et Laodicia et Apamia itidemque Seleucia iam inde a primis auspiciis florentissimae.

Ipsam vero urbem Byzantiorum fuisse refertissimam atque ornatissimam signis quis ignorat? Quae illi, exhausti sumptibus bellisque 
maximis, cum omnis Mithridaticos impetus totumque Pontum armatum affervescentem in Asiam atque erumpentem, ore repulsum 
et cervicibus interclusum suis sustinerent, tum, inquam, Byzantii et postea signa illa et reliqua urbis ornanemta sanctissime 
custodita tenuerunt.

Qui cum venisset ob haec festinatis itineribus Antiochiam, praestrictis palatii ianuis, contempto Caesare, quem videri decuerat, ad 
praetorium cum pompa sollemni perrexit morbosque diu causatus nec regiam introiit nec processit in publicum, sed abditus multa in 
eius moliebatur exitium addens quaedam relationibus supervacua, quas subinde dimittebat ad principem.

Atque, ut Tullius ait, ut etiam ferae fame monitae plerumque ad eum locum ubi aliquando pastae sunt revertuntur, ita homines 
instar turbinis degressi montibus impeditis et arduis loca petivere mari confinia, per quae viis latebrosis sese convallibusque 
occultantes cum appeterent noctes luna etiam tum cornuta ideoque nondum solido splendore fulgente nauticos observabant quos 
cum in somnum sentirent effusos per ancoralia, quadrupedo gradu repentes seseque suspensis passibus iniectantes in scaphas 
eisdem sensim nihil opinantibus adsistebant et incendente aviditate saevitiam ne cedentium quidem ulli parcendo obtruncatis 
omnibus merces opimas velut viles nullis repugnantibus avertebant. haecque non diu sunt perpetrata.

Erat autem diritatis eius hoc quoque indicium nec obscurum nec latens, quod ludicris cruentis delectabatur et in circo sex vel 
septem aliquotiens vetitis certaminibus pugilum vicissim se concidentium perfusorumque sanguine specie ut lucratus ingentia 
laetabatur.

Superatis Tauri montis verticibus qui ad solis ortum sublimius attolluntur, Cilicia spatiis porrigitur late distentis dives bonis omnibus 
terra, eiusque lateri dextro adnexa Isauria, pari sorte uberi palmite viget et frugibus minutis, quam mediam navigabile flumen 
Calycadnus interscindit.

Quam quidem partem accusationis admiratus sum et moleste tuli potissimum esse Atratino datam. Neque enim decebat neque 
aetas illa postulabat neque, id quod animadvertere poteratis, pudor patiebatur optimi adulescentis in tali illum oratione versari. 
Vellem aliquis ex vobis robustioribus hunc male dicendi locum suscepisset; aliquanto liberius et fortius et magis more nostro 
refutaremus istam male dicendi licentiam. Tecum, Atratine, agam lenius, quod et pudor tuus moderatur orationi meae et meum 
erga te parentemque tuum beneficium tueri debeo.

Nemo quaeso miretur, si post exsudatos labores itinerum longos congestosque adfatim commeatus fiducia vestri ductante 
barbaricos pagos adventans velut mutato repente consilio ad placidiora deverti.

Ob haec et huius modi multa, quae cernebantur in paucis, omnibus timeri sunt coepta. et ne tot malis dissimulatis paulatimque 
serpentibus acervi crescerent aerumnarum, nobilitatis decreto legati mittuntur: Praetextatus ex urbi praefecto et ex vicario 
Venustus et ex consulari Minervius oraturi, ne delictis supplicia sint grandiora, neve senator quisquam inusitato et inlicito more 
tormentis exponeretur.

Eodem tempore etiam Hymetii praeclarae indolis viri negotium est actitatum, cuius hunc novimus esse textum. cum Africam pro 
consule regeret Carthaginiensibus victus inopia iam lassatis, ex horreis Romano populo destinatis frumentum dedit, pauloque 
postea cum provenisset segetum copia, integre sine ulla restituit mora.


\section{De l'enflure}

Dein Syria per speciosam interpatet diffusa planitiem. hanc nobilitat Antiochia, mundo cognita civitas, cui non certaverit alia 
advecticiis ita adfluere copiis et internis, et Laodicia et Apamia itidemque Seleucia iam inde a primis auspiciis florentissimae.

Ipsam vero urbem Byzantiorum fuisse refertissimam atque ornatissimam signis quis ignorat? Quae illi, exhausti sumptibus bellisque 
maximis, cum omnis Mithridaticos impetus totumque Pontum armatum affervescentem in Asiam atque erumpentem, ore repulsum 
et cervicibus interclusum suis sustinerent, tum, inquam, Byzantii et postea signa illa et reliqua urbis ornanemta sanctissime 
custodita tenuerunt.

Qui cum venisset ob haec festinatis itineribus Antiochiam, praestrictis palatii ianuis, contempto Caesare, quem videri decuerat, ad 
praetorium cum pompa sollemni perrexit morbosque diu causatus nec regiam introiit nec processit in publicum, sed abditus multa in 
eius moliebatur exitium addens quaedam relationibus supervacua, quas subinde dimittebat ad principem.

Atque, ut Tullius ait, ut etiam ferae fame monitae plerumque ad eum locum ubi aliquando pastae sunt revertuntur, ita homines 
instar turbinis degressi montibus impeditis et arduis loca petivere mari confinia, per quae viis latebrosis sese convallibusque 
occultantes cum appeterent noctes luna etiam tum cornuta ideoque nondum solido splendore fulgente nauticos observabant quos 
cum in somnum sentirent effusos per ancoralia, quadrupedo gradu repentes seseque suspensis passibus iniectantes in scaphas 
eisdem sensim nihil opinantibus adsistebant et incendente aviditate saevitiam ne cedentium quidem ulli parcendo obtruncatis 
omnibus merces opimas velut viles nullis repugnantibus avertebant. haecque non diu sunt perpetrata.

Erat autem diritatis eius hoc quoque indicium nec obscurum nec latens, quod ludicris cruentis delectabatur et in circo sex vel 
septem aliquotiens vetitis certaminibus pugilum vicissim se concidentium perfusorumque sanguine specie ut lucratus ingentia 
laetabatur.

Superatis Tauri montis verticibus qui ad solis ortum sublimius attolluntur, Cilicia spatiis porrigitur late distentis dives bonis omnibus 
terra, eiusque lateri dextro adnexa Isauria, pari sorte uberi palmite viget et frugibus minutis, quam mediam navigabile flumen 
Calycadnus interscindit.

Quam quidem partem accusationis admiratus sum et moleste tuli potissimum esse Atratino datam. Neque enim decebat neque 
aetas illa postulabat neque, id quod animadvertere poteratis, pudor patiebatur optimi adulescentis in tali illum oratione versari. 
Vellem aliquis ex vobis robustioribus hunc male dicendi locum suscepisset; aliquanto liberius et fortius et magis more nostro 
refutaremus istam male dicendi licentiam. Tecum, Atratine, agam lenius, quod et pudor tuus moderatur orationi meae et meum 
erga te parentemque tuum beneficium tueri debeo.

Nemo quaeso miretur, si post exsudatos labores itinerum longos congestosque adfatim commeatus fiducia vestri ductante 
barbaricos pagos adventans velut mutato repente consilio ad placidiora deverti.

Ob haec et huius modi multa, quae cernebantur in paucis, omnibus timeri sunt coepta. et ne tot malis dissimulatis paulatimque 
serpentibus acervi crescerent aerumnarum, nobilitatis decreto legati mittuntur: Praetextatus ex urbi praefecto et ex vicario 
Venustus et ex consulari Minervius oraturi, ne delictis supplicia sint grandiora, neve senator quisquam inusitato et inlicito more 
tormentis exponeretur.

Eodem tempore etiam Hymetii praeclarae indolis viri negotium est actitatum, cuius hunc novimus esse textum. cum Africam pro 
consule regeret Carthaginiensibus victus inopia iam lassatis, ex horreis Romano populo destinatis frumentum dedit, pauloque 
postea cum provenisset segetum copia, integre sine ulla restituit mora.


\section{De la diction propre au sujet}

Dein Syria per speciosam interpatet diffusa planitiem. hanc nobilitat Antiochia, mundo cognita civitas, cui non certaverit alia 
advecticiis ita adfluere copiis et internis, et Laodicia et Apamia itidemque Seleucia iam inde a primis auspiciis florentissimae.

Ipsam vero urbem Byzantiorum fuisse refertissimam atque ornatissimam signis quis ignorat? Quae illi, exhausti sumptibus bellisque 
maximis, cum omnis Mithridaticos impetus totumque Pontum armatum affervescentem in Asiam atque erumpentem, ore repulsum 
et cervicibus interclusum suis sustinerent, tum, inquam, Byzantii et postea signa illa et reliqua urbis ornanemta sanctissime 
custodita tenuerunt.

Qui cum venisset ob haec festinatis itineribus Antiochiam, praestrictis palatii ianuis, contempto Caesare, quem videri decuerat, ad 
praetorium cum pompa sollemni perrexit morbosque diu causatus nec regiam introiit nec processit in publicum, sed abditus multa in 
eius moliebatur exitium addens quaedam relationibus supervacua, quas subinde dimittebat ad principem.

Atque, ut Tullius ait, ut etiam ferae fame monitae plerumque ad eum locum ubi aliquando pastae sunt revertuntur, ita homines 
instar turbinis degressi montibus impeditis et arduis loca petivere mari confinia, per quae viis latebrosis sese convallibusque 
occultantes cum appeterent noctes luna etiam tum cornuta ideoque nondum solido splendore fulgente nauticos observabant quos 
cum in somnum sentirent effusos per ancoralia, quadrupedo gradu repentes seseque suspensis passibus iniectantes in scaphas 
eisdem sensim nihil opinantibus adsistebant et incendente aviditate saevitiam ne cedentium quidem ulli parcendo obtruncatis 
omnibus merces opimas velut viles nullis repugnantibus avertebant. haecque non diu sunt perpetrata.

Erat autem diritatis eius hoc quoque indicium nec obscurum nec latens, quod ludicris cruentis delectabatur et in circo sex vel 
septem aliquotiens vetitis certaminibus pugilum vicissim se concidentium perfusorumque sanguine specie ut lucratus ingentia 
laetabatur.

Superatis Tauri montis verticibus qui ad solis ortum sublimius attolluntur, Cilicia spatiis porrigitur late distentis dives bonis omnibus 
terra, eiusque lateri dextro adnexa Isauria, pari sorte uberi palmite viget et frugibus minutis, quam mediam navigabile flumen 
Calycadnus interscindit.

Quam quidem partem accusationis admiratus sum et moleste tuli potissimum esse Atratino datam. Neque enim decebat neque 
aetas illa postulabat neque, id quod animadvertere poteratis, pudor patiebatur optimi adulescentis in tali illum oratione versari. 
Vellem aliquis ex vobis robustioribus hunc male dicendi locum suscepisset; aliquanto liberius et fortius et magis more nostro 
refutaremus istam male dicendi licentiam. Tecum, Atratine, agam lenius, quod et pudor tuus moderatur orationi meae et meum 
erga te parentemque tuum beneficium tueri debeo.

Nemo quaeso miretur, si post exsudatos labores itinerum longos congestosque adfatim commeatus fiducia vestri ductante 
barbaricos pagos adventans velut mutato repente consilio ad placidiora deverti.

Ob haec et huius modi multa, quae cernebantur in paucis, omnibus timeri sunt coepta. et ne tot malis dissimulatis paulatimque 
serpentibus acervi crescerent aerumnarum, nobilitatis decreto legati mittuntur: Praetextatus ex urbi praefecto et ex vicario 
Venustus et ex consulari Minervius oraturi, ne delictis supplicia sint grandiora, neve senator quisquam inusitato et inlicito more 
tormentis exponeretur.

Eodem tempore etiam Hymetii praeclarae indolis viri negotium est actitatum, cuius hunc novimus esse textum. cum Africam pro 
consule regeret Carthaginiensibus victus inopia iam lassatis, ex horreis Romano populo destinatis frumentum dedit, pauloque 
postea cum provenisset segetum copia, integre sine ulla restituit mora.


\section{Du nombre}

Dein Syria per speciosam interpatet diffusa planitiem. hanc nobilitat Antiochia, mundo cognita civitas, cui non certaverit alia 
advecticiis ita adfluere copiis et internis, et Laodicia et Apamia itidemque Seleucia iam inde a primis auspiciis florentissimae.

Ipsam vero urbem Byzantiorum fuisse refertissimam atque ornatissimam signis quis ignorat? Quae illi, exhausti sumptibus bellisque 
maximis, cum omnis Mithridaticos impetus totumque Pontum armatum affervescentem in Asiam atque erumpentem, ore repulsum 
et cervicibus interclusum suis sustinerent, tum, inquam, Byzantii et postea signa illa et reliqua urbis ornanemta sanctissime 
custodita tenuerunt.

Qui cum venisset ob haec festinatis itineribus Antiochiam, praestrictis palatii ianuis, contempto Caesare, quem videri decuerat, ad 
praetorium cum pompa sollemni perrexit morbosque diu causatus nec regiam introiit nec processit in publicum, sed abditus multa in 
eius moliebatur exitium addens quaedam relationibus supervacua, quas subinde dimittebat ad principem.

Atque, ut Tullius ait, ut etiam ferae fame monitae plerumque ad eum locum ubi aliquando pastae sunt revertuntur, ita homines 
instar turbinis degressi montibus impeditis et arduis loca petivere mari confinia, per quae viis latebrosis sese convallibusque 
occultantes cum appeterent noctes luna etiam tum cornuta ideoque nondum solido splendore fulgente nauticos observabant quos 
cum in somnum sentirent effusos per ancoralia, quadrupedo gradu repentes seseque suspensis passibus iniectantes in scaphas 
eisdem sensim nihil opinantibus adsistebant et incendente aviditate saevitiam ne cedentium quidem ulli parcendo obtruncatis 
omnibus merces opimas velut viles nullis repugnantibus avertebant. haecque non diu sunt perpetrata.

Erat autem diritatis eius hoc quoque indicium nec obscurum nec latens, quod ludicris cruentis delectabatur et in circo sex vel 
septem aliquotiens vetitis certaminibus pugilum vicissim se concidentium perfusorumque sanguine specie ut lucratus ingentia 
laetabatur.

Superatis Tauri montis verticibus qui ad solis ortum sublimius attolluntur, Cilicia spatiis porrigitur late distentis dives bonis omnibus 
terra, eiusque lateri dextro adnexa Isauria, pari sorte uberi palmite viget et frugibus minutis, quam mediam navigabile flumen 
Calycadnus interscindit.

Quam quidem partem accusationis admiratus sum et moleste tuli potissimum esse Atratino datam. Neque enim decebat neque 
aetas illa postulabat neque, id quod animadvertere poteratis, pudor patiebatur optimi adulescentis in tali illum oratione versari. 
Vellem aliquis ex vobis robustioribus hunc male dicendi locum suscepisset; aliquanto liberius et fortius et magis more nostro 
refutaremus istam male dicendi licentiam. Tecum, Atratine, agam lenius, quod et pudor tuus moderatur orationi meae et meum 
erga te parentemque tuum beneficium tueri debeo.

Nemo quaeso miretur, si post exsudatos labores itinerum longos congestosque adfatim commeatus fiducia vestri ductante 
barbaricos pagos adventans velut mutato repente consilio ad placidiora deverti.

Ob haec et huius modi multa, quae cernebantur in paucis, omnibus timeri sunt coepta. et ne tot malis dissimulatis paulatimque 
serpentibus acervi crescerent aerumnarum, nobilitatis decreto legati mittuntur: Praetextatus ex urbi praefecto et ex vicario 
Venustus et ex consulari Minervius oraturi, ne delictis supplicia sint grandiora, neve senator quisquam inusitato et inlicito more 
tormentis exponeretur.

Eodem tempore etiam Hymetii praeclarae indolis viri negotium est actitatum, cuius hunc novimus esse textum. cum Africam pro 
consule regeret Carthaginiensibus victus inopia iam lassatis, ex horreis Romano populo destinatis frumentum dedit, pauloque 
postea cum provenisset segetum copia, integre sine ulla restituit mora.


\section{Qu'il y a deux sortes d'élocution}

Dein Syria per speciosam interpatet diffusa planitiem. hanc nobilitat Antiochia, mundo cognita civitas, cui non certaverit alia 
advecticiis ita adfluere copiis et internis, et Laodicia et Apamia itidemque Seleucia iam inde a primis auspiciis florentissimae.

Ipsam vero urbem Byzantiorum fuisse refertissimam atque ornatissimam signis quis ignorat? Quae illi, exhausti sumptibus bellisque 
maximis, cum omnis Mithridaticos impetus totumque Pontum armatum affervescentem in Asiam atque erumpentem, ore repulsum 
et cervicibus interclusum suis sustinerent, tum, inquam, Byzantii et postea signa illa et reliqua urbis ornanemta sanctissime 
custodita tenuerunt.

Qui cum venisset ob haec festinatis itineribus Antiochiam, praestrictis palatii ianuis, contempto Caesare, quem videri decuerat, ad 
praetorium cum pompa sollemni perrexit morbosque diu causatus nec regiam introiit nec processit in publicum, sed abditus multa in 
eius moliebatur exitium addens quaedam relationibus supervacua, quas subinde dimittebat ad principem.

Atque, ut Tullius ait, ut etiam ferae fame monitae plerumque ad eum locum ubi aliquando pastae sunt revertuntur, ita homines 
instar turbinis degressi montibus impeditis et arduis loca petivere mari confinia, per quae viis latebrosis sese convallibusque 
occultantes cum appeterent noctes luna etiam tum cornuta ideoque nondum solido splendore fulgente nauticos observabant quos 
cum in somnum sentirent effusos per ancoralia, quadrupedo gradu repentes seseque suspensis passibus iniectantes in scaphas 
eisdem sensim nihil opinantibus adsistebant et incendente aviditate saevitiam ne cedentium quidem ulli parcendo obtruncatis 
omnibus merces opimas velut viles nullis repugnantibus avertebant. haecque non diu sunt perpetrata.

Erat autem diritatis eius hoc quoque indicium nec obscurum nec latens, quod ludicris cruentis delectabatur et in circo sex vel 
septem aliquotiens vetitis certaminibus pugilum vicissim se concidentium perfusorumque sanguine specie ut lucratus ingentia 
laetabatur.

Superatis Tauri montis verticibus qui ad solis ortum sublimius attolluntur, Cilicia spatiis porrigitur late distentis dives bonis omnibus 
terra, eiusque lateri dextro adnexa Isauria, pari sorte uberi palmite viget et frugibus minutis, quam mediam navigabile flumen 
Calycadnus interscindit.

Quam quidem partem accusationis admiratus sum et moleste tuli potissimum esse Atratino datam. Neque enim decebat neque 
aetas illa postulabat neque, id quod animadvertere poteratis, pudor patiebatur optimi adulescentis in tali illum oratione versari. 
Vellem aliquis ex vobis robustioribus hunc male dicendi locum suscepisset; aliquanto liberius et fortius et magis more nostro 
refutaremus istam male dicendi licentiam. Tecum, Atratine, agam lenius, quod et pudor tuus moderatur orationi meae et meum 
erga te parentemque tuum beneficium tueri debeo.

Nemo quaeso miretur, si post exsudatos labores itinerum longos congestosque adfatim commeatus fiducia vestri ductante 
barbaricos pagos adventans velut mutato repente consilio ad placidiora deverti.

Ob haec et huius modi multa, quae cernebantur in paucis, omnibus timeri sunt coepta. et ne tot malis dissimulatis paulatimque 
serpentibus acervi crescerent aerumnarum, nobilitatis decreto legati mittuntur: Praetextatus ex urbi praefecto et ex vicario 
Venustus et ex consulari Minervius oraturi, ne delictis supplicia sint grandiora, neve senator quisquam inusitato et inlicito more 
tormentis exponeretur.

Eodem tempore etiam Hymetii praeclarae indolis viri negotium est actitatum, cuius hunc novimus esse textum. cum Africam pro 
consule regeret Carthaginiensibus victus inopia iam lassatis, ex horreis Romano populo destinatis frumentum dedit, pauloque 
postea cum provenisset segetum copia, integre sine ulla restituit mora.


\section{Pour dire les choses spirituellement}

Dein Syria per speciosam interpatet diffusa planitiem. hanc nobilitat Antiochia, mundo cognita civitas, cui non certaverit alia 
advecticiis ita adfluere copiis et internis, et Laodicia et Apamia itidemque Seleucia iam inde a primis auspiciis florentissimae.

Ipsam vero urbem Byzantiorum fuisse refertissimam atque ornatissimam signis quis ignorat? Quae illi, exhausti sumptibus bellisque 
maximis, cum omnis Mithridaticos impetus totumque Pontum armatum affervescentem in Asiam atque erumpentem, ore repulsum 
et cervicibus interclusum suis sustinerent, tum, inquam, Byzantii et postea signa illa et reliqua urbis ornanemta sanctissime 
custodita tenuerunt.

Qui cum venisset ob haec festinatis itineribus Antiochiam, praestrictis palatii ianuis, contempto Caesare, quem videri decuerat, ad 
praetorium cum pompa sollemni perrexit morbosque diu causatus nec regiam introiit nec processit in publicum, sed abditus multa in 
eius moliebatur exitium addens quaedam relationibus supervacua, quas subinde dimittebat ad principem.

Atque, ut Tullius ait, ut etiam ferae fame monitae plerumque ad eum locum ubi aliquando pastae sunt revertuntur, ita homines 
instar turbinis degressi montibus impeditis et arduis loca petivere mari confinia, per quae viis latebrosis sese convallibusque 
occultantes cum appeterent noctes luna etiam tum cornuta ideoque nondum solido splendore fulgente nauticos observabant quos 
cum in somnum sentirent effusos per ancoralia, quadrupedo gradu repentes seseque suspensis passibus iniectantes in scaphas 
eisdem sensim nihil opinantibus adsistebant et incendente aviditate saevitiam ne cedentium quidem ulli parcendo obtruncatis 
omnibus merces opimas velut viles nullis repugnantibus avertebant. haecque non diu sunt perpetrata.

Erat autem diritatis eius hoc quoque indicium nec obscurum nec latens, quod ludicris cruentis delectabatur et in circo sex vel 
septem aliquotiens vetitis certaminibus pugilum vicissim se concidentium perfusorumque sanguine specie ut lucratus ingentia 
laetabatur.

Superatis Tauri montis verticibus qui ad solis ortum sublimius attolluntur, Cilicia spatiis porrigitur late distentis dives bonis omnibus 
terra, eiusque lateri dextro adnexa Isauria, pari sorte uberi palmite viget et frugibus minutis, quam mediam navigabile flumen 
Calycadnus interscindit.

Quam quidem partem accusationis admiratus sum et moleste tuli potissimum esse Atratino datam. Neque enim decebat neque 
aetas illa postulabat neque, id quod animadvertere poteratis, pudor patiebatur optimi adulescentis in tali illum oratione versari. 
Vellem aliquis ex vobis robustioribus hunc male dicendi locum suscepisset; aliquanto liberius et fortius et magis more nostro 
refutaremus istam male dicendi licentiam. Tecum, Atratine, agam lenius, quod et pudor tuus moderatur orationi meae et meum 
erga te parentemque tuum beneficium tueri debeo.

Nemo quaeso miretur, si post exsudatos labores itinerum longos congestosque adfatim commeatus fiducia vestri ductante 
barbaricos pagos adventans velut mutato repente consilio ad placidiora deverti.

Ob haec et huius modi multa, quae cernebantur in paucis, omnibus timeri sunt coepta. et ne tot malis dissimulatis paulatimque 
serpentibus acervi crescerent aerumnarum, nobilitatis decreto legati mittuntur: Praetextatus ex urbi praefecto et ex vicario 
Venustus et ex consulari Minervius oraturi, ne delictis supplicia sint grandiora, neve senator quisquam inusitato et inlicito more 
tormentis exponeretur.

Eodem tempore etiam Hymetii praeclarae indolis viri negotium est actitatum, cuius hunc novimus esse textum. cum Africam pro 
consule regeret Carthaginiensibus victus inopia iam lassatis, ex horreis Romano populo destinatis frumentum dedit, pauloque 
postea cum provenisset segetum copia, integre sine ulla restituit mora.


\section{Ce que c'est qu'énergie, et mettre une chose devant les yeux}

Dein Syria per speciosam interpatet diffusa planitiem. hanc nobilitat Antiochia, mundo cognita civitas, cui non certaverit alia 
advecticiis ita adfluere copiis et internis, et Laodicia et Apamia itidemque Seleucia iam inde a primis auspiciis florentissimae.

Ipsam vero urbem Byzantiorum fuisse refertissimam atque ornatissimam signis quis ignorat? Quae illi, exhausti sumptibus bellisque 
maximis, cum omnis Mithridaticos impetus totumque Pontum armatum affervescentem in Asiam atque erumpentem, ore repulsum 
et cervicibus interclusum suis sustinerent, tum, inquam, Byzantii et postea signa illa et reliqua urbis ornanemta sanctissime 
custodita tenuerunt.

Qui cum venisset ob haec festinatis itineribus Antiochiam, praestrictis palatii ianuis, contempto Caesare, quem videri decuerat, ad 
praetorium cum pompa sollemni perrexit morbosque diu causatus nec regiam introiit nec processit in publicum, sed abditus multa in 
eius moliebatur exitium addens quaedam relationibus supervacua, quas subinde dimittebat ad principem.

Atque, ut Tullius ait, ut etiam ferae fame monitae plerumque ad eum locum ubi aliquando pastae sunt revertuntur, ita homines 
instar turbinis degressi montibus impeditis et arduis loca petivere mari confinia, per quae viis latebrosis sese convallibusque 
occultantes cum appeterent noctes luna etiam tum cornuta ideoque nondum solido splendore fulgente nauticos observabant quos 
cum in somnum sentirent effusos per ancoralia, quadrupedo gradu repentes seseque suspensis passibus iniectantes in scaphas 
eisdem sensim nihil opinantibus adsistebant et incendente aviditate saevitiam ne cedentium quidem ulli parcendo obtruncatis 
omnibus merces opimas velut viles nullis repugnantibus avertebant. haecque non diu sunt perpetrata.

Erat autem diritatis eius hoc quoque indicium nec obscurum nec latens, quod ludicris cruentis delectabatur et in circo sex vel 
septem aliquotiens vetitis certaminibus pugilum vicissim se concidentium perfusorumque sanguine specie ut lucratus ingentia 
laetabatur.

Superatis Tauri montis verticibus qui ad solis ortum sublimius attolluntur, Cilicia spatiis porrigitur late distentis dives bonis omnibus 
terra, eiusque lateri dextro adnexa Isauria, pari sorte uberi palmite viget et frugibus minutis, quam mediam navigabile flumen 
Calycadnus interscindit.

Quam quidem partem accusationis admiratus sum et moleste tuli potissimum esse Atratino datam. Neque enim decebat neque 
aetas illa postulabat neque, id quod animadvertere poteratis, pudor patiebatur optimi adulescentis in tali illum oratione versari. 
Vellem aliquis ex vobis robustioribus hunc male dicendi locum suscepisset; aliquanto liberius et fortius et magis more nostro 
refutaremus istam male dicendi licentiam. Tecum, Atratine, agam lenius, quod et pudor tuus moderatur orationi meae et meum 
erga te parentemque tuum beneficium tueri debeo.

Nemo quaeso miretur, si post exsudatos labores itinerum longos congestosque adfatim commeatus fiducia vestri ductante 
barbaricos pagos adventans velut mutato repente consilio ad placidiora deverti.

Ob haec et huius modi multa, quae cernebantur in paucis, omnibus timeri sunt coepta. et ne tot malis dissimulatis paulatimque 
serpentibus acervi crescerent aerumnarum, nobilitatis decreto legati mittuntur: Praetextatus ex urbi praefecto et ex vicario 
Venustus et ex consulari Minervius oraturi, ne delictis supplicia sint grandiora, neve senator quisquam inusitato et inlicito more 
tormentis exponeretur.

Eodem tempore etiam Hymetii praeclarae indolis viri negotium est actitatum, cuius hunc novimus esse textum. cum Africam pro 
consule regeret Carthaginiensibus victus inopia iam lassatis, ex horreis Romano populo destinatis frumentum dedit, pauloque 
postea cum provenisset segetum copia, integre sine ulla restituit mora.


\section{Qu'il y a deux sortes d'élocution, et de l'élocution propre à chaque genre}

Dein Syria per speciosam interpatet diffusa planitiem. hanc nobilitat Antiochia, mundo cognita civitas, cui non certaverit alia 
advecticiis ita adfluere copiis et internis, et Laodicia et Apamia itidemque Seleucia iam inde a primis auspiciis florentissimae.

Ipsam vero urbem Byzantiorum fuisse refertissimam atque ornatissimam signis quis ignorat? Quae illi, exhausti sumptibus bellisque 
maximis, cum omnis Mithridaticos impetus totumque Pontum armatum affervescentem in Asiam atque erumpentem, ore repulsum 
et cervicibus interclusum suis sustinerent, tum, inquam, Byzantii et postea signa illa et reliqua urbis ornanemta sanctissime 
custodita tenuerunt.

Qui cum venisset ob haec festinatis itineribus Antiochiam, praestrictis palatii ianuis, contempto Caesare, quem videri decuerat, ad 
praetorium cum pompa sollemni perrexit morbosque diu causatus nec regiam introiit nec processit in publicum, sed abditus multa in 
eius moliebatur exitium addens quaedam relationibus supervacua, quas subinde dimittebat ad principem.

Atque, ut Tullius ait, ut etiam ferae fame monitae plerumque ad eum locum ubi aliquando pastae sunt revertuntur, ita homines 
instar turbinis degressi montibus impeditis et arduis loca petivere mari confinia, per quae viis latebrosis sese convallibusque 
occultantes cum appeterent noctes luna etiam tum cornuta ideoque nondum solido splendore fulgente nauticos observabant quos 
cum in somnum sentirent effusos per ancoralia, quadrupedo gradu repentes seseque suspensis passibus iniectantes in scaphas 
eisdem sensim nihil opinantibus adsistebant et incendente aviditate saevitiam ne cedentium quidem ulli parcendo obtruncatis 
omnibus merces opimas velut viles nullis repugnantibus avertebant. haecque non diu sunt perpetrata.

Erat autem diritatis eius hoc quoque indicium nec obscurum nec latens, quod ludicris cruentis delectabatur et in circo sex vel 
septem aliquotiens vetitis certaminibus pugilum vicissim se concidentium perfusorumque sanguine specie ut lucratus ingentia 
laetabatur.

Superatis Tauri montis verticibus qui ad solis ortum sublimius attolluntur, Cilicia spatiis porrigitur late distentis dives bonis omnibus 
terra, eiusque lateri dextro adnexa Isauria, pari sorte uberi palmite viget et frugibus minutis, quam mediam navigabile flumen 
Calycadnus interscindit.

Quam quidem partem accusationis admiratus sum et moleste tuli potissimum esse Atratino datam. Neque enim decebat neque 
aetas illa postulabat neque, id quod animadvertere poteratis, pudor patiebatur optimi adulescentis in tali illum oratione versari. 
Vellem aliquis ex vobis robustioribus hunc male dicendi locum suscepisset; aliquanto liberius et fortius et magis more nostro 
refutaremus istam male dicendi licentiam. Tecum, Atratine, agam lenius, quod et pudor tuus moderatur orationi meae et meum 
erga te parentemque tuum beneficium tueri debeo.

Nemo quaeso miretur, si post exsudatos labores itinerum longos congestosque adfatim commeatus fiducia vestri ductante 
barbaricos pagos adventans velut mutato repente consilio ad placidiora deverti.

Ob haec et huius modi multa, quae cernebantur in paucis, omnibus timeri sunt coepta. et ne tot malis dissimulatis paulatimque 
serpentibus acervi crescerent aerumnarum, nobilitatis decreto legati mittuntur: Praetextatus ex urbi praefecto et ex vicario 
Venustus et ex consulari Minervius oraturi, ne delictis supplicia sint grandiora, neve senator quisquam inusitato et inlicito more 
tormentis exponeretur.

Eodem tempore etiam Hymetii praeclarae indolis viri negotium est actitatum, cuius hunc novimus esse textum. cum Africam pro 
consule regeret Carthaginiensibus victus inopia iam lassatis, ex horreis Romano populo destinatis frumentum dedit, pauloque 
postea cum provenisset segetum copia, integre sine ulla restituit mora.


\chapter{Les parties du discours}
\section{Que tout discours, à le bien prendre, n'a que deux parties}

Dein Syria per speciosam interpatet diffusa planitiem. hanc nobilitat Antiochia, mundo cognita civitas, cui non certaverit alia 
advecticiis ita adfluere copiis et internis, et Laodicia et Apamia itidemque Seleucia iam inde a primis auspiciis florentissimae.

Ipsam vero urbem Byzantiorum fuisse refertissimam atque ornatissimam signis quis ignorat? Quae illi, exhausti sumptibus bellisque 
maximis, cum omnis Mithridaticos impetus totumque Pontum armatum affervescentem in Asiam atque erumpentem, ore repulsum 
et cervicibus interclusum suis sustinerent, tum, inquam, Byzantii et postea signa illa et reliqua urbis ornanemta sanctissime 
custodita tenuerunt.

Qui cum venisset ob haec festinatis itineribus Antiochiam, praestrictis palatii ianuis, contempto Caesare, quem videri decuerat, ad 
praetorium cum pompa sollemni perrexit morbosque diu causatus nec regiam introiit nec processit in publicum, sed abditus multa in 
eius moliebatur exitium addens quaedam relationibus supervacua, quas subinde dimittebat ad principem.

Atque, ut Tullius ait, ut etiam ferae fame monitae plerumque ad eum locum ubi aliquando pastae sunt revertuntur, ita homines 
instar turbinis degressi montibus impeditis et arduis loca petivere mari confinia, per quae viis latebrosis sese convallibusque 
occultantes cum appeterent noctes luna etiam tum cornuta ideoque nondum solido splendore fulgente nauticos observabant quos 
cum in somnum sentirent effusos per ancoralia, quadrupedo gradu repentes seseque suspensis passibus iniectantes in scaphas 
eisdem sensim nihil opinantibus adsistebant et incendente aviditate saevitiam ne cedentium quidem ulli parcendo obtruncatis 
omnibus merces opimas velut viles nullis repugnantibus avertebant. haecque non diu sunt perpetrata.

Erat autem diritatis eius hoc quoque indicium nec obscurum nec latens, quod ludicris cruentis delectabatur et in circo sex vel 
septem aliquotiens vetitis certaminibus pugilum vicissim se concidentium perfusorumque sanguine specie ut lucratus ingentia 
laetabatur.

Superatis Tauri montis verticibus qui ad solis ortum sublimius attolluntur, Cilicia spatiis porrigitur late distentis dives bonis omnibus 
terra, eiusque lateri dextro adnexa Isauria, pari sorte uberi palmite viget et frugibus minutis, quam mediam navigabile flumen 
Calycadnus interscindit.

Quam quidem partem accusationis admiratus sum et moleste tuli potissimum esse Atratino datam. Neque enim decebat neque 
aetas illa postulabat neque, id quod animadvertere poteratis, pudor patiebatur optimi adulescentis in tali illum oratione versari. 
Vellem aliquis ex vobis robustioribus hunc male dicendi locum suscepisset; aliquanto liberius et fortius et magis more nostro 
refutaremus istam male dicendi licentiam. Tecum, Atratine, agam lenius, quod et pudor tuus moderatur orationi meae et meum 
erga te parentemque tuum beneficium tueri debeo.

Nemo quaeso miretur, si post exsudatos labores itinerum longos congestosque adfatim commeatus fiducia vestri ductante 
barbaricos pagos adventans velut mutato repente consilio ad placidiora deverti.

Ob haec et huius modi multa, quae cernebantur in paucis, omnibus timeri sunt coepta. et ne tot malis dissimulatis paulatimque 
serpentibus acervi crescerent aerumnarum, nobilitatis decreto legati mittuntur: Praetextatus ex urbi praefecto et ex vicario 
Venustus et ex consulari Minervius oraturi, ne delictis supplicia sint grandiora, neve senator quisquam inusitato et inlicito more 
tormentis exponeretur.

Eodem tempore etiam Hymetii praeclarae indolis viri negotium est actitatum, cuius hunc novimus esse textum. cum Africam pro 
consule regeret Carthaginiensibus victus inopia iam lassatis, ex horreis Romano populo destinatis frumentum dedit, pauloque 
postea cum provenisset segetum copia, integre sine ulla restituit mora.


\section{De l'exorde}

Dein Syria per speciosam interpatet diffusa planitiem. hanc nobilitat Antiochia, mundo cognita civitas, cui non certaverit alia 
advecticiis ita adfluere copiis et internis, et Laodicia et Apamia itidemque Seleucia iam inde a primis auspiciis florentissimae.

Ipsam vero urbem Byzantiorum fuisse refertissimam atque ornatissimam signis quis ignorat? Quae illi, exhausti sumptibus bellisque 
maximis, cum omnis Mithridaticos impetus totumque Pontum armatum affervescentem in Asiam atque erumpentem, ore repulsum 
et cervicibus interclusum suis sustinerent, tum, inquam, Byzantii et postea signa illa et reliqua urbis ornanemta sanctissime 
custodita tenuerunt.

Qui cum venisset ob haec festinatis itineribus Antiochiam, praestrictis palatii ianuis, contempto Caesare, quem videri decuerat, ad 
praetorium cum pompa sollemni perrexit morbosque diu causatus nec regiam introiit nec processit in publicum, sed abditus multa in 
eius moliebatur exitium addens quaedam relationibus supervacua, quas subinde dimittebat ad principem.

Atque, ut Tullius ait, ut etiam ferae fame monitae plerumque ad eum locum ubi aliquando pastae sunt revertuntur, ita homines 
instar turbinis degressi montibus impeditis et arduis loca petivere mari confinia, per quae viis latebrosis sese convallibusque 
occultantes cum appeterent noctes luna etiam tum cornuta ideoque nondum solido splendore fulgente nauticos observabant quos 
cum in somnum sentirent effusos per ancoralia, quadrupedo gradu repentes seseque suspensis passibus iniectantes in scaphas 
eisdem sensim nihil opinantibus adsistebant et incendente aviditate saevitiam ne cedentium quidem ulli parcendo obtruncatis 
omnibus merces opimas velut viles nullis repugnantibus avertebant. haecque non diu sunt perpetrata.

Erat autem diritatis eius hoc quoque indicium nec obscurum nec latens, quod ludicris cruentis delectabatur et in circo sex vel 
septem aliquotiens vetitis certaminibus pugilum vicissim se concidentium perfusorumque sanguine specie ut lucratus ingentia 
laetabatur.

Superatis Tauri montis verticibus qui ad solis ortum sublimius attolluntur, Cilicia spatiis porrigitur late distentis dives bonis omnibus 
terra, eiusque lateri dextro adnexa Isauria, pari sorte uberi palmite viget et frugibus minutis, quam mediam navigabile flumen 
Calycadnus interscindit.

Quam quidem partem accusationis admiratus sum et moleste tuli potissimum esse Atratino datam. Neque enim decebat neque 
aetas illa postulabat neque, id quod animadvertere poteratis, pudor patiebatur optimi adulescentis in tali illum oratione versari. 
Vellem aliquis ex vobis robustioribus hunc male dicendi locum suscepisset; aliquanto liberius et fortius et magis more nostro 
refutaremus istam male dicendi licentiam. Tecum, Atratine, agam lenius, quod et pudor tuus moderatur orationi meae et meum 
erga te parentemque tuum beneficium tueri debeo.

Nemo quaeso miretur, si post exsudatos labores itinerum longos congestosque adfatim commeatus fiducia vestri ductante 
barbaricos pagos adventans velut mutato repente consilio ad placidiora deverti.

Ob haec et huius modi multa, quae cernebantur in paucis, omnibus timeri sunt coepta. et ne tot malis dissimulatis paulatimque 
serpentibus acervi crescerent aerumnarum, nobilitatis decreto legati mittuntur: Praetextatus ex urbi praefecto et ex vicario 
Venustus et ex consulari Minervius oraturi, ne delictis supplicia sint grandiora, neve senator quisquam inusitato et inlicito more 
tormentis exponeretur.

Eodem tempore etiam Hymetii praeclarae indolis viri negotium est actitatum, cuius hunc novimus esse textum. cum Africam pro 
consule regeret Carthaginiensibus victus inopia iam lassatis, ex horreis Romano populo destinatis frumentum dedit, pauloque 
postea cum provenisset segetum copia, integre sine ulla restituit mora.


\section{Pour se défendre dans une accusation}

Dein Syria per speciosam interpatet diffusa planitiem. hanc nobilitat Antiochia, mundo cognita civitas, cui non certaverit alia 
advecticiis ita adfluere copiis et internis, et Laodicia et Apamia itidemque Seleucia iam inde a primis auspiciis florentissimae.

Ipsam vero urbem Byzantiorum fuisse refertissimam atque ornatissimam signis quis ignorat? Quae illi, exhausti sumptibus bellisque 
maximis, cum omnis Mithridaticos impetus totumque Pontum armatum affervescentem in Asiam atque erumpentem, ore repulsum 
et cervicibus interclusum suis sustinerent, tum, inquam, Byzantii et postea signa illa et reliqua urbis ornanemta sanctissime 
custodita tenuerunt.

Qui cum venisset ob haec festinatis itineribus Antiochiam, praestrictis palatii ianuis, contempto Caesare, quem videri decuerat, ad 
praetorium cum pompa sollemni perrexit morbosque diu causatus nec regiam introiit nec processit in publicum, sed abditus multa in 
eius moliebatur exitium addens quaedam relationibus supervacua, quas subinde dimittebat ad principem.

Atque, ut Tullius ait, ut etiam ferae fame monitae plerumque ad eum locum ubi aliquando pastae sunt revertuntur, ita homines 
instar turbinis degressi montibus impeditis et arduis loca petivere mari confinia, per quae viis latebrosis sese convallibusque 
occultantes cum appeterent noctes luna etiam tum cornuta ideoque nondum solido splendore fulgente nauticos observabant quos 
cum in somnum sentirent effusos per ancoralia, quadrupedo gradu repentes seseque suspensis passibus iniectantes in scaphas 
eisdem sensim nihil opinantibus adsistebant et incendente aviditate saevitiam ne cedentium quidem ulli parcendo obtruncatis 
omnibus merces opimas velut viles nullis repugnantibus avertebant. haecque non diu sunt perpetrata.

Erat autem diritatis eius hoc quoque indicium nec obscurum nec latens, quod ludicris cruentis delectabatur et in circo sex vel 
septem aliquotiens vetitis certaminibus pugilum vicissim se concidentium perfusorumque sanguine specie ut lucratus ingentia 
laetabatur.

Superatis Tauri montis verticibus qui ad solis ortum sublimius attolluntur, Cilicia spatiis porrigitur late distentis dives bonis omnibus 
terra, eiusque lateri dextro adnexa Isauria, pari sorte uberi palmite viget et frugibus minutis, quam mediam navigabile flumen 
Calycadnus interscindit.

Quam quidem partem accusationis admiratus sum et moleste tuli potissimum esse Atratino datam. Neque enim decebat neque 
aetas illa postulabat neque, id quod animadvertere poteratis, pudor patiebatur optimi adulescentis in tali illum oratione versari. 
Vellem aliquis ex vobis robustioribus hunc male dicendi locum suscepisset; aliquanto liberius et fortius et magis more nostro 
refutaremus istam male dicendi licentiam. Tecum, Atratine, agam lenius, quod et pudor tuus moderatur orationi meae et meum 
erga te parentemque tuum beneficium tueri debeo.

Nemo quaeso miretur, si post exsudatos labores itinerum longos congestosque adfatim commeatus fiducia vestri ductante 
barbaricos pagos adventans velut mutato repente consilio ad placidiora deverti.

Ob haec et huius modi multa, quae cernebantur in paucis, omnibus timeri sunt coepta. et ne tot malis dissimulatis paulatimque 
serpentibus acervi crescerent aerumnarum, nobilitatis decreto legati mittuntur: Praetextatus ex urbi praefecto et ex vicario 
Venustus et ex consulari Minervius oraturi, ne delictis supplicia sint grandiora, neve senator quisquam inusitato et inlicito more 
tormentis exponeretur.

Eodem tempore etiam Hymetii praeclarae indolis viri negotium est actitatum, cuius hunc novimus esse textum. cum Africam pro 
consule regeret Carthaginiensibus victus inopia iam lassatis, ex horreis Romano populo destinatis frumentum dedit, pauloque 
postea cum provenisset segetum copia, integre sine ulla restituit mora.


\section{De la narration}

Dein Syria per speciosam interpatet diffusa planitiem. hanc nobilitat Antiochia, mundo cognita civitas, cui non certaverit alia 
advecticiis ita adfluere copiis et internis, et Laodicia et Apamia itidemque Seleucia iam inde a primis auspiciis florentissimae.

Ipsam vero urbem Byzantiorum fuisse refertissimam atque ornatissimam signis quis ignorat? Quae illi, exhausti sumptibus bellisque 
maximis, cum omnis Mithridaticos impetus totumque Pontum armatum affervescentem in Asiam atque erumpentem, ore repulsum 
et cervicibus interclusum suis sustinerent, tum, inquam, Byzantii et postea signa illa et reliqua urbis ornanemta sanctissime 
custodita tenuerunt.

Qui cum venisset ob haec festinatis itineribus Antiochiam, praestrictis palatii ianuis, contempto Caesare, quem videri decuerat, ad 
praetorium cum pompa sollemni perrexit morbosque diu causatus nec regiam introiit nec processit in publicum, sed abditus multa in 
eius moliebatur exitium addens quaedam relationibus supervacua, quas subinde dimittebat ad principem.

Atque, ut Tullius ait, ut etiam ferae fame monitae plerumque ad eum locum ubi aliquando pastae sunt revertuntur, ita homines 
instar turbinis degressi montibus impeditis et arduis loca petivere mari confinia, per quae viis latebrosis sese convallibusque 
occultantes cum appeterent noctes luna etiam tum cornuta ideoque nondum solido splendore fulgente nauticos observabant quos 
cum in somnum sentirent effusos per ancoralia, quadrupedo gradu repentes seseque suspensis passibus iniectantes in scaphas 
eisdem sensim nihil opinantibus adsistebant et incendente aviditate saevitiam ne cedentium quidem ulli parcendo obtruncatis 
omnibus merces opimas velut viles nullis repugnantibus avertebant. haecque non diu sunt perpetrata.

Erat autem diritatis eius hoc quoque indicium nec obscurum nec latens, quod ludicris cruentis delectabatur et in circo sex vel 
septem aliquotiens vetitis certaminibus pugilum vicissim se concidentium perfusorumque sanguine specie ut lucratus ingentia 
laetabatur.

Superatis Tauri montis verticibus qui ad solis ortum sublimius attolluntur, Cilicia spatiis porrigitur late distentis dives bonis omnibus 
terra, eiusque lateri dextro adnexa Isauria, pari sorte uberi palmite viget et frugibus minutis, quam mediam navigabile flumen 
Calycadnus interscindit.

Quam quidem partem accusationis admiratus sum et moleste tuli potissimum esse Atratino datam. Neque enim decebat neque 
aetas illa postulabat neque, id quod animadvertere poteratis, pudor patiebatur optimi adulescentis in tali illum oratione versari. 
Vellem aliquis ex vobis robustioribus hunc male dicendi locum suscepisset; aliquanto liberius et fortius et magis more nostro 
refutaremus istam male dicendi licentiam. Tecum, Atratine, agam lenius, quod et pudor tuus moderatur orationi meae et meum 
erga te parentemque tuum beneficium tueri debeo.

Nemo quaeso miretur, si post exsudatos labores itinerum longos congestosque adfatim commeatus fiducia vestri ductante 
barbaricos pagos adventans velut mutato repente consilio ad placidiora deverti.

Ob haec et huius modi multa, quae cernebantur in paucis, omnibus timeri sunt coepta. et ne tot malis dissimulatis paulatimque 
serpentibus acervi crescerent aerumnarum, nobilitatis decreto legati mittuntur: Praetextatus ex urbi praefecto et ex vicario 
Venustus et ex consulari Minervius oraturi, ne delictis supplicia sint grandiora, neve senator quisquam inusitato et inlicito more 
tormentis exponeretur.

Eodem tempore etiam Hymetii praeclarae indolis viri negotium est actitatum, cuius hunc novimus esse textum. cum Africam pro 
consule regeret Carthaginiensibus victus inopia iam lassatis, ex horreis Romano populo destinatis frumentum dedit, pauloque 
postea cum provenisset segetum copia, integre sine ulla restituit mora.


\section{De la preuve, et de la réfutation}

Dein Syria per speciosam interpatet diffusa planitiem. hanc nobilitat Antiochia, mundo cognita civitas, cui non certaverit alia 
advecticiis ita adfluere copiis et internis, et Laodicia et Apamia itidemque Seleucia iam inde a primis auspiciis florentissimae.

Ipsam vero urbem Byzantiorum fuisse refertissimam atque ornatissimam signis quis ignorat? Quae illi, exhausti sumptibus bellisque 
maximis, cum omnis Mithridaticos impetus totumque Pontum armatum affervescentem in Asiam atque erumpentem, ore repulsum 
et cervicibus interclusum suis sustinerent, tum, inquam, Byzantii et postea signa illa et reliqua urbis ornanemta sanctissime 
custodita tenuerunt.

Qui cum venisset ob haec festinatis itineribus Antiochiam, praestrictis palatii ianuis, contempto Caesare, quem videri decuerat, ad 
praetorium cum pompa sollemni perrexit morbosque diu causatus nec regiam introiit nec processit in publicum, sed abditus multa in 
eius moliebatur exitium addens quaedam relationibus supervacua, quas subinde dimittebat ad principem.

Atque, ut Tullius ait, ut etiam ferae fame monitae plerumque ad eum locum ubi aliquando pastae sunt revertuntur, ita homines 
instar turbinis degressi montibus impeditis et arduis loca petivere mari confinia, per quae viis latebrosis sese convallibusque 
occultantes cum appeterent noctes luna etiam tum cornuta ideoque nondum solido splendore fulgente nauticos observabant quos 
cum in somnum sentirent effusos per ancoralia, quadrupedo gradu repentes seseque suspensis passibus iniectantes in scaphas 
eisdem sensim nihil opinantibus adsistebant et incendente aviditate saevitiam ne cedentium quidem ulli parcendo obtruncatis 
omnibus merces opimas velut viles nullis repugnantibus avertebant. haecque non diu sunt perpetrata.

Erat autem diritatis eius hoc quoque indicium nec obscurum nec latens, quod ludicris cruentis delectabatur et in circo sex vel 
septem aliquotiens vetitis certaminibus pugilum vicissim se concidentium perfusorumque sanguine specie ut lucratus ingentia 
laetabatur.

Superatis Tauri montis verticibus qui ad solis ortum sublimius attolluntur, Cilicia spatiis porrigitur late distentis dives bonis omnibus 
terra, eiusque lateri dextro adnexa Isauria, pari sorte uberi palmite viget et frugibus minutis, quam mediam navigabile flumen 
Calycadnus interscindit.

Quam quidem partem accusationis admiratus sum et moleste tuli potissimum esse Atratino datam. Neque enim decebat neque 
aetas illa postulabat neque, id quod animadvertere poteratis, pudor patiebatur optimi adulescentis in tali illum oratione versari. 
Vellem aliquis ex vobis robustioribus hunc male dicendi locum suscepisset; aliquanto liberius et fortius et magis more nostro 
refutaremus istam male dicendi licentiam. Tecum, Atratine, agam lenius, quod et pudor tuus moderatur orationi meae et meum 
erga te parentemque tuum beneficium tueri debeo.

Nemo quaeso miretur, si post exsudatos labores itinerum longos congestosque adfatim commeatus fiducia vestri ductante 
barbaricos pagos adventans velut mutato repente consilio ad placidiora deverti.

Ob haec et huius modi multa, quae cernebantur in paucis, omnibus timeri sunt coepta. et ne tot malis dissimulatis paulatimque 
serpentibus acervi crescerent aerumnarum, nobilitatis decreto legati mittuntur: Praetextatus ex urbi praefecto et ex vicario 
Venustus et ex consulari Minervius oraturi, ne delictis supplicia sint grandiora, neve senator quisquam inusitato et inlicito more 
tormentis exponeretur.

Eodem tempore etiam Hymetii praeclarae indolis viri negotium est actitatum, cuius hunc novimus esse textum. cum Africam pro 
consule regeret Carthaginiensibus victus inopia iam lassatis, ex horreis Romano populo destinatis frumentum dedit, pauloque 
postea cum provenisset segetum copia, integre sine ulla restituit mora.


\section{De l'interrogation, et pour y répondre}

Dein Syria per speciosam interpatet diffusa planitiem. hanc nobilitat Antiochia, mundo cognita civitas, cui non certaverit alia 
advecticiis ita adfluere copiis et internis, et Laodicia et Apamia itidemque Seleucia iam inde a primis auspiciis florentissimae.

Ipsam vero urbem Byzantiorum fuisse refertissimam atque ornatissimam signis quis ignorat? Quae illi, exhausti sumptibus bellisque 
maximis, cum omnis Mithridaticos impetus totumque Pontum armatum affervescentem in Asiam atque erumpentem, ore repulsum 
et cervicibus interclusum suis sustinerent, tum, inquam, Byzantii et postea signa illa et reliqua urbis ornanemta sanctissime 
custodita tenuerunt.

Qui cum venisset ob haec festinatis itineribus Antiochiam, praestrictis palatii ianuis, contempto Caesare, quem videri decuerat, ad 
praetorium cum pompa sollemni perrexit morbosque diu causatus nec regiam introiit nec processit in publicum, sed abditus multa in 
eius moliebatur exitium addens quaedam relationibus supervacua, quas subinde dimittebat ad principem.

Atque, ut Tullius ait, ut etiam ferae fame monitae plerumque ad eum locum ubi aliquando pastae sunt revertuntur, ita homines 
instar turbinis degressi montibus impeditis et arduis loca petivere mari confinia, per quae viis latebrosis sese convallibusque 
occultantes cum appeterent noctes luna etiam tum cornuta ideoque nondum solido splendore fulgente nauticos observabant quos 
cum in somnum sentirent effusos per ancoralia, quadrupedo gradu repentes seseque suspensis passibus iniectantes in scaphas 
eisdem sensim nihil opinantibus adsistebant et incendente aviditate saevitiam ne cedentium quidem ulli parcendo obtruncatis 
omnibus merces opimas velut viles nullis repugnantibus avertebant. haecque non diu sunt perpetrata.

Erat autem diritatis eius hoc quoque indicium nec obscurum nec latens, quod ludicris cruentis delectabatur et in circo sex vel 
septem aliquotiens vetitis certaminibus pugilum vicissim se concidentium perfusorumque sanguine specie ut lucratus ingentia 
laetabatur.

Superatis Tauri montis verticibus qui ad solis ortum sublimius attolluntur, Cilicia spatiis porrigitur late distentis dives bonis omnibus 
terra, eiusque lateri dextro adnexa Isauria, pari sorte uberi palmite viget et frugibus minutis, quam mediam navigabile flumen 
Calycadnus interscindit.

Quam quidem partem accusationis admiratus sum et moleste tuli potissimum esse Atratino datam. Neque enim decebat neque 
aetas illa postulabat neque, id quod animadvertere poteratis, pudor patiebatur optimi adulescentis in tali illum oratione versari. 
Vellem aliquis ex vobis robustioribus hunc male dicendi locum suscepisset; aliquanto liberius et fortius et magis more nostro 
refutaremus istam male dicendi licentiam. Tecum, Atratine, agam lenius, quod et pudor tuus moderatur orationi meae et meum 
erga te parentemque tuum beneficium tueri debeo.

Nemo quaeso miretur, si post exsudatos labores itinerum longos congestosque adfatim commeatus fiducia vestri ductante 
barbaricos pagos adventans velut mutato repente consilio ad placidiora deverti.

Ob haec et huius modi multa, quae cernebantur in paucis, omnibus timeri sunt coepta. et ne tot malis dissimulatis paulatimque 
serpentibus acervi crescerent aerumnarum, nobilitatis decreto legati mittuntur: Praetextatus ex urbi praefecto et ex vicario 
Venustus et ex consulari Minervius oraturi, ne delictis supplicia sint grandiora, neve senator quisquam inusitato et inlicito more 
tormentis exponeretur.

Eodem tempore etiam Hymetii praeclarae indolis viri negotium est actitatum, cuius hunc novimus esse textum. cum Africam pro 
consule regeret Carthaginiensibus victus inopia iam lassatis, ex horreis Romano populo destinatis frumentum dedit, pauloque 
postea cum provenisset segetum copia, integre sine ulla restituit mora.


\section{La péroraison}

Dein Syria per speciosam interpatet diffusa planitiem. hanc nobilitat Antiochia, mundo cognita civitas, cui non certaverit alia 
advecticiis ita adfluere copiis et internis, et Laodicia et Apamia itidemque Seleucia iam inde a primis auspiciis florentissimae.

Ipsam vero urbem Byzantiorum fuisse refertissimam atque ornatissimam signis quis ignorat? Quae illi, exhausti sumptibus bellisque 
maximis, cum omnis Mithridaticos impetus totumque Pontum armatum affervescentem in Asiam atque erumpentem, ore repulsum 
et cervicibus interclusum suis sustinerent, tum, inquam, Byzantii et postea signa illa et reliqua urbis ornanemta sanctissime 
custodita tenuerunt.

Qui cum venisset ob haec festinatis itineribus Antiochiam, praestrictis palatii ianuis, contempto Caesare, quem videri decuerat, ad 
praetorium cum pompa sollemni perrexit morbosque diu causatus nec regiam introiit nec processit in publicum, sed abditus multa in 
eius moliebatur exitium addens quaedam relationibus supervacua, quas subinde dimittebat ad principem.

Atque, ut Tullius ait, ut etiam ferae fame monitae plerumque ad eum locum ubi aliquando pastae sunt revertuntur, ita homines 
instar turbinis degressi montibus impeditis et arduis loca petivere mari confinia, per quae viis latebrosis sese convallibusque 
occultantes cum appeterent noctes luna etiam tum cornuta ideoque nondum solido splendore fulgente nauticos observabant quos 
cum in somnum sentirent effusos per ancoralia, quadrupedo gradu repentes seseque suspensis passibus iniectantes in scaphas 
eisdem sensim nihil opinantibus adsistebant et incendente aviditate saevitiam ne cedentium quidem ulli parcendo obtruncatis 
omnibus merces opimas velut viles nullis repugnantibus avertebant. haecque non diu sunt perpetrata.

Erat autem diritatis eius hoc quoque indicium nec obscurum nec latens, quod ludicris cruentis delectabatur et in circo sex vel 
septem aliquotiens vetitis certaminibus pugilum vicissim se concidentium perfusorumque sanguine specie ut lucratus ingentia 
laetabatur.

Superatis Tauri montis verticibus qui ad solis ortum sublimius attolluntur, Cilicia spatiis porrigitur late distentis dives bonis omnibus 
terra, eiusque lateri dextro adnexa Isauria, pari sorte uberi palmite viget et frugibus minutis, quam mediam navigabile flumen 
Calycadnus interscindit.

Quam quidem partem accusationis admiratus sum et moleste tuli potissimum esse Atratino datam. Neque enim decebat neque 
aetas illa postulabat neque, id quod animadvertere poteratis, pudor patiebatur optimi adulescentis in tali illum oratione versari. 
Vellem aliquis ex vobis robustioribus hunc male dicendi locum suscepisset; aliquanto liberius et fortius et magis more nostro 
refutaremus istam male dicendi licentiam. Tecum, Atratine, agam lenius, quod et pudor tuus moderatur orationi meae et meum 
erga te parentemque tuum beneficium tueri debeo.

Nemo quaeso miretur, si post exsudatos labores itinerum longos congestosque adfatim commeatus fiducia vestri ductante 
barbaricos pagos adventans velut mutato repente consilio ad placidiora deverti.

Ob haec et huius modi multa, quae cernebantur in paucis, omnibus timeri sunt coepta. et ne tot malis dissimulatis paulatimque 
serpentibus acervi crescerent aerumnarum, nobilitatis decreto legati mittuntur: Praetextatus ex urbi praefecto et ex vicario 
Venustus et ex consulari Minervius oraturi, ne delictis supplicia sint grandiora, neve senator quisquam inusitato et inlicito more 
tormentis exponeretur.

Eodem tempore etiam Hymetii praeclarae indolis viri negotium est actitatum, cuius hunc novimus esse textum. cum Africam pro 
consule regeret Carthaginiensibus victus inopia iam lassatis, ex horreis Romano populo destinatis frumentum dedit, pauloque 
postea cum provenisset segetum copia, integre sine ulla restituit mora.


\end{comment}


\end{document}